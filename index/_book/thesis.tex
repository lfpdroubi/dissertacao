% ------------------------------------------------------------------------
% ------------------------------------------------------------------------
% Modelo UFSC para Trabalhos Academicos (tese de doutorado, dissertação de
% mestrado) utilizando a classe abntex2
%
% Autor: Alisson Lopes Furlani
% 	Modificações:
%	- 27/08/2019: Alisson L. Furlani, add pacote 'glossaries' para listas
% - 30/10/2019: Alisson L. Furlani, adjusted some spacing errors and changed math fonts
% - 17/01/2019: Alisson L. Furlani, updated certification page
% - 03/03/2020: Luiz F. P. Droubi, change file to be used as a template with R.
% ------------------------------------------------------------------------
% ------------------------------------------------------------------------

\documentclass[
	% -- opções da classe memoir --
	12pt,				% tamanho da fonte
	%openright,			% capítulos começam em pág ímpar (insere página vazia caso preciso)
	oneside,			% para impressão no anverso. Oposto a twoside
	a4paper,			% tamanho do papel.
	% -- opções da classe abntex2 --
	chapter=TITLE,		% títulos de capítulos convertidos em letras maiúsculas
	section=TITLE,		% títulos de seções convertidos em letras maiúsculas
	%subsection=TITLE,	% títulos de subseções convertidos em letras maiúsculas
	%subsubsection=TITLE,% títulos de subsubseções convertidos em letras maiúsculas
	% -- opções do pacote babel --
	english,			% idioma adicional para hifenização
	%french,				% idioma adicional para hifenização
	%spanish,			% idioma adicional para hifenização
	brazil				% o último idioma é o principal do documento
	]{abntex2}

\usepackage{setup/ufscthesisA4-alf}

\addbibresource{bib/cap1.bib}
\addbibresource{bib/cap2.bib}
\addbibresource{bib/cap3.bib}
\addbibresource{bib/cap4.bib}
\addbibresource{bib/cap5.bib}
\addbibresource{bib/cap6.bib}
\addbibresource{bib/pkgs.bib}

\usepackage[table]{xcolor}
\let\newfloat\undefined
\usepackage{floatrow}
\floatsetup[table]{capposition=top}
\floatsetup[figure]{capposition=top}

\newcommand{\pkg}[1]{{\normalfont\fontseries{b}\selectfont #1}}
\let\proglang=\textsf
\let\code=\texttt


\newcommand{\bcenter}{\begin{center}}
\newcommand{\ecenter}{\end{center}}

\newcommand{\bapendices}{\begin{apendicesenv}}
\newcommand{\eapendices}{\end{apendicesenv}}

\newcommand{\banexos}{\begin{anexosenv}}
\newcommand{\eanexos}{\end{anexosenv}}

% ---
% Filtering and Mapping Bibliographies
% ---
\DeclareSourcemap{
	\maps[datatype=bibtex]{
		% remove fields that are always useless
		\map{
			\step[fieldset=abstract, null]
			\step[fieldset=pagetotal, null]
		}
		% remove URLs for types that are primarily printed
%		\map{
%			\pernottype{software}
%			\pernottype{online}
%			\pernottype{report}
%			\pernottype{techreport}
%			\pernottype{standard}
%			\pernottype{manual}
%			\pernottype{misc}
%			\step[fieldset=url, null]
%			\step[fieldset=urldate, null]
%		}
		\map{
			\pertype{inproceedings}
			% remove mostly redundant conference information
			\step[fieldset=venue, null]
			\step[fieldset=eventdate, null]
			\step[fieldset=eventtitle, null]
			% do not show ISBN for proceedings
			\step[fieldset=isbn, null]
			% Citavi bug
			\step[fieldset=volume, null]
		}
	}
}
% ---

% ---
% Informações de dados para CAPA e FOLHA DE ROSTO
% ---
% FIXME Substituir 'Nome completo do autor' pelo seu nome.
\autor{LUIZ FERNANDO PALIN DROUBI}
% FIXME Substituir 'Título do trabalho' pelo título da trabalho.
\titulo{O Mercado Imobiliário e a Economia}
% FIXME Substituir 'Subtítulo (se houver)' pelo subtítulo da trabalho.
% Caso não tenha substítulo, comente a linha a seguir.
  \subtitulo{Qualificação}
% FIXME Substituir 'XXXXXX' pelo nome do seu
% orientador.
\orientador{Norberto Hochheim}
% FIXME Se for orientado por uma mulher, comente a linha acima e descomente a linha a seguir.
% \orientador[Orientadora]{Nome da orientadora, Dra.}
% FIXME Substituir 'XXXXXX' pelo nome do seu
% coorientador. Caso não tenha coorientador, comente a linha a seguir.
% FIXME Se for coorientado por uma mulher, comente a linha acima e descomente a linha a seguir.
% \coorientador[Coorientadora]{XXXXXX, Dra.}
% FIXME Substituir '[ano]' pelo ano (ano) em que seu trabalho foi defendido.
\ano{2020}
% FIXME Substituir '[dia] de [mês] de [ano]' pela data em que ocorreu sua defesa.
\data{31 de Março de 2020}
% FIXME Substituir 'Local' pela cidade em que ocorreu sua defesa.
\local{Florianópolis}
\instituicaosigla{UFSC}
\instituicao{Universidade Federal de Santa Catarina}
% FIXME Substituir 'Dissertação/Tese' pelo tipo de trabalho (Tese, Dissertação).
\tipotrabalho{Dissertação}
% FIXME Substituir '[mestre/doutor] em XXXXXX' pela grau adequado.
\formacao{Mestre em Engenharia de Transportes e Gestão Territorial}
% FIXME Substituir '[mestrado/doutorado]' pelo nivel adequado.
\nivel{mestrado}
% FIXME Substituir 'Programa de Pós-Graduação em XXXXXX' pela curso adequado.
\programa{Programa de Pós-Graduação em Engenharia de Transportes e Gestão Territorial}
% FIXME Substituir 'Campus XXXXXX ou Centro de XXXXXX' pelo campus ou centro adequado.
\centro{CTC - CENTRO TECNOLÓGICO}
\preambulo
{%
\imprimirtipotrabalho~submetida~ao~\imprimirprograma~da~\imprimirinstituicao~para~a~obtenção~do~título~de~\imprimirformacao.
}
% ---

% ---
% Configurações de aparência do PDF final
% ---
% alterando o aspecto da cor azul
\definecolor{blue}{RGB}{41,5,195}
% informações do PDF
\makeatletter
\hypersetup{
     	%pagebackref=true,
		pdftitle={\@title},
		pdfauthor={\@author},
    	pdfsubject={\imprimirpreambulo},
	    pdfcreator={LaTeX with abnTeX2},
		pdfkeywords={ufsc, latex, abntex2},
		colorlinks=true,       		% false: boxed links; true: colored links
    	linkcolor=black,%blue,          	% color of internal links
    	citecolor=black,%blue,        		% color of links to bibliography
    	filecolor=black,%magenta,      		% color of file links
		urlcolor=black,%blue,
		bookmarksdepth=4
}
\makeatother
% ---

% ---
% compila a lista de abreviaturas e siglas e a lista de símbolos
% ---

% Declaração das siglas
\siglalista{ABNT}{Associação Brasileira de Normas Técnicas}
\siglalista{Bacen}{Banco Central do Brasil}
\siglalista{FED}{Federal Reserve Bank}
\siglalista{FRED}{Federal Reserve Economic Data}
\siglalista{FMI}{Fundo Monetário Internacional}
\siglalista{CDO}{Collateralized Debt Obligation}
\siglalista{CMO}{Collateralized Mortgage Obligation}
\siglalista{ABECIP}{Associação Brasileira das Entidades de Crédito Imobiliário e Poupança}
\siglalista{IBRE}{Instituto Brasileiro de Economia}
\siglalista{FGV}{Fundação Getúlio Vargas}
\siglalista{SPV}{Special-purpose Vehicle}
\siglalista{ABS}{Asset Backed Security}
\siglalista{MBS}{Mortgage Backed Security}
\siglalista{CDS}{Credit Default Swaps}


% Declaração dos simbolos
\simbololista{R_e}{\ensuremath{R_e}}{Taxa mínima de atratividade}
\simbololista{R_f}{\ensuremath{R_f}}{Taxa livre de risco}
\simbololista{R_m}{\ensuremath{R_m}}{Taxa de risco do mercado}
\simbololista{hp}{\ensuremath{hp}}{Preço dos imóveis (índice)}
\simbololista{long}{\ensuremath{long}}{Taxa de juros de longo prazo}
\simbololista{short}{\ensuremath{short}}{Taxa de juros de curto prazo}
\simbololista{constr}{\ensuremath{constr}}{Nível dos custos de construção}
\simbololista{EA}{\ensuremath{EA}}{Nível de atividade econômica}
\simbololista{D}{\ensuremath{D}}{Nível de Demanda}
\simbololista{S}{\ensuremath{S}}{Nível de Oferta}
\simbololista{R}{\ensuremath{R}}{Renda (bruta) recorrente auferida com aluguel do imóvel}
\simbololista{R_L}{\ensuremath{R_L}}{Renda líquida recorrente auferida com aluguel do imóvel}
\simbololista{C}{\ensuremath{C}}{Custos recorrentes de manutenção do imóvel}
\simbololista{y_r}{\ensuremath{y_r}}{Taxa de rendimento do aluguel de um imóvel}

% Glossário
\newglossaryentry{glo:subprime}
{
  name=subprime,
  description={is a programmable machine that receives input,
               stores and manipulates data, and provides
               output in a useful format}
}

% compila a lista de abreviaturas e siglas e a lista de símbolos
\makenoidxglossaries

% ---

% ---
% compila o indice
% ---
\makeindex
% ---

\DeclareRobustCommand{\firstsecond}[2]{#1}

% ----
% Início do documento
% ----
\begin{document}

% Seleciona o idioma do documento (conforme pacotes do babel)
%\selectlanguage{english}
\selectlanguage{brazil}

% Retira espaço extra obsoleto entre as frases.
\frenchspacing

% Espaçamento 1.5 entre linhas
\OnehalfSpacing

% Corrige justificação
%\sloppy

% ----------------------------------------------------------
% ELEMENTOS PRÉ-TEXTUAIS
% ----------------------------------------------------------
% \pretextual %a macro \pretextual é acionado automaticamente no início de \begin{document}
% ---
% Capa, folha de rosto, ficha bibliografica, errata, folha de apróvação
% Dedicatória, agradecimentos, epígrafe, resumos, listas
% ---
% ---
% Capa
% ---
\imprimircapa
% ---

% ---
% Folha de rosto
% (o * indica que haverá a ficha bibliográfica)
% ---
\imprimirfolhaderosto*
% ---

% ---
% Inserir a ficha bibliografica
% ---
% http://ficha.bu.ufsc.br/
\begin{fichacatalografica}
	\includepdf{Ficha_Catalografica.pdf}
\end{fichacatalografica}
% ---

% ---
% Inserir folha de aprovação
% ---
\begin{folhadeaprovacao}
	\OnehalfSpacing
	\centering
	\imprimirautor\\%
	\vspace*{10pt}		
	\textbf{\imprimirtitulo}%
	\ifnotempty{\imprimirsubtitulo}{:~\imprimirsubtitulo}\\%
	%		\vspace*{31.5pt}%3\baselineskip
	\vspace*{\baselineskip}
	%\begin{minipage}{\textwidth}
	O presente trabalho em nível de \imprimirnivel~foi avaliado e aprovado por banca examinadora composta pelos seguintes membros:\\
	%\end{minipage}%
	\vspace*{\baselineskip}
    Prof. Everton da Silva, Dr.\\
  Universidade Federal de Santa Catarina - UFSC\\
  \vspace*{\baselineskip}
    Prof. Examinador 2, Dr.\\
  Fédération Internationale des Géomètres - FIG\\
  \vspace*{\baselineskip}
    
	\vspace*{2\baselineskip}
	\begin{minipage}{\textwidth}
		Certificamos que esta é a \textbf{versão original e final} do trabalho de conclusão que foi julgado adequado para obtenção do título de \imprimirformacao.\\
	\end{minipage}
	%    \vspace{-0.7cm}
	\vspace*{\fill}
	\assinatura{\OnehalfSpacing Ana Maria Bencciveni Franzoni \\ Coordenação do Programa de Pós-Graduação}
	\vspace*{\fill}
	\assinatura{\OnehalfSpacing\imprimirorientador \\ \imprimirorientadorRotulo}
	%	\ifnotempty{\imprimircoorientador}{
	%	\assinatura{\imprimircoorientador \\ \imprimircoorientadorRotulo \\
	%		\imprimirinstituicao~--~\imprimirinstituicaosigla}
	%	}
	% \newpage
	\vspace*{\fill}
	\imprimirlocal, \imprimirano.
\end{folhadeaprovacao}
% ---

% ---
% Dedicatória
% ---
\begin{dedicatoria}
	\vspace*{\fill}
	\noindent
	\begin{adjustwidth*}{}{5.5cm} 
		\raggedleft       
		À Bibi.
	\end{adjustwidth*}
\end{dedicatoria}
% ---

% ---
% Agradecimentos
% ---
\begin{agradecimentos}
	Gostaria de agradecer sinceramente a todos os que colaboraram à execução\\
deste trabalho.\\
Aos colegas da UFSC.\\
Aos professores do PPGTG.\\
Em especial ao meu orientador, pela paciência.\\
E a minha querida esposa pela compreensão.
\end{agradecimentos}
% ---

% ---
% Epígrafe
% ---
\begin{epigrafe}
	\vspace*{\fill}
	\begin{flushright}
		\textit{``Eppur si muove!''\\
(Galileu Galilei, 1633)}
	\end{flushright}
\end{epigrafe}
% ---

% ---
% RESUMOS
% ---

% resumo em português
\setlength{\absparsep}{18pt} % ajusta o espaçamento dos parágrafos do resumo
\begin{resumo}
	\SingleSpacing
  Este trabalho tem como objetivo efetuar um estudo do histórico e de tendências
  dos mercados imobiliários brasileiro e mundial e seus efeitos não apenas no
  setor habitacional, mas também na Economia como um todo. Esta pesquisa tenta
  compreender especialmente os eventos recentes do mercado imobiliário, como a
  formação e o estouro de bolhas imobiliárias e os efeitos destes eventos sobre a
  Economia com foco, claro, no setor habitacional. Preliminarmente o texto mostra
  uma análise histórica do fenômeno da urbanização, considerando a experiência de
  vários países. Então, é realizada uma análise dos mercados imobiliários e a
  interação destes com as variáveis macroeconômicas através da análise de séries
  históricas, buscando contextualizar os eventos de formação e estouro de bolhas
  imobiliárias e seus efeitos. Finalmente, o trabalho faz uma análise dos
  principais problemas decorrentes da financeirização do setor habitacional
  ocorrida nas últimas décadas, inicialmente através da análise das evidências
  anedóticas, advindas dos países desenvolvidos, da grave crise habitacional por
  que passam, depois através da análise de séries estatísticas históricas
  relacionadas ao setor habitacional nestes países. Por fim, este trabalho propõe
  a implementação de uma política habitacional coordenada, através da utilização
  dos diversos instrumentos de política urbana previstos no Estatuto da cidade, de
  maneira a enfrentar o problema da moradia precária no Brasil, de maneira a
  melhorar as condições de saneamento, saúde pública e bem-estar do povo
  brasileiro. 
  
  \textbf{Palavras-chave}: 
    Mercado Imobiliário.
    Macroeconomia.
    Microeconomia.
  \end{resumo}
% resumo em inglês
\begin{resumo}[Abstract]
	\SingleSpacing
	\begin{otherlanguage*}{english}
		Resumo traduzido para outros idiomas, neste caso, inglês. Segue o formato do resumo feito na língua vernácula. As palavras-chave traduzidas, versão em língua estrangeira, são colocadas abaixo do texto precedidas pela expressão `Keywords', separadas por ponto.
		
		\textbf{Keywords}:
	      Real Estate.
        Macroeconomics.
        Microeconomics.
    	\end{otherlanguage*}
\end{resumo}
%% resumo em francês 
%\begin{resumo}[Résumé]
% \begin{otherlanguage*}{french}
%    Il s'agit d'un résumé en français.
% 
%   \textbf{Mots-clés}: latex. abntex. publication de textes.
% \end{otherlanguage*}
%\end{resumo}
%
%% resumo em espanhol
%\begin{resumo}[Resumen]
% \begin{otherlanguage*}{spanish}
%   Este es el resumen en español.
%  
%   \textbf{Palabras clave}: latex. abntex. publicación de textos.
% \end{otherlanguage*}
%\end{resumo}
%% ---

{%hidelinks
	\hypersetup{hidelinks}
	% ---
	% inserir lista de ilustrações
	% ---
	\pdfbookmark[0]{\listfigurename}{lof}
	\listoffigures*
	\cleardoublepage
	% ---
	
	% ---
	% inserir lista de quadros
	% ---
	\pdfbookmark[0]{\listofquadrosname}{loq}
	\listofquadros*
	\cleardoublepage
	% ---
	
	% ---
	% inserir lista de tabelas
	% ---
	\pdfbookmark[0]{\listtablename}{lot}
	\listoftables*
	\cleardoublepage
	% ---
	
	% ---
	% inserir lista de abreviaturas e siglas (devem ser declarados no preambulo)
	% ---
	\imprimirlistadesiglas
	% ---
	
	% ---
	% inserir lista de símbolos (devem ser declarados no preambulo)
	% ---
	\imprimirlistadesimbolos
	% ---
	
	% ---
	% inserir o sumario
	% ---
	\pdfbookmark[0]{\contentsname}{toc}
	\tableofcontents*
	\cleardoublepage
	
}%hidelinks
% ---

% ---

% ----------------------------------------------------------
% ELEMENTOS TEXTUAIS
% ----------------------------------------------------------
\textual

\hypertarget{intro}{%
\chapter{Introdução}\label{intro}}

O território é a base das atividades econômicas e sociais do país. Um eficiente
planejamento territorial passa por desenvolver políticas públicas que garantam
um desenvolvimento equânime de todo o território. A proposta desta pesquisa é
aprofundar o conhecimento referente à dinâmica do mercado imobiliário e das
políticas de solo, de maneira a planejar a utilização dos instrumentos de
política do solo no sentido de resolver os problemas do crônico déficit
habitacional brasileiro, onde milhões de pessoas vivem em situação precária.

O mundo vive uma situação de crise habitacional que se agrava desde a crise
financeiro-habitacional de 2008, por diversos motivos. No caso brasileiro, a
crise habitacional é crônica.

A precariedade de grande parte do estoque de moradias atualmente no Brasil,
a falta de infraestrutura básica, e o grande número de pessoas dividindo o mesmo
cômodo é uma situação que afeta a saúde pública em tempos de normalidade e se
agrava fortemente em tempos de crises epidêmicas, com a atual pandemia de
COVID-19.

\hypertarget{o-conceito-de-terra-e-a-importuxe2ncia-do-territuxf3rio}{%
\section{O conceito de terra e a importância do território}\label{o-conceito-de-terra-e-a-importuxe2ncia-do-territuxf3rio}}

De acordo com \textcite{realestate}, a terra é a base das atividades econômicas e sociais
de um povo, essencial para a vida e a sociedade, sendo assunto de diversas
disciplinas, como o Direito, Economia, Finanças, Sociologia e a Geografia.

No Direito, a terra é abordada no direito de propriedade e uso social da terra.
Na Economia, a terra é considerada um dos fatores de produção, ao lado do
trabalho e do capital. Nas Finanças, a terra é considerada como um bem
suscetível de ser dado em garantia, em troca de capital financeiro, visando
propiciar o desenvolvimento. Na perspectiva da Sociologia, a terra é um bem
comum de todos, que deve ser utilizado com fins de obtenção de uma sociedade
melhor. Já a Geografia foca em descrever os elementos físicos da terra e as
atividades humanas das pessoas que as habitam.

A Engenharia de Avaliações se preocupa em reconhecer os atributos que atuam na
formação de valor dos bens imóveis, um conceito ligado à Economia. Para isto, os
Avaliadores devem entender o mercado no qual estes bens estão inseridos. É o
mercado que reflete a atitude dos atores econômicos em resposta às forças
sociais e econômicas e às restrições da lei e dos ônus legais \autocite[10]{realestate}.

\hypertarget{o-mercado-imobiliuxe1rio}{%
\section{O Mercado Imobiliário}\label{o-mercado-imobiliuxe1rio}}

O mercado imobiliário urbano pode ser dividido basicamente em:
\begin{enumerate}
\def\labelenumi{\arabic{enumi}.}
\tightlist
\item
  Mercado de imóveis residenciais e;
\item
  Mercado de imóveis comerciais.
\end{enumerate}
\textcite{Case2000} identificaram quatro canais em que o Mercado Imobiliário se
conecta à macroeconomia:
\begin{enumerate}
\def\labelenumi{\arabic{enumi}.}
\tightlist
\item
  Através do setor de construção civil, pelo seu papel na atividade econômica
  como um todo, apesar de, segundo os autores, este setor não ser grande o
  suficiente para orientar toda a economia de um país;
\item
  Através da formação da demanda agregada, já que existe uma conexão entre a
  propensão marginal a consumir com o efeito riqueza do Mercado Imobiliário;
\item
  Através do setor bancário, haja vista que os bancos possuem grande parte de
  seu portfolio ligado a produtos conectados de alguma forma com o mercado
  imobiliário
\item
  E, por fim, porém não menos importante, o \gls{MI} desempenha um relevante
  papel na Macroeconomia através da sua capacidade de alocar os trabalhadores de
  maneira eficiente, ou seja, em regiões de maior produtividade do trabalho, o que
  tem um potencial de aumentar a produtividade da Economia como um todo.
\end{enumerate}
Especialmente devido a sua relação com o setor bancário, o \gls{MI} tem um papel
importante na estabilidade financeira e na calibragem da política monetária
\autocite{Zhu}. Esta razão, por si só, seria suficiente para justificar intervenções
por parte dos governos no \gls{MI}. Mas ainda, pelo importante papel do \gls{MI}
na eficiente alocação dos trabalhadores em áreas de maior produtividade de
Economia, esta intervenção estatal no \gls{MI} deve ser justificada, já que isto
afeta a renda e o bem-estar de toda a população, propiciando um funcionamento
mais eficiente das engrenagens econômicas.

\hypertarget{a-economia-e-o-mercado-imobiliuxe1rio}{%
\section{A Economia e o mercado imobiliário}\label{a-economia-e-o-mercado-imobiliuxe1rio}}

Uma questão sempre presente nas discussões econômicas diz respeito ao papel do
Estado na Economia, seja como ofertante, através de empresas públicas,
geralmente criadas em setores onde não há atratividade para empresas privadas,
como demandante -- em muitos casos o Estado é o maior demandante do setor -- ou
ainda como interventor/regulador. Em relação ao \gls{MI}, então, qual deveria
ser o papel do Estado junto a este mercado? Deveria o Estado intervir de alguma
forma, seja como ofertante, demandante ou interventor, ou deveria o Estado
tentar permanecer neutro em relação ao \gls{MI}, deixando que o mercado se
auto-regule, através do equilíbrio da demanda e da oferta por bens imóveis?

Diversos autores tem manifestado que o \gls{MI}, por sua pujança e
particularidades, não deveria ser estudado apenas como mais um mercado de bens
da Economia, haja vista sua forte interação com o bem-estar da população, da
saúde pública, e do seu impacto nas variáveis macroeconômicas e \emph{vice-versa}.

De acordo com \textcite[p.~149-150]{Case2000}, a análise econômica padrão sugeriria que a
terra é uma mercadoria como qualquer outra e o seu preço reflete as forças de
mercado de oferta e demanda, e que por este motivo o Estado não deveria intervir
neste mercado, assim como no mercado de outros produtos. No entanto, \textcite{Case2000}
ponderam que o alto preço dos imóveis em alguns dos mercados mais aquecidos tem
menos a ver com as forças de mercado do que com políticas regulatórias
governamentais, como o zoneamento urbano, que restringem muito a possibilidade
de novos empreendimentos e acabam por valorizar demais os imóveis já
construídos, muitas vezes sem motivos técnicos que as justifiquem.

Ainda há de se considerar a inelasticidade da oferta e a inércia dos preços como
fatores preponderantes que devem levar a uma necessária intervenção do Estado no
\gls{MI}. Devido a estes fatores, choques macroeconômicos afetam diferentemente
o lado da oferta e o lado da demanda no \gls{MI}, podendo levar a formação de
bolhas especulativas.

Historicamente, nos países de mercado de capitais menos desenvolvidos, como o
Brasil, o \gls{MI} é utilizado como reserva de valor, o que tende a se agravar
em tempos de crise econômica, em que os investidores tendem a procurar
investimentos de menor risco.

Nos países desenvolvidos do sistema, os mercados de capitais mais desenvolvidos
propiciam alternativas de investimento que possibilitam aos grandes investidores
institucionais diversos produtos de investimento que tornam praticamente
desnecessário e contraproducente o investimento direto no \gls{MI}, haja vista
todas as taxas de negociação envolvidas, o que costuma espantar os investidores
profissionais. Isto não significa, no entanto, que o \gls{MI} destes países
esteja imune à especulação do capital financeiro, o que ficou claro com a crise
financeiro-imobiliária de 2008. Também o setor financeiro não está livre dos
impactos causados por fortes variações de preços no \gls{MI} e choques de preços
como o de 2007/2008 nos EUA mostraram que, numa economia capitalista
desenvolvida, crises setoriais que antes se restringiriam ao setor atingido,
atualmente tendem a se propagar rapidamente pelos diversos setores da Economia,
especialmente as crises advindas do mercado imobiliário.

\hypertarget{crise-habitacional-cruxf4nica}{%
\section{Crise habitacional crônica}\label{crise-habitacional-cruxf4nica}}

Segundo \textcite{ritchie_urbanization_2018}, em 1990 em torno de 36,7\% da população
urbana brasileira vivia em favelas. É verdade que a proporção da população
urbana vivendo em favelas vem diminuindo desde então. Em 2000, 31,5\% da
população urbana ocupava as favelas, ao passo que em 2014 esse número diminuiu
para 22,3\% da população urbana.

No entanto, com o crescimento da população brasileira e o aumento da porcentagem
de população urbana desde então, os números totais estão praticamente estagnados,
o que pode ser observado na tabela abaixo.
\begin{longtable}[]{@{}rrrrrr@{}}
\caption{População habitando favelas no Brasil.}\tabularnewline
\toprule
\begin{minipage}[b]{0.04\columnwidth}\raggedleft
Ano\strut
\end{minipage} & \begin{minipage}[b]{0.20\columnwidth}\raggedleft
Pop. total (milhões hab.)\strut
\end{minipage} & \begin{minipage}[b]{0.13\columnwidth}\raggedleft
Pop. urbana (\%)\strut
\end{minipage} & \begin{minipage}[b]{0.21\columnwidth}\raggedleft
Pop. urbana (milhões hab.)\strut
\end{minipage} & \begin{minipage}[b]{0.10\columnwidth}\raggedleft
Favelas (\%)\strut
\end{minipage} & \begin{minipage}[b]{0.17\columnwidth}\raggedleft
Favelas (milhões hab.)\strut
\end{minipage}\tabularnewline
\midrule
\endfirsthead
\toprule
\begin{minipage}[b]{0.04\columnwidth}\raggedleft
Ano\strut
\end{minipage} & \begin{minipage}[b]{0.20\columnwidth}\raggedleft
Pop. total (milhões hab.)\strut
\end{minipage} & \begin{minipage}[b]{0.13\columnwidth}\raggedleft
Pop. urbana (\%)\strut
\end{minipage} & \begin{minipage}[b]{0.21\columnwidth}\raggedleft
Pop. urbana (milhões hab.)\strut
\end{minipage} & \begin{minipage}[b]{0.10\columnwidth}\raggedleft
Favelas (\%)\strut
\end{minipage} & \begin{minipage}[b]{0.17\columnwidth}\raggedleft
Favelas (milhões hab.)\strut
\end{minipage}\tabularnewline
\midrule
\endhead
\begin{minipage}[t]{0.04\columnwidth}\raggedleft
1990\strut
\end{minipage} & \begin{minipage}[t]{0.20\columnwidth}\raggedleft
149,0\strut
\end{minipage} & \begin{minipage}[t]{0.13\columnwidth}\raggedleft
73,90\strut
\end{minipage} & \begin{minipage}[t]{0.21\columnwidth}\raggedleft
110,1\strut
\end{minipage} & \begin{minipage}[t]{0.10\columnwidth}\raggedleft
36,7\strut
\end{minipage} & \begin{minipage}[t]{0.17\columnwidth}\raggedleft
40,4\strut
\end{minipage}\tabularnewline
\begin{minipage}[t]{0.04\columnwidth}\raggedleft
1995\strut
\end{minipage} & \begin{minipage}[t]{0.20\columnwidth}\raggedleft
162,0\strut
\end{minipage} & \begin{minipage}[t]{0.13\columnwidth}\raggedleft
77,61\strut
\end{minipage} & \begin{minipage}[t]{0.21\columnwidth}\raggedleft
125,7\strut
\end{minipage} & \begin{minipage}[t]{0.10\columnwidth}\raggedleft
34,1\strut
\end{minipage} & \begin{minipage}[t]{0.17\columnwidth}\raggedleft
42,9\strut
\end{minipage}\tabularnewline
\begin{minipage}[t]{0.04\columnwidth}\raggedleft
2000\strut
\end{minipage} & \begin{minipage}[t]{0.20\columnwidth}\raggedleft
174,8\strut
\end{minipage} & \begin{minipage}[t]{0.13\columnwidth}\raggedleft
81,20\strut
\end{minipage} & \begin{minipage}[t]{0.21\columnwidth}\raggedleft
141,9\strut
\end{minipage} & \begin{minipage}[t]{0.10\columnwidth}\raggedleft
31,5\strut
\end{minipage} & \begin{minipage}[t]{0.17\columnwidth}\raggedleft
44,7\strut
\end{minipage}\tabularnewline
\begin{minipage}[t]{0.04\columnwidth}\raggedleft
2005\strut
\end{minipage} & \begin{minipage}[t]{0.20\columnwidth}\raggedleft
186,1\strut
\end{minipage} & \begin{minipage}[t]{0.13\columnwidth}\raggedleft
82,80\strut
\end{minipage} & \begin{minipage}[t]{0.21\columnwidth}\raggedleft
154,2\strut
\end{minipage} & \begin{minipage}[t]{0.10\columnwidth}\raggedleft
29,0\strut
\end{minipage} & \begin{minipage}[t]{0.17\columnwidth}\raggedleft
44,7\strut
\end{minipage}\tabularnewline
\begin{minipage}[t]{0.04\columnwidth}\raggedleft
2009\strut
\end{minipage} & \begin{minipage}[t]{0.20\columnwidth}\raggedleft
193,9\strut
\end{minipage} & \begin{minipage}[t]{0.13\columnwidth}\raggedleft
84,04\strut
\end{minipage} & \begin{minipage}[t]{0.21\columnwidth}\raggedleft
162,9\strut
\end{minipage} & \begin{minipage}[t]{0.10\columnwidth}\raggedleft
26,9\strut
\end{minipage} & \begin{minipage}[t]{0.17\columnwidth}\raggedleft
43,8\strut
\end{minipage}\tabularnewline
\begin{minipage}[t]{0.04\columnwidth}\raggedleft
2014\strut
\end{minipage} & \begin{minipage}[t]{0.20\columnwidth}\raggedleft
202,8\strut
\end{minipage} & \begin{minipage}[t]{0.13\columnwidth}\raggedleft
85,90\strut
\end{minipage} & \begin{minipage}[t]{0.21\columnwidth}\raggedleft
174,3\strut
\end{minipage} & \begin{minipage}[t]{0.10\columnwidth}\raggedleft
22,3\strut
\end{minipage} & \begin{minipage}[t]{0.17\columnwidth}\raggedleft
38,9\strut
\end{minipage}\tabularnewline
\bottomrule
\end{longtable}
\bcenter

Fonte: Do autor, a partir de dados do Banco Mundial.
\ecenter

A figura \ref{fig:slums} mostra graficamente a evolução da população urbana em
favelas (em vermelho) e a população urbana total (em azul). Nota-se que o
aumento da população urbana ocorreu com a quase estagnada população das
favelas.

A situação econômica do país deve ter um peso na composição da população
habitando as favelas: a década de 90 e o início da década de 2000 são períodos
sabidamente de crescimento baixo e intermitente, intercalados por graves
recessões na economia brasileira. Já a baixa considerável desta população
desfavorecida entre os períodos de 2009 e 2014 (aprox. 11\%) sugere que houve um
impacto do crescimento econômico do período\footnote{Especialmente no que tange à manutenção de baixas taxas de desemprego.}, que veio acompanhado de
programas habitacionais destinados à população de baixa renda, como o Minha Casa
Minha Vida e, ainda que mais timidamente, do instituto da regularização
fundiária. Deve-se lembrar que neste período houve aumento da proporção de
população urbana, ou seja, a diminuição da população favelada com certeza não
reflete um fenômeno de migração da cidade para o campo.
\begin{figure}[H]

{\centering \includegraphics[width=0.7\linewidth]{images/urban-pop-in-out-of-slums} 

}

\caption{População urbana e população em favelas.}\label{fig:slums}
\end{figure}
\bcenter

Fonte: \textcite{ritchie_urbanization_2018}
\ecenter

Já para o IBGE \autocite{ibge}, segundo dados do censo de 2010, o Brasil possuía 11,4
milhões de pessoas morando em aglomerados subnormais, ou favelas.
\begin{citacao}
Para o IBGE, os 'aglomerados subnormais' representam um conjunto constituído de,
no mínimo, 51 unidades habitacionais ocupando ou tendo ocupado, até período
recente, terreno de propriedade alheia (pública ou particular) dispostas, em
geral, de forma desordenada e densa, e apresentando carência em serviços
básicos.
\cite{ibge2}
\end{citacao}
A discrepância entre os números do IBGE e os números internacionais pode ser
explicado por falhas metodológicas. Segundo MARICATO (2002, p.~154) \autocite[\emph{apud}][9]{silva}:
\begin{citacao}
Não há números gerais, confiáveis, sobre a ocorrência de favelas ou de
loteamentos ilegais em todo o Brasil. Por falhas metodológicas ou ainda por uma
dificuldade óbvia de conhecer a titularidade da terra sobre a qual as favelas se
instalam, o IBGE apresenta dados bastante subdimensionados. A busca de números
mais rigorosos nos conduz a alguns diagnósticos elaborados por governos
municipais, teses acadêmicas ou organismos estaduais que, entretanto, fornecem
dados localizados e restritos.
\end{citacao}
\hypertarget{a-tenduxeancia-ao-agravamento-da-crise-habitacional}{%
\section{A tendência ao agravamento da crise habitacional}\label{a-tenduxeancia-ao-agravamento-da-crise-habitacional}}

Deixado o problema sujeito às forças de mercado, a tendência é de agravamento do
problema. A diminuição das taxas de juros a níveis muito baixos causa o
outrora raro problema da armadilha da liquidez, problema que vem atormentando os
países desenvolvidos há alguns anos e agora tem encontrado lugar nos países em
desenvolvimento, como o Brasil \autocite{krugman-emergentes}.

A reação dos bancos centrais do mundo a este problema é o de encontrar formas
alternativas de injetar mais liquidez ao sistema, o que pode acabar por
propiciar a formação de novas bolhas especulativas e o setor imobiliário não
está imune a isto. Pelo contrário, em tempos de incertezas econômicas, o
mercado imobiliário é um dos destinos preferenciais para os capitais em busca
de reserva de valor. O aumento do preço da terra que pode estar por vir, então,
acompanhado do alto desemprego que se tem gerado em todo o mundo pela pandemia
de COVID-19, tende a agravar sobremaneira o problema da crise habitacional no
mundo e, em especial, onde esta já se tornou crônica, como no Brasil e em outros
países da \gls{AL}.

\hypertarget{efeitos-do-mercado-imobiliuxe1rio-na-economia-em-geral}{%
\section{Efeitos do mercado imobiliário na Economia em geral}\label{efeitos-do-mercado-imobiliuxe1rio-na-economia-em-geral}}

Além da crise habitacional proveniente deste forte aumento dos preços dos
imóveis, inflar o setor habitacional também não tem dado bons resultados em
termos econômicos e políticos mais gerais:
\begin{citacao}
Os políticos tradicionalmente gostam quando os preços das casas aumentam. As
pessoas se sentem mais ricas e portanto, emprestam e gastam mais, dando um bom
impulso à economia, eles pensam. Quando todo mundo está se sentindo bem com sua
situação financeira, os políticos têm maior chance de reeleição. Mas existe um
outro lado. Habitações caras são inequivocamente ruins para a crescente
população de locatários do mundo rico, forçando-os a reduzir os gastos com
outros bens e serviços. E uma política econômica baseada em compradores de casas
em grandes dívidas não é sustentável. A curto prazo, encontra um estudo do FMI,
o aumento da dívida das famílias aumenta o crescimento econômico e o emprego.
Mas as famílias precisam controlar os gastos para pagar seus empréstimos, então
em três a cinco anos, esses efeitos são revertidos: o crescimento se torna mais
lento do que seria de outra forma e as chances de uma crise financeira aumentam.
\cite{economist-housing-2020}
\end{citacao}
\hypertarget{objetivos}{%
\section{Objetivos}\label{objetivos}}

\hypertarget{objetivo-geral}{%
\subsection{Objetivo Geral}\label{objetivo-geral}}

Estudar o comportamento histórico e recente do mercado imobiliário à luz das
variáveis macroeconômicas, de maneira que se possa propor novas políticas
públicas para orientar o setor habitacional.

\hypertarget{objetivos-especuxedficos}{%
\subsection{Objetivos Específicos}\label{objetivos-especuxedficos}}
\begin{enumerate}
\def\labelenumi{\arabic{enumi}.}
\tightlist
\item
  Pesquisar as raízes do grande déficit habitacional brasileiro;
\item
  Efetuar análise comparativa entre o desenvolvimento econômico brasileiro e o
  de outros países, buscando aplicar aqui soluções que eventualmente já tenham
  sido aplicadas em outros países com desenvolvimento similiar ao brasileiro;
\item
  Compreender e descrever o funcionamento do mercado imobiliário;
\item
  Abstrair das diferentes estruturas de mercado as relações entre as variáveis
  macroeconômicas e o seu impacto no mercado de terras e no incentivo à construção
  de moradias.
\item
  Propor, à partir de uma abordagem econômica, novas políticas de solo que
  visem uma melhor regulamentação do setor habitacional, visando que a aplicação
  dos diversas instrumentos previstos no Estatuto da Cidade, de maneira coordenada
  e focada na solução do problema do déficit habitacional venha a produzir
  resultados de maneira que este seja zerado em um período de tempo razoável.
\end{enumerate}
\hypertarget{estrutura-do-trabalho}{%
\section{Estrutura do trabalho}\label{estrutura-do-trabalho}}

O \textbf{Capítulo 1 (Introdução)} apresenta os objetivos, justificativas e
estrutura do trabalho.

O \textbf{Capítulo 2 (Método)} traz considerações a respeito do método de análise e
de pesquisa.

O \textbf{Capítulo \ref{historico} (Aspectos Históricos)} faz uma contextualização histórica do
problema do acesso à terra e moradia no Brasil, dado o modelo de desenvolvimento
econômico que aqui se instalou, e compara com o ocorrido em outros países do
mundo ocidental.

O \textbf{Capítulo \ref{economia} (O Mercado Imobiliário e a Economia)} aborda os aspectos
teóricos e conceituais mais modernos relevantes à interligação do mercado
imobiliário com as variáveis macroecômicas do país, explicando como o \gls{MI}
influencia na Economia do país e vice-versa.

O \textbf{Capítulo \ref{indices} (Índices de preços de imóveis e correlação com variáveis macroeconômicas)}
traz um apanhado teórico sobre a confecção de índices de preços de imóvei, dá
diretrizes para a eventual construção de um índice brasileiro e também traz um
histórico recente dos índices de preços de imóveis e das taxas de juros de longo
prazo nos EUA, fazendo considerações a respeito da correlação entre estas
variáveis.

O \textbf{Capítulo \ref{crise2008} (A crise imobiliário-financeira de 2007-2008 revisitada)} faz
uma reanálise da crise financeira-habitacional de 2008 a luz dos recentes
acontecimentos e suas consequências até os tempos presentes, passando pelas
origens da crise, a queda acentuada das taxas de lucros da economia capitalista,
os erros de análise de risco com a precifição dos derivativos de crédito
imobiliário, e, finalmente, o pós-crise, ou seja, que medidas tomaram os países
mais atingidos pela crise e como ela continua a produzir seus efeitos até os
dias atuais.

O \textbf{Capítulo \ref{politicas} (Políticas do solo)} faz um histórico a
respeito das principais abordagens utilizadas no mundo ocidental para solução da
crise habitacional de fins do século XIX/início do século XX e faz considerações
a respeito da atual crise habitacional e como ela vem sendo enfrentada no mundo.

Finalmente, o \textbf{Capítulo \ref{conclusao} (Conclusão)} traz as propostas de regulação baseadas
nos resultados obtidos.

\hypertarget{historico}{%
\chapter{Aspectos históricos}\label{historico}}
\begin{epigrafe}
	\vspace*{\fill}
	\begin{flushright}
		\textit{``Do ponto de vista social, todos os fatores se resumem\\ 
		em um `recurso' elementar: o homem. Logo, não é possível seguir\\ 
		conceptualmente o processo de industrialização se não sabemos como\\ 
		o homem aplicava antes o seu tempo de trabalho, como o aplica depois,\\ 
		o que ocorre quando passa de um modo de produzir a outra e em que\\ 
		condições realiza essa passagem.[ \ldots ] Considerando que na estrutura\\ 
		da economia que precede a industrialização quase toda a população está\\ 
		na `agricultura', é preciso estudar detidamente a organização deste setor.\\ 
		Em outras palavras, se o problema da `agricultura' não foi entendido,\\ 
		tampouco será possível compreender o problema da `indústria', ou manufatura,\\ 
		nem o papel que os serviços desempenham. Falando de modo sucinto, a \\
		`manufatura' e os serviços são novas formas de aplicação de parte do\\ 
		tempo de trabalho da população que antes estava na `agricultura'. Mas,\\
		por sua vez, a própria `agricultura' se reorganiza quando a transferência ocorre.''\\
		(RANGEL, 1954)}
	\end{flushright}
\end{epigrafe}
O Capitalismo é um sistema político-econômico que, historicamente, substitui o
Feudalismo, sistema em que a população encontrava-se toda concentrada no campo.

Nas sociedades pré-capitalistas, a população predominante rural organizava-se no
chamado `Complexo Rural', ou seja, num ambiente rural onde eram produzidos não
apenas os produtos agrícolas, mas onde também eram produzidos, pelos próprios
camponeses, em uma muito baixa produtividade, todo o ferramental necessário para
as suas atividades agrícolas, assim como suas vestes, utensílios domésticos e
outros itens.

A passagem do sistema feudal para o sistema capitalista ocorre com a \emph{divisão
social do trabalho}, ou seja, com o desenvolvimento de indústrias que vão aos
poucos absorver as atividades não-agrícolas realizadas no campo.
\begin{citacao}
Numa economia em expansão, com crescente industrialização, comercialização e
urbanização, numerosos processos anteriormente levados a efeito antes dentro da
casa da família ou unidade (econômica) familiar, ou são completamente
abandonados ou substituídos por processos semelhantes em bases
comerciais. \cite[p. 41]{kuznets} \textit{apud} \cite[p. 218]{rangel1956}.
\end{citacao}
O desenvolvimento do capitalismo brasileiro no século XX se deu pela chamada
``via prussiana'' ou \emph{junker} \autocite[155]{rangel1988}, que é um tipo de reforma agrária
que consiste na substituição do latifúndio feudal pelo latifúndio capitalista.
Este tipo de desenvolvimento tem como característica se dar sem a execução
prévia da reforma agrária no sentido clássico, \emph{i.e.} no sentido da distribuição
dos latifúndios em pequenas propriedades, a chamada via clássica ou democrática.
Apesar de permitir imprimir um ``impulso extraordinário e energético'' à
industrialização, a via prussiana ``promove uma distribuição muito desigual da
renda'' \autocite[155]{rangel1988}. A característica talvez mais perniciosa do
desenvolvimento capitalista por esta via se dá pela formação de um ``exército
industrial de reserva'' demasiado grande, ou seja, um aumento da população urbana
desproporcional à necessidade de mão-de-obra necessária nas indústrias do
capitalismo nascente nas cidades. O resultado é o crescimento acelerado e
desordenado das cidades, com a inevitável formação dos cortiços e favelas para
acomodar a parte mais carente da população que, expulsa do campo, vai se
aglomerar nos grandes centros urbanos em busca da sua sobrevivência.

Dados compilados pelas Nacões Unidas foram organizados na tabela
\ref{tab:pop-table} com o intuito de demonstrar a evolução e o atual tamanho
deste problema.
\begin{table}[H]

\caption{\label{tab:pop-table}População Urbana (\%).}
\centering
\begin{tabular}[t]{lrrrrrr}
\toprule
\multicolumn{1}{c}{} & \multicolumn{6}{c}{Ano} \\
\cmidrule(l{3pt}r{3pt}){2-7}
Entity & 1960 & 1970 & 1980 & 1990 & 2000 & 2014\\
\midrule
\rowcolor{gray!6}  \addlinespace[0.3em]
\multicolumn{7}{l}{\textbf{Mundo}}\\
\hspace{1em}World & 33,8 & 36,6 & 39,3 & 43,0 & 46,7 & 53,5\\
\hspace{1em}More developed regions & 61,1 & 66,8 & 70,3 & 72,4 & 74,2 & 78,0\\
\hspace{1em}Less developed regions & 21,9 & 25,3 & 29,4 & 34,9 & 40,1 & 48,4\\
\rowcolor{gray!6}  \addlinespace[0.3em]
\multicolumn{7}{l}{\textbf{Europa}}\\
\hspace{1em}Europe & 57,4 & 63,1 & 67,6 & 69,9 & 71,1 & 73,7\\
\hspace{1em}Eastern Europe & 48,9 & 56,6 & 63,8 & 68,0 & 68,2 & 69,2\\
\hspace{1em}Western Europe & 68,6 & 72,1 & 73,4 & 74,0 & 76,0 & 79,2\\
\rowcolor{gray!6}  \addlinespace[0.3em]
\multicolumn{7}{l}{\textbf{América}}\\
\hspace{1em}Latin America and the Caribbean & 49,4 & 57,3 & 64,6 & 70,7 & 75,5 & 79,7\\
\hspace{1em}South America & 51,8 & 60,0 & 67,6 & 74,2 & 79,6 & 83,3\\
\hspace{1em}Central America & 46,4 & 53,7 & 60,3 & 65,0 & 68,7 & 73,4\\
\hspace{1em}Northern America & 69,9 & 73,8 & 73,9 & 75,4 & 79,1 & 81,5\\
\hspace{1em}United States & 70,0 & 73,6 & 73,7 & 75,3 & 79,1 & 81,5\\
\hspace{1em}Argentina & 73,6 & 78,9 & 82,9 & 87,0 & 89,1 & 91,4\\
\hspace{1em}Brazil & 46,1 & 55,9 & 65,5 & 73,9 & 81,2 & 85,5\\
\bottomrule
\end{tabular}
\end{table}
\bcenter

Fonte: \textcite{doi:10.1177/0959683609356587}
\ecenter

Ainda, para melhor ilustrar, foram elaborados os gráficos das figuras
\ref{fig:pop-urb-mundo} a \ref{fig:pop-urb-brazil-brics}.

Na figura \ref{fig:pop-urb-mundo}, pode-se notar que a população urbana no
Brasil vem aumentando, desde 1950, numa taxa superior à média dos países em
desenvolvimento (\emph{Less developed regions}), atingindo uma proporção superior
inclusive à dos países mais desenvolvidos (\emph{More developed regions}).
\begin{figure}[H]

{\centering \includegraphics[width=0.8\linewidth]{images/pop-urb-mundo-1} 

}

\caption{População Urbana - Brasil vs. Mundo.}\label{fig:pop-urb-mundo}
\end{figure}
\bcenter

Fonte: \textcite{doi:10.1177/0959683609356587}
\ecenter

Na figura \ref{fig:pop-urb-continents} pode-se ver as séries da população
urbana em diversos continentes desde 1800. Percebe-se neste gráfico também uma
maior aceleração do crescimento da população urbana na América Latina e Caribe a
partir de meados da década de 40, chegando esta região a ultrapassar a população
urbana da Europa Ocidental no início do século corrente.
\begin{figure}[H]

{\centering \includegraphics[width=0.8\linewidth]{images/pop-urb-continents-1} 

}

\caption{População Urbana - Nos diferentes Continentes.}\label{fig:pop-urb-continents}
\end{figure}
\bcenter

Fonte: \textcite{doi:10.1177/0959683609356587}
\ecenter

A figura \ref{fig:pop-urb-brazil-latinAmerica} mostra a evolução da população
urbana no Brasil em comparação com o continente sul-americano e a América Latina,
dando destaque para alguns vizinhos.
\begin{figure}[H]

{\centering \includegraphics[width=0.8\linewidth]{images/pop-urb-brazil-latinAmerica-1} 

}

\caption{População Urbana - Brasil vs. AL.}\label{fig:pop-urb-brazil-latinAmerica}
\end{figure}
\bcenter

Fonte: \textcite{doi:10.1177/0959683609356587}
\ecenter

A figura \ref{fig:pop-urb-brazil-developed} mostra o comparativo da população
urbana no Brasil com uma seleção de países desenvolvidos desde 1800. Quanto aos
países desenvolvidos, nota-se que tiveram, primeiramente, uma ascenção um pouco
mais lenta da população urbana (excessão para a Grã-Bretanha, primeira nação a
industrializar-se), que essa ascenção teve lugar já na década de 1850 e que
houve uma estabilização gradual, por volta da década de 1970. Já quanto ao
Brasil nota-se uma grande aceleração no crescimento da população urbana
brasileira após a década de 1950, o que reflete o nascimento tardio do
capitalismo por aqui e, por fim, que, ao contrário dos países desenvolvidos, não
houve ainda uma estabilização da proporção de população urbana e esta segue em
crescimento, tendo chegado a níveis maiores aqui do que no resto do mundo
desenvolvido.
\begin{figure}[H]

{\centering \includegraphics[width=0.8\linewidth]{images/pop-urb-brazil-developed-1} 

}

\caption{População Urbana - Brasil vs. Países Desenvolvidos.}\label{fig:pop-urb-brazil-developed}
\end{figure}
\bcenter

Fonte: \textcite{doi:10.1177/0959683609356587}
\ecenter

E a figura \ref{fig:pop-urb-brazil-brics} mostra a comparação dos dados do
Brasil com os outros países do grupo dos BRICS.
\begin{figure}[H]

{\centering \includegraphics[width=0.8\linewidth]{images/pop-urb-brazil-brics-1} 

}

\caption{População Urbana - Brasil vs. BRICS.}\label{fig:pop-urb-brazil-brics}
\end{figure}
\bcenter

Fonte: \textcite{doi:10.1177/0959683609356587}
\ecenter

Em meados dos anos 60, apenas 46,1\% da população brasileira era urbana, uma
proporção bem menor do que a dos países do então \emph{primeiro mundo} (EUA e Europa
Ocidental), hoje ditos \emph{desenvolvidos}, que girava então em torno dos 70\% da
população.

Em apenas 10 anos, já em meados da década de 70, este número sofria um aumento
vertiginoso de quase 10 pontos percentuais, com 55,9\% da população urbana. A
população urbana brasileira equiparava-se à da Europa Oriental.
Já na década de 80 a população urbana no Brasil ultrapassaria a da Europa
Oriental, chegando à valores próximos da média para o continente europeu como um
todo (ocidental e oriental), enquanto a população urbana no mundo desenvolvido
se estagnava.

Chegado os anos 90, a população urbana brasileira atingiu notáveis 73,9\% da
população brasileira, número equiparado ao da população urbana do mundo
desenvolvido (74\% na Europa Ocidental). Em meados dos anos 2000, já então no
século atual, ela ultrapassou, em proporção, a população urbana da Europa
Ocidental e a dos EUA, chegando ao último dado de 2015, com 85,8\% da população
brasileira vivendo nas cidades.

Há de se levar em consideração, ainda, que este ``êxodo rural'' ainda foi
acompanhado de um crescimento demográfico expressivo.

Todo este crescimento expressivo seria salutar se tivesse se dado no contexto do
rápido desenvolvimento da economia nacional. Isto, porém, não ocorreu durante
todo o período analisado. O crescimento da economia brasileira acelerou-se na
segunda quadra da década de 60 e manteve-se alto até fins da década seguinte,
porém estagnou-se na década de 80, a chamada década perdida, sem que com isso a
população urbana deixasse de crescer vertiginosamente.

Para Rangel \autocite*[151]{rangel1986a}:
\begin{citacao}
"essa redistribuição da população entre os quadros urbano e rural não tem, em si
mesma, nada de anormal.[...] A urbanização, em si mesma, é um fenômeno
perfeitamente normal, numa economia em processo de industrialização. O que não é
normal é o ritmo que imprimimos ao \emph{nosso} processo de urbanização, que
implica criar, nas cidades, uma oferta de mão-de-obra em descompasso com a
demanda que a industrialização vai criando."
\end{citacao}
Todo este processo só poderia, então, ter desaguado no inchaço das principais
cidades brasileiras. Desnecessário dizer que o planejamento urbano nestas
condições é praticamente inviável. As administrações municipais, nem que fossem
as mais eficientes, teriam capacidade de planejar e disciplinar o uso do solo
urbano nesta ``velocidade migratória''.

Segundo Rangel, com o desenvolvimento da indústria pesada no Brasil, a crise
agrária, antes cíclica, tornou-se crônica, criando um \autocite*[156-157]{rangel1988}:
\begin{citacao}
"vasto deslocamento de população, na direção geral campo-cidade. Esse movimento
se faz escalonadamente, das áreas rurais para as cidades pequenas; destas para
as médias e grandes, e posteriormente para as metrópoles gigantes. No fim da
linha, portanto, vamos encontrar as cidades de São Paulo e do Rio de Janeiro".
\end{citacao}
De fato, os dados parecem mostrar a pertinência desta análise.

Enfim, para Rangel, a origem deste ``multitudinário deslocamento demográfico'',
está ``o modo como o país preparou sua estrutura agrária para a industrialização''.

\hypertarget{a-questuxe3o-agruxe1ria}{%
\section{A questão agrária}\label{a-questuxe3o-agruxe1ria}}

Segundo Rangel, a Questão Agrária, embora nascida na área rural, é um fenômeno
urbano. Com isto Rangel quer dizer que a crise agrária, a crise que se dá na
transição do feudalismo para o capitalismo, começa no campo, onde se passa o
enredo do feudalismo, para a cidade moderna, onde se desenvolve o capitalismo.

Para uma melhor compreensão da questão se faz mister compreender os mecanismos
de funcionamento dos sistemas citados, isto é, do feudalismo e do capitalismo,
especialmente no que tange a transição entre eles, nos motivos que levam ao
fim de um sistema e desembocam quase que inequivocamente (excetos raras
exceções) no outro.

\hypertarget{feudalismo}{%
\subsection{Feudalismo}\label{feudalismo}}

As ``leis'', ou princípios, ou ainda os ``motores primários'' do feudalismo são
\autocite[126]{rangel1985}:
\begin{itemize}
\tightlist
\item
  \emph{All land is king's land}
\item
  \emph{Nulle terre sans seigneur}
\end{itemize}
O primeiro princípio, \emph{all land is king's land}, ou ``toda a terra pertence ao
rei'', quer dizer, mais precisamente, que todo o domínio da terra está
concentrada nas mãos do rei, que as explora através dos laços de suserania e
vassalagem, típicos do feudalismo. Já o segundo princípio, segundo Rangel
\autocite*[219]{rangel1961}, \emph{nulle terre sans seigneur}, quer dizer que ``a existência
de terra livre é incompatível com o feudalismo'', ou seja, toda a terra deve ter
um senhor, que a administra a serviço da Coroa e lhe paga tributo. Na existência
de terra livre, como será visto, o feudalismo não se pode desenvolver, e a
tendência é que haja ou a formação de comunidades em estado tribal, ou que sejam
estabelecidas formas de escravidão. Ou seja, a terra, ``nas condições feudais,
não tem preço e é, de fato ou de direito, inalienável'' \autocite[206]{rangel1960}.

\hypertarget{o-feudalismo-no-brasil}{%
\subsubsection{O feudalismo no Brasil}\label{o-feudalismo-no-brasil}}

Segundo Rangel \autocite*[206]{rangel1956}, a atitude do economista do país
subdesenvolvido não pode ser a mesma do economista dos países mais
desenvolvidos, que, \emph{tendo vivido o processo histórico completo, assistiram
simultaneamente à morte do ser antigo e à sua representação}.
\begin{citacao} 
a absorção sem crítica do \emph{dernier cri} em matéria de ciência econômica por
ele lhe será fatal, porque implica mudar o reflexo ideal da realidade sem que
essa realidade mesma tenha mudado, ou sem que tenha mudado senão em parte. Para
nós, o pensamento dos antigos guarda muito de sua primitiva validade porque
reflete uma realidade que, em certa medida, continua a ser a nossa
\cite[p.~206-207]{rangel1956}.
\end{citacao}
Em outras palavras, para \textcite{rangel1956}, os economistas dos países
subdesenvolvidos, ou mais modernamente, países ``em desenvolvimento'', devem
utilizar em sua análise as teorias clássicas, neoclássicas, keynesianas, à
medida que subsistem nestes países características próprias da realidade
econômica que imperavam no Velho Mundo quando elas foram concebidas.

Desta maneira, o feudalismo tal como concebido na Europa não teve a mesma
estrutura que o feudalismo no Brasil, assim como o sistema feudal brasileiro foi
não-concomitante com o sistema feudal europeu.

Segundo Rangel \autocite*[726]{rangel1989}, através da bula papal de Alexandre IV, de
4 de maio de 1493 (ainda que tenha sido depois alterada pelo tratado de
Tordesilhas), toda a terra onde hoje encontra-se a América Latina era declarada
propriedade do rei. Isto é, estava satisfeito o primeiro princípio para a
implantação do feudalismo nos trópicos: \emph{all land is king's land}. A propriedade
sobre as terras era total, de maneira que podesse dizer que, juridicamente, em
nenhum momento a propriedade fundiária esteve mais concentrada do que naquele
primeiro momento.

O segundo princípio, no entanto, \emph{nulle terre sans seigneur}, ou seja, o
princípio de que não deve haver terra sem senhor, também indispensável para a
existência do feudalismo -- no surgimento do feudalismo na Europa, sem que todas
as terras social e economicamente significativas estivessem apropriadas, a
tendência natural do escravo liberto seria o retorno às condições de vida
tribal -- não era possível em território tão vasto e inexplorado como era o
território latino-americano naquele momento \autocite[726]{rangel1989}.

Desta maneira, os feudos que aqui se iam estabelecendo, através do instituto da
enfiteuse \autocite[726]{rangel1989}, os pactos de suserania-vassalagem que iam do servo
do gleba ao rei, passando por diversos patamares, muito diferiam dos feudos
europeus da Alta Idade Média, que ao contrário dos pactos aqui estabelecidos,
começavam a ser constituídos pela base, convertendo os escravos libertos em
servos e constituindo a pequena e a grande nobreza, ``tendendo afinal a, com o
tempo, colocar no píncaro o rei'' \autocite[727]{rangel1989}.

A esse respeito também escreveu Alceu Amoroso Lima \autocite*[51]{amoroso}, na grande
obra organizada por Vicente Licínio Cardoso:
\begin{citacao}
Foi-se vendo pouco a pouco – e até hoje o vemos ainda com surpresa, por vezes –
que o Brasil se formara às avessas, começara pelo fim. Tivera Coroa antes de ter
Povo. Tivera parlamentarismo antes de ter eleições. Tivera escolas superiores
antes de ter alfabetismo. Tivera bancos antes de ter economias. Tivera salões
antes de ter educação popular. Tivera artistas antes de ter arte. Tivera
conceito exterior antes de ter consciência interna. Fizera empréstimos antes de
ter riqueza consolidada. Aspirara a potência mundial antes de ter a paz e a
força interior. Começara em quase tudo pelo fim. Fora uma obra de inversão.
\end{citacao}
Segundo Rangel \autocite*[729]{rangel1989}, as condições em que operavam os nossos
feudos mais se assemelhavam às vigentes na República Romana e nos primeiros
tempos do Império, o que quer dizer que, aqui, internamente, até que o monopólio
da terra estivesse garantido, somente haveria viabilidade para o sistema
escravagista.

Com efeito, é sabido que foi necessário importar o escravo africano, que era
socialmente mais avançado que os índios que aqui habitavam, fazendo-o
prisioneiro do latifúndio, haja vista que o índio estava habituado a prover o
seu sustento de forma natural nas terras que aqui habitavam.

A Coroa portuguesa \autocite[731]{rangel1989}:
\begin{citacao}
não tinha pressa em dispor de todas as suas terras, mas apenas das suficientes
para implantar fazendas e estâncias, deixando aberta a porta para novas doações,
que comprassem novas vassalagens, aumentando o poder, a riqueza e a glória da
Coroa.
\end{citacao}
Assim, sobravam terras entre uma fazenda e outra, o que impossibilitava o modo
de produção feudal (pela não satisfação do princípio \emph{nulle terre sans
seigneur}), mas apenas o modo de produção escravista. Exceto por algumas regiões
do Brasil onde a pecuária extensiva logrou ocupar uma vasta extensão contínua de
terra, como no Rio Grande do Sul, o feudalismo só viria a se estabelecer muito
tempo depois, com a abolição da escravidão (1888) e a Proclamação da República
(1889) \autocite[732-733]{rangel1989}.

Porém, para que fosse possível o fim da escravidão sem que houvesse retorno às
formas primitivas de produção pré-escravagistas, foi necessário um longo
processo que teve início com a Lei do Tráfico e a Lei de Terras, ambas de 1850
\autocite[732]{rangel1989}.

Enquanto a Lei do Tráfico levaria inevitavelmente ao fim da escravidão em algum
ponto futuro, já que a ``lei demográfica peculiar ao escravismo é a reprodução
restrita, o que supõe aportes constantes de mão de obra alienígena'', a Lei de
Terras preparava o território para o novo regime que teria lugar, o feudalismo,
através da promoção da efetiva ocupação do território, ou seja, de todas as
terras acessíves, habitáveis e agricultáveis \autocite[732-733]{rangel1989}.

Vale dizer que, onde a condição \emph{nulle terre sans seigneur} não logrou após a
abolição da escravidão, como no estado do Maranhão, houve retrocesso a relações
de produção pré-escravistas \autocite[733-734]{rangel1989}.

\hypertarget{a-crise-do-feudalismo}{%
\subsection{A crise do feudalismo}\label{a-crise-do-feudalismo}}

A crise clássica da sociedade feudal ocorre quando a produção agrícola não
consegue suprir a demanda da superpopulação gerada. Segundo Rangel
\autocite*[219]{rangel1961}:
\begin{citacao} 
"tempo houve em que a expansão do estoque populacional era objetivamente a
maneira mais eficaz de expandir as forças produtivas e o produto social. Nesse
tempo (regime feudal), a riqueza dos príncipes se media pelas almas dos seus
domínios, e aumentar o número destas era a maneira óbvia de expandir aquela
riqueza e também a do corpo social. Este foi forjando para si uma ética, um
direito e uma política conducentes a esse resultado".
\end{citacao}
A crise do feudalismo, sistema eminentemente agrário, e o consequente surgimento
do capitalismo, com o surgimento das cidades modernas, se dá no contexto da
dissolução do Complexo Rural, o que descreve-se nas seções a seguir.

\hypertarget{a-crise-do-feudalismo-no-brasil}{%
\subsection{A crise do feudalismo no Brasil}\label{a-crise-do-feudalismo-no-brasil}}

O feudalismo no Brasil desenvolveu-se a partir da Abolição-República,
concomitantemente com a implantação, especialmente no quadro urbano, de uma
vigorosa economia capitalista. No campo, ao lado do velho latifúndio feudal,
logo surgiu outro latifúndio que, em vez de distribuir lotes entre os agregados
-- como seria natural na desintegração do feudalismo clássico -- empreendeu, ele
próprio, a atividade agrícola, usando mão-de-obra assalariada \autocite[
738-739]{rangel1989}.

Segundo Rangel \autocite*[739]{rangel1989}, o latifúndio feudal, então, percebendo-se
que havia tendência de seus agregados deixarem de lado o trabalho nos lotes
que haviam recebido no processo de abolição da escravidão, para trabalhar nas
novas fazendas capitalistas, logo comecou a deslocar esses agregados, dando
origem ao processo do êxodo rural.

\hypertarget{o-complexo-rural}{%
\subsubsection{O Complexo Rural}\label{o-complexo-rural}}

Segundo Rangel \autocite*[p.98]{rangel1956}, a unidade agrícola fechada é
\begin{citacao}
um microcosmo econômico no qual as pessoas distribuem seu tempo entre numerosas
atividades. Cada uma dessas atividades representa o estado rudimentar daquilo
que, com o desenvolvimento, se tornará uma 'indústria' (...) É evidente que o
camponês não tem consciência da multiplicidade de suas atividades. Ele considera
que elas formam um todo indivisível. Essa inespecialização é sua especialidade.
(\ldots) Chamaremos esse microcosmo econômico, essa 'matriz de insumo-produto' em
miniatura, de 'complexo rural'.
\end{citacao}
O estudo do desenvolvimento do capitalismo não pode ser feito sem o estudo das
bases para o seu desenvolvimento. O capitalismo é um sistema político-econômico
que tem surgimento com a queda do feudalismo, outro sistema político-econômico
cujo enredo se passa, basicamente, no campo. A classe burguesa, aliás, como diz
a história, era formada inicialmente pelos habitantes dos burgos, que se
localizavam no entorno dos feudos. Estes formavam, no entanto, uma minoria.
Durante a idade média, a maior parte da população vivia nos feudos, que se
constituiam de grandes áreas cercadas e isoladas umas das outras, com economia
quase auto-suficiente.

Na economia feudal, portanto, não existia grande grau de especialização das
atividades econômicas, como há hoje. Devido à precariedade do comércio, era
praticamente imperioso que, no interior de cada feudo todas as atividades
econômicas fossem executadas para a própria sustentabilidade do mesmo.

Segundo Lenin (\emph{apud} \textcite[p.~99]{rangel1954}), mesmo após o surgimento do capitalismo,
nos países periféricos, esta realidade feudal ou quase-feudal, deve ser levada
em consideração:
\begin{citacao} 
A população de um país de economia mercantil debilmente desenvolvida (ou não 
desenvolvida de todo) é quase exclusivamente agrícola. Todavia, não se deve 
deduzir daí que ela se ocupa só da agricultura. Significa apenas que a população 
ocupada na agricultura transforma, ela mesma, os produtos da terra, sendo quase 
inexistentes o intercâmbio e a divisão do trabalho. 
\end{citacao}
\hypertarget{condiuxe7uxf5es-e-muxe9todos-de-abertura-do-complexo-rural}{%
\subsubsection{Condições e Métodos de abertura do Complexo Rural}\label{condiuxe7uxf5es-e-muxe9todos-de-abertura-do-complexo-rural}}

Para a abertura do Complexo Rural é necessário que haja vantajosidade para a
economia de mercado e para a economia natural do próprio Complexo.
\begin{citacao} 
A Abertura do Complexo Rural não é uma operação momentânea, mas sim um largo
processo, com altos e baixos e problemas sempre novos. Sua história está muito
longe de ser idílica. Ao contrário, está cheia de violência. Uma planificação
econômica que não resolva preliminarmente este problema é inconcebível.
Alternadamente, pode conduzir à liberação de mais fatores que aqueles que os
setores não agrícolas podem usar, fazendo toda a economia submergir em uma crise
profunda, ou condenar esses setores à estagnação por insuficiência de
fatores \cite[p.~118]{rangel1954}
\end{citacao}
As medidas tendentes a romper o complexo rural podem ser classificadas em dois
grupos \autocite[113]{rangel1954}:
\begin{enumerate}
\def\labelenumi{\alph{enumi}.}
\item
  as que oferecem um incentivo positivo para a incorporação, à economia de
  mercado, dos fatores usados pelo complexo e;
\item
  as que buscam forçar a abertura do complexo a partir de dentro, provocando
  uma deterioração da produtividade das atividades manufatureiras dentro do
  complexo.
\end{enumerate}
As medidas do tipo a) tem seu exemplo mais típico nos EUA e também na França,
enquanto as medidas do tipo b) predominaram na Inglaterra, Alemanha e Japão
\autocite[114-115]{rangel1954}.

\hypertarget{uxeaxodo-rural-e-industrializauxe7uxe3o}{%
\subsection{Êxodo rural e industrialização}\label{uxeaxodo-rural-e-industrializauxe7uxe3o}}
\begin{citacao}
A revolução democrático-burguesa, nos casos em que a gleba feudal é -- como
aconteceu na Europa Ocidental (principalmente, na França) e nos Estados Unidos
-- substituída pela pequena propriedade familiar ou *homestead*, ao fortalecer
as bases da economia natural ou de autoconsumo, resolve satisfatoriamente o
problema na absorção dos excedentes de mão-de-obra no seio da própria economia
camponesa, estancando ou reduzindo drasticamente o fluxo populacional
responsável pelo êxodo campo-cidade \cite[p.~133]{rangel1986b}. 
\end{citacao}
Segundo Rangel \autocite*[133]{rangel1986b}, no entanto, ``esse tipo de superação das
relações de produção feudais'', ou seja, a revolução democrático-burguesa, ``não é
característico do Brasil. Sem embargo do surgimento de algumas `ilhas' de
pequena propriedade camponesa, notadamente nas áreas de colonização européia e
japonesa nos estados do Sul, que mais confirmam a regra.''

Pelo contrário, ``o modelo de desenvolvimento do capitalismo na agricultura
brasileira foi e é a grande exploração capitalista, cada dia mais propensa ao
uso de mão-de-obra assalariada e tendendo sempre ao desmantelamento das bases da
economia natural, causando por isso mesmo, o fenômeno do \textbf{êxodo rural}''
\autocite[134, grifo nosso]{rangel1986b}.

O caso brasileiro, porém, não é único: ``a industrialização da Inglaterra fez-se
também, originariamente, nas condições de um enorme excedente de mão de obra,
causado pelo \emph{enclosure}\footnote{\emph{Enclosure} - literalmente, cercamento. Movimento pelo qual os pequenos
  agricultores ingleses foram expulsos de terras, convertidas estas à pecuária, e
  amontoados nos \emph{slums}, ou favelas das cidades industriais nascentes, na
  primeira metade do século XIX.} \ldots{}''. No caso inglês, porém, ``o motor primário'' do
desenvolvimento foi a produção manufatureira para exportação, enquanto no Brasil
a industrialização teve seu desenvolvimento estimulado, ``nas condições de uma
crônica crise cambial'', pela política de substituição de importações
\autocite[43-44]{rangel1962}.

Ocorre que, de acordo com Rangel \autocite*[134]{rangel1986b}, ``a superabundância e a
barateza da mão-de-obra não costumam ser bons condicionantes do processo de
industrialização, dado que desestimulam a formação de capital, isto é, o
investimento. Ora, numa economia capitalista, o investimento é o motor primário
do desenvolvimento \ldots{}''.

Por este motivo, a ``economia brasileira, nas condições de uma crise agrária
profunda e crônica que, entre outras coisas, \textbf{causava uma urbanização
monstruosa}, sem comparação possível com a demanda de mão-de-obra que a
indústria e os serviços não-agrícolas estavam suscitando nas cidades (perto de
três milhões de novos citadinos a cada ano)\ldots{}'' \autocite[134]{rangel1986b}.

\hypertarget{o-uxeaxodo-rural-como-obstuxe1culo-ao-desenvolvimento}{%
\subsubsection{O êxodo rural como obstáculo ao desenvolvimento}\label{o-uxeaxodo-rural-como-obstuxe1culo-ao-desenvolvimento}}
\begin{citacao}
Ordinariamentem a industrialização pressupõe certa escassez latente de 
mão-de-obra, levando o empresário capitalista a buscar, pelo emprego de bens
modernos de equipamento, economizar o fator trabalho. O resultado é a elevação 
da taxa de investimento, o aumento da procura de bens de capital e de novas
construções, para o que se torna mister empregar mais mão-de-obra 
\cite[p.~43]{rangel1962}.
\end{citacao}
No capitalismo, é conhecido o papel do investimento ou formação de capital nas
taxas de desemprego. Segundo Rangel \autocite*[156]{rangel1988}, ``por um lado, via
efeito multiplicador (efeito para trás), o investimento cria emprego de
mão-de-obra; por outro lado, via implementação de nova tecnologia, promove
dispensa de mão-de-obra (efeito para frente)''.

Segundo Rangel \autocite[142]{rangel1986c}, um ``\,`exército industrial de reserva'
limitado, isto é, algum desemprego, pode ser considerado útil, do ponto de vista
da produção capitalista, porque serve de instrumento de coerção para os
trabalhadores livres, fortalecendo assim a disciplina no trabalho''. No entanto,
quando este torna-se excessivo, ``pode converter-se em obstáculo ao
desenvolvimento da própria economia capitalista. Ora, aqui está o nosso
problema, dado que o `exército industrial de reserva' brasileiro tornou-se
teratologicamente grande. Por isso mesmo, a questão agrária, que se exprime
precipuamente pela formação desse `exército', não interessa apenas aos
camponeses, mas à sociedade como um todo.''

De acordo com Rangel \autocite*[156]{rangel1988}, ``a via democrática -- divisão dos
latifúndios em pequenas propriedades -- ao favorecer uma distribuição menos
desigualitária de renda, cria condições para um vigoroso efeito multiplicador
dos investimentos, isto é, forte efeito para trás. Inversamente, a via
prussiana, ao promover uma distribuição de renda mais desigualitária, debilita o
efeito multiplicador, isto é, para trás, mas, por força da concentração de
renda, aumenta o peso relativo dos investimentos dispensando mão-de-obra e, por
isso mesmo, aumentando o efeito para diante.''

\hypertarget{reforma-agruxe1ria}{%
\subsection{Reforma agrária}\label{reforma-agruxe1ria}}

Como foi visto, o assunto é complexo e requer uma análise de todo o contexto
econômico, social e político vigente. A tão sonhada ``reforma agrária'' a que
normalmente se refere a mídia, os movimentos sociais ou a população em geral,
deveria ter tido lugar ainda na década de 1930, ou seja, em fase anterior ao
início da industrialização brasileira. Para Rangel \autocite[154]{rangel1986a}, a
``reforma agrária, no sentido convencional da expressão, isto é, a implantação de
propriedade familiar suficientemente ampla, para permitir, ao lado da produção
agrícola para o mercado, uma produção complementar agrícola e não-agrícola, isto
é, para autoconsumo, pode justificar-se em certos casos, especialmente quando
seja possível o renascimento da policultura tradicional e onde a fazenda
capitalista, mono ou oligoculturista, ainda não tenha aparecido''. No atual
contexto econômico, porém, esta reforma, com a dissolução do latifúndio
produtivo para assentamento de famílias, levaria a uma regressão tecnológica no
campo, o que seria altamente prejudicial para a economia brasileira como um
todo.

Isto dito, Rangel então propõe então que, no Brasil, com a agricultura
capitalista plenamente desenvolvida, uma segunda variante de reforma agrária,
``\emph{não necessariamente rural}'', com a finalidade de ``recompor a economia natural
onde quer que isto seja possível'', viabilizando ``uma produção complementar,
deixando a produção agrícola para o mercado a cargo da fazenda capitalista com
mão-de-obra assalariada'',seja implementada \autocite[155]{rangel1986a}.

\hypertarget{considerauxe7uxf5es-a-respeito-da-valorizauxe7uxe3o-da-terra-no-brasil}{%
\section{Considerações a respeito da valorização da terra no Brasil}\label{considerauxe7uxf5es-a-respeito-da-valorizauxe7uxe3o-da-terra-no-brasil}}

Para Rangel \autocite*[138-139]{rangel1986b}, o problema da terra é uma questão
financeira. Quer com isso dizer que, ultrapassados os problemas jurídicos da
nossa legislação pré-capitalista (Lei de Terras de 1850), que dificultava a
comercialização da terra, o problema do acesso à terra resume-se ao problema da
capacidade do camponês de comprá-la, o que deveria ter se tornado possível
devido ao esperado declínio do preço da terra que viria com a expansão das
fronteiras agrícolas, mas que não ocorreu, devido à \emph{demanda especulativa}, o
que é um \emph{problema financeiro}.

\hypertarget{a-tenduxeancia-uxe0-elevauxe7uxe3o-de-preuxe7os-nos-peruxedodos-de-recessuxe3o-econuxf4mica}{%
\subsection{A tendência à elevação de preços nos períodos de recessão econômica}\label{a-tenduxeancia-uxe0-elevauxe7uxe3o-de-preuxe7os-nos-peruxedodos-de-recessuxe3o-econuxf4mica}}

Considerando-se que a \emph{demanda especulativa} aumenta nos períodos de recessão,
quando não há melhores oportunidades de investimento, a tendência é que o preço
da terra varie inversamente à \emph{eficácia marginal do capital} (ver \ref{capital}).

Isto se explica pelo motivo que, em períodos de recessão econômica, a
atratividade dos investimentos na economia real diminui. Os capitais sobrantes
do período anterior de expansão/acumulação, então, na falta de boas
oportunidades de aplicação, torna-se `excessivo' e ocasionando a diminuição da
\emph{eficácia marginal do capital}, o que se reflete na taxa de juros básica da
economia. Esta diminuição da rentabilidade do capital faz com que os
investidores procurem formas alternativas de alocação financeira destes
capitais, ou seja, há um aumento da \emph{demanda especulativa}, seja no mercado
imobiliário, seja no mercado mobiliário.

Outros fatores também podem agravar o problema. A entrada do Estado no mercado
fundiário como comprador com fins de realização de reforma agrária, por exemplo,
de acordo com Rangel \autocite*[128]{rangel1985}, promoveria a elevação do preço da
terra, o que aumentaria ainda mais o problema agrário.

Desta forma, a reforma agrária viria naturalmente com a solução do problema
financeiro da economia, ou seja, com a abertura de novas possibilidades de
investimentos que diminuíssem a demanda especulativa sobre a terra. Como a
demanda de terra para cultivo e construção seria pequena em relação à demanda
especulativa, o preço da terra tenderia a cair naturalmente \autocite[139]{rangel1986b}.

\hypertarget{pressupostos-da-anuxe1lise-de-rangel}{%
\section{Pressupostos da análise de Rangel}\label{pressupostos-da-anuxe1lise-de-rangel}}

Ora, lá se vão quase 35 anos da análise de Rangel. Seria no mínimo questionável
ainda, senão a validade da análise, mas pelo menos a sua atualidade.

Em primeiro lugar, deve-se ter em conta a questão da inflação brasileira. É de
Rangel também a tese de que a inflação brasileira se exacerba(va) em períodos de
conjuntura fraca, ao contrário do senso comum, de que a inflação é um fenômeno
advindo do superaquecimento da economia \autocite{inflacao-brasileira}. Segundo \textcite{rangel},
a inflação brasileira desempenhava ``\emph{um papel bloqueador das quedas de
conjuntura, via penalização da liquidez, que induzia investimentos que, de outro
modo, isto é, nas condições de moeda estável, não se fariam}''. Quer dizer com
isso que, dada a baixa eficácia marginal do capital em períodos de baixa
conjuntura (ver Apêndice \ref{capital}), o incentivo aos investimentos
vinham através das taxas reais de juros negativas (pelo efeito da alta inflação)
que induziam os investidores a fugir da liquidez, já que a inflação
desvalorizava a moeda, tornando rentável aplicar recursos em investimentos nem
tão rentáveis, talvez até com retornos negativos, porém que acarretariam perda
menor do que o investidor teria se mantivesse a sua posição em moeda.

Desta forma, a inflação era um grande indutor da imobilização de capital em
terras, seja para pura reserva de valor, seja por motivos especulativos,
aguardando a sua valorização. Segundo \textcite{rangel1986b}, estudos levados a cabo pelo
IPEA \emph{a posteriori} confirmaram a sua hipótese da variação do preço da terra em
sentido inverso da eficácia marginal do capital que, \emph{coeteris paribus},
determina o preço dos valores mobiliários.

Com o advento da estabilidade da moeda, no entanto, a análise de Rangel caducou.
Essa demanda por terra para proteção do processo inflacionário não mais se
justifica.

Por outro lado, não consta que a demanda especulativa sobre a terra tenha caído,
pelo menos não a ponto do preço da terra cair a um nível que possibilitasse uma
verdadeira reforma agrária, conforme previa Rangel. Deve-se buscar, portanto os
motivos por trás desta nova onda de especulação em torno do valor da terra, o
que será feito no capítulo \ref{economia}.

Deve-se levar em conta, no entanto, que a análise de Rangel é prévia à abertura
das contas de capital do balanço de pagamento, ou seja, do estabelecimento do
livre (in)fluxo de capital estrangeiro no país, o que deve ter sido o
responsável pela manutenção da demanda especulativa sobre o preço da terra até
os tempos recentes.

Ainda há de se considerar que a crise agrária crônica não cessou. Pelo contrário,
se agravou e vem se agravando cada vez mais, tendo o Brasil atingido uma
proporção de população urbana muito maior do que o seu grau de desenvolvimento
econômico possibilitaria.

No entanto, outro fator que deve ser levado em conta é estrutural: as taxas de
juros, sejam de curto ou longo prazo, atingiram os patamares mais baixos da
série histórica, desde 1954, como será visto no capítulo \ref{economia}.

A grave crise econômica de 2008 e seus efeitos que o mundo vive até os dias
atuais teve grandes consequências sobre o preço dos ativos imobiliários mundo
afora. A crise de 2008 gerou uma resposta por parte dos Bancos Centrais de
praticamente todo o mundo, mas em especial os Bancos Centrais dos países
desenvolvidos, no sentido de um aumento nunca antes visto da liquidez dos
sistemas financeiros mundiais, através da redução aos limites mínimos das taxas
de juros de curto prazo, mas também da aplicação de medidas de recompra de
títulos de mais longo prazo, reduzindo-se assim também a níveis baixíssimos toda
a estrutura a termo da taxa de juros. Talvez Rangel nem tivesse cogitado que as
taxas de juros um dia chegariam a patamares tão baixos que levariam a enormes
valorizações do preço da terra de maneira não-especulativa, mas de acordo com os
fundamentos econômicos.

Estas taxas de juros muito baixas, tanto a curto quanto a longo prazo, tem
estimulado financiamentos imobiliários a taxas de juros negativas no mundo
desenvolvido \autocite{serapicos}, criando um aumento vertiginoso dos preços dos
imóveis, muito acima dos fundamentos econômicos de um mundo em estagnação
econômica crônica, o que vem levando os governos destes países a tomarem medidas
não-usuais para a regulação do mercado imobiliário, especialmente no que se
refere ao controle dos preços dos aluguéis, devido à crise habitacional que se
instalou nestes países desde a crise de 2008 \autocite{berlim,londres,california,suecia,newzeland,finlandia}.

\hypertarget{efeitos-da-falta-de-reforma-agruxe1ria-no-cadastro-urbano}{%
\section{Efeitos da falta de reforma agrária no cadastro urbano}\label{efeitos-da-falta-de-reforma-agruxe1ria-no-cadastro-urbano}}

As altas proporções da população urbana nos países ditos \emph{em desenvolvimento} em
comparação aos \emph{países desenvolvidos} não nos permite imaginar que as
ferramentas de planejamento urbano desenvolvidas no primeiro mundo surtam o
efeito esperado em outras regiões menos desenvolvidas do globo, pelo menos não
na atual realidade econômico-social. E isto também se aplica, é claro, ao
cadastro territorial multifinalitário.

Enquanto a reforma agrária ideal almejada por muitos infelizmente não tem lugar,
são o cadastro e as outras ferramentas de planejamento urbano que devem se
adaptar a essa outra realidade particular da paisagem dos países
subdesenvolvidos. Tentar, pelo contrário, promover a fórceps a modificação da
terrível paisagem urbana destes locais para que se enquadrem nos modelos
teóricos do mundo desenvolvido passaria, necessariamente, pela expulsão das
classes menos favorecidas das grandes cidades, sem que exista do outro lado uma
porta de saída.

Obviamente, compreendido este contexto histórico do desenvolvimento do
capitalismo no Brasil, que não difere muito do desenvolvimento capitalista dos
outros países da América Latina, não seria de se esperar que o cadastro
territorial multifinalitário, assim como outras ferramentas do planejamento
urbano, como concebidos nos países desenvolvidos, possam ser replicadas, sem as
devidas adaptações, nessa outra realidade, quase que completamente oposta.

Desta maneira, vem em boa hora a implantação de núcleos de estudos específicos
para o estudo e disseminação do cadastro na América Latina \autocite{lalan}, em que
entende-se que deve-se, contudo, concentrar os esforços destes núcleos na
adaptação das ferramentas clássicas do CTM à realidade regional.

\hypertarget{economia}{%
\chapter{O Mercado Imobiliário e a Economia}\label{economia}}
\begin{epigrafe}
	\vspace*{\fill}
	\begin{flushright}
    \textit{``a economia é uma ciência histórica por excelência -\\
    qualidade que partilha com outras ciências sociais.\\
    Quer isso dizer que está submetida a um duplo processo\\
    evolutivo: o fenomenal (como representação, como ideia \\
    da coisa, como `coisa para nós', no sentido kantiano)\\
    e o numenal (como objeto, coisa representada, `coisa em si')\\
    \ldots e não pode ser estudada senão nesse duplo contexto.''\\
    (Rangel, 1956, p. 204)}
	\end{flushright}
\end{epigrafe}
\hypertarget{os-vuxe1rios-significados-do-termo}{%
\section{\texorpdfstring{Os vários significados do termo \emph{economia}}{Os vários significados do termo }}\label{os-vuxe1rios-significados-do-termo}}

A economia é uma ciência social. Apesar do desenvolvimento recente das ciências
econômicas ter levado muitos dos economistas da atualidade, em sua prática, a se
assemelharem muito mais aos profissionais das ciências exatas, como um cientista
de dados, um estatístico, a Economia como ciência continua, a luz dos últimos
fatos da história recente, uma ciência histórica por excelência, como dizia
Rangel. O `fim da história' \autocite{fukuyama}, \emph{i.e.} a ingênua tese de Francis
Fukuyama sobre a naturalização do capitalismo neoliberal, sobre a forma racional
de vida finalmente encontrada \autocite[401]{zizek2011}, indiscutivelmente não se
concretizou de fato: o próprio Fukuyama hoje admite este fato
\autocite{menand_francis_nodate}.

Para \textcite{rangel1956}, ``ciência é classificação e medida - não apenas medida, como
pode se depreender do lema econométrico. Se ciência fosse medida não haveria
ciência em Aristóteles ou Hegel'' \autocite[204]{rangel1956}.

Desta maneira, pode-se traçar um paralelo das noções sobre o termo ``economia'',
com as noções sobre outros termos, como religião e ideologia, como neste
parágrafo de Zizek, onde o autor trata do conceito de ``ideologia'', a partir das
noções de Marx sobre ``religião'':
\begin{citacao}
A propósito da religião (que, para Marx, era a ideologia por excelência), Hegel
distinguiu três momentos: \emph{doutrina, crença e ritual}; assim, fica-se
tentado a distribuir em torno desses três eixos a multiplicidade de ideias
associadas com o termo `ideologia': a ideologia como um complexo de ideias
(teorias, convicções, crenças, métodos de argumentação); a ideologia em seu
aspecto externo, ou seja, a materialidade da ideologia, os Aparelhos Ideológicos
de Estado; e por fim, o campo mais fugidio, a ideologia `espontânea' que atua
no cerne da própria `realidade' social.
\cite[p.~15]{zizek}
\end{citacao}
Da mesma forma, para Singer \autocite*[7]{singer}, é possível distinguir pelo menos três
significados do termo \emph{economia}:
\begin{itemize}
\tightlist
\item
  a qualidade de ser estrito ou austero no uso de recursos ou valores;
\item
  a característica comum de uma ampla gama de atividades que compõe a \emph{economia}
  de um país, de uma cidade, etc.
\item
  a ciência que tem por objeto a atividade que dá o segundo significado.
\end{itemize}
A economia (ciência, ``coisa para nós''), então, é a sistematização do
conhecimento sobre a economia (atividade, ``coisa em si''). Assim, em suma, quando
a ``coisa em si'' muda, também deve mudar a ``coisa para nós'', ou seja, o resultado
contingente do processo histórico inacabado continua a mudar as ciências
econômicas hoje, tal qual ocorreu em diversos outros momentos, como ensina
Rangel, a respeito das várias fases do desenvolvimento da ciência Econômica,
ocorridas de acordo com as mudanças estruturais ou materiais (mudança da
``coisa em si''):
\begin{citacao}
Quando a oferta de mão de obra, por efeito do próprio desenvolvimento do 
capitalismo, se tornou relativamente inelástica, a análise clássica pereceu, 
porque ela supunha uma oferta perfeitamente elástica desse fator. A morte da 
coisa em si acarretou a morte da coisa para nós.
Quando se tornou patente que o fluxo de pagamento aos fatores não estava gerando
mais uma procura efetiva da mesma magnitude que a oferta efetiva que o emprego
destes fatores possibilitava, e que as discrepâncias não eram transitórias,
capazes de corrigir-se progressivamente pelo processo da crise, foi preciso 
abandonar outro suposto dos clássicos, que havia passado aos neoclássicos, 
implicitamente, sob a forma da `lei dos mercados'. A problemática econômica
mudara inteiramente e, para resolver os novos problemas, tivemos a teoria
keynesiana.
Simultaneamente, quando, graças a certas mudanças institucionais trazidas pela
Revolução Russa, criaram-se condições concretas que permitem tornar a procura
efetiva cada vez mais independente do preço dos fatores, o planejamento 
econômico tornou-se possível e tivemos as teorias que correspondem à nova
problemática. O economista deixara de ser um `meteorologista' da conjuntura
para fazer-se fautor da sua própria conjuntura.
\cite[p.~206]{rangel1956}
\end{citacao}
Nos próximos itens será visto como as diferentes escolas de pensamento, a saber,
as escolas \emph{marxista} e \emph{marginalista}, abordam os diferentes conceitos do termo
``economia''.

\hypertarget{a-economia-como-atividade}{%
\subsection{A economia como atividade}\label{a-economia-como-atividade}}

A ciência se divide a respeito da definição de economia como atividade, entre
social (escola \emph{marxista}) e individual (escola \emph{marginalista}) \autocite[9]{singer}.

Enquanto para os \emph{marxistas} a atividade econômica é sempre coletiva, praticada
mediante a divisão social do trabalho, para os \emph{marginalistas} a atividade
econômica é em sua essência individual, que atuam autonomamente, tendo em vista
apenas seus desejos ou suas necessidades \autocite[10]{singer}.

\hypertarget{a-economia-como-ciuxeancia}{%
\subsection{A economia como ciência}\label{a-economia-como-ciuxeancia}}

Também diferem os \emph{marxistas} e os \emph{marginalistas} quanto a definição de
economia como ciência.

Enquanto para os \emph{marxistas} a economia política é a ciência do social,
abrangendo em seu campo de estudo o conjunto de atividades que formam a vida
econômica da sociedade \autocite[14]{singer}, para os \emph{marginalistas} a
ciência econômica tem como modelo as ciências da natureza, onde cada uma das
quais tem como objeto próprio um determinado ``setor'' do universo físico.
Analogamente, as ciências do homem teriam como objeto um ``setor'' do universo
humano \autocite[15]{singer}.

\hypertarget{a-economia-como-ciuxeancia-social}{%
\subsection{A Economia como Ciência Social}\label{a-economia-como-ciuxeancia-social}}

Como Ciência Social, a Economia pode ser definida como a ciência:
\begin{citacao}
que estuda como as pessoas e a sociedade decidem empregar recursos escassos, que
poderiam ter utilização alternativa, na produção de bens e serviços de modo a
distribuí-los  entre as várias pessoas e grupos da sociedade, a fim de
satisfazer as necessidades humanas
\cite[p.~5]{passosnogami}.
\end{citacao}
Para Rosseti \autocite*[31]{rossetti}, no entanto, a Economia não é uma ciência com
limites nitidamente definidos, assim como as demais ciências sociais:
\begin{citacao}
À semelhança do que ocorre com os demais ramos das ciências sociais, não se pode
considerar a economia como fechada em torno de si mesma. Pelas implicações da
ação econômica sobre outros aspectos da vida humana, o estudo da economia
implica a abertura de suas fronteiras às demais áreas das ciências humanas. Esta
abertura se dá em dupla direção, assumindo assim caráter \textbf{biunívoco}.
\end{citacao}
Segundo Rosseti \autocite*[32]{rossetti}, a separação das ciências sociais em
especialidades distintas é não-rigorosa, ou, ao contrário, estas especialidades
estão entremeadas:
\begin{citacao}
Em síntese, pode-se inferir que as interfaces da economia com outros ramos do
conhecimento social decorrem de que as relações humanas e os problemas nelas
implícitos ou delas decorrentes não são facilmente separáveis segundo níveis de
referência rigorosamente pré-classificados. O referencial econômico deve ser
visto apenas como uma abstração útil, para que se analisem aspectos específicos
da luta humana pela sobrevivência, prosperidade, bem-estar individual e
bem-comum. Ocorre, todavia, que essa mesma luta não se esgota nos limites do que
se convencionou chamar de relações econômicas. Vai muito além, abrangendo
aspectos que dizem respeito à postura ético-religiosa, às formas de organização
política, aos modos de relacionamento social, à estruturação da ordem jurídica,
aos padrões das conquistas tecnológicas, às limitações impostas pelas condições
do meio ambiente e, mais abrangentemente, à formação cultural da sociedade.
\end{citacao}
Conforme visto no capítulo anterior, a análise de Rangel a respeito da questão
agrária não mais pode ser considerada atual, porém esta análise está longe de
poder ser considerada errada. Como ciência histórica, a Economia está sujeita a
este tipo de caducidade de teorias que, antes de mudanças drásticas ou mesmo de
uma lenta evolução natural da `Economia em si', eram válidas.

Não teria sido a vez, então, da própria análise de Rangel à respeito do
aprofundamento da crise agrária (que como vimos reflete nas cidades) ``morrer''
com a estabilidade da moeda alcançada pela economia brasileira desde meados da
década de 90 do século passado, ou ao menos mais recentemente, nos últimos anos,
quando a inflação brasileira parece ancorada, a despeito da prática de taxas de
juros internas (e externas) muito próximas de zero? Pelo menos enquanto a
perdurar a estabilidade da moeda, ao menos, é esta a realidade.

Na atualidade, a economia brasileira pode estar no caminho de experimentar um
fenômeno novo: a armadilha da liquidez \autocite[ver][]{krugman-emergentes}, problema que
aflige os países de economia desenvolvida já há alguns anos, é uma espécie de
`incapacidade' da economia gerar inflação suficiente para que a taxa de juros
básica real da economia chegue a patamares baixos o suficientes para que iguale
a eficácia marginal média do sistema. Em outras palavras, a inflação deixa de
ser um fator corretivo da conjuntura, passando a ser insuficiente para tal. No
entanto, a análise de Rangel não deixa de ser útil, na medida em que explica
como chegamos a situação presente. Como explicar este déficit habitacional
crônico senão através da crise agrária? Como explicar o aumento do preço da
terra em períodos de baixa conjuntura, que agravava a situação agrária, levando
milhares de pessoas a se aglomerarem em situação precária nas cidades
brasileiras? Por que não houve a esperada queda do preço da terra, como
esperado? Que tipo de soluções podem ser aplicadas ao caso brasileiro?

Estes questionamentos, porém, não podem ser respondidos antes de uma análise do
atual comportamento do \gls{MI} no Brasil. É o que se pretende fazer neste
capítulo. No entanto, se uma análise estatística direta do comportamento do
\gls{MI} no Brasil dificilmente poderá ser feita, pela ausência de séries
históricas de preços de imóveis, uma análise econômica conceitual, aos moldes da
análise de Rangel, pode ter espaço. Estudos futuros tratarão da validade desta
análise conceitual, através da análise dos dados, que porventura estarão
disponíveis.

\hypertarget{o-mercado-imobiliuxe1rio}{%
\section{O Mercado Imobiliário}\label{o-mercado-imobiliuxe1rio}}

Segundo \textcite[p.~188-189]{shelter}, o funcionamento do \gls{MI} residencial pode
ser compreendido de acordo com o diagrama da Figura \ref{fig:MIResidencial}:
existe um mercado de residências que pode ser considerado competitivo. No
processo de produção, contudo, a presença de poucos incorporadores e
proprietários pode lhes dar um grande poder de mercado, sobretudo se eles também
controlarem o mercado de outros insumos, porém isto não é comum. No Mercado de
Insumos, a propriedade pode ser tão concentrada que os proprietários podem ter
poder de fixar os preços, grandes economias de escala podem tornar a produção de
alguns insumos um monopólio natural e a regulação governamental pode restringir
a alocação competitiva de recursos, especialmente os financeiros.
\begin{figure}[H]

{\centering \includegraphics[width=0.7\linewidth]{images/MIResidencial-1} 

}

\caption{Diagrama esquemático do funcionamento do \gls{MI} residencial.}\label{fig:MIResidencial}
\end{figure}
\bcenter

Fonte: Adaptado de \textcite[p.~189]{shelter}
\ecenter

Para \textcite[p.~189]{shelter}, então, o problema dos \gls{MI} residenciais usualmente
se concentram no mercado de insumos, onde as intervenções governamentais
deveriam se concentrar. Contudo, nem sempre é isto que ocorre: muitos governos
decidem intervir diretamente no processo de produção ou no controle de preços ao
consumidor final, como o controle de aluguéis, no entanto isto apenas distorce
os sinais enviados ao mercado e podem exacerbar o problema original.

\hypertarget{macro}{%
\section{O Mercado Imobiliário e a Macroeconomia}\label{macro}}

Segundo \textcite{LEUNG}, há um reconhecimento relativamente recente e crescente
sobre a importância da interação entre os diversos mercados imobiliários entre
si e entre o \gls{MI} como um todo e a macroeconomia.

Pesquisas em economia habitacional convencional e em economia urbana
praticamente ignoram as interações com a macroeconomia. Na melhor das hipóteses,
algumas análises teóricas e empíricas da economia urbana e habitacional incluem
variáveis macroeconômicas (como inflação, crescimento econômico, PIB, taxa de
desemprego etc.) como `variáveis de controle' exógenas \autocite[3]{LEUNG}.

Por sua vez, os livros de Economia tradicionais ou tratam o mercado imobiliário
como apenas um dos muitos mercados de bens de consumo ou o negligenciam como um
todo. A Macroeconomia convencional ignora completamente o mercado imobiliário
\autocite[3]{LEUNG}, embora o mercado habitacional constitua uma grande parte da
Macroeconomia \autocite[5]{LEUNG}. \textcite{krugman} afirmou que um dos segredos da
política monetária reside no fato que a política monetária funciona através do
mercado imobiliário, tendo pouco impacto direto no investimento em negócios.

Segundo Greenwood e Hercowitz \autocite[\emph{apud}][5]{LEUNG}, o estoque de capital
imobiliário é maior do que o capital de negócios e, em geral, o valor de mercado
investido anualmente no mercado habitacional é maior do que o investimento em
negócios, o que claramente faz do segmento habitacional muito mais do que apenas
um outro mercado de bens de consumo.

Davis e Heathcote (2001) \autocite[\emph{apud}][6]{LEUNG} afirmam que o valor de mercado das
propriedades imobiliárias em estoque nos EUA é aproximadamente igual ao PIB
médio anual. Segundo a revista britânica The Economist \autocite{economist}, a maior
classe de ativos no mundo é a propriedade residencial, com valor estimado de 200
trilhões de dólares, o que equivale a 3 vezes mais o valor de todas as ações
negociadas em bolsa.

No Brasil, segundo \textcites{fnogueira}[155]{fnogueira}, a participação dos ativos de
base imobiliária era cerca de 40\% dos bens e direitos declarados na Declaração
do Imposto de Renda da Pessoa Física. Deve-se levar em conta, no entanto, que
os valores declarados são os valores do momento da aquisição dos imóveis, que
não são atualizados para fins de declaração de imposto de renda, portanto estes
são usualmente menores do que os valores de mercado.

Segundo \textcite[p.~4]{LEUNG}, no entanto, mais recentemente tem havido um pequeno porém
crescente esforço de pesquisa para preencher a lacuna entre as duas literaturas
e lançar luz sobre questões que são conjuntamente conseqüentes para a
macroeconomia e a habitação.

O mercado imobiliário, diferentemente de outros mercados de capitais, exibe uma
flutuação de valores baixa e não muda imediatamente após a mudança do noticiário
econômico \autocite[3]{ADAMS2010}. O mercado imobiliário residencial, em particular,
mostra forte rigidez pra baixo, porque os donos de imóveis residenciais tendem a
resistir a venda da propriedade sob períodos de recessão econômica \autocite[
129]{Case2000}.

\hypertarget{correlauxe7uxe3o-entre-os-mercados-imobiliuxe1rios}{%
\subsection{Correlação entre os mercados imobiliários}\label{correlauxe7uxe3o-entre-os-mercados-imobiliuxe1rios}}

Em geral, é costume entre os economistas dizer que os mercados imobiliários são
mercados locais, isto é, devido a heterogeneidade dos bens imóveis e,
principalmente, devido à localização espacial, que faz com que cada imóvel seja
único, ou seja, bens imóveis em diferentes mercados não podem ser
considerados bens substitutos.

O comportamento destes mercados, no entanto, não está imune ao que ocorre em
outros locais diferentes do mercado estudado. De fato, estudos mostram que os
\gls{MI} de diferentes países estão altamente correlacionados, dependendo do
grau de abertura econômica de cada país \autocite{Case2000,ADAMS2010}.

A disponibilidade de dados é usualmente um problema na análise dos \gls{MI}. No
entanto, dada a alta correlação demonstrada entre os diversos mercados,
\textcite{ADAMS2010} lograram elaborar um modelo de dados em painéis cointegrados
para diversos mercados, o que mostra a importância do estudo do comportamento do
\gls{MI} de outros países, especialmente na falta de dados disponíveis, como é
o caso do Brasil.

A figura \ref{fig:adams2} mostra que a alta de preços de imóveis nos países
estudados por \textcite{ADAMS2010} são semelhantes e diferem mais pela questão de uma
assincronia, do que pela tendência, \emph{i. e.} a tendência em geral é a mesma para
todos os países, porém há um \emph{lag} entre o início de uma tendência nos diversos
mercados, devido a diferenças nos mecanismos de propagação das variáveis
macroeconômicas em cada país.
\begin{figure}[H]

{\centering \includegraphics[width=\textwidth]{./images/adams2_crop} 

}

\caption{Interconexão dos preços dos imóveis em diversos países.}\label{fig:adams2}
\end{figure}
\bcenter

Fonte: \textcite{ADAMS2010}.
\ecenter

A figura \ref{fig:nhpi} ilustra o comportamento do \gls{MI} residencial
norte-americano desde 1990.
\begin{figure}[H]

{\centering \includegraphics[width=0.7\linewidth]{images/nhpi-1} 

}

\caption{Home Price Index (HPI), de Case e Shiller, em termos nominais.}\label{fig:nhpi}
\end{figure}
\bcenter

Fonte: O autor, à partir de dados de \textcite{QuandlWIKI}.
\ecenter

Nota-se que após o derretimento do \gls{MI} norte-americano depois da crise
imobiliário-financeira de 2007-2008, este mercado voltou a se aquecer fortemente
desde Janeiro de 2012, tendo o nível de preços já ultrapassado o último pico,
atingido em julho de 2006.

\textcite{regulation} também notaram um aumento generalizado nos preços dos imóveis em
diversos países, especialmente após a segunda metade da década de 90, com
exceção do Japão e da Alemanha, conforme pode ser visto na Figura
\ref{fig:several-indices}.
\begin{figure}[H]

{\centering \includegraphics[width=0.7\linewidth]{images/several_indices_regulations} 

}

\caption{Evolução dos índices de preços de imóveis em diversos países.}\label{fig:several-indices}
\end{figure}
\bcenter

Fonte: \textcite[p.~3]{regulation}
\ecenter

A comparação de índices de preços dos diversos países, no entanto, deve ser
feita com cautela, haja vista que os países compõem os seus índices de maneira
diferente, além das estruturas de oferta e demanda destes países serem
diferentes \autocites[ver][p.~4]{regulation}{silver}.

Segundo \textcite[p.~4-5]{regulation}, o aumento do preço dos imóveis está fortemente
ligado ao aumento do preço da terra, ou seja, os custos de construção tem um
papel secundário no aumento do preço dos imóveis. Segundo Davis e Palumbo
\autocite[\emph{apud}][p.~4]{regulation}, é o preço da terra que governa o aumento dos preços
dos imóveis nos EUA, o que se confirma também na China, como pode ser visto na
Figura \ref{fig:land}.
\begin{figure}[H]

{\centering \includegraphics[width=0.7\linewidth]{images/land} 

}

\caption{Aumento nominal e real do valor da terra na China.}\label{fig:land}
\end{figure}
\bcenter

Fonte: \textcite[p.~5]{regulation}
\ecenter

A alta volatilidade do \gls{MI} tal como apresentada nos últimos anos, como será
visto, não é comum. No capítulo \ref{indices} serão analisados os motivos que
levaram o \gls{MI} norte-americano e mundial a este forte aumento de
volatilidade e os efeitos desta grande volatilidade, não apenas no \gls{MI}, mas
na Economia como um todo. A próxima seção apresenta os modelos teóricos por trás
das análises que serão efetuadas no próximo capítulo.

\hypertarget{modelos-economuxe9tricos-para-o}{%
\subsection{\texorpdfstring{Modelos econométricos para o \gls{MI}}{Modelos econométricos para o }}\label{modelos-economuxe9tricos-para-o}}

A elaboração de modelos econométricos ajuda a explicar o comportamento dos
preços no \gls{MI}. Geralmente são elaborados modelos estáticos, que visam
descobrir o preço de equilíbrio do \gls{MI} no longo prazo. Estes modelos
inicialmente aplicavam-se apenas a países onde estavam disponíveis séries
temporais de longo prazo de preços de imóveis, como os EUA e o Reino Unido, já
que as abordagens padrão para cointegração de dados em painel de diversos países
requerem uma quantidade grande de dados para testar as relações de longo prazo
\autocite[2]{ADAMS2010}.

Para contornar o problema da falta de séries temporais de mais longo prazo,
\textcite{ADAMS2010} aplicaram uma abordagem de integração de dados em painel proposta por
Pedroni e mostraram, com a observação de dados em painel de 15 países por um
período de 30 anos, que variáveis macroeconômicas apresentam um significativo
impacto no preço de imóveis residenciais em longo prazo. \textcite[p.~51]{FAN201937}
também chegaram a essa conclusão para mercados de imóveis em 5 diferentes
regiões da China, utilizando modelos dinâmicos.

Segundo \textcite[p.~18]{ADAMS2010}, particularmente variáveis como emprego, produção
industrial e aumento da base monetária demonstraram-se propensas a aumentar a
demanda por imóveis residenciais, aumentando assim os seus preços. Além disto,
um aumento na taxa de juros de curto prazo também afeta positivamente o preço
dos imóveis residenciais, pelo efeito do aumento dos custos de financiamento e
pelo desaquecimento do setor de construção, o que ocasiona também um aumento no
preço dos aluguéis, que também puxa para cima o preço dos imóveis.

Por outro lado, um aumento nas taxas de juros de longo prazo leva a diminuição
da demanda por imóveis residenciais. Por causa da maior atratividade nos
investimentos de renda fixa oriundos do aumento das taxas de longo prazo,
reduz-se a demanda por (investimento em) imóveis residenciais, o que
por conseguinte reduz os seus preços \autocite[19]{ADAMS2010}.

\textcite{ADAMS2010} notaram também que, devido a diferenças no nível regulatório e nas
características do mercado hipotecário, houve um alto grau de variação entre os
países, embora os resultados sejam muito similares mesmo com a variação dos
métodos de estimação.

\textcite{goodhart2008} foram além e mostram a existência de uma ligação multidirecional
entre os preços dos imóveis, a base monetária (ampla), o crédito, e a
macroeconomia.
\begin{citacao}
O aumento da base monetária tem um efeito significativo nos preços das
residências e crédito, o crédito influencia a base monetária e os preços das 
residências e os preços das residências influenciam crédito e a base monetária. 
Este link é considerado mais forte em relação a um sub-amostra de 1985 
a 2006 do que em uma amostra mais longa que remonta ao início dos anos 1970, uma 
descoberta que provavelmente reflete os efeitos das liberalizações do sistema 
financeiro em países industrializados durante os anos 1970 e início dos anos 
1980. Devido à grandes bandas de confiança das respostas ao impulso, este 
resultado não é, no entanto, estatisticamente significante. Os resultados 
sugerem ainda que choques nos preços das residências, crédito e base monetária, 
todos tem repercussões significativas sobre a atividade econômica e a inflação 
agregada de preços. Choques no PIB, IPC e taxa de juros, por sua vez, têm 
efeitos significantes sobre preços das residências, dinheiro e crédito.
\cite[31]{goodhart2008}
\end{citacao}
\textcite{macroHousing} mostraram, no entanto, que a relação entre algumas variáveis
macroeconômicas e o mercado imobiliário podem ter modificado depois da crise
de 2008, o que ainda deve ser confirmado com novas pesquisas, no entanto.

\hypertarget{modelo-de-equiluxedbrio-do-no-longo-prazo}{%
\subsubsection{\texorpdfstring{Modelo de equilíbrio do \gls{MI} no longo prazo}{Modelo de equilíbrio do  no longo prazo}}\label{modelo-de-equiluxedbrio-do-no-longo-prazo}}

Para \textcite[p.~6]{ADAMS2010}, pode-se elaborar um modelo para demanda de bens imóveis
em diferentes países de acordo com a equação \eqref{eq:demanda-geral}, onde
\(x_t^D\) é um vetor de variáveis macroeconômicas afetando a demanda, \(z_t^D\) é
um vetor de características específicas de cada país com influência na demanda
por bens imóveis, como características do mercado de hipotecas, impostos
incidentes e depreciação.
\begin{equation}
\gls{D}_t = \alpha - \beta' x_t^D + \delta' z_t^D + \varepsilon_t
\label{eq:demanda-geral}
\end{equation}
A oferta por bens imóveis, por sua vez, é modelada conforme a equação
\eqref{eq:oferta-geral}, onde \(x_t^S\) é um vetor de variáveis macroeconômicas
afetando a oferta, \(z_t^S\) é um vetor de características específicas de cada
país com influência na oferta por bens imóveis, como provisões de moradias
sociais pelo governo e outras.
\begin{equation}
\gls{S}_t = \eta - \gamma' x_t^S + \lambda' z_t^S + \upsilon_t
\label{eq:oferta-geral}
\end{equation}
Para um número de países grande o suficiente, pode-se assumir que os efeitos das
variáveis \(z_t^D\) e \(z_t^S\) são absorvidas pelos termos de erros \(\varepsilon_t\)
e \(\upsilon_t\).

Assim, o equilíbrio no mercado de bens imobiliários, assim como em outros
mercados de bens, se dá no ponto onde a curva de oferta de bens imóveis encontra
com a curva da demanda por estes bens, \emph{i.e.} no ponto onde as equações de
oferta \eqref{eq:oferta-bens} e demanda \eqref{eq:demanda-bens} por bens
imóveis se encontram é atingido um preço, dito preço de equilíbrio, a que
corresponde uma quantidade de bens produzidos.
\begin{equation}
\gls{D}_t = \alpha - \beta_1 \gls{hp}_t + \beta_2 \gls{EA}_t - 
\beta_3 \gls{long}_t - \beta_4 \gls{short}_t + \tilde \varepsilon_t
\label{eq:demanda-bens}
\end{equation}
Na equação \eqref{eq:demanda-bens}, presume-se que a demanda por bens imóveis
\(\gls{D}_t\) seja afetada negativamente pelos preços dos imóves (\(hp_t\)), pela
taxa de juros de longo prazo (\(long_t\)) e pela taxa de juros de curto prazo
(\(short_t\)) e positivamente pelo nível da atividade econômica (\(EA_t\)).

De acordo com \textcite[p.~7-8]{ADAMS2010}, uma taxa de juros de longo prazo mais alta
tende a desviar a demanda por investimento em imóveis para títulos da dívida
público, de menor risco, ou seja, um aumento da taxa de juros de longo prazo
tende a diminuir a atratividade pelo investimento em imóveis. Já a taxa de juros
de curto prazo, segundo \textcite[p.~7]{ADAMS2010}, deve diminuir a demanda por moradia
por causa do aumento que acarreta nas taxas das hipotecas.

A oferta do mercado de bens, por sua vez, pode ser modelada de acordo com a
equação \eqref{eq:oferta-bens}.
\begin{equation}
\gls{S}_t = \eta + \gamma_1 \gls{hp}_t - \gamma_2 \gls{short}_t - 
\gamma_3 \gls{constr}_t + \tilde \upsilon_t
\label{eq:oferta-bens}
\end{equation}
Para \textcite[p.~7-8]{ADAMS2010}, um aumento da taxa de juros de curto prazo tende a
aumentar os custos de financiamento para construção de moradias, o que impacta
negativamente a oferta, o que é modelado na equação \eqref{eq:oferta-bens} pelo
sinal negativo do termo \(\gls{short}_t\). Outra variável que impacta
negativamente na oferta de bens imóveis é a variável custos de construção,
representada na equação \eqref{eq:oferta-bens} pela variável \(\gls{constr}_t\).
Finalmente, a variável \(\gls{hp}_t\), que representa o preço médio dos imóveis,
normalmente obtida de algum índice de preços do mercado que se está a modelar,
tem um impacto positivo na quantidade ofertada de imóveis (\(\gls{S}_t\)).

O equilíbrio do mercado se dá no ponto de encontro entre as equações de oferta e
procura, ou seja, o equilíbrio é atingido quando o preço dos imóveis é obtido de
acordo com a equação \eqref{eq:equilibrio-bens}, onde o índice \(i\) foi incluso
para simbolizar a estrutura dos dados em painel (ou seja, o índice \(i\)
representa cada país da amostra, enquanto o índice \(t\) representa cada intervalo
de tempo amostrado).
\begin{equation}
\gls{hp}_{it} = \alpha^*_i - \beta_{2i}^* \gls{EA}_{it} + 
\gamma^*_{2i} \gls{short}_{it} + \gamma^*_{3i} \gls{constr}_{it} - 
\beta^*_{3i} \gls{long}_{it} + \hat \epsilon^*_{it}
\label{eq:equilibrio-bens}
\end{equation}
onde \(\alpha^*_i = \dfrac{\alpha_i - \eta_i}{\gamma_{1i} - \beta_{1i}}\),
\(\gamma^*_{2i} = \dfrac{\gamma_{2i} - \beta_{4i}}{\gamma_{1i} - \beta_{1i}}\),
\(\gamma^*_{3i} = \dfrac{\gamma_{3i}}{\gamma_{1i}-\beta_{1i}}\),
\(\epsilon^*_{it} = \hat \epsilon_{it} - \hat \eta_{it}\) e\\
\(\beta^*_{ji} = \dfrac{\beta_{ji}}{\gamma_{1i} - \beta_{1i}} \quad para \quad j = 2, 3\)

\textcite{ADAMS2010} salientam que, pela análise das equações acima, o efeito total da taxa
de juros de curto prazo é ambíguo, dependendo da força relativa dos coeficientes
\(\gamma_2\) e \(\beta_4\). Para o mundo desenvolvido, segundo \textcite[p.~15]{ADAMS2010},
a magnitude do coeficiente \(\gamma_2\) é maior do que a do coeficiente \(\beta_4\),
ou seja, o aumento de custo impacta mais a oferta do que o aumento das hipotecas
afeta a demanda, exceção apenas para EUA e Itália, onde as taxas hipotecárias
preferidas são as ajustáveis.

A figura \ref{fig:adams} mostra como, teoricamente, uma mudança nas variáveis
macroeconômicas impactam na formação de preços do mercado imobiliário. Por
exemplo, no painel central é ilustrado como se propaga para o mercado
imobiliário um aumento na taxa de juros de longo prazo: um aumento nos juros de
longo prazo diminui os preços dos imóveis, que por sua vez diminuem o apetite
dos investidores por imóveis, fazendo diminuir o ritmo das construções, o que
por sua vez tende a diminuir o estoque de imóveis (baixa entrada de imóveis
novos no mercado), o que por sua vez tende a aumentar o valor dos aluguéis.
\begin{figure}[H]

{\centering \includegraphics[width=\textwidth]{./images/adams_crop} 

}

\caption{O impacto das variáveis macroeconômicas no preço dos imóveis residenciais.}\label{fig:adams}
\end{figure}
\bcenter

Fonte: \textcite{ADAMS2010}.
\ecenter

\hypertarget{possuxedveis-adequauxe7uxf5es-necessuxe1rias-ao-modelo-no-brasil}{%
\paragraph{Possíveis adequações necessárias ao modelo no Brasil}\label{possuxedveis-adequauxe7uxf5es-necessuxe1rias-ao-modelo-no-brasil}}

É importante notar que na equação \eqref{eq:demanda-bens} as taxa de juros de
curto e longo prazo afetam negativamente a demanda por bens imóveis, exatamente
como previsto por \textcite{rangel1986b} e confirmado pelo IPEA.

No entanto, segundo \textcite{rangel1986b}, no Brasil a atividade econômica tinha efeito
inverso, ou seja, em períodos de fraca conjuntura a demanda por bens imóveis
tendia a aumentar e a diminuir em períodos de aquecimento da atividade
econômica. Entende-se que este comportamente devia estar relacionado ao
comportamento da inflação brasileira, que tinha o efeito de penalizar a
liquidez, incentivando as imobilizações de capital. Com a estabilidade de moeda,
esta aberração da economia brasileira provavelmente desapareceu ou se reduziu
a um papel secundário, ou seja, atualmente é esperado que o aquecimento da
atividade da Economia brasileira venha acompanhado de aumento dos preços dos
imóveis e \emph{vice-versa}.

O que provavelmente deve impactar a forma dos modelos no Brasil é a estrutura a
termo da taxa de juros. Ao contrário do mundo desenvolvido, no Brasil a
estrutura a termo da taxa de juros é invertida, pois há concentração da dívida
pública no curto prazo. Outra hipótese a ser verificada é quanto ao impacto das
taxas de juros de curto prazo nos juros hipotecários, o que parece não ocorrer
no Brasil.

\hypertarget{modelos-dinuxe2micos}{%
\subsubsection{Modelos dinâmicos}\label{modelos-dinuxe2micos}}

O mercado imobiliário, assim como acontece com o mercado mobiliário, deve
seguir os fundamentos da Economia no longo prazo. No entanto, eventuais
descolamentos dos preços no mercado imobiliário dos fundamentos econômicos podem
ocorrer, especialmente pela inelasticidade de oferta em alguns \gls{MI} e pela
inércia nos preços.

Desta maneira, pode ser conveniente a elaboração de modelos dinâmicos para o
\gls{MI}, que modelem o comportamento dos preços devido à mudanças de conjuntura
antes que o mercado volte a se equilibrar, o que pode demorar muitos anos, o que
varia, é claro, de mercado para mercado. Segundo \textcite[p.~4]{regulation}, um ciclo de
preços no mercado imobiliário pode durar décadas. Existe, dessa forma uma
propensão à formação de bolhas especulativas, atribuídas a diversos motivos,
como ``expectativas exuberantes'' e problemas de informações de preços, que são
difíceis devido às particularidades do \gls{MI} \autocite[3]{ADAMS2010}.

\hypertarget{micro}{%
\section{O Mercado Imobiliário e a Microeconomia}\label{micro}}

O mercado imobiliário, como visto na seção anterior, não pode ser
considerado um simples mercado de bens, como outros bens de consumo em geral,
como automóveis, móveis ou eletrodomésticos. Sua análise em nível macroeconômico,
portanto, requer um estudo mais aprofundado das variáveis da Economia do país,
o que não quer dizer que não se dispense de analisar também o seu comportamento
microeconômico, o que aliás a NBR 14.653-01 \autocite*{NBR1465301} parece ter acabado de
perceber e recomendar.

\hypertarget{estruturas-buxe1sicas-de-mercado}{%
\subsection{Estruturas básicas de mercado}\label{estruturas-buxe1sicas-de-mercado}}

A nova versão da NBR 14.653-01 \autocite*[ix]{NBR1465301} traz uma evolução quanto à
consideração das diversas estruturas de mercado possíveis para a avaliação de um
bem.
\begin{quote}
As estruturas básicas do mercado podem ser, resumidamente:
\begin{itemize}
\item
  \textbf{Concorrência Perfeita}: situação em que o número de vendedores e de
  compradores é suficientemente elevado para que um agente isolado não seja
  capaz de influenciar o comportamento dos preços;
\item
  \textbf{Monopólio}: É constituído por um único vendedor;
\item
  \textbf{Monopsônio}: é constituído por um único comprador;
\item
  \textbf{Oligopólio}: é constituído por um número pequeno de vendedores;
\item
  \textbf{Oligopsônio}: é constituído por um número pequeno de compradores.
\end{itemize}
\end{quote}
No entanto, a NBR 14.653-01 \autocite*{NBR1465301} não deixa claro de que modo cada
estrutura de mercado deve impactar o valor final de mercado do imóvel.

Outro problema com a NBR 14.653-01 é que os mercados não deveriam ser
classificados apenas em relação à sua estrutura concorrencial, mas também de
acordo com as suas restrições geográficas e/ou regulatórias: existem mercados em
que não é possível aumentar a oferta, mesmo com o aumento da demanda, porque
existem limitações geográficas que impedem que isto ocorra, ou ao menos impedem
que isto seja feito sem que antes sejam feitas grandes intervenções urbanas que,
inicialmente, permitam que a área seja loteada. Em outros mercados, existem
questões regulatórias fortes que impedem um aumento da oferta de bens imóveis,
como é o caso de Brasília, por exemplo: na região do plano-piloto existem
questões de gabarito que impedem uma maior verticalização e também existe um
grande número de imóveis em propriedade do Estado brasileiro, que distorcem o
funcionamento do mercado.

\hypertarget{equiluxedbrio-de-mercado}{%
\subsubsection{Equilíbrio de mercado}\label{equiluxedbrio-de-mercado}}

O preço de equilíbrio de um bem se dá, geometricamente falando, no cruzamento
das curvas de oferta e demanda, ou seja, o prelço de equilíbrio, em qualquer
mercado, é o ponto onde a oferta e a demanda do mercado são iguais \autocite[p.~310]{varian}.

Há de se discernir, no entanto, entre equilíbrio de curto e de longo prazo.

A Figura \ref{fig:equilibrioMercado} ilustra o equilíbrio num mercado
hipotético, modelado pelas curvas \textbf{O} (oferta) e \textbf{D} (demanda). Para este
mercado, o equilíbrio se dá no ponto de coordenadas (\(\tilde Q, \tilde P\)). Às
quantidades \(\tilde P\) e \(\tilde Q\) dão-se o nome de preço e quantidade de
equilíbrio.
\begin{figure}[H]

{\centering \includegraphics[width=0.7\linewidth]{images/equilibrioMercado-1} 

}

\caption{Equilíbrio de mercado.}\label{fig:equilibrioMercado}
\end{figure}
O equilíbrio de um mercado, no entanto, denominado equilíbrio estático ou de
curto prazo, está sujeito a choques que podem perturbar este equilíbrio.

De acordo com \textcite[p.~310]{varian}, um mercado competitivo (concorrência perfeita),
é um mercado em que cada agente econômico, ofertante ou demandante, considera os
preços como dados, \emph{i.e.} fora de seu controle. Não há, portanto, num mercado
competitivo, restrições à entrada de novas empresas \autocite[p.~433]{varian}. Neste
mercado, o que determina o preço é a ação conjunta de todos os agentes do
mercado.

\hypertarget{elasticidade-da-demanda}{%
\subsubsection{Elasticidade da demanda}\label{elasticidade-da-demanda}}

\hypertarget{elasticidade-preuxe7o-da-demanda}{%
\paragraph{Elasticidade-preço da demanda}\label{elasticidade-preuxe7o-da-demanda}}

A elasticidade-preço da demanda \(\varepsilon_p\), definida conforme a equação
\eqref{eq:epdemanda} \autocite[p.~302]{varian}, mede a quantidade marginal que será
demandada do mercado se o preço de uma mercadoria aumenta ou diminui em uma
unidade de preço. Geralmente a elasticidade-preço da oferta é, então, um número
negativo \autocite[289]{varian}
\begin{equation}
\varepsilon_p = \frac{p}{q}\frac{\mathrm{d} q}{\mathrm{d} p}
\label{eq:epdemanda}
\end{equation}
Uma maneira mais conveniente (e exata) de se escrever a equação da elasticidade
é a apresentada na equação \eqref{eq:elasticidadeLN}, em função do logaritmo
natural das variáveis \autocite[p.307-308]{varian}:
\begin{equation}
\varepsilon_p = \frac{\mathrm{d} \ln(q)}{\mathrm{d} \ln(p)}
\label{eq:elasticidadeLN}
\end{equation}
Segundo Mayo (1981) e Olsen (1986) \autocite[\emph{apud}][p.23]{Malpezzi2002TheRO}, uma boa
estimativa para o valor da elasticidade-preço da demanda é algo em torno de -0,8.

\hypertarget{elasticidade-renda}{%
\paragraph{Elasticidade-renda}\label{elasticidade-renda}}

A elasticidade-renda da demanda \(\varepsilon_r\) mede como muda a quantidade
demandada quando a renda do consumidor varia \autocite[p.300]{varian}. De acordo com a
elasticidade-renda os bens podem ser classificados em bens normais, \emph{i.e.} bens
cuja demanda aumenta com o aumento da renda; bens inferiores, \emph{i.e.} bens cuja
demanda diminui com o aumento da renda; e bens de luxo (\(\varepsilon_r > 1\)),
\emph{i.e.} bens cujo aumento na demanda é mais do que proporcional ao aumento da
renda (um aumento de 1\% na renda conduz a um aumento de mais de 1\% da demanda).

\hypertarget{elasticidade-preuxe7o-da-oferta}{%
\subsubsection{Elasticidade-preço da oferta}\label{elasticidade-preuxe7o-da-oferta}}

Na Figura \ref{fig:equilibrioMercado} foi apresentado o conceito de equilíbrio
em um determinado mercado, a partir de curvas de oferta e de demanda.

Outros tipos de curvas, porém, podem ocorrer. Segundo \textcite[p.~311-312]{varian}, dois
tipos diferentes de oferta ocorrem com frequência: a oferta fixa ou
perfeitamente inelástica (Figura \ref{fig:casosEspeciais} A), onde a quantidade
de equilíbrio é determinada inteiramente pelas condições de oferta e o preço de
equilíbrio é determinado inteiramente pelas condições de demanda; e a oferta
perfeitamente elástica, onde qualquer quantidade desejada do bem é ofertada a um
preço constante, determinado pela oferta e a quantidade de equilíbrio é
determinada pela curva de demanda (Figura \ref{fig:casosEspeciais} B).
\begin{figure}[H]

{\centering \includegraphics[width=1\linewidth]{images/casosEspeciais-1} 

}

\caption{Equilíbrio de mercado.}\label{fig:casosEspeciais}
\end{figure}
Na prática, o valor da elasticidade-preço da oferta, segundo pesquisas efetuadas
em diferentes \gls{MI} varia de praticamente zero (perfeitamente inelástica), em
mercados fortemente restritos \autocite{malaysia} a praticamente infinito (perfeitamente
elástica) em outros mercados. \textcite{supplyelasticity} argumentam que, no entanto, esta
grande variação encontrada em diversos trabalhos pode ser explicada pelas
diferentes janelas de tempo utilizadas para o cálculo da elasticidade e que há
evidências de que, no longo prazo, para a maioria das regiões metropolitanas dos
EUA, as elasticidades da oferta tendem a ser altas (\textgreater10).

\hypertarget{equauxe7uxf5es-de-oferta-e-demanda}{%
\subsubsection{Equações de oferta e demanda}\label{equauxe7uxf5es-de-oferta-e-demanda}}

Hipóteses sobre as elasticidades de ofeta e demanda de um \gls{MI} em específico
podem ser testadas a partir de modelos de fluxo do \gls{MI} em análise como o
apresentado nas equações \eqref{eq:flowModel1} a \eqref{eq:flowModel3}, onde \(Q_S\)
é a quantidade ofertada, \(Q_D\) é a quantidade demandanda, \(P_h\) é o preço
relativo das residências, \(Y\) é a renda e \(N\) é a população
\autocite[ p.~282]{longrunsupplyelasticity}.
\begin{align} 
Q_D &= \alpha_0 + \alpha_1 P_h + \alpha_2 Y + \alpha_3 N \label{eq:flowModel1}\\ 
Q_S &= \beta_0 + \beta_1 P_h \label{eq:flowModel2}\\
Q_D &= Q_S \label{eq:flowModel3}
\end{align}
De acordo com \textcite[p.~282]{longrunsupplyelasticity}, se os modelos forem elaborados
com o logaritmo natural das variáveis (logaritmo dos preços, logaritmo das
quantidades ofertadas e demandadas, logaritmo da renda e logaritmo da
população), os coeficientes \(\alpha_i\) e \(\beta_i\) podem ser interpretados
como as elasticidades do mercado.

O modelo das equações \eqref{eq:flowModel1} a \eqref{eq:flowModel3} pode ser
escrito em sua forma reduzida, de acordo com a equação \eqref{eq:flowModel}
\autocite[ p.~283]{longrunsupplyelasticity}:
\begin{equation}
P_h = \frac{\alpha_0 - \beta_0}{\beta_1 - \alpha_1} + \frac{\alpha2}{\beta_1 - \alpha_1}Y + \frac{\alpha3}{\beta_1 - \alpha_1} N
\label{eq:flowModel}
\end{equation}
Segundo \textcite[p.~285]{longrunsupplyelasticity}, sobre este modelo podem ser feitas
hipóteses simplificadoras, tal como a hipótese de que o estoque \(K\) de
residências não se ajusta completamente em cada período, o que está implícito
no modelo acima, mas de acordo com um parâmetro de ajuste do estoque \(\delta\)
(usualmente igual a 0,3 \autocite[p.~285]{longrunsupplyelasticity}). Assim:
\begin{align} 
Q_D &= \delta (K^* - K_{-1}) \label{eq:flowModela1}\\ 
K^* = \alpha_0 + \alpha_1 P_h + \alpha_2 Y + \alpha_3 N \label{eq:flowModela2}\\ 
Q_S &= \beta_0 + \beta_1 P_h \label{eq:flowModela3}\\
Q_D &= Q_S \label{eq:flowModela4}
\end{align}
O modelo das equações \eqref{eq:flowModela1} a \eqref{eq:flowModela4} pode ser
então escrito de maneira reduzida, conforme equação \eqref{eq:flowModela}:
\begin{equation}
P_h = \frac{\delta \alpha_0 - \beta_0}{\beta_1 - \delta \alpha_1} + \frac{\delta \alpha2}{\beta_1 - \delta \alpha_1}Y + \frac{\delta \alpha3}{\beta_1 - \delta \alpha_1} N - \frac{\delta}{\beta_1 - \delta \alpha_1}K_{-1}
\label{eq:flowModela}
\end{equation}
E pode ser estimado de acordo com a equação \eqref{eq:flowModelEstimate}:
\begin{equation}
P_h = \gamma_0 + \gamma_1 Y + \gamma_2 N + \gamma_3 K_{-1} + \varepsilon
\label{eq:flowModelEstimate}
\end{equation}
Segundo \textcite[p.~285]{longrunsupplyelasticity}, a elasticidade-preço da oferta do
\gls{MI} residencial pode ser, então, estimada a partir da equação abaixo,
parametricamente em função de \(\alpha_1\) e \(\alpha_2\), que a princípio são
desconhecidos:

\[\beta_1 = \frac{\delta \alpha_2}{\gamma_1} + \delta \alpha_1\]

\hypertarget{choques-no-mercado}{%
\subsubsection{Choques no mercado}\label{choques-no-mercado}}

Qualquer mercado está sujeito a choques, seja de oferta ou de demanda. Um choque
de oferta se daria com a quebra de um dos ofertantes, por exemplo, situação em
que a quantidade ofertada poderia vir a ser diminuída abruptamente, dependendo
da estrutura do mercado. Já choque de demanda ocorre quando há, por exemplo, por
fatores que são exógenos ao \gls{MI}, um aumento abrupto da demanda, que pode
ocorrer pela abertura de linhas de créditos antes não existentes ou pelo aumento
da renda da população. Quaisquer que sejam as causas que levem a um choque no
\gls{MI}, seus efeitos devem ser investigados. Se o mercado impactado tiver uma
estrutura fortemente concorrencial, a tendência é que um choque de demanda seja
absorvido a médio e longo prazo sem grandes flutuações de preços, devido à
entrada de novos ofertantes no mercado. Por outro lado, se o mercado em análise
sofre de algum tipo de restrição geográfica ou mesmo econômica que dificulte a
entrada de novos atores como ofertantes, a tendência é que os choques de demanda
tenham maior impacto. Os detalhes por trás deste raciocínio serão melhor
esclarecidos no capítulo \ref{crise2008}.

\hypertarget{particularidades-do-mercado-imobiliuxe1rio}{%
\subsection{Particularidades do mercado imobiliário}\label{particularidades-do-mercado-imobiliuxe1rio}}

Também em nível microeconômico o mercado imobiliário é um mercado diferenciado
dos outros mercados de bens de consumo. Apesar de toda a indústria da construção
ter evoluído muito ao longo das últimas décadas, ainda persiste no Brasil um
forte componente artesanal na construção civil, o que implica em bens imóveis
de características muito heterogêneas, a depender da mão-de-obra aplicada na sua
execução. Também a questão do projeto arquitetônico implica numa singularidade
para cada bem imobiliário. Mas mais importante ainda é a questão da localização,
que torna cada imóvel único. Isto não ocorre em qualquer outro mercado de bens
de consumo. A não ser por questões de natureza sentimental, um carro, uma
geladeira, ou praticamente qualquer outro bem de consumo é produzido em série:
existem milhares de itens iguais no mercado. Isto nunca ocorre com os imóveis.
Mesmo apartamento vizinhos, em um mesmo prédio, tem características diferentes,
dada a sua posição solar, localização do andar em relação ao prédio, a vista
que cada um possui, entre outras questões.

Segundo a NBR 14.653-01 \autocite*[x]{NBR1465301}, ``o mercado imobiliário caracteriza-se
como um `mercado imperfeito', com bens não homogêneos, estoque limitado,
liquidez diferenciada e grande influência de fatores externos.''

De acordo com \textcite[p.~3]{ADAMS2010}, a forte inércia dos preços do mercado
imobiliário influencia o comportamento do mercado durante os \emph{booms} econômicos,
já que a exuberância das expectativas (exuberância irracional) dos investidores
e proprietários de imóveis facilita a formação de bolhas nestes mercados. Além
disto, a falta de informação a respeito de preços no mercado imobiliário, pelo
motivo deste ser um mercado segmentado, ou seja, os preços obedecem uma lógica
local, também é uma facilitadora da formação de bolhas.

\hypertarget{diagnuxf3stico-de-mercado}{%
\subsection{Diagnóstico de mercado}\label{diagnuxf3stico-de-mercado}}

A NBR 14.653-01 \autocite*[12]{NBR1465301}, estabelece que, ``o profissional, conforme o
tipo de bem, as condições de contratação, o método empregado e a finalidade da
avaliação, pode \textbf{tecer considerações sobre o mercado do bem avaliando}, de
forma a indicar, tanto quanto possível, \textbf{a estrutura, a conduta e o desempenho
do mercado}.''

\hypertarget{o-imuxf3vel-visto-como-um-investimento}{%
\subsection{O imóvel visto como um investimento}\label{o-imuxf3vel-visto-como-um-investimento}}

Segundo \textcite{Malpezzi2002TheRO}, um imóvel é um ativo que rende um fluxo de serviços
ao longo do tempo. O Quadro \ref{qua:Quadro-1} mostra a diferenciação entre os
conceitos de Estoque e Fluxo, muito utilizados na Economia.
\begin{quadro}[htb]
	\centering
	\caption{\label{qua:Quadro-1}Distinção entre estoque e fluxo.}	
	\begin{tabular}{|l|l|}
		\hline
		\textbf{Riqueza}  & \textbf{Fluxo}\\ \hline
		 Riqueza          & Renda         \\ \hline
		 Dívida Pública   & Déficit       \\ \hline
		 Valor do imóvel  & Aluguel       \\ \hline
	\end{tabular}
	\fonte{\textcite[3]{Malpezzi2002TheRO}.}
\end{quadro}
O quadro acima poderia ser facilmente expandido para incorporar outras formas de
investimentos, como títulos (que rendem coupons), ações (que rendem dividendos)
e outros.

Na ótica do investidor, o imóvel é como um título de longo prazo. Racionalmente
ou não, o comprador de um imóvel com fins de investimento espera que o imóvel
comprado vá gerar um fluxo de aluguéis (constantes ou não) ao longo do tempo, de
maneira que este fluxo de aluguéis compense o investimento inicial na compra do
imóvel.

Segundo \textcite[p.~4]{Malpezzi2002TheRO}, o valor presente \(V\) de um imóvel pode ser
calculado conforme a equação \eqref{eq:VPImovel}, onde \(\gls{R}_t\) é a renda
bruta dos aluguéis, \(C_t\) é o custo recorrente com a manutenção do imóvel e \(i\)
é a taxa de desconto.
\begin{equation}
V = \sum_{t=0}^T \frac{\mathbb{E}[R_t-C_t]}{(1+i)^t}
\label{eq:VPImovel}
\end{equation}
Diferentemente do que hoje ocorre com os investimentos capitalistas, onde o
\emph{payback} esperado gira em torno de 5 a 10 anos, o comprador de um imóvel
esperar que este gere um fluxo de renda ao longo de décadas.

Assim, a compra de um imóvel assemelha-se à compra dos títulos de renda fixa de
maior \emph{duration} disponíveis no mercado.

Ora, como se sabe, o valor de face destes títulos, ou seja, o valor do resgate
destes títulos no vencimento, é dado. Porém, estes títulos são negociados no
mercado secundário a valor de mercado, sendo que os títulos de longo prazo são
altamente sensíveis a variações nas taxas de juros de longo prazo. A saber, o
preço destes títulos é inversamente proporcional às taxas de juros, ou seja,
quanto menor as taxas, maior o valor presente dos títulos, ou valor de mercado,
e vice-versa.

\hypertarget{o-papel-da-especulauxe7uxe3o-no}{%
\subsection{\texorpdfstring{O papel da especulação no \gls{MI}}{O papel da especulação no }}\label{o-papel-da-especulauxe7uxe3o-no}}

\textcite[p.~5]{Malpezzi2002TheRO} fazem um bom apanhado de diversos conceitos relacionados à
``especulação''. Primeiramente, argumentam que é polêmica a definição do termo
\emph{especulador}, especialmente se comparado ao termo \emph{investidor}: seria o
especulador um investidor que negocia o bem em prazos mais curtos? Não existe
uma definição para tal.

Então \textcite[p.~12]{Malpezzi2002TheRO} apresentam diversas maneiras de se modelar as
expectativas (expectativas racionais, adaptativas, etc.), e apresentam (p.~14),
um modelo do valor presente de um imóvel calculado de acordo com o valor
esperado da renda líquida com aluguéis futuros, \(\gls{R_L}\), trazidos ao Valor
Presente através da aplicação de uma taxa de desconto \(i\), de acordo com a
expressão vista na equação \eqref{eq:valor-presente}:
\begin{equation}
V = \sum_{t = 0}^T \frac{\mathbb{E}[R_{Lt}]}{(1+i)^t}
\label{eq:valor-presente}
\end{equation}
Supondo que se possa calcular um valor presente para um imóvel baseado apenas
nos fundamentos econômicos \(V^*\), o valor em torno do qual se negociaria o
imóvel com expectativas racionais, com expectativas adaptativas\footnote{Expectativas adaptativas assumem que os agentes do mercado presumem que os
  ganhos futuros serão iguais aos do passado recente analisado.}, um imóvel
com preço igual ao da equação \eqref{eq:valor-presente-adptativo} também seria
negociado `racionalmente', caso o comprador considerasse que o grau de
supervalorização do imóvel continuasse a crescer nos períodos subsequentes por
uma taxa maior ou igual à taxa de desconto \autocite[15]{Malpezzi2002TheRO}.
\begin{equation}
V_t = V_t^* + b_t, \qquad com \quad \mathbb{E}_t[b_{t+1}] = (1+i)b_t
\label{eq:valor-presente-adptativo}
\end{equation}
Em suma, isto implica que a autocorrelação serial nos aumentos de preços dos
imóveis é uma condição necessária para a formação de bolhas. Segundo
\textcite[p.~15]{Malpezzi2002TheRO}, muitos estudos documentam a existência de correlação
serial em preços, porém a evidência de que isso leva, por si só, a formação de
bolhas é controversa. No capítulo \ref{crise2008} serão vistos com mais detalhes os
fatores que levam à formação de bolhas.

\hypertarget{o-papel-da-regulauxe7uxe3o-do-na-prevenuxe7uxe3o-da-formauxe7uxe3o-de-bolhas-especulativas}{%
\subsection{\texorpdfstring{O papel da regulação do \gls{MI} na prevenção da formação de bolhas especulativas}{O papel da regulação do  na prevenção da formação de bolhas especulativas}}\label{o-papel-da-regulauxe7uxe3o-do-na-prevenuxe7uxe3o-da-formauxe7uxe3o-de-bolhas-especulativas}}

Conforme será visto no capítulo \ref{crise2008}, para \textcite[p.~26]{Malpezzi2002TheRO}, o
efeito da especulação na volatilidade do \gls{MI} depende das condições de
oferta em cada mercado. Reformas adequadas na regulação do \gls{MI} com base
nesta hipótese serão apresentadas no capítulo \ref{politicas}.

\hypertarget{formauxe7uxe3o-de-bolhas-especulativas}{%
\subsection{Formação de bolhas especulativas}\label{formauxe7uxe3o-de-bolhas-especulativas}}

Na análise de \textcite{Malpezzi2002TheRO} a respeito da formação de bolhas, a variável
taxa de desconto (\(i\)) é considerada como uma variável exógena, isto é, não
foram considerados os efeitos do preço dos imóveis sobre esta variável, tampouco
foram consideradas os efeitos de possíveis mudanças no patamar desta variável
sobre o preço dos imóveis.

Como visto no \ref{qua:Quadro-1}, um imóvel pode ser visto como um investimento.
O seu valor presente depende de um fluxo estimado de aluguéis trazidos a
valor presente através da aplicação de uma taxa de desconto, assim como ocorre
com outros ativos financeiros, como os títulos da dívida pública, por exemplo.

Os títulos da dívida pública são vistos, normalmente, como os ativos livres de
risco da economia. Por isso, a taxa de desconto destes ativos é chamada de taxa
livre de risco. A taxa de desconto de quaisquer outros ativos na Economia de um
país será maior do que essa taxa livre de risco, pois nela estará embutida uma
taxa devido ao risco de carregamento daquele investimento, que se soma à taxa de
livre de risco da economia\footnote{O capítulo \ref{crise2008} ilustra como as taxas praticadas no mercado de
  crédito de hipotecas, por exemplo, são compostas de uma taxa adicional de risco
  em relação à taxa livre de risco.}. Dessa maneira, o que ocorre quando a taxa livre
de risco oscila é que também oscilam os Valores Presentes dos imóveis e dos
outros investimentos.

Se estas oscilações de preços dos imóveis devido às taxas de juros podem causar
bolhas especulativas, no entanto, é uma outra história. Este assunto será abordado com mais profundidade no capítulo \ref{crise2008}.

\hypertarget{rendimentos-de-aluguel}{%
\subsection{Rendimentos de aluguel}\label{rendimentos-de-aluguel}}

A taxa de rendimento do aluguel de um imóvel é obtida invertendo-se os termos
da equação \eqref{eq:valor-presente}, de maneira a isolar a taxa de desconto.

Para facilitar os cálculos, é conveniente a utilização da simplificação da
equação \eqref{eq:valor-presente}, fazendo a hipótese de que a série de
pagamentos é tão longa, que pode ser considerada perpétua. Geralmente esta é
uma boa hipótese no caso de imóveis, pois para séries de pagamentos constantes
com duração superior a 20 anos a hipótese da perpetuidade é considerada razoável,
haja vista que a vida útil de qualquer imóvel costuma ultrapassar esse prazo.

Assim, o Valor Presente \(VP\) de uma série de pagamentos perpétua de valor
periódico \(\gls{R_L}\), descontados de uma taxa de juros \(i\) é igual a:
\begin{equation}
VP = \lim_{n \to \infty} R_L \frac{(1+i)^t-1}{i(1+i)^t}= \frac{R_L}{i}
\label{eq:perpetua}
\end{equation}
Os rendimentos de aluguel \(\gls{y_r}\) são taxas brutas de retorno dos
recebimento de aluguéis comparados ao valor de venda de um imóvel.

Por exemplo, para um imóvel avaliado em R\$1.000.000,00, com um rendimento de
aluguéis de R\$1.500,00 reais mensais, o rendimento bruto do aluguel
deste imóvel seria:

\[y_r = \frac{12*1.500}{1.000.000} = 3,6\% \ a.a.\]
Esta taxa tem sido utilizada para comparar o rendimento do investimento em
imóveis em todo o planeta, através de sítios de internet especializados em
investimento em imóveis \autocite{gpg}.

\hypertarget{duration-de-um-tuxedtulo-de-renda-fixa}{%
\subsubsection{Duration de um título de renda fixa}\label{duration-de-um-tuxedtulo-de-renda-fixa}}

Segundo \textcite{marins1}, a \emph{duration} de um título, ou de um conjunto de títulos de
renda fixa pode ser calculada de acordo com a equação \eqref{eq:duration},
de Macaulay:
\begin{equation}
D = \frac{\sum\limits_{t = 1}^{n} t \times \frac{F_t}{(1+i)^t}}{\sum\limits_{t = 1}^{n} \frac{F_t}{(1+i)^t}}
\label{eq:duration}
\end{equation}
A medida tem por finalidade comparar títulos com diferentes características. Por
exemplo, existem títulos no mercado de renda fixa que não fazem pagamentos
constantes de coupons, sendo que o retorno integral do capital investido e juros
associado a este título só retornam ao investidor no vencimento do título
(normalmente são títulos de vencimento mais curto), enquanto outros títulos
fazem pagamentos constantes de coupons (semestralmente, p.ex.), ou seja,
uma parte do valor investido no título volta ao investidor na forma de coupons,
e no vencimento do título é resgatado o valor principal (normalmente são títulos
com vencimentos mais longos).

Para comparar então um título com vencimento em 5 anos que não faz pagamento de
coupons e um título com vencimento em 7 anos que faz pagamentos de coupons,
utiliza-se a equação \eqref{eq:duration} para calcular a \emph{duration} de cada
título.

O investimento em imóveis será, via de regra, o investimento de maior \emph{duration}
do mercado, haja vista que um imóvel pode ter vida útil de 50 anos ou mais,
enquanto os títulos de renda fixa normalmente tem prazo máximo de 30 anos.

\hypertarget{cuxe1lculo-do-valor-justo-de-um-imuxf3vel-em-funuxe7uxe3o-da-taxa-de-desconto}{%
\subsubsection{Cálculo do valor justo de um imóvel em função da taxa de desconto}\label{cuxe1lculo-do-valor-justo-de-um-imuxf3vel-em-funuxe7uxe3o-da-taxa-de-desconto}}

O cálculo do valor justo de um imóvel pode ser feito considerando-se o método
do fluxo de caixa descontado, assim como é feito o \emph{valuation} de uma empresa
capitalista.

Por exemplo, se um investidor estima que os rendimentos líquidos de um
determinado imóvel (aqui definida como o valor dos aluguéis descontados de
taxas, custos de manutenção e outras despesas) será de R\$2.000,00 mensais, a uma
taxa de juros de 3\% ao ano, o valor presente do imóvel, considerando-se que este
fluxo seja constante ao longo de toda a vida útil do imóvel (não menor do que 20
anos), é:

\[VP = \frac{12 \times 2.000}{0,03} = 800.000\]

Uma queda moderada da taxa de juros de longo prazo, digamos, para 2\% a.a.,
teria o seguinte impacto no valor presente deste imóvel:

\[VP = \frac{12 \times 2.000}{0,02} = 1.200.000\]

Já uma queda da taxa de juros de longo prazo mais agressiva, digamos para 1\%
a.a., teria o seguinte impacto:

\[VP = \frac{12 \times 2.000}{0,01} = 2.400.000\]

A Figura \ref{fig:valores-juros} mostra como varia, \emph{coeteris paribus}, o valor
justo de um imóvel em função da taxa de juros de longo prazo.
\begin{figure}[H]

{\centering \includegraphics[width=0.7\linewidth]{images/valores-juros-1} 

}

\caption{Variação do valor justo de um imóvel em função da taxa de juros.}\label{fig:valores-juros}
\end{figure}
\bcenter

Fonte: Do autor.
\ecenter

A Tabela \ref{tab:tabela-valor-justo} mostra alguns valores calculados de
acordo com a equação \eqref{eq:valor-presente} para um imóvel, supondo receita
líquida de aluguel mensal constante de R\$2.000,00, em função da taxa de
desconto.
\begin{table}[H]

\caption{\label{tab:tabela-valor-justo}Valor justo de um imóvel em função do valor do aluguel.}
\centering
\begin{tabular}[t]{rr}
\toprule
Taxa de Juros(\%) & Valor Justo (R\$)\\
\midrule
0,10 & 24.000.000\\
0,25 & 9.600.000\\
0,50 & 4.800.000\\
0,75 & 3.200.000\\
1,00 & 2.400.000\\
\addlinespace
1,50 & 1.600.000\\
2,00 & 1.200.000\\
3,00 & 800.000\\
4,00 & 600.000\\
5,00 & 480.000\\
\addlinespace
6,00 & 400.000\\
\bottomrule
\multicolumn{2}{l}{\textit{Notas:}}\\
\multicolumn{2}{l}{Supondo um aluguel constante de R\$2.000/mês.}\\
\multicolumn{2}{l}{Taxas de juros anuais.}\\
\end{tabular}
\end{table}
\bcenter

Fonte: Do autor.
\ecenter

Prentende-se mostrar com a exposição dos valores da Tabela
\ref{tab:tabela-valor-justo}, assim como com o gráfico da Figura
\ref{fig:valores-juros} que a taxa de juros de referência impacta fortemente
na variação dos preços dos imóveis a partir de um determinado patamar. Isto se
deve, é claro, à grande vida útil (ou grande \emph{duration}) de um imóvels e à
capacidade de extração de renda deste imóvel durante esta vida útil. Uma
diminuição na taxa de juros de longo prazo de referência de apenas 0,25 p.p.
pode acarretar numa mudança do valor presente (ou valor justo) de um imóvel da
ordem de milhões de reais. Isto impacta o \gls{MI} de uma maneira muito forte e,
como será mostrado no capítulo \ref{crise2008}, sem precedentes históricos.

\hypertarget{MI-e-o-setor-bancario}{%
\section{O Mercado Imobiliário e o setor bancário}\label{MI-e-o-setor-bancario}}

A participação dos produtos financeiros relacionados ao setor imobiliário
constituem uma grande parte dos portfolios bancários. Justamente por isto,
crises bancárias estão frequentemente associadas a superexposição do setor
bancário no mercado imobiliário \autocite[148]{Case2000}. Segundo Claessens \emph{et al.}
(2010) \autocite[\emph{apud}][3]{silver}, de 46 crises bancárias sistêmicas para quais há dados
disponíveis, mais de dois terços foram precedidas por padrões de aumento-estouro
de preços de imóveis.

Não apenas pelo estouro de uma bolha especulativa pode haver quedas nos preços
dos bens imóveis, mas também por conta de uma inversão dos fundamentos
econômicos que sustentavam os preços \autocite[129]{Case2000}, como aliás parece ser o
caso no momento, devido a um fator exógeno, a saber, a pandemia do corona vírus.

(\textbf{AUMENTAR})

\hypertarget{indices}{%
\chapter{\texorpdfstring{Índices de preços e indicadores de performance do \gls{MI}}{Índices de preços e indicadores de performance do }}\label{indices}}

\hypertarget{uxedndices-de-preuxe7os-de-imuxf3veis}{%
\section{Índices de preços de imóveis}\label{uxedndices-de-preuxe7os-de-imuxf3veis}}

Diversos países hoje contam com índices de preços de imóveis nacionais, oficiais
ou não, no entanto o modo de construção destes índices de preços difere de país
para país, assim como variam as estruturas de oferta e demanda em cada um deles
\autocite[3-4]{regulation}.

No Brasil recentemente foram criados alguns índices de preços de imóveis,
contudo, deve-se ter em conta, na criação de um índice de preços qualquer, a que
objetivo ele se destina e, como se verá, isto parece não ter sido levado em
conta na criação dos índices de preços de imóveis brasileiros.

De acordo com o \textcite{rppi}, índices de preços de imóveis são importantes e podem ser
úteis ou necessários:
\begin{itemize}
\tightlist
\item
  como um indicador macroeconômio da atividade econômica;
\item
  para uso na calibração da política monetária;
\item
  como uma ferramenta para estimar o valor de um dos componentes de riqueza;
\item
  como um indicador da estabilidade financeira ou de solidez da exposição ao
  risco;
\item
  como um deflator nas contas nacionais;
\item
  para auxílio para a tomada de decisões dos cidadãos na negociação de um imóvel;
\item
  como um dos componentes de um índice de preços ao consumidor; e
\item
  para uso em comparações inter-setoriais ou internacionais.
\end{itemize}
Ainda segundo o \textcite{rppi}, grosseiramente falando, os índices de preços podem ser
divididos em dois grupos:
\begin{itemize}
\tightlist
\item
  índices para a aferição do \emph{estoque} de propriedade imobiliária em um
  determinado tempo; e
\item
  índices para a aferição das \emph{vendas} de propriedade imobiliária em um
  determinado período de tempo.
\end{itemize}
Em síntese, índices do primeiro tipo devem conter informação tanto das
propriedades existentes e das propriedades recém-construídas, enquanto que para
índices do segundo tipo apenas, destinado a medir o investimento imobiliário,
deve se ater às vendas de novos imóveis ou imóveis recém convertidos em imóveis
de novo tipo \autocite[155]{rppi}.

Contudo, deve-se ter em conta que para aferir o investimento bruto no \gls{MI}
deve-se primeiro decompor o componente estrutural do componente
referente ao preço da terra no valor de venda dos imóveis \autocite[155]{rppi}. Esta
decomposição também deve ser efetuada para utilização do índice como um
componente de um índice de preços ao consumidor \autocite[156]{rppi}.

No Brasil existe uma grande dificuldade para a confecção de plantas genéricas
para a constituição da base de cálculo do \gls{IPTU}. Em diversas cidades é
comum que as \gls{PVG} não sejam atualizadas por muitos anos, o que acaba
distorcendo muito a \gls{PVG} em relação aos valores reais de mercado. O que
agrava ainda mais a situação é que as bases de cálculo acabam sendo atualizadas
por índices de preços ao consumidor, como o \gls{IPCA}. O efeito disto é que, em
períodos em que os preços dos imóveis aumentam mais do que o \gls{IPCA} em média
(ou qualquer quer seja o índice de preço adotado), a base da tributação acaba
deprimida e vice-versa, se o aumento do \gls{IPCA} é muito alto, o aumento da
base de cálculo do \gls{IPTU} acaba por sobrecarregar ainda mais os
proprietários. A elaboração de índices de preços de imóveis pode ajudar a
resolver este problema.

\hypertarget{metodologias-para-confecuxe7uxe3o-de-uxedndices-de-preuxe7os-de-imuxf3veis}{%
\subsection{Metodologias para confecção de índices de preços de imóveis}\label{metodologias-para-confecuxe7uxe3o-de-uxedndices-de-preuxe7os-de-imuxf3veis}}

Diferentes metodologias são conhecidas para a confecção de índices de preços de
imóveis, que vão desde as mais simples, como o método de vendas repetidas,
proposto por Case e Shiller \autocite*{repeatedSales}, até métodos mais complexos, com
modelos hedônicos, com os quais é possível decompor os preços dos imóveis em
preços de terra e preços de estruturas.

A opção pela aplicação de cada método não se dá apenas pela maior ou menor
precisão ou complexidade de cada índice, mas deve se dar pelos aspectos do
\gls{MI} em referência e em relação à finalidade do índice. O índice de Case e
Shiller, um dos mais conhecidos, se aplica perfeitamente ao \gls{MI} americano,
dado este é um mercado muito dinâmico, onde a compra e venda de imóveis é feita
com considerável frequência, o que permite a utilização do índice, que se baseia
em vendas repetidas de um mesmo imóvel. É provável, portanto, que este índice
não se adapte à países com \gls{MI} menos complexos, como o Brasil, onde ainda
existe grande informalidade e as transações muitas vezes nem são comunicadas às
entidades reguladoras, dado que diversas transações são feitas baseadas em
simples contratos particulares de compra e venda.

\hypertarget{muxe9todos-de-estratificauxe7uxe3o}{%
\subsubsection{Métodos de estratificação}\label{muxe9todos-de-estratificauxe7uxe3o}}

São os métodos mais simples de medição de mudanças nos índices de preços de
imóveis. É feita pela aferição da mudança de uma medida de tendência central
(usualmente média ou mediana) em um determinado período de tempo, dividindo-se
os imóveis em sub-amostras de características similares.

Um dos problemas da utilização deste tipo de metodologia é que ele é incapaz
de capturar eventuais mudanças quanto a qualidade do estoque de imóveis ao longo
do tempo, ficando viesado para cima, por exemplo, quando a qualidade dos imóveis
aumenta ao longo do tempo e vice-versa. O índice tambem pode ficar viesado se
houver alteração no \emph{mix} de imóveis entre dois períodos, por exemplo, se num
determinado período houver um maior número de imóveis de alto padrão do que
num período anterior, o índice ficará viesado para cima.

Segundo o Eurostat \autocite*[158]{rppi}, o método da estratificação é recomendado quando o
volume de vendas é grande o suficiente e a informação das características dos
imóveis é detalhada o suficiente para uma classificação detalhada das
propriedades, sendo este método de confecção de índice somente apropriado quando:
\begin{itemize}
\tightlist
\item
  um nível apropriado de detalhe escolhido é factível para as células
\item
  uma das variáveis de estratificação é a variável \emph{idade} da edificação; e
\item
  a decomposição do índice em terra e estrutura não é necessária.
\end{itemize}
\hypertarget{muxe9todos-heduxf4nicos}{%
\subsubsection{Métodos hedônicos}\label{muxe9todos-heduxf4nicos}}

É o método mais eficiente de construção de um índice quando há dados
disponíveis. Além de explorar o máximo os dados disponíveis, permite a
desagragação dos preços das estruturas do preço da terra e pode ser elaborado de
forma a levar em conta a depreciação dos imóveis, razão pela qual tal tipo
índice pode ser utilizado tanto para o monitoramento do valor dos estoques de
imóveis quanto para aferir o investimento no setor imobiliário.

Consiste na aplicação de modelos hedônicos, basicamente de duas maneiras distintas:

\hypertarget{modelos-com-variuxe1veis-dicotuxf4micas-de-tempo}{%
\paragraph{Modelos com variáveis dicotômicas de tempo}\label{modelos-com-variuxe1veis-dicotuxf4micas-de-tempo}}

Trata-se de um método que utiliza um modelo de regressão onde as variáveis
hedônicas dos imóveis são adicionadas de variáveis dicotômicas de cada período
\autocite[158]{rppi}. Apesar da simplicidade, uma vez que um novo período é adicionado,
com o novo ajuste do modelo, os índices dos períodos anteriores são modificados.
Para contornar este problema, utiliza-se uma janela de tempo móvel, efetuando um
ajuste de modelo para os últimos N períodos.

\hypertarget{modelos-de-preuxe7os-caracteruxedsticos}{%
\paragraph{Modelos de preços característicos}\label{modelos-de-preuxe7os-caracteruxedsticos}}

Nestes tipos de índices, um modelo hedônico é ajustado a cada período e o índice
é construído através dos preços previstos pelos modelos \autocite[53]{rppi}. Assim, o
índice é construído através da razão entre os preços previstos nos dois
períodos, como mostra a equação \eqref{eq:cp-approach}.
\begin{equation}
\frac{\hat p^t}{\hat p^0} = \frac{\hat \beta_0^t + \sum \hat \beta_k^t z_k^*}{\hat \beta_0^0 + \sum \hat \beta_k^0 z_k^*}
\label{eq:cp-approach}
\end{equation}
onde as características dos imóveis \(z_k^*\) pode ser estabelecida de diversas
maneiras, como a média das características do período base (\(z_k^* = \bar z_k^0\)),
denominado índice de Laspeyres, como média das características do período
posterior (\(z_k^* = \bar z_k^t\)), denominado índice de Paasche, ou ainda tomando
a média geométrica dos períodos anterior e posterior, denominado índice do tipo
Fisher.

\hypertarget{muxe9todo-de-vendas-repetidas-repeated-sales}{%
\subsubsection{Método de vendas repetidas (Repeated Sales)}\label{muxe9todo-de-vendas-repetidas-repeated-sales}}

Embora seja o método mais ``natural'' para a confecção de um índice, e de ser
também o método do índice talvez mais famoso, o de Case e Shiller, o método de
vendas repetidas não é tão eficiente quanto o método de modelos hedônicos, pois
o método de vendas repetidas, ao utilizar apenas os dados de imóveis
transacionados mais de uma vez, por conseguinte descarta muitos dados de
transação que poderiam ser úteis para a construção de um índice mais robusto
\autocite[160]{rppi}.

\hypertarget{muxe9todos-baseados-em-avaliauxe7uxe3o-sales-price-appraisal-ratio-spar}{%
\subsubsection{Métodos baseados em avaliação -- Sales Price Appraisal Ratio (SPAR)}\label{muxe9todos-baseados-em-avaliauxe7uxe3o-sales-price-appraisal-ratio-spar}}

Este tipo de índice utiliza de dados de avaliação feitas para propósito de
tributação ou outros tipos de avaliação, usualmente utilizadas para bens
similares, com o intuito de resolver alguns problemas relacionados ao modelo de
vendas repetidas (pequeno número de dados repetidos e a suscetibilidade ao viés).
Matematicamente, o índice é calculado de acordo com a equação \eqref{eq:AP}
\autocite[75]{rppi}.
\begin{equation}
P_{AP}^{0t} = \sum_{n \in S(t)} w_n^0(t) \left ( \frac{p_n^t}{a_n^0} \right )
\label{eq:AP}
\end{equation}
onde \(p_n^t\) é o preço de venda no período \(t\), \(a_n^0\) é o valor da avaliação
do bem no período base, \(S(t)\) é a amostra do período \(t\) e \(w_n^0(t)\) são os
pesos utilizados para a confecção do índice.

Ou seja, este método, como uma maneira de contornar o baixo volume de dados de
venda de bens nos diferentes períodos, compensa essa falta de dados com a
utilização de dados de avaliação, nos períodos onde não há dados de venda.

Como o método acima não utiliza valores observados no período base, os valores
neste período, como acima calculados, serão diferentes de 1. Como isto é
indesejável, uma normalização é feita, obtendo-se assim o \emph{Sales Price Appraisal
Ratio} ou SPAR, cuja formulação pode ser vista na equação \eqref{eq:SPAR}
\autocite[75]{rppi}:
\begin{equation}
P_{SPAR}^{0t} = \frac{\sum\limits_{n \in S(t)} p_n^t}{\sum\limits_{n \in S(t)} a_n^0} \left [ \frac{\sum\limits_{n \in S(0)} p_n^0}{\sum\limits_{n \in S(0)} a_0^n} \right ]^{-1}
\label{eq:SPAR}
\end{equation}
Apesar de ser simples, necessitar apenas de dados de preços e avaliações, o que
simplifica a sua computação, os métodos baseados em avaliações tem a desvantagem
de não lidar adequadamente com mudanças de qualidade entre os períodos (reparos
ou depreciação), além de ser dependente da qualidade das avaliações.

Uma característica interessante deste tipo de índices é que, quando o cadastro
está atualizado e podem ser utilizados, os dados utilizados para a confecção do
índice representam a totalidade da população e não apenas uma amostra, como de
costume, ou seja, não há erro amostral, apenas o erro devido à avaliação em si.

Uma limitação importante, porém, é que índices elaborados com tais métodos não
podem decompor o valor total da propriedade imobiliária em valor de terra e
valor de estruturas.

\hypertarget{uxedndices-no-mundo}{%
\subsection{Índices no mundo}\label{uxedndices-no-mundo}}

O \textcite{rppi} elencou uma série de índices ao redor do mundo. Nesta subseção, alguns destes índice são descritos.

\hypertarget{uxedndices-globais}{%
\subsubsection{Índices globais}\label{uxedndices-globais}}

\hypertarget{global-real-house-price-index}{%
\paragraph{Global Real House Price Index}\label{global-real-house-price-index}}

O FMI, através do observatório Global Housing Watch\ldots{}

\hypertarget{uxedndices-em-pauxedses-desenvolvidos}{%
\subsubsection{Índices em países desenvolvidos}\label{uxedndices-em-pauxedses-desenvolvidos}}
\begin{enumerate}
\def\labelenumi{\alph{enumi}.}
\tightlist
\item
  Índice de Case-Shiller
\end{enumerate}
\textcite{NBERw2506} desenvolveram\ldots{}

\hypertarget{uxedndices-em-pauxedses-em-desenvolvimento}{%
\subsubsection{Índices em países em desenvolvimento}\label{uxedndices-em-pauxedses-em-desenvolvimento}}

Nos países em desenvolvimento, segundo o o Eurostat \autocite*[110]{rppi}, uma proporção
significante dos imóveis advém de construção própria, em terrenos de família,
muitas vezes incompletas, sendo que a construção de uma casa pode durar anos, as
fontes de financiamento são incertas, não há avaliações disponíveis, o que faz
ser impossível o cálculo dos valores de aquisição do imóvel.

Nestes casos, segundo o o Eurostat \autocite*[110]{rppi}, a única opção prática para
estimar o valor dos imóveis é através da equivalência dos valores de aluguéis.

\hypertarget{uxedndices-na-amuxe9rica-latina}{%
\subsubsection{Índices na América Latina}\label{uxedndices-na-amuxe9rica-latina}}

\hypertarget{ipvu-coluxf4mbia}{%
\paragraph{IPVU (Colômbia)}\label{ipvu-coluxf4mbia}}

Desde 2003 \autocite[130]{rppi}

\hypertarget{uxedndices-no-brasil}{%
\subsubsection{Índices no Brasil}\label{uxedndices-no-brasil}}

\hypertarget{uxedndice-fipezap}{%
\paragraph{Índice FipeZap}\label{uxedndice-fipezap}}

O índice de maior longevidade que se tem conhecimento no Brasil foi elaborado em
conjunto pela Fundação Instituto de Pesquisas Econômicas (FIPE) e pelo portal de
anúncios de imóveis, o chamado índice FipeZap.

Segundo a \textcite{fipezap}, a metodologia utilizada para o cálculo do índice é a
estratificação. Para o índice FipeZap, apenas dois critérios de estratificação
foram adotados, no caso dos imóveis residenciais, a saber, o número de
dormitórios e a área de ponderação, definida pelos critérios do IBGE. No caso de
imóveis comerciais, o único critério de estratificação é a localização
\autocite[7]{fipezap}.

De acordo com a \textcite{fipezap}, não foi possível a inclusão da variável idade da
edificação, por esta não ser uma variável obrigatória nos anúncios da plataforma
zap imóveis.

Infelizmente, devido à sua metodologia e composição, o índice FipeZap não se
presta nem a aferir o estoque de riqueza, nem o investimento bruto no mercado
imobiliário, já que a variável idade não foi inserida na estratificação e não é
possível a decomposição do índice na parte estrutural e preço da terra, devido à
própria metodologia.

Outro problema relacionado a este índice é que ele não é composto de transações
de imóveis, porém de anúncios de imóveis, o que além de poder ser problemático é
questionável, devido aos diferentes descontos praticados no mercado em
diferentes períodos e pelos diferentes atores, que neste caso não são
mensuráveis. Segundo o o Eurostat \autocite*[126]{rppi}:
\begin{citacao}
Although not related to the issue of timing, a disadvantage of advertised
prices and mortgage approvals is that not all of the prices included end in
transactions, and in the former case, the price will tend to be higher than the
fnal negotiated transaction price.
\end{citacao}
Outros problemas relacionados ao índice FipeZap, como se basear exclusivamente
em anúncios em websites, também são elencados pelo Eurostat \autocite*[104]{rppi}:
\begin{citacao}
Information collected on a seller’s asking price cannot always be easily
verifed and, as well as depending on a balanced and representative sample,
relies on the honesty and knowledge of those being surveyed and when drawn from
advertisements, the accuracy of the information, especially when it is from a
website [...] It has also been argued that websites will tend to be biased
towards properties that have a competitive asking price to entice potential
sellers. All this is, of course, speculation but it does bring home some of the
potential dificulties associated with these sources.
\end{citacao}
\hypertarget{uxedndices-igmi-r-e-igmi-c}{%
\paragraph{Índices IGMI-R e IGMI-C}\label{uxedndices-igmi-r-e-igmi-c}}

Trata-se de um novo conjunto de índices construídos pela \gls{ABECIP} em
convênio com o \gls{IBRE}/\gls{FGV}, provavelmente através de um método baseado
em avaliações, ``levando em conta os laudos de imóveis financiados pelos bancos''.
\begin{citacao}
Nos Estados Unidos, país de grande dimensão territorial e elevada mobilidade
da população, o índice Case-Schiller faz o cálculo através da metodologia das
transações repetidas. Outros países, que têm uma realidade diferente da
americana, buscam alternativas para fazer o cálculo.
\end{citacao}
\hypertarget{indicadores-de-performance-do-setor-imobiliuxe1rio-residencial}{%
\section{Indicadores de performance do setor imobiliário residencial}\label{indicadores-de-performance-do-setor-imobiliuxe1rio-residencial}}

O Quadro \ref{qua:Quadro-2} traz uma coleção de dez indicadores que podem
ser utilizados para o diagnóstico e um comparativo do \gls{MI} residencial em
um país com outros mercados em perspectiva \autocite[7]{indicators}.
\begin{quadro}[htb]
	\centering
	\caption{\label{qua:Quadro-2}Dez indicadores chave da performance do \gls{MI} residencial.}	
	\begin{tabular}{|p{5cm}|p{10cm}|}
		\hline
		\textbf{Indicador}        & \textbf{Definição}\\ \hline
		 Razão preço/renda        & Razão do preço mediano de uma casa e a renda 
		 mediana doméstica anual  \\ \hline
		 Razão aluguel/renda      & Razão do aluguel mediano anual de uma casa e a 
		 renda mediana anual doméstica dos locadores       \\ \hline
		 Produção de unidades     & Número total de unidades residenciais (formais e 
		 informais) produzida por ano por 1.000 habitantes       \\ \hline
		 Investimento residencial & Investimento total no \gls{MI} residencial, em 
		 porcentagem do \gls{PIB}       \\ \hline
		 Área útil por habitante  & Área útil total disponível por habitante       \\ \hline
		 Estruturas permanentes   & Porcentagem de unidades residenciais em 
		 estruturas construídas por material permanente       \\ \hline
		 Estruturas irregulares   & Porcentagem do estoque total residencial na área 
		 urbana em desacordo com a regulamentação corrente.       \\ \hline
		 Carteira de crédito residencial & Razão do crédito total hipotecários em 
		 relação ao crédito total (instituições comerciais e governamentais)       \\ \hline
		 Multiplicador de terra loteada & Raiz média entre o preço mediano da terra 
		 loteada nos extremos urbanos em um loteamento típico e o preço mediano da 
		 terra nua, não loteada, próxima ao perímetro urbano       \\ \hline
		 Gastos com infraestrutura per capita & Razão dos gastos totais com 
		 infraestrutura durante o ano e a população urbana       \\ \hline
	\end{tabular}
	\fonte{\textcite[4]{indicators}.}
\end{quadro}
\hypertarget{variuxe1veis-macroeconuxf4micas-e-variuxe1veis-do-mercado-imobiliuxe1rio}{%
\section{Variáveis macroeconômicas e variáveis do mercado imobiliário}\label{variuxe1veis-macroeconuxf4micas-e-variuxe1veis-do-mercado-imobiliuxe1rio}}

Outros indicares importantes de performance do mercado imobiliário foram
elencadas por \textcite{macroHousing}.

\textcite{macroHousing} classificaram as variáveis importantes para a aferição do
comportamento do \gls{MI} e sua correlação com a Macroeconomia em variáveis
macroeconômicas -- \gls{MV} -- e variáveis do mercado residencial -- \gls{HMV}.
Entre as \gls{MV} estão as variáveis macroeconômicas mais convencionais, como o
\gls{PIB}, a taxa de desemprego, taxa de inflação e outras, mas também o que
\textcite{macroHousing} chamaram de variáveis macro-financeiras, como as taxas de juros
básicas, sejam nominais ou reais, os índices das bolsas de valores (S\&P 500),
etc. Entre as \gls{HMV} encontram-se não apenas os índices de preços, mas também
variáveis como o número de novas residências vendidas, a taxa de vacância e o
investimento residencial.

A Figura \ref{fig:vacancyRate} mostra a taxa de vacância de imóveis próprios (A)
e de aluguel (B) para os EUA.
\begin{figure}[H]

{\centering \includegraphics[width=1\linewidth]{images/vacancyRate-1} 

}

\caption{Home Price Index, de Case e Shiller, em termos reais.}\label{fig:vacancyRate}
\end{figure}
\bcenter

Fonte: O autor, com dados obtidos do FRED \autocite{RHVRUSQ156N,RRVRUSQ156N}.
\ecenter

Outra \gls{HMV} importante não citada por \textcite{macroHousing} é a \emph{Homeownership rate},
\emph{i.e} a proporção da população que vive em imóveis próprios. A Figura
\ref{fig:USHOWN} mostra a evolução da \emph{Homeownership rate} desde 1984.
\begin{figure}[H]

{\centering \includegraphics[width=0.7\linewidth]{images/USHOWN-1} 

}

\caption{Proporção da população vivendo em imóveis próprios (EUA).}\label{fig:USHOWN}
\end{figure}
\bcenter

Fonte: O autor, com dados obtidos do FRED \autocite{RSAHORUSQ156S}.
\ecenter

É usual encontrar análises sobre o \gls{MI} com estas séries plotadas num
único gráfico \autocites[ver][]{FRED2014}{FRED2016}{FRED2018}, como mostrado na Figura
\ref{fig:twoseries}.
\begin{figure}[H]

{\centering \includegraphics[width=1\linewidth]{images/twoseries-1} 

}

\caption{Análise comparativa de séries de \gls{HMV}.}\label{fig:twoseries}
\end{figure}
\bcenter

Fonte: O autor, com dados obtidos do FRED \autocite{RSAHORUSQ156S,CSUSHPINSA}.
\ecenter

\hypertarget{comportamento-recente-dos-uxedndices-de-preuxe7os-e-relauxe7uxe3o-com-e}{%
\section{\texorpdfstring{Comportamento recente dos índices de preços e relação com \gls{MV} e \gls{HMV}}{Comportamento recente dos índices de preços e relação com  e }}\label{comportamento-recente-dos-uxedndices-de-preuxe7os-e-relauxe7uxe3o-com-e}}

No Brasil não existem séries históricas de prazo satisfatórios para uma análise
de longo prazo referente ao preço dos imóveis. No entanto, como explicitado
teoricamente no capítulo \ref{historico}, e mostrado por \textcite{ADAMS2010} numa
série de países, num mundo globalizado, em que há livre fluxo de capitais, a
tendência é que o comportamento dos preços siga uma mesma tendência, portanto a
análise do comportamento dos preços nos países centrais da economia mundial
tende a representar o que ocorrerá, com maior ou menor \emph{lag}, nos outros países,
influenciados pelas variáveis e decisões ocorridas nestes países.

\hypertarget{uma-anuxe1lise-do-hpi-de-case-e-shiller}{%
\subsection{Uma análise do HPI de Case e Shiller}\label{uma-anuxe1lise-do-hpi-de-case-e-shiller}}

A figura \ref{fig:nhpi} mostra a evolução do \emph{Home Price Index} de Case e
Shiller, desde 1990.

É possível notar pela análise do índice que os preços dos imóveis que, após o
estouro da bolha do \gls{glo:subprime}, os preços sofrem uma nítida tendência de
alta, desde janeiro de 2012, ultrapassando já, em termos nominais, os níveis de
preços anteriores à crise de 2008.

Em julho de 2006 o índice atingiu o valor de 184,62
pontos, entrando então em tendência de baixa, até atingir os
134,16 pontos, em janeiro de 2012, uma queda de
50,45
pontos, quando se iniciou nova tendência de alta, que perdura até os dias atuais.
No momento em que se escreve esta dissertação o índice se encontra com
212,20 pontos, maior valor da série histórica, uma alta de
78,04
pontos.

Em termos reais, no entanto, os preços ainda são inferiores ao pico registrado,
mas há uma nítida tendência de alta, conforme pode ser visto na figura
\ref{fig:rhpi}.
\begin{figure}[H]

{\centering \includegraphics[width=0.7\linewidth]{images/rhpi-1} 

}

\caption{Home Price Index (HPI), de Case e Shiller, em termos reais.}\label{fig:rhpi}
\end{figure}
\bcenter

Fonte: O autor, à partir de dados de \textcite{QuandlWIKI}.
\ecenter

Na atualidade este índice se encontra em 176,84 pontos, sendo
que o maior valor da série histórica se deu em dezembro de 2005, quando atingiu
195,83 pontos.

Em janeiro de 1997, o índice estava em 112,26 pontos.
Desta data em diante, o índice entrou em franca tendência de alta até atingir os
195,83 pontos, em dezembro de 2005, auge, em termos
reais, da bolha imobiliária, que no entanto só viria a estourar, nominalmente
falando, em julho de 2006, como se pode ver na figura \ref{fig:nhpi}.

É importante salientar que o fenômeno da alta tão relevante dos preços dos
imóveis é relativamente recente. O gráfico da Figura \ref{fig:rhpi2} ilustra
isto: apesar de alguns períodos de picos e vales relevantes, no longo prazo, não
há uma tendência clara dos preços, até fins do século XX. Isto contrasta com as
séries de preços de ativos de renda variável, por exemplo. Em séries temporais,
usa-se o termo passei aleatório (\emph{random walk}) para se referir à séries como o
HPI (até fins do século XX), onde não é possível estabelecer uma padrão claro de
ciclos ou uma tendência. Já para as séries de ativos de renda variável (ações)
ou do \gls{PIB}, há evidências que há uma tendência de longo prazo (\emph{drift}) na
série, apesar da aleatoriedade.

De fato, \textcite[p.~290-291]{supplyelasticity} mostraram que não há, ou melhor, não
havia, até a publicação daquele trabalho, tendência de longo prazo para o preço
dos imóveis nos EUA. No entanto, o que se tem observado num período que vai
desde meados de princípios do século atual é que, ao menos graficamente, existe
uma clara e forte tendência de alta de preços, não apenas nos EUA, como mostram
a análise do HPI de Case e Shiller, mas também em nível global, como mostra a
Figura \ref{fig:global-rhpi}, do índice de preços global de imóveis do
\gls{FMI}, que será apresentado na próxima seção.

A possível presença de uma tendência de longo prazo para os preços dos imóveis
deve ser investigada. No capítulo \ref{crise2008} isto será abordado.
\begin{figure}[H]

{\centering \includegraphics[width=0.7\linewidth]{images/rhpi2-1} 

}

\caption{HPI, em termos reais.}\label{fig:rhpi2}
\end{figure}
\bcenter

Fonte: O autor, à partir de dados de \textcite{QuandlWIKI}.
\ecenter

O valor mínimo da série histórica (não mostrado no gráfico) foi atingido em\\
Nov de
1921, quando o índice atingiu a marca de
65,61 pontos.

É interessante notar que, em um século, desde 1890, o índice de preços reais de
imóveis atingiu um valor máximo de apenas 130,99
pontos, o que ocorreu em agosto de 1989.

Ou seja, em relação ao topo histórico de um século (agosto de 1989) à partir do
início da coleta do índice (1890), o índice hoje se encontra
45,85 pontos percentuais acima daquela
marca e a apenas -45,85 pontos
percentuais abaixo de atingir o topo histórico de dezembro de 2005.

Em janeiro de 2012, no fundo do vale da curva do índice real, este atingiu
126,65 pontos, apenas
4,34 pontos abaixo do topo
histórico de um século à partir de 1890.

\hypertarget{o-global-real-house-price-index}{%
\subsection{O Global Real House Price Index}\label{o-global-real-house-price-index}}

O \emph{Global Real House Price Index} do \gls{FMI}, que se constitui de uma média
simples de preços reais de residências em 57 países \autocite{fmitwa}, mostra também uma
nítida e forte tendência de alta, como pode ser observado na Figura
\ref{fig:global-rhpi}.
\begin{figure}[H]

{\centering \includegraphics[width=0.7\linewidth]{images/global-rhpi-1} 

}

\caption{HPI real do FMI.}\label{fig:global-rhpi}
\end{figure}
\bcenter

Fonte: O autor, à partir de dados de \textcite{QuandlWIKI}.
\ecenter

Deve-se reparar que, ao contrário do que aconteceu nos EUA, o pico em termos
globais só foi alcançado no último trimestre de 2007. Deve ser observado ainda
que a queda, em termos globais, foi muito menos abrupta que a queda ocorrida
nos EUA. O ponto de retomada, no entanto, tem grande coincidência: enquanto nos
EUA a retomada começa à partir de Jan/2012, tanto para o índice nominal quanto
para o índice real, globalmente esta retomada ocorre à partir do final do
primeiro trimestre de 2012. Se levar-se em conta que, diferentemente dos índices
norte-americanos, o índice global é trimestral, pode-se dizer que a retomada
global da alta dos preços dos imóveis é praticamente simultânea.

No entanto, deve-se ter em conta que o preço dos imóveis nos EUA afeta o índice
global.

\hypertarget{crise2008}{%
\chapter{Ciclos, bolhas especulativas e a crise imobiliário-financeira de 2007-2008}\label{crise2008}}

No capítulo \ref{economia} foi mostrado um modelo econométrico que correlaciona
os preços residenciais a variáveis macroeconômicas. Estes modelos podem ser
elaborados relacionando os preços dos imóveis a diversas variáveis
macroeconômicas como o \gls{PIB}, à atividade econômica \gls{EA} ou à índices de
mercados de capitais, como o \gls{IBOVESPA} ou o \gls{SP500}. Estes modelos
testa estatisticamente a significância destas diversas variáveis na explicação
do preço dos imóveis, num espaço razoável de tempo (longo prazo). No entanto,
descolamentos momentâneos dos preços dos imóveis do preço de equilíbrio de
longo prazo podem acontecer, iniciando um ciclo no \gls{MI}. A partir do início
do ciclo, oferta e demanda vão se ajustar, até que um novo equilíbrio do mercado
seja atingido. Segundo \textcite[p.~18]{regulation} este novo equilíbrio pode levar ``uma
década ou mais''.

Os ciclos do \gls{MI}, portanto, muitas vezes são atribuídos a um único fator
exógeno a este mercado. Por exemplo, pode-se atribuir a criação de um ciclo a um
incentivo fiscal do governo ou a um choque de demanda na economia. Porém,
segundo \textcite[p.~209-210]{wheaton1999}, diferentes tipos de imóveis apresentam
diferentes tipos de comportamento cíclico. Alguns tipos de imóveis apresentam
movimentos de preços mais ligados à economia enquanto outros apresentam períodos
de oscilação muito mais longos e apresentam quase nenhuma relação com oscilação
da economia.

Neste capítulo serão abordados as características dos diversos \gls{MI}, como
se constituem os seus ciclos e como pode ocorrer o surgimento de bolhas
especulativas neste mercados, a depender do impacto de variáveis exógenas, assim
como das características intrínsecas destes mercados e dos seus aspectos
regulatórios.

\hypertarget{ciclos-do}{%
\section{\texorpdfstring{Ciclos do \gls{MI}}{Ciclos do }}\label{ciclos-do}}

De acordo com \textcite[p.~212]{wheaton1999}, alguns mercados parecem não apresentar um
componente cíclico intrínseco: o investimento nestes imóveis aumenta ou diminui
de acordo com os choques econômicos, como no mercado de apartamentos ou no
mercado de imóveis industriais nos EUA. Já para outros mercados, a oscilação
natural dos investimentos parece não ter ligação alguma com a economia: existe
uma oscilação, mas que parece ser uma característica intrínseca daqueles
mercados, como o mercado de imóveis de escritório e o mercado de imóveis para o
varejo (\emph{shopping centers}, p.~ex.) nos EUA.

\textcite{wheaton1999} elaborou então um modelo econométrico que leva em conta o tempo de
desenvolvimento de um projeto nos diferentes mercados, assim como a taxa de
depreciação em cada um deles, assim como outras variáveis como a
elasticidade-preço da demanda e da oferta nestes mercados e concluiu que se a
demanda é mais elástica do que a oferta (ou mesmo que ainda menos elástica,
porém com tempo de desenvolvimento curto e a taxa de depreciação baixa),
existe uma estabilidade de preços nestes mercados. Contudo, se a oferta é mais
elástica que a demanda e o tempo de desenvolvimento é longo, o mercado
apresenta instabilidade.

Para \textcite[p.~228]{wheaton1999}, com isto é possível explicar a diferença dos ciclos
nos diversos mercados, já que o tempo de desenvolvimento geralmente é curto nos
mercados de apartamentos e industrial, enquanto no mercado de escritórios e se
imóveis comerciais o tempo de desenvolvimento é maior.

Quanto às elasticidades, no mercado residencial geralmente a demanda é
inelástica, porém também é a oferta. Já para o caso dos escritórios, a demanda
também é inelástica, porém estudos sugerem haver relativa elasticidade da oferta,
o que explica os vários casos de \emph{overbuilding}, ou seja, de construção além do
que o mercado pode absorver \autocite[p.~228]{wheaton1999}.

Finalmente, segundo \textcite[p.~228]{wheaton1999}, a única coisa em comum nos vários
mercados é o alto grau de durabilidade dos bens. Porém mesmo pequenas diferenças
na durabilidade entre os diversos mercados podem fazer uma diferença
considerável na estabilidade desses mercados, assim como os diferentes prazos de
desenvolvimento dos projetos e das elasticidades de oferta e demanda.

\hypertarget{bolhas-especulativas-no-mercado-imobiliuxe1rio}{%
\section{Bolhas Especulativas no Mercado Imobiliário}\label{bolhas-especulativas-no-mercado-imobiliuxe1rio}}

Conforme antecipado no capítulo \ref{economia}, mecanismos regulatórios para o
\gls{MI} podem ser utilizados visando uma melhora da curva da oferta, tornando-a
mais elástica, de maneira a evitar grandes oscilações de preços quando estes
mercados forem atingidos por choques de demanda.

No entanto, em alguns mercados isto não é tão simples: muitos mercados estão
limitados por condições geográficas e simplesmente não é possível tornar a
oferta elástica nestes tipos de mercado. Em outros mercados, apesar da oferta
ser relativamente elástica, o choque de demanda pode ser tão forte que mesmo com
o aumento da oferta, os preços oscilem muito. Ademais, em alguns mercados, como
visto no item anterior, o tempo de desenvolvimento é maior, o que faz com que
um aumento de demanda leve alguns anos até que possa ser absorvido pelo mercado.

Estas grandes oscilações de preços são indesejáveis por uma série de motivos.
A dinâmica do mercado pode ser assim descrita:
\begin{enumerate}
\def\labelenumi{\arabic{enumi}.}
\tightlist
\item
  Com os preços acima (abaixo) dos fundamentos, ao longo do tempo, a oferta
  aumenta (diminui);
\item
  Há superinvestimento (subinvestimento) no setor de construção, devido aos
  preços acima (abaixo) dos fundamentos;
\item
  Com a oferta muito alta (baixa), os preços tendem a cair (subir);
\item
  Se a volatilidade é muito alta, estes movimentos podem ser bruscos: ao invés
  da bolha se dissolver, ela estoura;
\item
  Muitos investimentos no setor imobiliário são feitos por um empreendedor
  financiado por empréstimos bancários: uma grande queda nos preços pode levar
  ao desequilíbrio econômico financeiro do empreendedor, o que desencadeia uma
  série de problemas posteriores\footnote{Segundo \textcite[p.~2]{Malpezzi2002TheRO}, nas economias mais afetadas,o colapso
    dos preços no \gls{MI} é seguida por uma série de eventos, como uma crise
    bancária, uma crise no balanço de pagamentos, uma crise financeira e um estouro
    de um ciclo de negócios.}
\end{enumerate}
\hypertarget{o-papel-da-oferta-na-prevenuxe7uxe3o-de-bolhas}{%
\subsection{O papel da oferta na prevenção de bolhas}\label{o-papel-da-oferta-na-prevenuxe7uxe3o-de-bolhas}}

De acordo com \textcite[p.~11]{regulation}, os padrões de \emph{boom} e \emph{boost} no \gls{MI},
ou seja, a formação e estouro de bolhas, podem ser explicados por modelos
dinâmicos e tem maior relação com problemas na inelasticidade da oferta de
alguns mercados do que propriamente com a ação de especuladores, que são quase
sempre eleitos os culpados no noticiário econômico.

No entanto, para \textcite[p.~1]{regulation}, a oferta inelástica em alguns mercados é
uma condição necessária para a formação de bolhas, porém não suficiente: a
formação de bolhas também está relacionada a choques de demanda.

Devido a estes fatos, são importantes na análise do \gls{MI} os modelos
dinâmicos. Descrever estes modelos, no entanto, está além do escopo deste
trabalho. Alguns trabalhos importantes de se destacar, visando pesquisas
futuras são: \textcite{wheaton1999}, \textcite{FAN201937} e \textcite{Malpezzi2002TheRO}.

É possível compreender o problema da inelasticidade da oferta no \gls{MI},
contudo, através de simples gráficos de oferta e demanda, como pode ser visto
nas Figuras \ref{fig:choque} A e B: na Figura \ref{fig:choque} A, um mercado
inicialmente com preço de equilíbrio \(P_1\), com um aumento na demanda, modelado
com o deslocamento da curva da demanda \(D_1\) para a direita (curva \(D_2\)), leva
os preços de equilíbrio inicial \(P_1\) ao novo preço de equilíbrio \(P_2\), um
pouco maior que \(P_1\). Com a mudança subsequente da estrutura da oferta no longo
prazo, o preço de equilíbrio retorna ao ponto \(P_3\).

Já num mercado com uma oferta mais inelástica, a volatilidade é muito maior, o
que pode ser visto na Figura \ref{fig:choque} B: o deslocamento da curva da
demanda \(D_1\) para a curva \(D_2\) provoca um aumento mais lento da oferta, fazendo
com que o preço inicial de equilíbrio \(P_1\) aumente, muito mais pronunciadamente
do que no caso anterior, para \(P_2\), voltando no longo prazo a um preço de
equilíbrio \(P_3\), após a mudança da estrutura da oferta, muito menor do que
\(P_2\), o que configura um padrão de \emph{boom e bust}: no primeiro momento, como a
oferta é inelástica, ocorre um grande aumento dos preços, ou seja, a bolha é
inflada, para num segundo momento estourar, voltando para um preço de equilíbrio
muito menor do que o alcançado durante o \emph{boom}.
\begin{figure}[H]

{\centering \includegraphics[width=1\linewidth]{images/choque-1} 

}

\caption{Efeitos de um choque de demanda em diferentes mercados.}\label{fig:choque}
\end{figure}
\bcenter

Fonte: Adaptado de \textcite[p.~19]{Malpezzi2002TheRO}.
\ecenter

Para \textcite[p.~11]{regulation}, ainda, além do problema da oferta inelástica pode
ocorrer também o problema de uma resposta mal desenhada pelo governo ao aumento
de preços inicial, como ilustrado na Figura \ref{fig:choque2}: algumas medidas
podem funcionar a médio prazo, como as grandes intervenções governamentais que
se destinam a promover uma grande oferta de imóveis de uma só vez (programas do
tipo ``1 milhão de casas''), mas seu efeito é equivalente a deslocar a curva de
oferta inicial \(O_1\) para \(O_2\). Aumentando a oferta temporariamente, evita-se
uma grande alta dos preços, que são estabilizados no preço de equilíbrio \(P_3\).
No longo prazo, porém, um posterior novo choque de demanda pode vir a elevar os
preços a níveis muito mais altos (\(P_4\)), superiores inclusive aos preços
atingidos durante o primeiro choque (\(P_2\)).
\begin{figure}[H]

{\centering \includegraphics[width=0.7\linewidth]{images/choque2-1} 

}

\caption{Efeitos de uma resposta governamental inadequada.}\label{fig:choque2}
\end{figure}
\bcenter

Fonte: \textcite[p.~19]{Malpezzi2002TheRO}
\ecenter

Segundo \textcite[p.~19]{Malpezzi2002TheRO}, o processo descrito na Figura
\ref{fig:choque2} não se trata de mera curiosidade, tendo sido documentado em
diversos estudos.

\hypertarget{modelo-dinuxe2mico}{%
\subsection{Modelo dinâmico}\label{modelo-dinuxe2mico}}

No capítulo \ref{economia} foi mostrado que é possível modelar o mercado a
partir de equações de oferta e demanda (equações \eqref{eq:flowModel1} a
\eqref{eq:flowModel3}).

Nas equações \eqref{eq:flow1} a \eqref{eq:flow4}\ldots{}
\begin{align}
Q_D &= \delta (K^* - K_{-1}) \label{eq:flow1}\\
K^* &= D + \alpha_1 P_t + \alpha_4 (P_t - P_{t-1}) \label{eq:flow2} \\
Q_S &= \beta_0 + \beta_1 P_t + \beta_2 P_{t-1} \label{eq:flow3} \\
Q_D &= Q_S \label{eq:flow4}
\end{align}
Supondo \(\beta_0 = 0\) e igualando a equação \eqref{eq:flow3}, tem-se a equação:

\[\beta_1 P_t + \beta_2 P_{t-1} = \delta[K^* - K_{-1}]\]

Substituindo \eqref{eq:flow2} em \eqref{eq:flow1}:

\[\beta_1 P_t + \beta_2 P_{t-1} = \delta[D +\alpha_1 P_t + \alpha_4 (P_t - P_{t-1}) - K_{-1}]\]
Isolando \(P_t\), tem-se:

\[\beta_1 P_t - \delta \alpha_1 P_t -\delta \alpha_4 P_t = \delta D - \delta \alpha_4 P_{t-1} - \beta_2 P_{t-1} - \delta K_{-1}\]

\[P_t = \frac{\delta}{\beta_1 - \delta \alpha_1 - \delta \alpha_4} D - \frac{\beta_2 + \delta \alpha_4}{\beta_1 - \delta \alpha_1 - \delta \alpha_4} P_{t-1} - \frac{\delta}{\beta_1 - \delta \alpha_1 - \delta \alpha_4} K_{-1}\]
Onde o ajuste do estoque é feito a cada período a partir das equações
\eqref{eq:stock} e \eqref{eq:supply}:
\begin{align}
K_t &= K_{t-1} + Q_S \label{eq:stock}\\
Q_S &= \beta_1 P_t + \beta_2 P_{t-1} \label{eq:supply}
\end{align}
Nas Figuras \ref{fig:TheRole1} e \ref{fig:TheRole2} são apresentadas as
variações dos preços ao longo dos anos subsequentes a um choque de demanda,
conforme simulado por \textcite{Malpezzi2002TheRO}. Segundo \textcite[p.~25]{Malpezzi2002TheRO},
como mostram as simulações, num mercado com oferta elástica a volatilidade dos
preços é muito menos, com ou sem especulação.
\begin{figure}[H]

{\centering \includegraphics[width=0.7\linewidth]{images/TheRole_1} 

}

\caption{Resposta do \gls{MI} com oferta elástica a um choque de demanda.}\label{fig:TheRole1}
\end{figure}
\bcenter

Fonte: \textcite[p.~24]{Malpezzi2002TheRO}
\ecenter
\begin{figure}[H]

{\centering \includegraphics[width=0.7\linewidth]{images/TheRole_2} 

}

\caption{Resposta do \gls{MI} com oferta inelástica a um choque de demanda.}\label{fig:TheRole2}
\end{figure}
\bcenter

Fonte: \textcite[p.~24]{Malpezzi2002TheRO}
\ecenter

Deve-se notar que, apesar do modelo ser teoricamente razoável, os resultados das
simulações não o são: especialmente para o mercado com oferta elástica, mesmo
sem especulação, a Figura \ref{fig:TheRole1} mostra um aumento da volatilidade
ao longo do tempo, o que não parece ser correto. Num \gls{MI} real, a tendência
deveria ser a dissipação do choque ao longo dos períodos.

Foram, então, elaboradas simulações próprias a partir do modelo proposto por
\textcite{Malpezzi2002TheRO}, com os mesmos parâmetros adotados. Os resultados das
simulações podem ser vistos nas Figuras \ref{fig:elastico} e
\ref{fig:inelastico}.
\begin{figure}[H]

{\centering \includegraphics[width=0.7\linewidth]{images/elastico-1} 

}

\caption{Simulações próprias para mercado com oferta elástica.}\label{fig:elastico}
\end{figure}
\bcenter

Fonte: O autor. Adaptado de \textcite{Malpezzi2002TheRO}.
\ecenter
\begin{figure}[H]

{\centering \includegraphics[width=0.7\linewidth]{images/inelastico-1} 

}

\caption{Simulações próprias para mercado com oferta inelástica.}\label{fig:inelastico}
\end{figure}
\bcenter

Fonte: O autor. Adaptado de \textcite{Malpezzi2002TheRO}.
\ecenter

\hypertarget{r-ambas-fig.capccc-out.width100-plot_gridpl1-pl2-labels-auto}{%
\chapter{\texorpdfstring{\texttt{\{r\ ambas,\ fig.cap="CCC",\ out.width="100\%"\}\ \#\ plot\_grid(pl1,\ pl2,\ labels\ =\ "AUTO")\ \#}}{\{r ambas, fig.cap="CCC", out.width="100\%"\} \# plot\_grid(pl1, pl2, labels = "AUTO") \#}}\label{r-ambas-fig.capccc-out.width100-plot_gridpl1-pl2-labels-auto}}

\hypertarget{medidas-macroprudenciais-de-prevenuxe7uxe3o-de-bolhas}{%
\subsection{Medidas macroprudenciais de prevenção de bolhas}\label{medidas-macroprudenciais-de-prevenuxe7uxe3o-de-bolhas}}

Isto poderia ser indício da formação de nova bolha imobiliária, nos moldes da
que estourou em meados de 2006, com consequências catastróficas para a economia
global?

Bolhas imobiliárias são difíceis de serem identificadas. Bolhas são fenômenos
de curto prazo que representam descolamentos dos preços em relação aos seus
fundamentos. Logo, para se afirmar que existe uma bolha, é necessário mostrar
que inexiste uma correlação entre os fundamentos e os preços dos ativos
imobiliários.

\textcite{fmitwa} elencam uma série de motivos que podem nos levar a conclusão que desta
vez é diferente, como a falta de sincronicidade em diversos países, o que
ocorreu durante a bolha dos anos 2000, e a maior vigilância por parte das
autoridades monetários no que tange às medidas macroprudenciais na prevenção,
que incluem \autocite{fmiem}:
\begin{itemize}
\tightlist
\item
  Limite a razão empréstimo/valor (\emph{loan-to-value ratio}), o que limita o valor
  da hipoteca relativa ao valor da propriedade.
\item
  Limites a razão dívida/renda, que limita o tamanho do pagamento do serviço da
  dívida a um limite fixo da renda do mutuário.
\item
  Requerimentos de capitais setoriais, que forçam os bancos a manter capital
  extra contra empréstimos a setores específicos, como o mercado imobiliário.
\item
  Requerimentos de provisionamento para pagamentos duvidosos.
\end{itemize}
Porém, além das medidas macroprudenciais citadas, considera-se que existem
outros fundamentos que indicam a não existência de uma bolha especulativa, mas
que indicam que o preço dos imóveis estão seguindo os fundamentos econômicos.

\hypertarget{bolhas-especulativas-e-taxas-de-juros}{%
\subsection{Bolhas Especulativas e taxas de juros}\label{bolhas-especulativas-e-taxas-de-juros}}

Como demonstraram \textcite{joebges}, um dos grandes dogmas da Economia, de que políticas
monetarárias frouxas causam bolhas no \gls{MI} não tem fundamentação científica,
tanto teorica, como empiricamente falando. Bolhas especulativas foram
documentadas tanto em países com política monetária frouxa quanto em países que
praticavam política monetária apertada. Isto não quer dizer, no entanto, que
as taxas de juros não tenham impacto sobre os preços dos imóveis, como mostram
diversos trabalhos. Para a ocorrência de bolhas especulativas, no entanto, é
necessário que existam outras características presentes: a inelasticidade da
oferta, como citado no capítulo \ref{economia}, a entrada no mercado de
grandes especuladores, geralmente fundos de hedge, a desregulamentação
financeira, com o lançamento de novos produtos financeiros de risco duvidosos,
etc.

No entanto, é sabido que os bancos centrais praticam (ou deveriam praticar)
políticas monetárias frouxas visando estimular uma economia em crise, visando
aumentar a liquidez monetária do mercado e com isso, fazer com que a economia
volte ao seu ritmo de crescimento `natural'. Ora, ocorre que nestes momentos em
que os bancos centrais estão praticando esta política monetária frouxa,
usualmente esta política monetária vem acompanhada de uma série de
desregulamentações financeiras, visando o aumento do volume de crédito, e também
que os grandes investidores, com as baixas taxas de juros, tendem a procurar
alternativas mais rentáveis de investimento, ao custo do aumento do risco destes
investimentos. A dinâmica que levou a crise imobiliário-financeira nos EUA foi
essa e será melhor detalhada no capítulo \ref{crise}.

Por ora o que interessa é o entendimento das razões da queda das taxas de lucro
da economia capitalista e porque isto é um fator estrutural, não conjuntural, o
que é o assunto do próximo item.

\hypertarget{a-tenduxeancia-a-queda-das-taxas-de-lucro}{%
\subsubsection{A tendência a queda das taxas de lucro}\label{a-tenduxeancia-a-queda-das-taxas-de-lucro}}

Como anteriormente mencionado, uma das variáveis macroeconômicas que afetam o
mercado imobiliário são as taxas de juros de longo prazo. Uma diminuição das
taxas de juros de longo prazo tende a reduzir a demanda por títulos públicos de
longo prazo e a aumentar a demanda por outros investimentos, ou seja, uma
diminuição nas taxas de juros da dívida pública aumenta a propensão ao risco dos
investidores, que são obrigados a procurar investimentos mais rentáveis, já que
a rentabilidade dos títulos é muito baixa.

Estas taxas de juros de mais longo prazo tem caído no mundo todo, especialmente
nas economias desenvolvidas, como ilustra o gráfico da figura \ref{fig:yields},
elaboradas a partir de dados obtidos do \textcite{fredgs10}, \textcite{fredgs20}, \textcite{fredgs30}. Esta
mostra as taxas de juros dos títulos da dívida do tesouro norte-americano,
em periodicidade mensal.

A causa da queda destas taxas de juros de mais longo prazo (assim como as de
curto prazo) são atribuídas a diversos fatores, o que vai muito além do escopo
deste trabalho. Em resumo, diversas escolas de pensamento econômico tem
diagnósticos diferentes: a escola marginalista atribui a queda das taxas de
juros à chamada estagnação secular da Economia dos países desenvolvidos. Uma boa
discussão pode ser vista em \textcite{bresser2018} e \textcite{krugman2020}. Já os marxistas
teorizam sobre uma queda natural das taxas de lucros no sistema capitalista,
o que já era previsto pelo próprio Marx \autocite{roberts2020}. A Figura \ref{fig:roberts}
ilustra o fenômeno desde meados do século XIX.
\begin{figure}[H]

{\centering \includegraphics[width=1\linewidth]{images/wrp-2} 

}

\caption{Diminuição da taxa de lucro na economia capitalista.}\label{fig:roberts}
\end{figure}
\bcenter

Fonte: \textcite{roberts2020}, adaptado de \textcite{crisis}.
\ecenter

A queda nas taxas de lucro obviamente impactam as taxas de juros dos títulos
públicos, que apenas refletem as taxas de lucro das empresas capitalistas, que
são compostas pelas taxas de juros livres de risco mais um prêmio de risco pela
alocação do capital naquele mercado.
\begin{figure}[H]

{\centering \includegraphics[width=0.7\linewidth]{images/yields-1} 

}

\caption{Taxas dos \emph{treasuries bonds} desde janeiro/1990.}\label{fig:yields}
\end{figure}
\bcenter

Fonte: O autor, à partir de dados do \gls{FRED}.
\ecenter

Mesmo uma análise mais ampla (desde 1953) das taxas de juros mostram que o
período recente é o período de menor taxa de juros em termos históricos, o que
pode ser visto na figura \ref{fig:fred}.
\begin{figure}[H]

{\centering \includegraphics[width=0.7\linewidth]{images/yields2-1} 

}

\caption{Taxas diárias dos \emph{treasuries bonds} desde janeiro/1990.}\label{fig:yields2}
\end{figure}
\bcenter

Fonte: O autor, à partir de dados do \gls{FRED}.
\ecenter

No momento em que se escreve deste trabalho, a taxa de juros dos títulos de mais
longa maturidade, ou seja, de 30 anos, está em 1,45\% a.a.

Para os títulos com vencimento em 20 anos, a taxa está em 1,23
\% a.a. e para os títulos com vencimento em 10 anos está em 0,88\% a.a.,
todos muito perto dos seus valores mínimos da série histórica, que foram
atingidos em 2020-03-09, quando os títulos
com maturidade de 10 anos atingiram impressionantes 0,54\% a.a.

Segundo Krugman \autocite*{krugman2020}, no entanto, isto não está a ocorrer apenas nos
EUA, mas em todo o mundo desenvolvido, com maior ou menor gravidade, já há
alguns anos, e não deve ser interpretado apenas como um efeito de curto prazo
de uma crise com a do Corona vírus, que só fez agravar ainda mais um quadro
que já vem de longo prazo:
\begin{citacao}
What this tells us is that the bond market isn’t just pricing in a global
recession driven by the coronavirus, but that it expects the Fed funds rate to
be near zero a lot of the time looking forward. That is, the market sees a
future of secular stagnation, in which the economy is in a liquidity trap, that
is, a situation in which monetary policy loses most of its traction, much if not
most of the time. We were in a liquidity trap for 8 of the past 12 years; the
market now appears to believe that something like this is the new normal.
\end{citacao}
\begin{figure}[H]

{\centering \includegraphics[width=\textwidth]{images/fred-1} 

}

\caption{Taxas mensais dos \emph{treasuries bonds} desde abril/1953.}\label{fig:fred}
\end{figure}
\bcenter

Fonte: O autor, à partir de dados do \gls{FRED}.
\ecenter

Esta longa e sustentada tendência de queda de longo prazo, com esta persistência
não pode ser atribuída a uma bolha especulativa, \emph{i.e.} um desvio de curto prazo
de uma tendência de longo prazo, mas sim a uma tendência estrutural, sistêmica.

As taxas de juros das hipotecas são influenciadas pelas taxas dos títulos da
dívida pública: os títulos da dívida pública são considerados ativos sem risco,
Por isto os investimentos em títulos públicos possuem as menores taxas do
mercado: o governo, que emite a dívida, também emite a moeda necessária para
pagá-la. Ou seja, não há risco verdadeiro quando um governo emite um título de
dívida, a não ser o risco da própria desvalorização da moeda (inflação). É
natural, portanto, que os investidores exijam, para que invistam em ativos
diferentes dos títulos da dívida pública, um prêmio de risco. Para o mercado
imobiliário, este risco é mínimo, pois o empréstimo se dá com o imóvel em
garantia. É esperado, portanto, que as taxas de juros das hipotecas sejam apenas
um pouco superiores aos títulos da dívida pública.
\begin{figure}[H]

{\centering \includegraphics[width=\textwidth]{images/mortagages-1} 

}

\caption{Juros hipotecários médios (30 anos) e taxas de títulos de maturidade constante de 30 anos.}\label{fig:mortagages}
\end{figure}
Esta alta queda das taxas de juros dos ativos financeiros de longo prazo é uma
das causadoras da elevação dos preços dos mesmos ativos, assim como a elevação
do preço dos imóveis.

De fato, \textcite{goodhart2008} encontraram uma ligação multidirecional entre \gls{MV}
relacionadas à oferta monetária (base monetária, crédito, entre outras) e o
preço dos imóveis.

A grande diferença então da recente alta dos preços dos imóveis (desde 2012) da
alta anterior (até Dez/2005) é que a alta mais recente é plenamente justificada
pela altíssima liquidez dos mercados. Ou seja, não se pode atribuir a recente
alta dos preços dos imóveis à especulação imobiliária, ou pelo menos não se pode
atribuir a recente alta somente a especulação: há fundamento para a alta, o que
se discute com mais propriedade na seção \ref{micro}.

Em um cenário de longo prazo, a tendência é que a crise de 2008 apenas tenha
acelerado em demasia, momentaneamente, o que já era uma tendência de longo
prazo, portanto estrutural, da economia.

\hypertarget{a-crise-imobiliuxe1rio-financeira-de-2008}{%
\section{A Crise imobiliário-financeira de 2008}\label{a-crise-imobiliuxe1rio-financeira-de-2008}}

\hypertarget{a-dinuxe2mica-da-crise}{%
\subsection{A dinâmica da crise}\label{a-dinuxe2mica-da-crise}}

Em 2001 o \gls{FED} reduziu a taxa básica da economia americana para 1,5\% a.a.,
para combater a recessão econômica, atingindo um dos níveis mais baixos da
história. Um dos grandes problemas que emergem com níveis tão baixos de juros é
que os maiores investidores institucionais do mercado de títulos de dívida
pública, como os fundos de pensão, possuem metas atuariais que só são atingidas
com taxas mais altas. Estes investidores, então, pressionados a atingir estas
metas atuariais, são obrigados a procurar opções de investimentos mais rentáveis
do que os títulos públicos. Estas instituições, no entanto, tem em seus
estatutos algumas limitações, como investir apenas em produtos financeiros acima
de determinados \emph{ratings}, ou seja, classificações dos produtos, que são
elaboradas pelas agências de classificação de risco.

Desta maneira, havia demanda no mercado para ativos de melhores classificação de
risco, porém com rentabilidade mais altas do que os ativos da dívida pública.
Para suprir esta demanda, houve o advento dos \gls{CDO}, que são derivativos
financeiros compostos de ativos financeiros, \gls{ABS}, que podem ser compostos
de diversos tipos de empréstimos. No caso do mercado imobiliário, estes ativos
são os \gls{MBS}, que são grupos de ativos constituídos de hipotecas. Este tipo
de derivativos, \gls{CDO}, tem uma estrutura que permite às instituições agrupar
uma série de ativos de menor \emph{rating}, criando assim um \gls{CDO} compostos de
ativos com maior e menor \emph{rating}. A estrutura destes derivativos se verá com
mais detalhes na próxima seção.

No entanto, em virtude das taxas muito baixas praticadas pelo \gls{FED}, esta
demanda estava tão grande, que os bancos se sentiram incentivados a produzir
cada vez mais empréstimos imobiliários, com o intuito de vender estes
empréstimos na forma de \gls{CDO}. Desta maneira, os bancos se livravam dos
riscos de crédito e liberavam mais capital para possibilitar a realização de
novos empréstimos.

Neste ínterim, porém, surgiram empréstimos exóticos, em que o mutuário pagava
por alguns anos apenas o valor dos juros. Depois deste período, o mutuário passa
a pagar também uma parte do principal. Este tipo de empréstimos eram chamados
de \emph{interest-only loans}. No caso específico do mercado imobiliário, eram
chamados de \emph{interest-only mortgages}. Convencionou-se chamar este tipo de
empréstimos também de empréstimos \emph{subprime}, o que é terminologicamente
incorreto. Tecnicamente, \emph{subprime} é o nome que se dá aos clientes cujo
histórico de crédito não é bom e, portanto, são mais propensos a \emph{default}. Como
o risco destes empréstimos é maior, os juros destes empréstimos também são
maiores. Com juros maiores, o risco de \emph{default} aumenta ainda mais. Os
empréstimos do tipo \emph{interest-only} ajudavam a possibilitar a este tipo de
cliente a compra de um imóvel, pois a prestação era bem diminuída no início e,
em tese, enquanto o mercado imobiliário continuasse subindo, o tomador do
empréstimo sempre poderia, em todo caso, revendê-lo com preço maior no futuro,
caso não conseguisse arcar com as prestações, especialmente quando do início do
pagamento do principal. Porém, neste tipo de empréstimo também era permitido aos
emprestadores ajustar a taxa de juros.

Com a grande liquidez inserida no mercado por estes empréstimos, o preço dos
imóveis tendia a continuar aumentando. Porém, em 2004 o \gls{FED} iniciou um
ciclo de alta dos juros para conter possíver surtos inflacionários. Em junho de
2006 a taxa básica do \gls{FED} chegou a 5,25\% a.a. Com o aumento das taxas de
juros, a demanda por residências começou a diminuir e os preços dos imóveis
começaram a cair. Os mutuários começaram a perceber que não conseguiriam
pagar as suas prestações e tampouco revender as suas casas a preços maiores. A
queda dos mercados em geral foi grande e com a queda dos mercados, ou seja, com
a queda dos valores das companhias, estas não conseguiam oferecer garantias
suficientes para conseguir novos empréstimos no mercado financeiro. Ainda, os
bancos começaram a ficar receosos de emprestar uns para os outros, por causa
da alta presença de ativos tóxicos nos seus balanços, os derivativos de crédito
que compraram durante o \emph{boom} imobiliário, o que causou uma crise de liquidez.

\hypertarget{a-raiz-da-crise}{%
\subsection{A raiz da crise}\label{a-raiz-da-crise}}

Segundo Donnelly e Embrechts \autocite*[3]{devil}, a raiz da crise financeira de 2008
estava nos derivativos de crédito, \gls{CDO}, lastreados em ativos de crédito
imobiliário, \gls{MBS}, que tinham por objetivo transferir o risco de uma
hipoteca dos emprestadores ao mercado financeiro em geral, ou seja, aos bancos,
fundos de \emph{hedge} e companhias seguradoras. Esta transferência de risco, ou
securitização, transformou o que seria apenas uma crise setorial das
instituições hipotecárias, numa crise financeira que atingiu praticamente todos
os setores financeiros e, por consequência, toda a economia real.

O propósito da venda destes seguros era a liberação de capital dos bancos: uma
vez que os bancos necessitam manter provisões de capital em seus balanços contra
os empréstimos realizados, a venda destes derivativos de risco a outros
investidores possibilitava a liberação desse capital, tornando possível aos
bancos efetuar mais empréstimos.

Esta dispersão do risco do setor bancário era vista como um fator de diminuição
da vulnerabilidade do setor aos choques econômicos. Na prática, porém, a
diminuição dos riscos levou os bancos a uma maior tolerância na análise de
risco dos tomadores de empréstimos, o que não foi percebido de imediato pelos
compradores dos \gls{CDO}s.

Segundo Donnelly e Embrechts \autocite*[5]{devil}:
\begin{citacao}
If a bank is not exposed to the risk of mortgage default, then it has no
incentive to control and maintain the quality of the loans it makes. To protect
against this, the theory was that the banks should retain the riskiest part of
the mortgage pool. In practice, this did not always happen, which led to a
reduction in lending standards [...] This possibility was foreseen some fifteen
years before the Crisis with remarkable prescience by Stiglitz, as he points out
in Stiglitz (2008).
\end{citacao}
No mercado financeiro é comum a utilização da expressão ``risco moral'' (\emph{moral
hazard}), para descrever os mecanismos que podem levar a um desequilíbrio de um
mercado. Na crise de 2008, este risco moral esteve presente não apenas neste
desincentivo aos emprestadores para a diminuição dos riscos dos empréstimos, já
que estes eram securitizandos, mas também na crença de que, em última instância,
os governos seriam obrigados a socorrer as instituições denominadas \emph{too big to
fail} (grandes demais para quebrar) \autocite[4-5]{devil}.

\hypertarget{derivativos-de-cruxe9dito}{%
\subsubsection{Derivativos de crédito}\label{derivativos-de-cruxe9dito}}

\gls{CDO} são um tipo de instrumento financeiro do grupo dos denominados
derivativos de crédito, lastreados em \gls{ABS}, construídos através do
empacotamento de ativos de crédito financeiros, com o intuito da diluição do
risco. Os \gls{CDO} tiveram papel central na eclosão da crise
imobilário-financeira de 2007-2008.

Segundo \textcite{watts}, \gls{MBS}, \emph{i.e.} seguros baseados em empréstimos hipotecários,
surgiram nos anos 70. Pouco tempo depois, vieram os \gls{CMO}, que eram
derivativos de crédito similares aos \gls{MBS}, porém com a característica de
serem fatiados em \emph{tranches}.

A diferença entre os \gls{CDO} e \gls{CMO} é que os \gls{CDO} podem ser baseados
em qualquer tipo de ativos ou dívidas, inclusive outros \gls{CDO}.

Este tipo de instrumento é utilizado tanto no mercado imobiliário como em geral
para a diluição do risco dos empréstimos. No caso do mercado imobiliário, das
hipotecas \autocite[2]{watts}.

Este tipo de instrumento é oferecido pelos emprestadores através de veículos de
propósito específicos, ou \gls{SPV}, que, apesar de patrocinados pelos bancos,
não transferem a ele o risco de falência institucional. Ou seja, a falência da
instituição bancária não afeta o risco de recebimento dos \gls{CDO}.
\begin{figure}[H]

{\centering \includegraphics[width=0.7\linewidth]{images/CDO-1} 

}

\caption{Esquemático do funcionamento de um CDO.}\label{fig:CDO}
\end{figure}
\bcenter

Fonte: Do autor. Adaptado de \textcite{devil}.
\ecenter

Na figura \ref{fig:CDO} pode ser visto um esquemático do funcionamento de um
CDO simples, dividido em três \emph{tranches} ou parcelas. Cada \emph{tranche} tem uma
prioridade no recebimento de cupons. Após o pagamento das despesas, a prioridade
é para o pagamento do \emph{Senior tranche}, depois do \emph{Mezzanine tranche} e então do
\emph{Equity tranche}. Se ocorrer um \emph{default} em um dos ativos lastreados, esta
perda é descontada primeiramente do \emph{Equity tranche}, que tem os valores de seus
cupons reduzidos. Depois de mais alguns \emph{defaults}, os cupons do \emph{Equity
tranche} vão a zero, e perdas adicionais serão descontadas do \emph{Mezzanine
tranche} e assim por diante \autocite[6]{devil}.

Desta maneira, a cada \emph{tranche} será atribuída pelas agências de risco uma nota
de crédito (\emph{rating}) diferente. Segundo \textcite{devil}, tipicamente um \gls{CDO} possui
composição de em torno de 80\% de \emph{senior tranches}, o que significa que
aproximadamente 20\% do portfolio de ativos base devem entrar em \emph{default} antes
do \emph{senior tranche} ser afetado. Na classificação dos \emph{tranches} pelas agências
de risco normalmente era atribuída ao \emph{senior tranche} uma nota AAA. Aos
\emph{mezzanine tranches} normalmente era atribuída uma nota BBB-, enquanto que aos
\emph{equity tranches} não era atribuída nenhuma classificação de risco, ou seja,
estas parcelas dos \gls{CDO} eram consideradas de grau especulativo.

Deve-se salientar que existem ainda outros tipos de \gls{CDO}, com maior número
de parcelas, sejam parcelas intermediárias (vários níveis de \emph{mezzanine
tranches}), ou parcelas mais seguras, com as \emph{super-senior tranches}, aos quais
eram atribuídas notas de crédito superiores à AAA, dos \emph{senior tranches}.

Segundo \textcite{watts}, a duração típica de um contrato de \gls{CDO} era de 5 anos.
\begin{longtable}[]{@{}lr@{}}
\caption{Parte da matriz de risco de default da Fitch.}\tabularnewline
\toprule
\textbf{Rating at issuance} & \textbf{5-yr default probability (\%)}\tabularnewline
\midrule
\endfirsthead
\toprule
\textbf{Rating at issuance} & \textbf{5-yr default probability (\%)}\tabularnewline
\midrule
\endhead
AAA & 0,05\tabularnewline
AA+ & 0,19\tabularnewline
AA & 0,26\tabularnewline
AA- & 0,36\tabularnewline
A+ & 0,56\tabularnewline
A & 0,62\tabularnewline
A- & 0,92\tabularnewline
BBB+ & 1,20\tabularnewline
BBB & 1,89\tabularnewline
BBB- & 3,63\tabularnewline
BB+ & 5,74\tabularnewline
BB & 8,11\tabularnewline
BB- & 12,50\tabularnewline
B+ & 17,09\tabularnewline
B & 21,36\tabularnewline
B- & 27,08\tabularnewline
CCC+ & 33,64\tabularnewline
CCC & 37,64\tabularnewline
\bottomrule
\end{longtable}
Fonte: \textcite{watts}.
\begin{longtable}[]{@{}lc@{}}
\caption{Exemplo de parcelamento de \gls{CDO}.}\tabularnewline
\toprule
\textbf{Tranche (Rating)} & \textbf{Attachment Points}\tabularnewline
\midrule
\endfirsthead
\toprule
\textbf{Tranche (Rating)} & \textbf{Attachment Points}\tabularnewline
\midrule
\endhead
Senior (AAA) & 42\% -- 100\%\tabularnewline
Mezzanine 1 (AA-) & 34\% -- 42\%\tabularnewline
Mezzanine 4 (A-) & 28\% -- 34\%\tabularnewline
Mezzanine 4 (BBB-) & 20\% -- 28\%\tabularnewline
Mezzanine 4 (BB-) & 11\% -- 20\%\tabularnewline
Equity (NA) & 0\% -- 11\%\tabularnewline
\bottomrule
\end{longtable}
Fonte: \textcite{watts}.

Para Donnelly e Embrechts \autocite[20]{devil}, os \emph{equity tranches} são produtos com
muito risco, portanto de pouco valor, o que interessa aos grandes especuladores,
como os \emph{hedge funds}, enquanto os \emph{senior tranches} eram vistos pelas grandes
instituições como ativos muito seguros. Já os \emph{mezzanine tranches} nem eram tão
seguros e nem tão baratos, não interessando nem aos investidores mais cautelosos,
nem aos grandes especuladores. Desta maneira, o \emph{mezzanine tranches} eram
também empacotados em outros \gls{CDO}, os chamados \gls{CDO}-squared, o que, em
tradução literal seria um \gls{CDO} ao quadrado. Este tipo de instrumento
financeiro, ainda mais complexo, é de precificação ainda mais difícil e isto
contribuiu ainda mais para aumentar o risco moral do mercado.

Segundo Donnelly e Embrechts \autocite[7]{devil}, o fato de que as grandes instituições
financeiras viam os \emph{senior tranches} como ativos muito seguros teve uma grande
influência na crise financeira de 2008. Segundo Donnelly e Embrechts
\autocite[24]{devil}, um executivo de uma subsidiária da AIG chegou a dizer, em agosto
de 2007 que era difícil vislumbrar um cenário onde a AIG poderia perder um dólar
sequer com estes ativos. Em 2008 o prejuízo líquido da AIG foi de
US\$ 99 bilhões, sendo US\$ 62 bi apenas no último trimestre, auge da crise do
\emph{subprime}.

\hypertarget{precificauxe7uxe3o}{%
\paragraph{Precificação}\label{precificauxe7uxe3o}}

De acordo com \textcite{watts}, apesar de terem sido criados há algum tempo, uma década e
meia teve que se passar para que o mercado de \gls{CDO} se tornasse grande. E o
motivo era que faltava um modelo que servisse de \emph{benchmark}, ou seja, um modelo
que permitisse ao mercado basilar o preço dos \gls{CDO} para que desse uma
relativa segurança aos investidores.

As condições para a precificação destes ativos vieram apenas na primeira década
do presente século, após a publicação do trabalho seminal de \textcite{Li}.

De acordo com Donnelly e Embrechts \autocite*[7]{devil}, a chave para a valorização dos
\gls{CDO} é a modelagem dos \emph{defaults}. Para isto, foi adotado pelo mercado
(agências de risco de crédito como Fitch, Moody's e Standard \& Poor's) o modelo
da Copula Gaussiana \autocite[14]{devil}, que tem algumas vantagens, como rapidez de
computação e facilidade de calibração. No entanto, foram desconsideradas pelo
mercado algumas desvantagens do modelo, a saber \autocite[15]{devil}:
\begin{itemize}
\tightlist
\item
  modelagem insuficiente do agrupamento de \emph{defaults} no portfolio (quando uma
  companhia quebra, é provável que outras companhias do mesmo setor também quebrem,
  isto não pode ser modelado pela Copula Gaussiana);
\item
  Diferentes correlações entre os \emph{tranches} do CDO, o que não ocorre na
  prática; e
\item
  ausência de modelagem dos fatores que levam aos \emph{defaults}.
\end{itemize}
Especialmente no que tange à primeira desvantagem, a Copula Gaussiana é
indesejável para a aferição de risco. Como enfatizam Donnelly e Embrechts
\autocite*[16]{devil}: não é sábio confiar em um modelo baseado na distribuição normal
para verificar a probabilidade de ocorrência de eventos extremos.

Pela própria estrutura dos \gls{CDO}, havia uma concepção que muito raramente
ocorreriam tantos \emph{defaults} simultâneos a ponto de um \emph{senior tranche} ser
afetado. Ocorre que pela própria característica da Copula Gaussiana, os eventos
extremos não são suficientemente bem representados, o que não ocorre com outros
modelos de Copulas, como a de Gumbel, Clayton ou Copula-t. A apresentação
destes modelos será vista com detalhe no capítulo \ref{copulas} ({[}O Método
Copulas{]}).

Em suma, os fatores que levaram à crise imobiliário financeira pode ser assim
resumidos:
\begin{itemize}
\item
  O \textbf{risco moral} presente na análise de risco de financiamento de imóveis
  devido ao surgimento de ativos financeiros (\gls{CDO}) que possibilitavam às
  instituições hipotecárias transferir o risco dos seus empréstimos aos
  compradores destes ativos, o que fez com que a análise de capacidade de
  pagamento dos mutuários fosse negligenciada;
\item
  O \textbf{risco moral} devido à longa cadeia entre os empréstimos originais e as
  pessoas/instituições que acabavam assumindo o risco final dos \emph{default}, o que
  tornou o entendimento do risco por parte dos compradores de \gls{CDO} quase
  impossível, tendo estes que confiar quase que cegamente nas classifações das
  agências de risco de crédito;
\item
  O \textbf{risco moral} nas agências de classificação de risco, por conflito de
  interesse, já que ao mesmo tempo em que aconselhavam seus clientes como
  securitizar os seus produtos, também faziam o \emph{rating} destes mesmos produtos;
\item
  O \textbf{risco moral} na crença de instituições consideradas grandes demais para
  quebrar, que seriam sempre socorridas pelos governos, em última instância;
\item
  O \textbf{excesso de liquidez} do sistema que levou à formação de uma bolha
  especulativa no setor imobiliário;
\item
  A \textbf{falta de controles adequados de risco} por parte de algumas instituições
  que negociavam os derivativos de crédito, assim como as agências de
  classificação de risco de crédito, que se basearam somente no modelo da Copula
  Gaussiana, subestimando a probabilidade de \emph{default} em \emph{clusters}, o que
  eventualmente levou à falta de liquidez pelo não pagamento dos coupons dos
  \emph{senior tranches} dos \gls{CDO}, e portanto à eclosão da crise financeira.
\end{itemize}
Este capítulo tem como objetivo demonstrar os problemas teóricos que levaram
a este último fator, que está na raiz da formação da bolha imobiliária.

Deve-se lembrar que, se houvesse controle adequado do risco, não haveria mercado
para os \gls{CDO}, especialmente para os lastreados em créditos do tipo
\emph{subprime}, que estão na raiz da crise. Sem possiblidade de transferência de
riscos, as instituições hipotecárias provavelmente não teriam efetuados tantos
empréstimos \emph{subprime}, pois sentiriam a necessidade de controlar o seu próprio
risco de \emph{default}. Por fim, se não houvesse tanta liquidez devido a todo o
crédito oferecido pelas instituições hipotecárias, muito provavelmente não teria
se formado uma bolha especulativa.

\hypertarget{politicas}{%
\chapter{Políticas do solo}\label{politicas}}

Os instrumentos tradicionais da política de solo historicamente não se mostraram
capazes de satisfazer a grande demanda por imóveis urbanos nos países da \gls{AL},
especialmente no Brasil. A grande falência das políticas de solo podem ser
vistas nas periferias das grandes cidades. É claro que essas políticas \autocite{suelo},
tem um papel em amenizar o problema -- através do combate à especulação
imobiliária, por exemplo -- que tem origem na própria história do
desenvolvimento dos países, mas elas não chegam a atacar o cerne desse problema,
já que não são concebidas para isto. Em linhas gerais, as políticas do solo
tradicionais tem uma função regulatória, no sentido de orientar o funcionamento
do mercado imobiliário. O problema, no entanto, é que o mercado imobiliário,
por si só, não tem a capacidade de resolver todos os problemas decorrentes da
crise agrária, como descrita no capítulo \ref{historico}. Intervenções estatais
de grande vulto se fazem necessárias, especialmente para as populações de baixa
renda. A regularização fundiária pode ter um papel neste sentido, mas na
atualidade esta política tem sido utilizada apenas timidamente, beneficiando
apenas um pequeno número de famílias.
\begin{figure}[H]

{\centering \includegraphics[width=0.7\linewidth]{images/injustica_SP} 

}

\caption{Condomínio de classe alta ao lado da favela de Paraisópolis em São Paulo/SP.}\label{fig:injustica}
\end{figure}
\bcenter

Fonte: \textcite{paraisopolis}.
\ecenter

Como tratado no Capítulo \ref{historico}, o cerne do problema habitacional
do Brasil (e provavelmente em toda a \gls{AL}), está na crise agrária,
\emph{i.e}, na passagem do modo de produção feudal para o modo de produção
capitalista.

Deve tratar o Brasil, portanto, assim como os outros países da \gls{AL},
de saber como os países avançados -- hoje ditos desenvolvidos -- cujas
transições para o capitalismo se deram através da chamada via prussiana,
trataram de resolver o problema habitacional nas grandes cidades, após a intensa
migração dos camponeses para as cidades.

Claro está que a solução vislumbrada por Rangel, de que a verdadeira reforma
agrária se daria através da queda ``natural'' do preço da terra, não se deu e mui
provavelmente nunca se dará, haja vista que Rangel dava como certo o aumento das
taxas de juros (mais precisamente, da eficiência marginal do capital), o que não
ocorreu e, conforme visto no capítulo \ref{economia}, não se pode prever que
acontecerá.

Claro está também que o pleno desenvolvimento do mercado financeiro, como
exposto nos capítulos \ref{economia} e \ref{copulas} não deve trazer consigo
uma solução para o problema. Pelo contrário, o maior desenvolvimento de
instrumentos financeiros, como foi visto, tende a gerar bolhas no mercado
imobiliário, o que tende a agravar o problema habitacional, prejudicando as
famílias de menor renda.

O que se nota hoje nos países desenvolvidos é uma grave crise habitacional,
apesar de todo o desenvolvimento dos seus mercados financeiros, da baixa
informalidade e dos baixos juros que sempre tiveram, se comparados aos países em
desenvolvimento \autocite{housing-europe,california}.

Considera-se, porém, que o problema especulativo não é a raiz do problema.
Segundo Zizek \autocite*[220-221]{zizek2005}, não é possível retirar a especulação de um
suposto capitalismo puro: a especulação é a própria alma do processo produtivo
no sistema capitalista. Sem a especulação imobiliária, inexistiria o mercado
imobiliário. As políticas de solo, portanto, que visam o combate da especulação
imobiliária não tem, portanto, papel maior do que o de paliativo, pois essas
políticas não podem ser radicais ao ponto de acabar com a especulação, haja
vista que acabariam também por extinguir os empreendimentos imobiliários. Aliás,
no Brasil, as reformas microeconômicas de 2004 tiveram um forte papel no
desenvolvimento do crédito imobiliário, através da melhoria das condições de
execuções de garantia, propiciadas pela Lei 10.931/2004. A alta no crédito
imobiliário fomentou a construção civil, setor responsável por puxar a economia
naquele ciclo. No entanto, o que se viu, desde então, foi um aumento expressivo
no valor dos imóveis, fazendo que, se por um lado houve aumento na oferta de
imóveis, devido à alta demanda reprimida, esse aumento da oferta veio
acompanhado do aumento do valor dos imóveis, o que dificultou ainda mais o
acesso das camadas de menor renda à aquisição da casa própria.

Ainda, mesmo que fosse possível separar mercado imobiliário de especulação
imobiliária, mantendo o sistema capitalista de produção, espera-se ter
demonstrado que, na atualidade, o problema ainda persistiria: está cada vez mais
nos fundamentos o problema dos preços elevados, não em grandes desvios de curto
prazo (bolhas especulativas) dos preços em relação aos fundamentos econômicos,
pois taxas de juros tão baixas tendem a valorizar ainda mais preço da terra.

Também a redução dos juros hipotecários, ao invés de ajudar, como seria de se
esperar, tende a agravar o problema, como foi visto no capítulo \ref{economia},
dado que o aumento do valor dos imóveis é exponencial com a queda das taxas de
juros. A redução dos juros, como se mostrou, está na ordem natural das
coisas, já que as taxas dos créditos hipotecários acompanham as taxas de juros
básicas.Ou seja, não se pode esperar que a solução para o problema habitacional
tenha lugar a partir de uma menor taxa de exploração dos mutuários.

Segundo \textcite{terraredonda}, ainda, esse aumento dos preços dos imóveis parece ter
efeito de beneficiar os proprietários, mas esses não são os principais
beneficiados:
\begin{citacao}
O rápido aumento dos preços dos imóveis parece beneficiar os proprietários, mas
os principais beneficiários são, de fato, os bancos, as instituições de crédito
e os grandes conglomerados e \emph{hedge funds} que aderiram ao jogo
especulativo.
Isso ficou evidente quando veio o \emph{crash}. Os bancos foram socorridos e os
donos das habitações foram jogados aos tubarões da bolsa de valores. Nos EUA,
milhões perderam suas casas para execução de hipotecas em 2007-10, enquanto no
setor de aluguéis o ritmo de despejos de populações de baixa renda acelerou em
todos os lugares, com consequências sociais devastadoras. Os \emph{hedge funds}
e as empresas de \emph{private equity} compraram casas hipotecadas a preços de
liquidação relâmpago e agora estão ganhando uma bolada em suas operações. No que
restou do setor público, a austeridade levou à falta de manutenção e à
deterioração do parque habitacional, até o ponto em que, como nos foi dito,
apenas a privatização melhoraria as coisas.
\end{citacao}
\textcite{terraredonda} nos dá uma pista de como os países desenvolvidos lidaram com este
problema no passado: habitação sob domínio público, ou simplesmente habitação
social. Claro, pois é inútil aguardar que a solução venha através do mercado.

Segundo \textcite{terraredonda}:
\begin{citacao}
Diferentes formas de valor sempre coexistiram, de modo desconfortável, com a
forma mercadoria. Sua coevolução na história recente dos mercados imobiliários
culminou no atual impasse, no qual a valorização especulativa determina que mais
da metade da população do planeta Terra não consiga encontrar um lugar decente
para viver em um ambiente de vida decente devido ao poder hegemônico do capital
sobre os mercados de terras e propriedades. Não precisa ser assim.

Ao terminar meu estudo recentemente, deparei-me com um folheto publicado pelo
"Conselho Metropolitano de Habitação de Nova York" em 1978. O título era
"Habitação sob o Domínio Público: A Única Solução". Em 1978, o "Departamento de
Habitação e Desenvolvimento Urbano" dos EUA tinha um orçamento de 83 bilhões de
dólares para ajudar a buscar essa solução. Cooperativas de capital limitado e
até fundos comunitários de terras estavam surgindo na maioria das grandes
cidades para oferecer soluções fora do mercado. Em 1983, o orçamento desse
Departamento havia sido reduzido para 18 bilhões de dólares, até ser abolido na
década de 1990 durante os anos Clinton. Quarenta anos depois, eu me pego
refletindo sobre as consequências desastrosas em todo o mundo de não se buscar
resolutamente a solução óbvia: habitação sob domínio público. O valor de uso
deve vir primeiro.
\end{citacao}
Segundo \textcite{rolnik}, no entanto, na raiz da atual crise habitacional do Reino Unido
está o descolamento do setor habitacional das políticas sociais, iniciado por
Margaret Thatcher. Com o desenvolvimento econômico, os países desenvolvidos
acabaram por abandonar estes programas sociais, em troca de soluções de
mercado.

\textcite{harvey} contextualiza a questão do direito à moradia dentro de uma conceito
maior, do direto à cidade. Segundo Harvey \autocite*[166]{harvey}:
\begin{citacao}
Da Californa à Grécia, a crise produziu perdas de direitos e novas valores dos
ativos urbanos para a maioria da população, junto com a extensão do poder
capitalista predatório sobre as populações de baixa renda e até hoje
marginalizadas. Em resumo, foi um ataque indiscriminado sobre os comuns
reprodutivos e ambientais. Vivendo com menos de 2 dólares por dia, uma população
global de por volta de 2 bilhões de pessoas está sendo ludibriada pelo sistema
de microfinanças, sendo "o \emph{subprime} de todas as formas de \emph{subprime}
de crédito", para deles extrair riquezas (como aconteceu no mercado imobiliário
dos Estados Unidos por empréstimos \emph{subprime} predatórios seguidos por
execuções de hipotecas) para tornar ainda mais opulentas as mansões dos ricos
[...]. Não surpreende, portanto, que os pobres não apenas estejam entre nós, mas
que seu número aumente com o tempo, em vez de diminuir. Enquanto a Índia vem
acumulando um respeitável índice de crescimento ao longo dessa crise, por
exemplo, o número de bilionários saltou de 26 para 69 nos últimos três anos,
enquanto o número de favelados quase dobrou na última década. Os impactos
urbanos são estarrecedores conforme condomínios fechados, luxuosos e com
ar-condicionado surgem em meio ao abandono da miséria urbana, em que os pobres
lutam para construir algum tipo aceitável de existência para si mesmos.
\end{citacao}
\hypertarget{breve-histuxf3rico-das-poluxedticas-habitacionais-e-urbanas}{%
\section{Breve histórico das políticas habitacionais e urbanas}\label{breve-histuxf3rico-das-poluxedticas-habitacionais-e-urbanas}}

Gonzalez \autocite*{jung2018} compara as políticas de habitação praticadas desde o
início do século XX no Brasil e na Inglaterra. De acordo com Gonzales
\autocite*[165]{jung2018}, a diferença principal entre os dois países, além dos aspectos
cronológicos, é que enquanto no Brasil houve uma opção clara pela construção de
grandes conjuntos habitacionais, geralmente periféricos, na Inglaterra optou-se
por um balanço entre grandes conjuntos e núcleos urbanísticos do tipo
Cidade-Jardim, \emph{new towns}, aluguel social e revitalização urbana. Ainda segundo
Gonzales, o que se observou também nos dois países é que em grandes conjuntos
habitacionais onde priorizam-se a quantidade de imóveis em detrimento da sua
qualidade, estes acabam por serem rejeitados pela população.

Segundo Rolnik \autocite[\emph{apud}][165-166]{jung2018}:
\begin{citacao}
A principal característica dos assentamentos precários é a ausência de
infraestrutura – água, luz, esgoto, rede elétrica, iluminação pública,
pavimentação, etc. Nesse sentido há algumas semelhanças entre a precariedade de
lá e a daqui. Mas nos países desenvolvidos as favelas aparecem aqui e ali,
residualmente. Raramente se permite que uma região fique abandonada por muito
tempo. Já aqui, é comum encontrar até neto dos primeiros moradores de uma favela
vivendo em condições similares às que viviam seus avós. Temos assentamentos
precários com até 50 anos de existência. A favela é elemento estrutural do
processo de urbanização. Ou seja, desde sua concepção, o modelo de urbanização
adotado não acolhe os pobres. Vou dar um exemplo. Na Inglaterra, desde a década
de 50, todo novo empreendimento imobiliário deve destinar parte de seu terreno
às moradias populares. Houve interesse, desde a aceleração do processo de
urbanização, de incluir quem não tinha acesso à moradia. No Brasil, não existe
uma política de uso do solo urbano desenhada para acolher a habitação social.
Aqui a política habitacional se limita ao crédito para a indústria da construção
civil. Mas de que adianta esse crédito se ele só viabiliza a construção de casas
de quinta categoria a 50 quilômetros da cidade? Estamos perpetuando a
precariedade
\end{citacao}
Apesar das políticas habitacionais avançadas executadas durante a primeira
metade do século XX, porém, à partir do \emph{Housing Act}, de 1980, que instituiu o
\emph{Right to buy}, ou direito de comprar (das habitações de interesse social), a
lógica do mercado, do neoliberalismo Tatcherista, voltou a prevalecer: o
resultado é que já em fins dos anos 80, com o aumento dos preços dos imóveis e a
ausência de intervenções públicas, o índice de desabrigados na cidade de Londres
já era novamente muito alto \autocite[72]{jung2018}.

De fato, a produção de novas unidades habitacionais diminui de maneira drástica,
especialmente pela abrupta diminuição das unidades construídas pelo setor
público, o que pode ser visto na figura \ref{fig:inglaterra-unidades}.
\begin{figure}[H]

{\centering \includegraphics[width=0.7\linewidth]{images/inglaterra-unidades} 

}

\caption{Construções de novas unidades habitacionais na Inglaterra, de 1974 a 1994.}\label{fig:inglaterra-unidades}
\end{figure}
\bcenter

Fonte: VARADY, 1998 \autocite[\emph{apud}][73]{jung2018}.
\ecenter

Como se pode perceber, ainda, a diminuição da produção de unidades habitacionais
pelo poder público não foi compensada por um aumento na produção de unidades
pelo setor privado.

Após o ano 2000 diversas tentativas foram feitas pelos governos trabalhistas de
Tony Blair, sem resultados animadores, contudo. Com o intuito de evitar criar
guetos, nos moldes atuais, é exigido dos empreendedores que ao menos 40\% das
unidades construídas sejam destinadas à habitação social (\emph{affordable housing}),
sendo que a qualidade construtiva destas habitações sociais deve ser a mesma das
demais unidades do empreendimento, com excessão da metragem quadrada, o que
parece ser uma tendência \autocite[77]{jung2018}.

\hypertarget{o-sanitarismo-como-propulsor-das-poluxedticas-habitacionais}{%
\subsection{O sanitarismo como propulsor das políticas habitacionais}\label{o-sanitarismo-como-propulsor-das-poluxedticas-habitacionais}}

Segundo Poleto \autocite*[36]{poleto}, dada a incapacidade do mercado, logo após a
revolução industrial, de suprir as condições de moradia digna para a classe
trabalhadora, a intervenção estatal foi necessária, ainda que de forma
inicialmente tímida e pontual.

Para Poleto \autocite*[37]{poleto}, na Holanda, país pioneiro neste tipo de intervenção
estatal para construção de habitações sociais, estas intervenções se deram
principalmente devido às crises da epidemia de cólera de meados do século XIX,
``que acometiam principalmente as regiões com habitações mais pobres, atrelando
dessa forma a condição de moradia a uma questão de saúde pública''.
\begin{citacao}
Na Holanda, de 1888 a 1895, diversos estudos e ações do Estado holandês, que até
hoje é monárquico, delegou e exigiu das municipalidades que se
responsabilizassem pela infraestrutura, incluindo o abastecimento de água, luz,
gás, transporte coletivo (bondes) e telefonia.
Decorrente daí, em 1894, foi elaborado um relatório sobre a questão da habitação
social, com o objetivo de alicerçar uma legislação que permitisse às
municipalidades desapropriarem terras e edifícios urbanos visando à erradicação
de cortiços, gerando, a partir desse Relatório, a Leis de Habitação, decretada
em 1902.
\cite[p.~317]{poleto}
\end{citacao}
Da mesma maneira, devido à atual pandemia de COVID-19, concomitante com esta
grave crise urbana que já se estava alimentando há alguns anos, é evidente que
a tendência é que haja mobilização pela volta da intervenção estatal para a
construção de habitações sociais, tanto no Brasil como no mundo.

A experiência de países como a Finlândia, mostram que a concessão
de habitação permanente aos desabrigados pode, inclusive, ser uma solução mais
rentável do que prover abrigos temporários e que esta intervenção deve ocorrer
da maneira mais rápida possível, pois a tendência é que as pessoas desabrigadas
venham a apresentar outros problemas decorrentes dessa condição, o que tende
a aumentar ainda mais os problemas sociais o que, em consequência, acaba por
aumentar os gastos sociais \autocite{finlandia}.

\hypertarget{cooperativas-habitacionais}{%
\subsection{Cooperativas habitacionais}\label{cooperativas-habitacionais}}

Nos primórdios do capitalismo nos países hoje dito desenvolvidos, dada a
impossibilidade do mercado suprir a demanda por habitações acessíveis para os
trabalhadores, houve uma série de intervenções do estado, ao lado da filantropia,
no sentido de tentar prover habitações em condições salubres aos operários.

Neste quesito teve pioneirismo a Holanda, especialmente após as crises de cólera
em meados do século XIX, que acometiam especialmente as regiões mais pobres e
insalubres.

Datam de 1868 a existência de cooperativas operárias que, em conjunto com as
administrações locais tentavam suprir as necessidades de habitações para a
classe trabalhadora.

A primeira legislação no sentido de regulamentar e incentivar a produção de
habitação atrelada à planejamentos municipais foi o Ato Habitacional
(\emph{Woningwet}), de 1901, na Holanda. Esta lei pioneira foi criada baseada em um
diagnóstico da Sociedade para o Bem-Estar Geral
(\emph{Maatschappij tot nut van t'algemeen}) de 1887, posteriormente publicados em
1894 no documento ``A Questão da Habitação Social''.

Em 1902 o Ato Habitacional foi aprovado pela coroa holandesa que atribuiu às
administrações locais o direito, dever e meios financeiros para o
estabelecimento de planos de extensão para desapropriar imóveis considerados
insalubres, adquirir áreas e construir diretamente ou oferecer empréstimos para
cooperativas ou sociedades de promoção de moradias populares.

Na Alemanha a legislação também fomentou o crescimento das cooperativas
habitacionais (em 1924 foi instituída uma taxa de 15\% sobre o aluguel das
habitações já construídas para alavancar os fundos federais para construção
habitacional) e as próprias administrações muncipais tinham a função de atuar no
desenho urbano para possibilitar a produção em massa de habitação. Esta
interação levou à construção de 130.000 unidades habitacionais entre 1924 e
1930, sendo que em algumas cidades como Berlim os conjuntos habitacionais foram
amplamente divulgados e reconhecidos nos meios arquitetônicos, adquirando valor
emblemático na história da arquitetura do século XX \autocite[90]{poleto}.
\begin{citacao}
Na Alemanha, também as prefeituras eram encarregadas de projetar, por 
seus próprios arquitetos, ou **supervisionar o trabalho das cooperativas e 
associações no projeto das habitações**, construí-las ou supervisionar a 
construção integralmente como foram planejadas, além de determinar o 
financiamento.
Tanto na Holanda, quanto na Alemanha, com o objetivo de minimizar ou 
impedir a inadimplência, **o financiamento para a casa jamais era concedido 
diretamente ao mutuário, como pessoa física, mas para cooperativas municipais**.
\cite[p.~317-318]{poleto}
\end{citacao}
\hypertarget{instrumentos-de-poluxedtica-do-solo}{%
\section{Instrumentos de política do solo}\label{instrumentos-de-poluxedtica-do-solo}}

\hypertarget{estatuto-da-cidade}{%
\subsection{Estatuto da Cidade}\label{estatuto-da-cidade}}

O Estatuto da Cidade (lei 10.257, de 10 de julho de 2001) prevê uma série de
instrumentos de política urbana. Segundo o Estatuto, os instrumentos são
classificados em seis categorias:

I. planos nacionais, regionais e estaduais de ordenação do território e de
desenvolvimento econômico e social;
\begin{enumerate}
\def\labelenumi{\Roman{enumi}.}
\setcounter{enumi}{1}
\item
  planejamento das regiões metropolitanas, aglomerações urbanas e
  microrregiões;
\item
  planejamento municipal;
\item
  institutos tributários e financeiros;
\end{enumerate}
V. institutos jurídicos e políticos;
\begin{enumerate}
\def\labelenumi{\Roman{enumi}.}
\setcounter{enumi}{5}
\tightlist
\item
  \gls{EIA} e \gls{EIV}.
\end{enumerate}
Dentre os instrumentos citados, destacamos os seguintes.

\hypertarget{iptu-progressivo}{%
\subsubsection{IPTU progressivo}\label{iptu-progressivo}}

O \gls{IPTU} progressivo no tempo tem como finalidade promover uma maior
efetividade da função social da propriedade.

\hypertarget{contribuiuxe7uxe3o-de-melhoria}{%
\subsubsection{Contribuição de melhoria}\label{contribuiuxe7uxe3o-de-melhoria}}

A escrever\ldots{}

\hypertarget{regularizauxe7uxe3o-fundiuxe1ria}{%
\subsubsection{Regularização Fundiária}\label{regularizauxe7uxe3o-fundiuxe1ria}}

A \gls{REURB} é um instrumento de política urbana essencial em países onde
existem altos índices de informalidade, como nos países da \gls{AL}, por
ser uma forma de aquisição de propriedade.

No Brasil, diversos avanços foram trazidos com a edição da lei 11.977, de 07 de
junho de 2009, que previa em seu capítulo III, artigos 46 a 71, a regularização
fundiária de assentamentos urbanos e, conjuntamente com a titulação do imóvel,
uma série de medidas associadas a condições dignas de moradia e acesso à
infraestrutura adequada. Porém, com a edição da nova lei 13465/2017, estas
medidas foram retiradas.

A nova lei sobre regularização fundiária (lei 13.465/2017) trata, além da
\gls{REURB-S} e da \gls{REURB-E}, da ``regularização fundiária'' rural e na
Amazônia Legal. Entre os pontos polêmicos desta nova lei está a regularização de
conjuntos habitacionais sem a obrigatoriedade do ``habite-se'' (artigos 60 e 63),
entre outros, como a possibilidade de regularização de imóveis em áreas
públicas, o que deverá ser discutida pelo \gls{STF} na \gls{ADI} 5771, proposta
pela \gls{PGR}.

O instituto da regularização fundiária de imóveis públicos já estava previsto na
lei 9.636, de 15 de maio de 1998, porém apenas para comunidades de baixa renda,
como se vê na redação do art. 6º-A da referida lei:
\begin{citacao}
Art. 6º-A No caso de cadastramento de ocupações para fins de moradia cujo
ocupante seja considerado carente ou de baixa renda, na forma do $\S$ 2º do art.
1º do Decreto-Lei nº 1.876, de 15 de julho de 1981, a União poderá proceder à
regularização fundiária da área, utilizando, entre outros, os instrumentos
previstos no art. 18, no inciso VI do art. 19 e nos arts. 22-A e 31 desta Lei.
\end{citacao}
\hypertarget{o-efeito-da-poluxedtica-de-limitauxe7uxe3o-do-valor-de-aluguuxe9is}{%
\subsection{O efeito da política de limitação do valor de aluguéis}\label{o-efeito-da-poluxedtica-de-limitauxe7uxe3o-do-valor-de-aluguuxe9is}}

A política de limitação do valor de aluguéis garante o controle da inflação,
controlando os custos de moradia para a população (aluguéis), garantindo assim
o direito à moradia, que não implica um direito à propriedade \autocite{fnogueira}.

Segundo \textcite[p.~194-195]{shelter}, \ldots{}

Obviamente que isto implica de num efeito balizador importante para o cálculo de
um preço justo dos imóveis, mas será que esta política apenas é uma política
capaz de regular com um mínimo de equidade o valor do solo urbano?

Qualquer instrumento financeiro, como um título público, uma ação de uma
empresa, ou um derivativo, terá um valor de mercado diferente do valor justo
calculado para o instrumento, em virtude dos movimentos do mercado (oferta vs.
demanda), que se dão não apenas pelos fundamentos econômicos, mas também pelas
expectativas dos diversos agentes econômicos em relação ao valor futuro daquele
instrumento. Em outras palavras, quem determina o preço é o mercado. No entanto,
um modelo de preços é utilizado para a determinação de um \emph{benchmark}, uma
referência de mercado. Assim como o \emph{valuation} serve para calcular o ``preço
justo'' de uma ação (ou \emph{target price}, ou ainda preço-alvo), o que por sua vez
permite às corretoras efetuarem recomendações (\emph{outperform} ou \emph{buy}, \emph{neutral}
ou \emph{hold}, \emph{underperform} ou \emph{sell}), assim como o método de Black\&Scholes
\autocite{marins2} permite o cálculo do preço justo das opções de compra e venda, as
séries perpétuas permitem um cálculo razoável do preço justo de um imóvel, haja
vista que permitem, sem maiores especulações, o cálculo do fluxo de caixa
descontado dos aluguéis, balizando assim as expectativas em torno dos preços dos
imóveis, ajudando a conter dessa maneira a especulação imobiliária.

Em períodos de normalidade econômica poder-se-ia dizer que o estabelecimento
de preços máximos de aluguéis seriam suficientes para conter uma especulação
imobiliária desenfreada.

No entanto, o mundo não vive tempos de normalidade econômica. Desde a crise de
2008 o mundo vive tempos de uma crise crônica, persistente, que tem sido
enfrentada pelos bancos centrais (especialmente os bancos centrais dos países
desenvolvidos, onde a estagnação é maior), por enormes aumentos de liquidez
do sistema financeiro.

Ou seja, a imposição de um valor máximo aos aluguéis pode ser uma boa política
para contenção da inflação, regulando os custos de moradia das classes menos
privilegiadas, impondo também alguma limitação da especulação imobiliária
desenfreada, porém em um cenário de juros baixíssimos e estagnação econômica
crônica como o que se avizinha e que cada vez mais se prevê de forma duradoura,
não será o suficiente para conter uma alta expressiva no valor dos imóveis.

Pode-se prever que, apenas com esta política de limitação de aluguéis, o mundo
estará fadado a dividir os cidadãos em proprietários e locatários, uma vez que o
preço do aluguel estará controlado, mas o valor dos imóveis tende a disparar.

\hypertarget{problemas-gerados-pelo-alto-valor-dos-imuxf3veis}{%
\section{Problemas gerados pelo alto valor dos imóveis}\label{problemas-gerados-pelo-alto-valor-dos-imuxf3veis}}

Além dos problemas relacionados ao setor bancário discutidos na seção
\ref{MI-e-o-setor-bancario}, existe um problema, normalmente desconsiderado
pela maioria, que o altos preços de imóveis e/ou aluguéis previnem que os
trabalhadores se mudem para cidades com maior produtividade marginal do
trabalho. Ou seja, algumas pessoas tendem a se manter em ocupações de baixa
produtividade em cidades pequenas, no interior, ao invés de se mudarem para
grandes cidades onde certamente teriam ocupações de maior produtividade, como
numa fábrica, por exemplo, pois o custo adicional de moradia que elas teriam não
seria compensado pelo maior salário que receberiam, devido à maior produtividade
da sua ocupação. Se fosse possível a redução dos custos de moradia em regiões de
maior produtividade da mão de obra, isto resultaria num maior Produto Interno
Bruto \autocites[149]{Case2000}{economist-housing-2020}.
\begin{citacao}
A habitação também é um grande motivo pelo qual muitas pessoas no mundo rico
sentem que o economia não funciona para eles. Enquanto os \emph{baby boomers}
tendem a possuir casas grandes e caras, os jovens precisam cada vez mais alugar
um lugar apertado com seus amigos, fomentando o ressentimento dos jovens
\emph{millenials} em relação aos idosos. O economista Thomas Piketty afirmou que
nas últimas décadas o retorno ao capital excedeu o que é pago ao trabalho na
forma de salários, aumentando a desigualdade. Mas outros criticaram os achados
de Piketty, apontando que o que realmente explica o aumento da participação no
capital está nos retornos crescentes sobre a habitação.
\cite{economist-housing-2020}
\end{citacao}
Mas o principal problema advindo de uma grande alta no valor dos imóveis é o
problema da reversão das expectativas: uma vez que as expectativas se revertem,
seja por um aumento da taxa de juros, seja pelo estouro de uma bolha de crédito,
como ocorreu em 2008, tendem a causar transtornos tanto para as pessoas físicas,
que podem perder seus imóveis, quanto para as instituições financeiras expostas
ao risco do mercado imobiliário, seja para os governos que são instados a
socorrer as empresas e as pessoas em dificuldades.

Segundo o conceituado semanário econômico \emph{The Economist}, entre 1960 e 2000 um
quarto das recessões no mundo rico estavam associadas com fortes declínios de
preços de moradia. Ainda segundo a \emph{The Economist}, recessões associadas com
restrições de créditos e estouros de preços de moradia foram mais profundas e
mais duradouras do que as outras recessões.\autocite{economist-housing-2020}.

Segundo \textcite{fmiera}, episódios com padrões de \emph{boom} seguidos de estouros (de
bolhas), ou \emph{boom-bust patterns}, precederam mais de dois terços das 50 mais
recentes crises bancárias sistêmicas.

Uma reversão de expectativas pode ocorrer devido a um fator exógeno. Por exemplo,
no momento em que se escreve esta dissertação, o mundo se encontra em meio a
uma pandemia de proporções globais e ainda sem qualquer perspectiva que seja
encontrada uma cura ou uma vacina contra o Corona vírus.

Estimativas recentes dão conta de que os níveis de desemprego podem aumentar
substancialmente em todo o mundo, o que pode gerar uma perspectiva de que a
demanda por aluguéis diminua muito, puxada pela diminuição da renda da
população devido ao desemprego, \emph{i.e.}, uma diminução no numerador da equação
\eqref{eq:perpetua}.

Por outro lado, a brusca queda das taxas de juros longas, \emph{i.e}, uma diminuição
no denominador da equação \eqref{eq:perpetua} deve mais que compensar a queda no
numerador.

Seja, por exemplo um imóvel em que, antes da crise, se imaginava que produziria
uma série de pagamentos de aluguéis de R\$2.000,00 mensais. A uma taxa de juros
de longo prazo de 2\% a.a., o seu valor justo, segundo a equação
\eqref{eq:perpetua} seria de R\$1.200.000,00. Imaginando que, com a crise do COVID
haja uma redução nas expectativas de receitas com o imóvel, passando para uma
prestação mensal de R\$1.000,00. Com a queda da taxa dos títulos para 1\% a.a.,
o preço do imóvel permaneceria constante.

Uma vez alternadas as expectativas, no entanto, ou seja, uma vez que se resolva
a crise sanitária da COVID-19, as taxas de juros podem subir repentinamente,
ainda que permanecendo baixa em níveis históricos, mas muito dificilmente
ocorrerá o mesmo no mercado de trabalho, que tem recuperação lenta. Este cenário
poderia vir a desencadear uma crise imobiliária similar ou ainda pior do que a
da década passada, haja vista que na atualidade, as ferramentas tradicionais de
política monetária já foram exauridas, além das ferramentas não-tradicionais.

\hypertarget{mercado-imobiliuxe1rio-e-populismo}{%
\section{Mercado imobiliário e populismo}\label{mercado-imobiliuxe1rio-e-populismo}}

Segundo a \emph{The Economist}, o mercado imobiliário e o populismo estão intimamente
ligados: novas pesquisas mostraram que pessoas que habitam locais onde o mercado
imobiliário está estagnado tem se aproximado mais de partidos e ideiais da
extrema-direita populista na Inglaterra e na França \autocite{economist-housing-2020}.

Ainda de acordo com a \emph{The Economist}, felizmente os governos estão começando a
perceber o estrago causado pelos erros da política habitacional desde o fim da
segunda guerra mundial. Para \textcite{economist-housing-2020}:
\begin{citacao}
sistemas de planejamento flexíveis, tributação e regulamentação financeira
apropriadas podem transformar a habitação em força para estabilidade social e
econômica. O sistema de habitação pública de Cingapura ajuda a melhorar inclusão
social; financiamento imobiliário na Alemanha ajudou o país a evitar a pior da
crise de 2008-10; O sistema de planejamento da Suíça ajuda bastante a explicar
porque o populismo até agora não chegou até lá. Governos em todo o mundo
precisam agir de forma decisiva e sem demora. Nada menos que a estabilidade 
econômica e política do mundo está em jogo.
\cite{economist-housing-2020}
\end{citacao}
\hypertarget{exemplo-de-regularizauxe7uxe3o-fundiuxe1ria-de-imuxf3vel-puxfablico-ponta-do-leal-florianuxf3polissc}{%
\paragraph{Exemplo de regularização fundiária de imóvel público -- Ponta do Leal, Florianópolis/SC}\label{exemplo-de-regularizauxe7uxe3o-fundiuxe1ria-de-imuxf3vel-puxfablico-ponta-do-leal-florianuxf3polissc}}

Um exemplo do bom uso da regularização fundiária em imóvel público é a
regularização da comunidade da Ponta do Leal, localizada no Balneário do
Estreito, em Florianópolis/SC.

O processo tramitou na \gls{SPU} em Santa Catarina sob nº 04972.000987/2010-14.
\begin{figure}[H]

{\centering \includegraphics[width=0.7\linewidth]{images/ponta-leal-antes} 

}

\caption{Ponta do Leal. Situação anterior: palafitas situadas em terrenos de marinha.}\label{fig:ponta-leal-antes}
\end{figure}
\bcenter

Fonte: \textcite{sigsc}.
\ecenter

Como pode ser visto na figura \ref{fig:ponta-leal-antes}, a comunidade ali
radicada habitava palafitas em condições degradantes, sem condições mínimas de
salubridade.

Com a intervenção do governo federal, através da \gls{SPU} em SC, responsável
pelos cadastramento das famílias, e com o auxílio da \gls{CEF}, foram
construídos quatro edifícios de apartamentos com o \gls{PMCMV}, beneficiando as
famílias ali existentes.
\begin{figure}[H]

{\centering \includegraphics[width=0.7\linewidth]{images/ponta-leal-depois} 

}

\caption{Ponta do Leal. Situação atual: construídos quatro edifícios de apartamentos.}\label{fig:ponta-leal-depois}
\end{figure}
\bcenter

Fonte: Imagens de satélite da ESRI.
\ecenter

A figura \ref{fig:ponta-leal-depois} mostra a área após a regularização, com os
edifício construídos e as palafitas devidamente removidas (a fota atualizada com
as palafitas removidas ainda não estão disponíveis, mas até o final da escrita
da dissertação devem estar).

\hypertarget{outros-instrumentos}{%
\subsection{Outros instrumentos}\label{outros-instrumentos}}

\hypertarget{controle-de-aluguuxe9is-residenciais}{%
\subsubsection{Controle de aluguéis residenciais}\label{controle-de-aluguuxe9is-residenciais}}

Diversos países utilizaram em algum momento a política de controle de aluguéis
com o intuito de conter o peso excessivo dos custos habitacionais sobre a
população em geral. Estas políticas intervencionistas do poder público sobre o
mercado, no entanto, vão contra a ideologia liberal predominante, que prega a
livre atuação das forças de mercado como solução para os mais diversos problemas,
não somente os habitacionais.

Na Alemanha houve controle de aluguéis entre 1917 e 1922, o que afastou a
iniciativa privada da produção habitacional. Por outro lado, o governo assumiu
pra si a responsabilidade pela construção das moradias, chegando a ser
utilizados fundos públicos para a construção de 70\% das habitações edificadas
\autocite[89]{poleto}.

Após, em 1990, o chamado \emph{Wohnungsgemeinnütziqkeit} limitou os reajustes dos
aluguéis em Berlim, após a reabertura econômica.

Atualmente, após um rápido e forte aumento dos preços dos imóveis e dos
aluguéis, devido ao forte aumento da procura por moradia na cidade, às
dificuldades regulatórias para a construção de novas moradias e o forte
desenvolvimento do setor de turismo que se estabeleceu na cidade, Berlim foi
obrigada a estabelecer um teto no valor dos aluguéis (\emph{Mietendeckel}), além do
congelamento do preço dos aluguéis dos imóveis construídos antes de 2013 e da
requisição pública de imóveis privatizados durante a década de 1990. \autocite{berlim}.

A crise habitacional atual tem levado uma série de cidades européias a implantar
o controle de aluguéis residenciais, tais como Barcelona, Amsterdam e Paris
\autocite{economist-rent-control}. A cidade de Londres também vem discutindo a
implementação de uma comissão para sistema de controle efetivo de aluguéis, a
criação de um sistema de registro universal de proprietários e o estabelecimento
de tetos para os aluguéis como medida de urgência, enquanto não se cria a
comissão referida \autocite{guardian}.

No Brasil houve uma tentativa de estabelecer um controle de aluguéis através do
decreto-lei do inquilinato, de 1942. No entanto, àquela época a adoção da do
controle de aluguéis parece ter tido mais motivações econômicas, no sentido de
destinar mais capitais para a indústria leve então em implantação no Brasil, num
contexto de falta de capitais de vulto para fazêlo, do que motivações no sentido
de promover uma boa política habitacional. De fato, o efeito do congelamento do
valor dos aluguéis então foi de paralisar o segmento de construção para fins
habitacionais, levando ao fim numa crise habitacional de grande monta.
\autocite{bonduki_origens_1994}

De toda maneira, entende-se que esta política de controle de aluguéis é muito
limitada e só teria um efeito positivo se mantida por um curto período de tempo,
de maneira que não desincentive a construção de novas moradias pelo mercado, e
que venha acompanhada de outras medidas que impulsionem a construção de
habitações sociais.

\hypertarget{intervenuxe7uxe3o-estatal-direta}{%
\subsubsection{Intervenção estatal direta}\label{intervenuxe7uxe3o-estatal-direta}}

Apesar da atual procura pelos governos por soluções de mercado que resolvam os
problemas sem a necessidade de uma intervenção estatal direta, parece que esta
é, ainda hoje, a única opção realmente eficaz no sentido de resolver o problema.

Apesar do relativo sucesso do \gls{PMCMV}, soluções como as da Ponta do Leal, em
que a implantação da habitação social se deu em área nobre, são uma exceção à
regra. A tendência neste tipo de programa é que os empreendimentos se viabilizem
apenas em áreas remotas, desprovidas de transporte público decente, escolas,
postos de saúde, segurança e outros, o que acaba por acarretar ou numa
necessidade maior de investimento em infraestrutura pública, ou na tendência de
favelização e abandono destas construções com o tempo.

O artigo 3º do Estatuto da Cidade prevê que cabe a União, entre outras
atribuições:
\begin{citacao}
III - promover, por iniciativa própria e em conjunto com os Estados, o Distrito
Federal e os Municípios, programas de construção de moradias e melhoria das
condições habitacionais, de saneamento básico, das calçadas, dos passeios
públicos, do mobiliário urbano e dos demais espaços de uso público;
\end{citacao}
No entanto, ainda é preciso definir como isso deve ser implementado.

O que o Brasil, assim como outros países da \gls{AL} necessitam, é não de
apenas a implementação solta de alguns dos instrumentos da política urbana
previstas no Estatuto da Cidade por parte de um ou outro ente municipal ou
federado. É necessário o estabelecimento de uma política pública totalmente
coordenada que, utilizando os diversos instrumentos e com objetivo definido,
ataque diretamente o problema e não aguarde que uma eventual melhora da
regulação do mercado imobiliário porventura obtida com a aplicação de um ou mais
instrumentos venha a resolver os problemas de falta de habitação decente através
da iniciativa privada. A história dos países com histórico de desenvolvimento
similar ao do Brasil e muitos outros países da \gls{AL}, com forte êxodo rural e
crescimento urbano vertiginoso, como a Inglaterra e a Alemanha, apenas mostram
que soluções como as de mercado não resolvem o problema do déficit habitacional,
especialmente para as classes menos favorecidas. As soluções de mercado se
aplicam aos países cujo desenvolvimento capitalista tenha se dado nas condições
de prévia reforma agrária, como os EUA ou a França, onde a migração do campo
para a cidade se deu de maneira mais gradual, devido às melhores condições do
campesinato quando da abertura do complexo rural.

\hypertarget{diretrizes-para-a-implementauxe7uxe3o-de-poluxedticas-puxfablicas-para-a-habitauxe7uxe3o-social}{%
\section{Diretrizes para a implementação de políticas públicas para a habitação social}\label{diretrizes-para-a-implementauxe7uxe3o-de-poluxedticas-puxfablicas-para-a-habitauxe7uxe3o-social}}

Políticas públicas devem ser traçadas no sentido de:
\begin{enumerate}
\def\labelenumi{\arabic{enumi}.}
\item
  Controle de aluguéis dos imóveis existentes por um breve período de tempo,
  diminuindo assim o grau de especulação do sistema. Esta medida, por si só, deve
  vir a ser necessária aqui, como tem se mostrado necessária no exterior, mas não
  é suficiente. Deverá vir acompanhada de medidas de estímulo à produção de novos
  imóveis para a solução do déficit habitacional no longo prazo;
\item
  Para os novos empreendimentos urge considerar adotar medidas como as da
  Inglaterra, como o estabelecimento de uma cota de habitações sociais. Estes
  empreendimentos deverão ser localizados preferencialmente em regiões de
  infraestrutura já consolidada. Para isto o poder pública municipal deverá
  utilizar dos instrumentos de desapropriação, caso necessário;
\item
  Ao mesmo tempo, o poder público deve iniciar a busca de novas fontes de
  financiamento, através do bom uso dos instrumentos disponíveis previstos no
  Estatuto da Cidade (IPTU progressivo, Contribuição de melhoria e outros). No
  entanto, acredita-se que apenas medidas sistêmicas e obrigatórias surtam
  efeito. Deixar que a possibilidade de aplicação destas medidas a cargo das
  municipalidades tende a inação pela lógica política local;
\item
  Os recursos advindos da aplicação destes instrumentos deve ter destinação
  certa: deve ser estudado um marco legal para isto, que preveja que os recursos
  arrecadados com estes instrumentos sejam revertidos para a atualização e
  manutenção dos cadastros municipais, revisão de plantas de valores, criação de
  bases de dados confiáveis de transações imobiliárias e aluguéis, confecção de
  índices imobiliários que possibilitem a aferição do estoque de moradias e o
  investimento em novas moradias, desapropriação de áreas para implantação de
  empreendimentos para habitação social, etc;
\item
  Devem ser criadas empresas municipais de urbanização que reúna os melhores
  engenheiros e arquitetos, que se especializem nas soluções adotadas por outros
  países, como as cidades-jardim, as \emph{new towns} e projetem conjuntos
  habitacionais dentro de um contexto maior de urbanismo para que não se deteriore
  facilmente, para que não se criem novos guetos;
\item
  Os empreendimentos devem vir acompanhados de infraestrutura pública,
  especialmente segurança e saneamento básico, para que se evite a propagação de
  novas epidemais;
\end{enumerate}
\hypertarget{conclusao}{%
\chapter{Conclusão}\label{conclusao}}

Os tempos que se aproximam são de uma tendência de queda brutal das taxas de
juros de longo prazo. Estas taxas apenas refletem uma imensa sobra de capital no
sistema, decorrente da falta de oportunidades de inversões em atividades
produtivas, o que tende a levar a eficácia marginal do capital a níveis muito
baixos, as vezes até negativos. As taxas de juros oficiais apenas refletem esta
realidade do mercado. Problemas como a armadilha de liquidez, antes restritos
aos países desenvolvidos, estão na ordem natural das coisas também para os
países emergentes \autocite{krugman-emergentes}.

Este capital abundante, sem oportunidades de inversão, tende a se procurar
reservas de valor, se alocando em grande parte no mercado imobiliário. Isto, por
si só já é um grande problema a ser resolvido, pois a tendência é que, como a
recessão econômica, novos empreendimentos não venham a surgir na mesma
velocidade de antes e os imóveis podem ficar parados por anos a espera de
oportunidades de inversão, tendo lugar apenas empreendimentos para as classes
mais altas, muito mais rentáveis. Porém, um problema ainda maior pode vir a
ocorrer com o enorme influxo de capital se movendo para o mercado imobiliário, a
saber, uma forte tendência de aumento dos preços. Esta possível tendência de
aumento dos preços pode vir a desencadear uma espiral de especulação
desenfreada, atraindo ainda mais capitais para o mercado imobiliário.

O problema destes juros muito baixos já se fez sentir nos países desenvolvidos,
onde o excesso de capitais teve papel principal na criação de bolhas
imobiliárias, como a do \gls{glo:subprime} norte-americano. Os principais prejudicados,
no final, são os mutuários, que perdem suas casas e suas poupanças, já que os
especuladores tendem a lucrar com a liquidação forçada dos imóveis,
revendendo-os depois de alguns anos com a recuperação dos preços, como foi
observado. Mas mesmo após o estouro da bolha, a elevação dos preços dos imóveis
se mostrou uma tendência nestes países e o resultado é que há crises
habitacionais em praticamente todos os países, pela falta de habitação a preços
acessíveis.

Nos países em desenvolvimento, esta tendência ao aumento do preço da terra em
períodos de recessão econômica, já prevista por Rangel, tende somente a agravar
a crise habitacional crônica em que vivem o Brasil e outros países da \gls{AL}
há décadas.

Neste contexto, os instrumentos tradicionais de política urbana previstos no
Estatuto da Cidade são insuficientes. Não quer isto dizer que sejam
desnecessários. Muito pelo contrário: o que se necessita é de uma política
de Estado que coloque em aplicação os diversos instrumentos previstos de maneira
coordenada, além da inclusão de outros instrumentos mais radicais não previstos,
dada a magnitude do problema.

Esta política habitacional deve ser coordenada em torno de um objetivo, que é
a construção de habitação social. A implementação destas políticas não pode ser
deixada a cargo de cada municipalidade: as administrações municipais devem ter
um papel de auxiliar na implantação da política habitacional, mas a coordenação
deve ser nacional, pois a lógica política local tende a abortar muitas
iniciativas.

Por fim, uma nova política habitacional poderia trazer consigo o estabelecimento
de um novo ciclo de crescimento econômico, pois tem um grande potencial para a
geração de empregos. Além disto, o investimento em habitação social pode vir a
diminuir os gastos sociais, pois melhora as condições de vida da população, e
e reduzir também os gastos com a saúde pública, dada a melhora no saneamento
básico para a população.

\postextual

\begingroup

\printbibliography[title=REFERÊNCIAS]

\endgroup

\markboth{Referências}{REFERÊNCIAS}

\hypertarget{refs}{}

\bapendices

\hypertarget{capital}{%
\chapter{Eficiência Marginal do Capital}\label{capital}}

\hypertarget{a-funuxe7uxe3o-investimento-e-a-eficiuxeancia-marginal-do-capital}{%
\section{A função investimento e a eficiência marginal do capital}\label{a-funuxe7uxe3o-investimento-e-a-eficiuxeancia-marginal-do-capital}}

Para Bresser-Pereira \autocite*[3]{Bresser-Pereira1973}, \emph{a determinação da variável
estratégica a determinar o volume de investimentos torna-se de extraordinária
importância}.

Segundo Bresser-Pereira \autocite*[3]{Bresser-Pereira1973}, \emph{a tradição clássica de dar
primazia a taxa de lucros foi abandonada pelos neoclássicos, que colocaram a
taxa de juros no centro do seu sistema}. Posteriormente, foi Keynes quem
\emph{restabeleceu, até um certo ponto, a importância da taxa de lucros, através do
conceito de eficiência marginal do capital}.

Para \textcite{Bresser-Pereira1973}, ``a teoria ortodoxa\footnote{Bresser define como economistas ortodoxos os economistas neoclássicos e os
  keynesianos, no contexto do trabalho citado.} sobre a função investimento
afirma que a acumulação de capital depende da taxa de lucro prevista (ou
eficiência marginal do capital) da taxa de juros, dado o nível da renda'', com
uma relação inversa, ou seja, à medida que aumenta o volume de investimentos,
cai a eficiência marginal do capital, conforme pode ser observado na
figura \ref{fig:eficienciamarginal} \autocite[4]{Bresser-Pereira1973}:
\begin{figure}[H]

{\centering \includegraphics[width=0.8\linewidth]{images/Page-4-Image-1} 

}

\caption{Eficiência Marginal do Capital e Investimento.}\label{fig:eficienciamarginal}
\end{figure}
Uma das possíveis explicações para esta relação inversa pode ser vista no trecho
abaixo:
\begin{citacao}
Há, portanto, uma relação inversa entre o volume dos investimentos e a
eficiência marginal do capital. Podemos, inclusive, imaginar que as empresas ou
os empresários disponham sempre de um "estoque" de projetos de investimentos,
com taxas diferentes e declinantes de lucro. Quanto maiores fossem os
investimentos efetivamente realizados, mais seria preciso descer na escala de
rentabilidade prevista dos projetos \ldots Será interessante para a empresa
investir enquanto ela puder esperar do novo investimento um retorno superior ou
pelo menos igual ao da taxa de juros do mercado
\cite[p.~5]{Bresser-Pereira1973}.
\end{citacao}
A citação acima implica que também haverá uma relação entre a taxa de juros de
mercado e o volume de investimentos, novamente em uma relação inversa, haja
vista que quanto menor for a taxa de juros de mercado, maior será o volume de
investimentos.

A diferença básica entre a taxa de juros de mercado e a taxa de lucros (ou
eficiência marginal do capital), segundo \textcite{Bresser-Pereira1973}, é que, enquanto a
taxa de lucros é dependente do volume de investimentos, a taxa de juros de
mercado é uma variável independente.
\begin{citacao}
Em outras palavras, é a variação dos investimentos que leva à variação da
eficiência marginal do capital, enquanto que é a variação da taxa de juros que
leva à variação do volume de investimentos
\cite{Bresser-Pereira1973}.
\end{citacao}
Segundo \textcite{Bresser-Pereira1973}, a eficiência marginal do capital
varia conforme o nível de otimismo dos empresários. A ``distinção entre a
eficiência marginal do capital, dado um determinado nível de otimismo dos
empresários, \(r\), e a eficiência marginal do capital com diferentes níveis de
otimismo, quanto às suas perspectivas de lucro, \(r’\)'', pode ser vista na
figura \ref{fig:eficienciamarginal2}: \textit{fixada uma taxa de juros em um
determinado nível $j_1$, podemos, então, deduzir graficamente uma nova função
investimento, relacionando positivamente o volume de investimentos, dado um
nível de renda, com a influência marginal do capital, $r’$, a diferentes níveis de
otimismo} \cite[p.~8]{Bresser-Pereira1973}:
\begin{figure}[h]
\begin{center}
\includegraphics[width=.8\textwidth]{images/Page-8-Image-3.png}
\includegraphics[width=.8\textwidth]{images/Page-8-Image-4.png}
\end{center}
\caption{A nova função Investimento.}
\label{fig:eficienciamarginal2}
\end{figure}
\begin{citacao}
Através dos mecanismos ortodoxos da política monetária e fiscal, e dos
mecanismos menos ortodoxos da política salarial, da política cambial, da
política fiscal ampliada, que inclui subsídios os mais variados, o Governo tem
condições crescentes de influenciar direta ou indiretamente as perspectivas de
lucro dos empresários. Por outro lado, as variações no nível de segurança
política para os investimentos, tão grandes no mundo moderno, devem também fazer
variar grandemente o nível de otimismo dos empresários em relação a suas
perspectivas de lucro
\cite[p.~9]{Bresser-Pereira1973}.
\end{citacao}
Segundo Rangel (\emph{apud} \textcite{pereira}), \emph{a eficácia marginal do capital das empresas
com capacidade ociosa é negativa e, pela lógica, é essa eficácia que deve 3
orientar a taxa de juros}.

\eapendices

\banexos

\hypertarget{artigo-valor-econuxf4mico}{%
\chapter{Artigo Valor Econômico}\label{artigo-valor-econuxf4mico}}

\eanexos

% ----------------------------------------------------------
% Glossário
% ----------------------------------------------------------
%
% Consulte o manual da classe abntex2 para orientações sobre o glossário.
%
\imprimirglossario

%---------------------------------------------------------------------
% INDICE REMISSIVO
%---------------------------------------------------------------------
\phantompart
\printindex
%---------------------------------------------------------------------

\end{document}
