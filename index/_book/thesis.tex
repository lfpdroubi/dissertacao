% ------------------------------------------------------------------------
% ------------------------------------------------------------------------
% Modelo UFSC para Trabalhos Academicos (tese de doutorado, dissertação de
% mestrado) utilizando a classe abntex2
%
% Autor: Alisson Lopes Furlani
% 	Modificações:
%	- 27/08/2019: Alisson L. Furlani, add pacote 'glossaries' para listas
% - 30/10/2019: Alisson L. Furlani, adjusted some spacing errors and changed math fonts
% - 17/01/2019: Alisson L. Furlani, updated certification page
% - 03/03/2020: Luiz F. P. Droubi, change file to be used as a template with R.
% ------------------------------------------------------------------------
% ------------------------------------------------------------------------

\documentclass[
	% -- opções da classe memoir --
	12pt,				% tamanho da fonte
	%openright,			% capítulos começam em pág ímpar (insere página vazia caso preciso)
	oneside,			% para impressão no anverso. Oposto a twoside
	a4paper,			% tamanho do papel.
	% -- opções da classe abntex2 --
	chapter=TITLE,		% títulos de capítulos convertidos em letras maiúsculas
	section=TITLE,		% títulos de seções convertidos em letras maiúsculas
	%subsection=TITLE,	% títulos de subseções convertidos em letras maiúsculas
	%subsubsection=TITLE,% títulos de subsubseções convertidos em letras maiúsculas
	% -- opções do pacote babel --
	english,			% idioma adicional para hifenização
	%french,				% idioma adicional para hifenização
	%spanish,			% idioma adicional para hifenização
	brazil				% o último idioma é o principal do documento
	]{abntex2}

\usepackage{setup/ufscthesisA4-alf}

\addbibresource{bib/cap1.bib}
\addbibresource{bib/cap2.bib}
\addbibresource{bib/cap3.bib}
\addbibresource{bib/cap4.bib}
\addbibresource{bib/cap5.bib}
\addbibresource{bib/pkgs.bib}

\usepackage[table]{xcolor}
\let\newfloat\undefined
\usepackage{floatrow}
\floatsetup[table]{capposition=top}
\floatsetup[figure]{capposition=top}

\newcommand{\pkg}[1]{{\normalfont\fontseries{b}\selectfont #1}}
\let\proglang=\textsf
\let\code=\texttt


\newcommand{\bcenter}{\begin{center}}
\newcommand{\ecenter}{\end{center}}

\newcommand{\bapendices}{\begin{apendicesenv}}
\newcommand{\eapendices}{\end{apendicesenv}}

\newcommand{\banexos}{\begin{anexosenv}}
\newcommand{\eanexos}{\end{anexosenv}}

% ---
% Filtering and Mapping Bibliographies
% ---
\DeclareSourcemap{
	\maps[datatype=bibtex]{
		% remove fields that are always useless
		\map{
			\step[fieldset=abstract, null]
			\step[fieldset=pagetotal, null]
		}
		% remove URLs for types that are primarily printed
%		\map{
%			\pernottype{software}
%			\pernottype{online}
%			\pernottype{report}
%			\pernottype{techreport}
%			\pernottype{standard}
%			\pernottype{manual}
%			\pernottype{misc}
%			\step[fieldset=url, null]
%			\step[fieldset=urldate, null]
%		}
		\map{
			\pertype{inproceedings}
			% remove mostly redundant conference information
			\step[fieldset=venue, null]
			\step[fieldset=eventdate, null]
			\step[fieldset=eventtitle, null]
			% do not show ISBN for proceedings
			\step[fieldset=isbn, null]
			% Citavi bug
			\step[fieldset=volume, null]
		}
	}
}
% ---

% ---
% Informações de dados para CAPA e FOLHA DE ROSTO
% ---
% FIXME Substituir 'Nome completo do autor' pelo seu nome.
\autor{LUIZ FERNANDO PALIN DROUBI}
% FIXME Substituir 'Título do trabalho' pelo título da trabalho.
\titulo{O Mercado Imobiliário e a Economia}
% FIXME Substituir 'Subtítulo (se houver)' pelo subtítulo da trabalho.
% Caso não tenha substítulo, comente a linha a seguir.
  \subtitulo{Qualificação}
% FIXME Substituir 'XXXXXX' pelo nome do seu
% orientador.
\orientador{Norberto Hochheim}
% FIXME Se for orientado por uma mulher, comente a linha acima e descomente a linha a seguir.
% \orientador[Orientadora]{Nome da orientadora, Dra.}
% FIXME Substituir 'XXXXXX' pelo nome do seu
% coorientador. Caso não tenha coorientador, comente a linha a seguir.
% FIXME Se for coorientado por uma mulher, comente a linha acima e descomente a linha a seguir.
% \coorientador[Coorientadora]{XXXXXX, Dra.}
% FIXME Substituir '[ano]' pelo ano (ano) em que seu trabalho foi defendido.
\ano{2020}
% FIXME Substituir '[dia] de [mês] de [ano]' pela data em que ocorreu sua defesa.
\data{31 de Março de 2020}
% FIXME Substituir 'Local' pela cidade em que ocorreu sua defesa.
\local{Florianópolis}
\instituicaosigla{UFSC}
\instituicao{Universidade Federal de Santa Catarina}
% FIXME Substituir 'Dissertação/Tese' pelo tipo de trabalho (Tese, Dissertação).
\tipotrabalho{Dissertação}
% FIXME Substituir '[mestre/doutor] em XXXXXX' pela grau adequado.
\formacao{Mestre em Engenharia de Transportes e Gestão Territorial}
% FIXME Substituir '[mestrado/doutorado]' pelo nivel adequado.
\nivel{mestrado}
% FIXME Substituir 'Programa de Pós-Graduação em XXXXXX' pela curso adequado.
\programa{Programa de Pós-Graduação em Engenharia de Transportes e Gestão
Territorial}
% FIXME Substituir 'Campus XXXXXX ou Centro de XXXXXX' pelo campus ou centro adequado.
\centro{CTC - CENTRO TECNOLÓGICO}
\preambulo
{%
\imprimirtipotrabalho~submetida~ao~\imprimirprograma~da~\imprimirinstituicao~para~a~obtenção~do~título~de~\imprimirformacao.
}
% ---

% ---
% Configurações de aparência do PDF final
% ---
% alterando o aspecto da cor azul
\definecolor{blue}{RGB}{41,5,195}
% informações do PDF
\makeatletter
\hypersetup{
     	%pagebackref=true,
		pdftitle={\@title},
		pdfauthor={\@author},
    	pdfsubject={\imprimirpreambulo},
	    pdfcreator={LaTeX with abnTeX2},
		pdfkeywords={ufsc, latex, abntex2},
		colorlinks=true,       		% false: boxed links; true: colored links
    	linkcolor=black,%blue,          	% color of internal links
    	citecolor=black,%blue,        		% color of links to bibliography
    	filecolor=black,%magenta,      		% color of file links
		urlcolor=black,%blue,
		bookmarksdepth=4
}
\makeatother
% ---

% ---
% compila a lista de abreviaturas e siglas e a lista de símbolos
% ---

% Declaração das siglas
\siglalista{ABNT}{Associação Brasileira de Normas Técnicas}
\siglalista{Bacen}{Banco Central do Brasil}
\siglalista{FED}{Federal Reserve Bank}
\siglalista{FRED}{Federal Reserve Economic Data}
\siglalista{FMI}{Fundo Monetário Internacional}
\siglalista{CDO}{Collateralized Debt Obligation}
\siglalista{CMO}{Collateralized Mortgage Obligation}
\siglalista{ABECIP}{Associação Brasileira das Entidades de Crédito Imobiliário e Poupança}
\siglalista{IBRE}{Instituto Brasileiro de Economia}
\siglalista{FGV}{Fundação Getúlio Vargas}
\siglalista{SPV}{Special-purpose Vehicle}
\siglalista{ABS}{Asset Backed Security}
\siglalista{MBS}{Mortgage Backed Security}
\siglalista{CDS}{Credit Default Swaps}


% Declaração dos simbolos
\simbololista{R_e}{\ensuremath{R_e}}{Taxa mínima de atratividade}
\simbololista{R_f}{\ensuremath{R_f}}{Taxa livre de risco}
\simbololista{R_m}{\ensuremath{R_m}}{Taxa de risco do mercado}
\simbololista{hp}{\ensuremath{hp}}{Preço dos imóveis (índice)}
\simbololista{long}{\ensuremath{long}}{Taxa de juros de longo prazo}
\simbololista{short}{\ensuremath{short}}{Taxa de juros de curto prazo}
\simbololista{constr}{\ensuremath{constr}}{Nível dos custos de construção}
\simbololista{EA}{\ensuremath{EA}}{Nível de atividade econômica}
\simbololista{D}{\ensuremath{D}}{Nível de Demanda}
\simbololista{S}{\ensuremath{S}}{Nível de Oferta}
\simbololista{R}{\ensuremath{R}}{Renda (bruta) recorrente auferida com aluguel do imóvel}
\simbololista{R_L}{\ensuremath{R_L}}{Renda líquida recorrente auferida com aluguel do imóvel}
\simbololista{C}{\ensuremath{C}}{Custos recorrentes de manutenção do imóvel}
\simbololista{y_r}{\ensuremath{y_r}}{Taxa de rendimento do aluguel de um imóvel}

% compila a lista de abreviaturas e siglas e a lista de símbolos
\makenoidxglossaries

% ---

% ---
% compila o indice
% ---
\makeindex
% ---

\DeclareRobustCommand{\firstsecond}[2]{#1}

% ----
% Início do documento
% ----
\begin{document}

% Seleciona o idioma do documento (conforme pacotes do babel)
%\selectlanguage{english}
\selectlanguage{brazil}

% Retira espaço extra obsoleto entre as frases.
\frenchspacing

% Espaçamento 1.5 entre linhas
\OnehalfSpacing

% Corrige justificação
%\sloppy

% ----------------------------------------------------------
% ELEMENTOS PRÉ-TEXTUAIS
% ----------------------------------------------------------
% \pretextual %a macro \pretextual é acionado automaticamente no início de \begin{document}
% ---
% Capa, folha de rosto, ficha bibliografica, errata, folha de apróvação
% Dedicatória, agradecimentos, epígrafe, resumos, listas
% ---
% ---
% Capa
% ---
\imprimircapa
% ---

% ---
% Folha de rosto
% (o * indica que haverá a ficha bibliográfica)
% ---
\imprimirfolhaderosto*
% ---

% ---
% Inserir a ficha bibliografica
% ---
% http://ficha.bu.ufsc.br/
\begin{fichacatalografica}
	\includepdf{Ficha_Catalografica.pdf}
\end{fichacatalografica}
% ---

% ---
% Inserir folha de aprovação
% ---
\begin{folhadeaprovacao}
	\OnehalfSpacing
	\centering
	\imprimirautor\\%
	\vspace*{10pt}		
	\textbf{\imprimirtitulo}%
	\ifnotempty{\imprimirsubtitulo}{:~\imprimirsubtitulo}\\%
	%		\vspace*{31.5pt}%3\baselineskip
	\vspace*{\baselineskip}
	%\begin{minipage}{\textwidth}
	O presente trabalho em nível de \imprimirnivel~foi avaliado e aprovado por banca examinadora composta pelos seguintes membros:\\
	%\end{minipage}%
	\vspace*{\baselineskip}
    Prof. Everton da Silva, Dr.\\
  Universidade Federal de Santa Catarina - UFSC\\
  \vspace*{\baselineskip}
    Prof. Examinador 2, Dr.\\
  Fédération Internationale des Géomètres - FIG\\
  \vspace*{\baselineskip}
    
	\vspace*{2\baselineskip}
	\begin{minipage}{\textwidth}
		Certificamos que esta é a \textbf{versão original e final} do trabalho de conclusão que foi julgado adequado para obtenção do título de \imprimirformacao.\\
	\end{minipage}
	%    \vspace{-0.7cm}
	\vspace*{\fill}
	\assinatura{\OnehalfSpacing Ana Maria Bencciveni Franzoni \\ Coordenação do Programa de Pós-Graduação}
	\vspace*{\fill}
	\assinatura{\OnehalfSpacing\imprimirorientador \\ \imprimirorientadorRotulo}
	%	\ifnotempty{\imprimircoorientador}{
	%	\assinatura{\imprimircoorientador \\ \imprimircoorientadorRotulo \\
	%		\imprimirinstituicao~--~\imprimirinstituicaosigla}
	%	}
	% \newpage
	\vspace*{\fill}
	\imprimirlocal, \imprimirano.
\end{folhadeaprovacao}
% ---

% ---
% Dedicatória
% ---
\begin{dedicatoria}
	\vspace*{\fill}
	\noindent
	\begin{adjustwidth*}{}{5.5cm} 
		\raggedleft       
		À Bibi.
	\end{adjustwidth*}
\end{dedicatoria}
% ---

% ---
% Agradecimentos
% ---
\begin{agradecimentos}
	Gostaria de agradecer sinceramente a todos os que colaboraram à
execução\\
deste trabalho.\\
Aos colegas da UFSC.\\
Aos professores do PPGTG.\\
Em especial ao meu orientador, pela paciência.\\
E a minha querida esposa pela compreensão.
\end{agradecimentos}
% ---

% ---
% Epígrafe
% ---
\begin{epigrafe}
	\vspace*{\fill}
	\begin{flushright}
		\textit{``Eppur si muove!''\\
(Galileu Galilei, 1633)}
	\end{flushright}
\end{epigrafe}
% ---

% ---
% RESUMOS
% ---

% resumo em português
\setlength{\absparsep}{18pt} % ajusta o espaçamento dos parágrafos do resumo
\begin{resumo}
	\SingleSpacing
  No resumo são ressaltados o objetivo da pesquisa, o método utilizado,
  as discussões e os resultados com destaque apenas para os pontos
  principais. O resumo deve ser significativo, composto de uma sequência
  de frases concisas, afirmativas, e não de uma enumeração de tópicos.
  Não deve conter citações. Deve usar o verbo na voz ativa e na
  terceira pessoa do singular. O texto do resumo deve ser digitado, em um
  único bloco, sem espaço de parágrafo. O espaçamento entre linhas é
  simples e o tamanho da fonte é 12. Abaixo do resumo, informar as
  palavras-chave (palavras ou expressões significativas retiradas do
  texto) ou, termos retirados de thesaurus da área. Deve conter de 150 a
  500 palavras. O resumo é elaborado de acordo com a NBR 6028. 
  
  \textbf{Palavras-chave}: 
    Mercado Imobiliário.
    Macroeconomia.
    Microeconomia.
  \end{resumo}
% resumo em inglês
\begin{resumo}[Abstract]
	\SingleSpacing
	\begin{otherlanguage*}{english}
		Resumo traduzido para outros idiomas, neste caso, inglês. Segue o
formato do resumo feito na língua vernácula. As palavras-chave
traduzidas, versão em língua estrangeira, são colocadas abaixo do
texto precedidas pela expressão `Keywords', separadas por ponto.
		
		\textbf{Keywords}:
	      Real Estate.
        Macroeconomics.
        Microeconomics.
    	\end{otherlanguage*}
\end{resumo}
%% resumo em francês 
%\begin{resumo}[Résumé]
% \begin{otherlanguage*}{french}
%    Il s'agit d'un résumé en français.
% 
%   \textbf{Mots-clés}: latex. abntex. publication de textes.
% \end{otherlanguage*}
%\end{resumo}
%
%% resumo em espanhol
%\begin{resumo}[Resumen]
% \begin{otherlanguage*}{spanish}
%   Este es el resumen en español.
%  
%   \textbf{Palabras clave}: latex. abntex. publicación de textos.
% \end{otherlanguage*}
%\end{resumo}
%% ---

{%hidelinks
	\hypersetup{hidelinks}
	% ---
	% inserir lista de ilustrações
	% ---
	\pdfbookmark[0]{\listfigurename}{lof}
	\listoffigures*
	\cleardoublepage
	% ---
	
	% ---
	% inserir lista de quadros
	% ---
	\pdfbookmark[0]{\listofquadrosname}{loq}
	\listofquadros*
	\cleardoublepage
	% ---
	
	% ---
	% inserir lista de tabelas
	% ---
	\pdfbookmark[0]{\listtablename}{lot}
	\listoftables*
	\cleardoublepage
	% ---
	
	% ---
	% inserir lista de abreviaturas e siglas (devem ser declarados no preambulo)
	% ---
	\imprimirlistadesiglas
	% ---
	
	% ---
	% inserir lista de símbolos (devem ser declarados no preambulo)
	% ---
	\imprimirlistadesimbolos
	% ---
	
	% ---
	% inserir o sumario
	% ---
	\pdfbookmark[0]{\contentsname}{toc}
	\tableofcontents*
	\cleardoublepage
	
}%hidelinks
% ---

% ---

% ----------------------------------------------------------
% ELEMENTOS TEXTUAIS
% ----------------------------------------------------------
\textual

\hypertarget{intro}{\chapter{Introdução}\label{intro}}

\section{O conceito de terra e a importância do
território}\label{o-conceito-de-terra-e-a-importuxe2ncia-do-territuxf3rio}

De acordo com \textcite{realestate}, a terra é a base das atividades
econômicas e sociais de um povo, essencial para a vida e a sociedade,
sendo assunto de diversas disciplinas, como o Direito, Economia,
Finanças, Sociologia e a Geografia.

No Direito, a terra é abordada no direito de propriedade e uso social da
terra. Na Economia, a terra é considerada um dos fatores de produção, ao
lado do trabalho e do capital. Nas Finanças, a terra é considerada como
um bem suscetível de ser dado em garantia, em troca de capital
financeiro, visando propiciar o desenvolvimento. Na perspectiva da
Sociologia, a terra é um bem comum de todos, que deve ser utilizado com
fins de obtenção de uma sociedade melhor. Já a Geografia foca em
descrever os elementos físicos da terra e as atividades humanas das
pessoas que as habitam.

A Engenharia de Avaliações se preocupa em reconhecer os atributos que
atuam na formação de valor dos bens imóveis, um conceito ligado à
Economia. Para isto, os Avaliadores devem entender o mercado no qual
estes bens estão inseridos. É o mercado que reflete a atitude dos atores
econômicos em resposta às forças sociais e econômicas e às restrições da
lei e dos ônus legais \autocite[10]{realestate}.

\section{O Mercado Imobiliário}\label{o-mercado-imobiliuxe1rio}

O mercado imobiliário urbano pode ser dividido basicamente em:
\begin{enumerate}
\def\labelenumi{\arabic{enumi}.}
\tightlist
\item
  Mercado de imóveis residencias e;
\item
  Mercado de imóveis comerciais.
\end{enumerate}
O Mercado Imobiliário se conecta à macroeconomia através:
\begin{enumerate}
\def\labelenumi{\arabic{enumi}.}
\tightlist
\item
  Do setor de construção civil;
\item
  Da demanda agregada, já que existe uma conexão entre a propensão
  marginal a consumir com o efeito riqueza do Mercado Imobiliário;
\item
  Do setor bancário, haja vista que os bancos possuem imóveis em seu
  portfolio, seja através da propriedade, ou através do financiamento
  imobiliário e hipotecas.
\end{enumerate}
\section{Objetivos}\label{objetivos}

\subsection{Objetivo Geral}\label{objetivo-geral}

Propor, através do estudo das diversas estruturas de mercado conhecidas
e elencadas na nova NBR 14.653-01 \autocite{NBR1465301} e do estudo das
relações entre as diversas variáveis macroeconômicas, novas aplicações
do método involutivo e, de acordo com os resultados, propor novas
políticas para a regulação do setor.

\subsection{Objetivos Específicos}\label{objetivos-especuxedficos}
\begin{enumerate}
\def\labelenumi{\arabic{enumi}.}
\tightlist
\item
  Compreender e descrever o funcionamento das diversas estruturas de
  mercado.
\item
  Abstrair das diferentes estruturas de mercado as relações entre as
  variáveis macroeconômicas e o seu impacto no mercado imobiliário.
\item
  De acordo com os resultados obtidos, propor políticas públicas que
  visem uma melhor regulamentação do mercado imobiliário, de maneira que
  este atenda melhor aos anseios da população.
\end{enumerate}
\section{Justificativa}\label{justificativa}

O território é a base das atividades econômicas e sociais do país. Um
eficiente planejamento territorial passa por desenvolver políticas
públicas que garantam um desenvolvimento equânime de todo o território.
A proposta desta pesquisa é aprofundar o conhecimento referente à
dinâmica cadastro territorial multifinalitário, ao mercado imobiliário e
às políticas de solo. Os resultados das pesquisas, além de serem
divulgados no meio acadêmico, embasam ações de extensão que proporcionam
a implementação de políticas de solo e a capacitação de pessoas que
atuam com atividades relacionadas ao território. O grupo é formado
majoritariamente por professores e estudantes do Programa de
Pós-Graduação em Engenharia de Transportes e Gestão Territorial e dos
departamentos de Geociências e Engenharia Civil da UFSC, mas participam
também professores, pesquisadores e estudantes de outros programas e
instituições nacionais e internacionais. Forma parte da Rede Acadêmica
de Cadastro Multifinalitário e atua em conjunto com o Centro de Estudios
Territoriales da Universidad Nacional de Córdoba, Argentina, no projeto
intitulado Valores del Suelo en América Latina.

\section{Estrutura do trabalho}\label{estrutura-do-trabalho}

O Capítulo \ref{intro} (\protect\hyperlink{intro}{Introdução}) apresenta
os objetivos, justificativas e estrutura do trabalho. O Capítulo
\ref{historico} (\protect\hyperlink{historico}{Aspectos Históricos}) faz
uma contextualização histórica do problema do acesso à terra e moradia
no Brasil e no mundo. O Capítulo \ref{economia}
(\protect\hyperlink{economia}{O Mercado Imobiliário e a Economia})
aborda os aspectos teóricos e conceituais mais modernos relevantes à
interligação do mercado imobiliário à Economia do país. O capítulo
\ref{involutivo} (\protect\hyperlink{involutivo}{O Método Involutivo})
apresenta o método involutivo de avaliação de imóveis. O Capítulo
\ref{copulas} (\protect\hyperlink{copulas}{O Método Copulas}) introduz o
método Copulas, essencial para a compreender as simulações para cada
cenário/estrutura de mercado descritas no Capítulo \ref{metodologia}
(\protect\hyperlink{metodologia}{Metodologia}). O Capítulo
\ref{resultados} (\protect\hyperlink{resultados}{Resultados}) demonstra
os resultados obtidos com as simulações e o Capítulo \ref{conclusao}
(\protect\hyperlink{conclusao}{Conclusão}) traz as propostas de
regulação baseadas nos resultados obtidos.

\hypertarget{historico}{\chapter{Aspectos históricos}\label{historico}}
\begin{epigrafe}
    \vspace*{\fill}
    \begin{flushright}
        \textit{``Do ponto de vista social, todos os fatores se resumem\\ 
        em um `recurso' elementar: o homem. Logo, não é possível seguir\\ 
        conceptualmente o processo de industrialização se não sabemos como\\ 
        o homem aplicava antes o seu tempo de trabalho, como o aplica depois,\\ 
        o que ocorre quando passa de um modo de produzir a outra e em que\\ 
        condições realiza essa passagem.[ \ldots ] Considerando que na estrutura\\ 
        da economia que precede a industrialização quase toda a população está\\ 
        na `agricultura', é preciso estudar detidamente a organização deste setor.\\ 
        Em outras palavras, se o problema da `agricultura' não foi entendido,\\ 
        tampouco será possível compreender o problema da `indústria', ou manufatura,\\ 
        nem o papel que os serviços desempenham. Falando de modo sucinto, a \\
        `manufatura' e os serviços são novas formas de aplicação de parte do\\ 
        tempo de trabalho da população que antes estava na `agricultura'. Mas,\\
        por sua vez, a própria `agricultura' se reorganiza quando a transferência ocorre.''\\
        (RANGEL, 1954)}
    \end{flushright}
\end{epigrafe}
O Capitalismo é um sistema político-econômico que, historicamente,
substitui o Feudalismo, sistema em que a população encontrava-se toda
concentrada no campo.

Nas sociedades pré-capitalistas, a população predominante rural
organizava-se no chamado `Complexo Rural', ou seja, num ambiente rural
onde eram produzidos não apenas os produtos agrícolas, mas onde também
eram produzidos, pelos próprios camponeses, em uma muito baixa
produtividade, todo o ferramental necessário para as suas atividades
agrícolas, assim como suas vestes, utensílios domésticos e outros itens.

A passagem do sistema feudal para o sistema capitalista ocorre com a
\emph{divisão social do trabalho}, ou seja, com o desenvolvimento de
indústrias que vão aos poucos absorver as atividades não-agrícolas
realizadas no campo.
\begin{citacao}
Numa economia em expansão, com crescente industrialização, comercialização e
urbanização, numerosos processos anteriormente levados a efeito antes dentro da
casa da família ou unidade (econômica) familiar, ou são completamente
abandonados ou substituídos por processos semelhantes em bases
comerciais. \cite[p. 41]{kuznets} \textit{apud} \cite[p. 218]{rangel1956}.
\end{citacao}
O desenvolvimento do capitalismo brasileiro no século XX se deu pela
chamada ``via prussiana'' ou \emph{junker} \autocite[155]{rangel1988},
que é um tipo de reforma agrária que consiste na substituição do
latifúndio feudal pelo latifúndio capitalista. Este tipo de
desenvolvimento tem como característica se dar sem a execução prévia da
reforma agrária no sentido clássico, \emph{i.e.} no sentido da
distribuição dos latifúndios em pequenas propriedades, a chamada via
clássica ou democrática. Apesar de permitir imprimir um ``impulso
extraordinário e energético'' à industrialização, a via prussiana
``promove uma distribuição muito desigual da renda''
\autocite[155]{rangel1988}. A característica talvez mais perniciosa do
desenvolvimento capitalista por esta via se dá pela formação de um
``exército industrial de reserva'' demasiado grande, ou seja, um aumento
da população urbana desproporcional à necessidade de mão-de-obra
necessária nas indústrias do capitalismo nascente nas cidades. O
resultado é o crescimento acelerado e desordenado das cidades, com a
inevitável formação dos cortiços e favelas para acomodar a parte mais
carente da população que, expulsa do campo, vai se aglomerar nos grandes
centros urbanos em busca da sua sobrevivência.

Dados compilados pelas Nacões Unidas foram organizados na tabela
\ref{tab:pop-table} com o intuito de demonstrar a evolução e o atual
tamanho deste problema.
\begin{table}[H]

\caption{\label{tab:pop-table}População Urbana (\%).}
\centering
\begin{tabular}[t]{lrrrrrr}
\toprule
\multicolumn{1}{c}{} & \multicolumn{6}{c}{Ano} \\
\cmidrule(l{3pt}r{3pt}){2-7}
Entity & 1960 & 1970 & 1980 & 1990 & 2000 & 2014\\
\midrule
\rowcolor{gray!6}  \addlinespace[0.3em]
\multicolumn{7}{l}{\textbf{Mundo}}\\
\hspace{1em}World & 33,8 & 36,6 & 39,3 & 43,0 & 46,7 & 53,5\\
\hspace{1em}More developed regions & 61,1 & 66,8 & 70,3 & 72,4 & 74,2 & 78,0\\
\hspace{1em}Less developed regions & 21,9 & 25,3 & 29,4 & 34,9 & 40,1 & 48,4\\
\rowcolor{gray!6}  \addlinespace[0.3em]
\multicolumn{7}{l}{\textbf{Europa}}\\
\hspace{1em}Europe & 57,4 & 63,1 & 67,6 & 69,9 & 71,1 & 73,7\\
\hspace{1em}Eastern Europe & 48,9 & 56,6 & 63,8 & 68,0 & 68,2 & 69,2\\
\hspace{1em}Western Europe & 68,6 & 72,1 & 73,4 & 74,0 & 76,0 & 79,2\\
\rowcolor{gray!6}  \addlinespace[0.3em]
\multicolumn{7}{l}{\textbf{América}}\\
\hspace{1em}Latin America and the Caribbean & 49,4 & 57,3 & 64,6 & 70,7 & 75,5 & 79,7\\
\hspace{1em}South America & 51,8 & 60,0 & 67,6 & 74,2 & 79,6 & 83,3\\
\hspace{1em}Central America & 46,4 & 53,7 & 60,3 & 65,0 & 68,7 & 73,4\\
\hspace{1em}Northern America & 69,9 & 73,8 & 73,9 & 75,4 & 79,1 & 81,5\\
\hspace{1em}United States & 70,0 & 73,6 & 73,7 & 75,3 & 79,1 & 81,5\\
\hspace{1em}Argentina & 73,6 & 78,9 & 82,9 & 87,0 & 89,1 & 91,4\\
\hspace{1em}Brazil & 46,1 & 55,9 & 65,5 & 73,9 & 81,2 & 85,5\\
\bottomrule
\end{tabular}
\end{table}
\bcenter
Fonte: \textcite{doi:10.1177/0959683609356587} \ecenter

Ainda, para melhor ilustrar, foram elaborados os gráficos das figuras
\ref{fig:pop-urb-mundo} a \ref{fig:pop-urb-brazil-brics}.

Na figura \ref{fig:pop-urb-mundo}, pode-se notar que a população urbana
no Brasil vem aumentando, desde 1950, numa taxa superior à média dos
países em desenvolvimento (\emph{Less developed regions}), atingindo uma
proporção superior inclusive à dos países mais desenvolvidos (\emph{More
developed regions}).
\begin{figure}[H]

{\centering \includegraphics[width=0.8\linewidth]{images/pop-urb-mundo-1} 

}

\caption{População Urbana - Brasil vs. Mundo.}\label{fig:pop-urb-mundo}
\end{figure}
\bcenter
Fonte: \textcite{doi:10.1177/0959683609356587} \ecenter

Na figura \ref{fig:pop-urb-continents} pode-se ver as séries da
população urbana em diversos continentes desde 1800. Percebe-se neste
gráfico também uma maior aceleração do crescimento da população urbana
na América Latina e Caribe a partir de meados da década de 40, chegando
esta região a ultrapassar a população urbana da Europa Ocidental no
início do século corrente.
\begin{figure}[H]

{\centering \includegraphics[width=0.8\linewidth]{images/pop-urb-continents-1} 

}

\caption{População Urbana - Nos diferentes Continentes.}\label{fig:pop-urb-continents}
\end{figure}
\bcenter
Fonte: \textcite{doi:10.1177/0959683609356587} \ecenter

A figura \ref{fig:pop-urb-brazil-latinAmerica} mostra a evolução da
população urbana no Brasil em comparação com o continente sul-americano
e a América Latina, dando destaque para alguns vizinhos.
\begin{figure}[H]

{\centering \includegraphics[width=0.8\linewidth]{images/pop-urb-brazil-latinAmerica-1} 

}

\caption{População Urbana - Brasil vs. AL.}\label{fig:pop-urb-brazil-latinAmerica}
\end{figure}
\bcenter
Fonte: \textcite{doi:10.1177/0959683609356587} \ecenter

A figura \ref{fig:pop-urb-brazil-developed} mostra o comparativo da
população urbana no Brasil com uma seleção de países desenvolvidos desde
1800. Quanto aos países desenvolvidos, nota-se que tiveram,
primeiramente, uma ascenção um pouco mais lenta da população urbana
(excessão para a Grã-Bretanha, primeira nação a industrializar-se), que
essa ascenção teve lugar já na década de 1850 e que houve uma
estabilização gradual, por volta da década de 1970. Já quanto ao Brasil
nota-se uma grande aceleração no crescimento da população urbana
brasileira após a década de 1950, o que reflete o nascimento tardio do
capitalismo por aqui e, por fim, que, ao contrário dos países
desenvolvidos, não houve ainda uma estabilização da proporção de
população urbana e esta segue em crescimento, tendo chegado a níveis
maiores aqui do que no resto do mundo desenvolvido.
\begin{figure}[H]

{\centering \includegraphics[width=0.8\linewidth]{images/pop-urb-brazil-developed-1} 

}

\caption{População Urbana - Brasil vs. Países Desenvolvidos.}\label{fig:pop-urb-brazil-developed}
\end{figure}
\bcenter
Fonte: \textcite{doi:10.1177/0959683609356587} \ecenter

E a figura \ref{fig:pop-urb-brazil-brics} mostra a comparação dos dados
do Brasil com os outros países do grupo dos BRICS.
\begin{figure}[H]

{\centering \includegraphics[width=0.8\linewidth]{images/pop-urb-brazil-brics-1} 

}

\caption{População Urbana - Brasil vs. BRICS.}\label{fig:pop-urb-brazil-brics}
\end{figure}
\bcenter
Fonte: \textcite{doi:10.1177/0959683609356587} \ecenter

Em meados dos anos 60, apenas 46,1\% da população brasileira era urbana,
uma proporção bem menor do que a dos países do então \emph{primeiro
mundo} (EUA e Europa Ocidental), hoje ditos \emph{desenvolvidos}, que
girava então em torno dos 70\% da população.

Em apenas 10 anos, já em meados da década de 70, este número sofria um
aumento vertiginoso de quase 10 pontos percentuais, com 55,9\% da
população urbana. A população urbana brasileira equiparava-se à da
Europa Oriental. Já na década de 80 a população urbana no Brasil
ultrapassaria a da Europa Oriental, chegando à valores próximos da média
para o continente europeu como um todo (ocidental e oriental), enquanto
a população urbana no mundo desenvolvido se estagnava.

Chegado os anos 90, a população urbana brasileira atingiu notáveis
73,9\% da população brasileira, número equiparado ao da população urbana
do mundo desenvolvido (74\% na Europa Ocidental).

Em meados dos anos 2000, já então no século atual, ousamos ultrapassar,
em proporção, a população urbana da Europa Ocidental e a dos EUA,
chegando ao último dado de 2015, com 85,8\% da população brasileira
vivendo nas cidades.

Há de se levar em consideração, ainda, que este ``êxodo rural'' ainda
foi acompanhado de um crescimento demográfico expressivo.

Todo este crescimento expressivo seria salutar se tivesse se dado no
contexto do rápido desenvolvimento da economia nacional. Isto, porém,
não ocorreu durante todo o período analisado. O crescimento da economia
brasileira acelerou-se na segunda quadra da década de 60 e manteve-se
alto até fins da década seguinte, porém estagnou-se na década de 80, a
chamada década perdida, sem que com isso a população urbana deixasse de
crescer vertiginosamente.

Para Rangel \autocite*[151]{rangel1986a}:
\begin{citacao}
"essa redistribuição da população entre os quadros urbano e rural não tem, em si
mesma, nada de anormal.[...] A urbanização, em si mesma, é um fenômeno
perfeitamente normal, numa economia em processo de industrialização. O que não é
normal é o ritmo que imprimimos ao \emph{nosso} processo de urbanização, que
implica criar, nas cidades, uma oferta de mão-de-obra em descompasso com a
demanda que a industrialização vai criando."
\end{citacao}
Todo este processo só poderia, então, ter desaguado no inchaço das
principais cidades brasileiras. Desnecessário dizer que o planejamento
urbano nestas condições é praticamente inviável. As administrações
municipais, nem que fossem as mais eficientes, teriam capacidade de
planejar e disciplinar o uso do solo urbano nesta ``velocidade
migratória''.

Segundo Rangel, com o desenvolvimento da indústria pesada no Brasil, a
crise agrária, antes cíclica, tornou-se crônica, criando um
\autocite*[156-157]{rangel1988}:
\begin{citacao}
"vasto deslocamento de população, na direção geral campo-cidade. Esse movimento
se faz escalonadamente, das áreas rurais para as cidades pequenas; destas para
as médias e grandes, e posteriormente para as metrópoles gigantes. No fim da
linha, portanto, vamos encontrar as cidades de São Paulo e do Rio de Janeiro".
\end{citacao}
Enfim, para Rangel, a origem deste ``multitudinário deslocamento
demográfico'', está ``o modo como o país preparou sua estrutura agrária
para a industrialização''.

\section{A questão agrária}\label{a-questuxe3o-agruxe1ria}

Segundo Rangel, a Questão Agrária, embora nascida na área rural, é um
fenômeno urbano. Com isto Rangel quer dizer que a crise agrária, a crise
que se dá na transição do feudalismo para o capitalismo, começa no
campo, onde se passa o enredo do feudalismo, para a cidade moderna, onde
se desenvolve o capitalismo.

Para uma melhor compreensão da questão se faz mister compreender os
mecanismos de funcionamento dos sistemas citados, isto é, do feudalismo
e do capitalismo, especialmente no que tange a transição entre eles, nos
motivos que levam ao fim de um sistema e desembocam quase que
inequivocamente (excetos raras exceções) no outro.

\subsection{Feudalismo}\label{feudalismo}

As ``leis'', ou princípios, ou ainda os ``motores primários'' do
feudalismo são \autocite[126]{rangel1985}:
\begin{itemize}
\tightlist
\item
  \emph{All land is king's land}
\item
  \emph{Nulle terre sans seigneur}
\end{itemize}
O primeiro princípio, \emph{all land is king's land}, ou ``toda a terra
pertence ao rei'', quer dizer, mais precisamente, que todo o domínio da
terra está concentrada nas mãos do rei, que as explora através dos laços
de suserania e vassalagem, típicos do feudalismo. Já o segundo
princípio, segundo Rangel \autocite*[219]{rangel1961}, \emph{nulle terre
sans seigneur}, quer dizer que ``a existência de terra livre é
incompatível com o feudalismo'', ou seja, toda a terra deve ter um
senhor, que a administra a serviço da Coroa e lhe paga tributo. Na
existência de terra livre, como será visto, o feudalismo não se pode
desenvolver, e a tendência é que haja ou a formação de comunidades em
estado tribal, ou que sejam estabelecidas formas de escravidão. Ou seja,
a terra, ``nas condições feudais, não tem preço e é, de fato ou de
direito, inalienável'' \autocite[206]{rangel1960}.

\subsubsection{O feudalismo no Brasil}\label{o-feudalismo-no-brasil}

Segundo Rangel \autocite*[206]{rangel1956}, a atitude do economista do
país subdesenvolvido não pode ser a mesma do economista dos países mais
desenvolvidos, que, \emph{tendo vivido o processo histórico completo,
assistiram simultaneamente à morte do ser antigo e à sua representação}.
\begin{citacao} 
a absorção sem crítica do \emph{dernier cri} em matéria de ciência econômica por
ele lhe será fatal, porque implica mudar o reflexo ideal da realidade sem que
essa realidade mesma tenha mudado, ou sem que tenha mudado senão em parte. Para
nós, o pensamento dos antigos guarda muito de sua primitiva validade porque
reflete uma realidade que, em certa medida, continua a ser a nossa
\cite[p.~206-207]{rangel1956}.
\end{citacao}
Em outras palavras, para \textcite{rangel1956}, os economistas dos
países subdesenvolvidos, ou mais modernamente, países ``em
desenvolvimento'', devem utilizar em sua análise as teorias clássicas,
neoclássicas, keynesianas, à medida que subsistem nestes países
características próprias da realidade econômica que imperavam no Velho
Mundo quando elas foram concebidas.

Desta maneira, o feudalismo tal como concebido na Europa não teve a
mesma estrutura que o feudalismo no Brasil, assim como o sistema feudal
brasileiro foi não-concomitante com o sistema feudal europeu.

Segundo Rangel \autocite*[726]{rangel1989}, através da bula papal de
Alexandre IV, de 4 de maio de 1493 (ainda que tenha sido depois alterada
pelo tratado de Tordesilhas), toda a terra onde hoje encontra-se a
América Latina era declarada propriedade do rei. Isto é, estava
satisfeito o primeiro princípio para a implantação do feudalismo nos
trópicos: \emph{all land is king's land}. A propriedade sobre as terras
era total, de maneira que podesse dizer que, juridicamente, em nenhum
momento a propriedade fundiária esteve mais concentrada do que naquele
primeiro momento.

O segundo princípio, no entanto, \emph{nulle terre sans seigneur}, ou
seja, o princípio de que não deve haver terra sem senhor, também
indispensável para a existência do feudalismo -- no surgimento do
feudalismo na Europa, sem que todas as terras social e economicamente
significativas estivessem apropriadas, a tendência natural do escravo
liberto seria o retorno às condições de vida tribal -- não era possível
em território tão vasto e inexplorado como era o território
latino-americano naquele momento \autocite[726]{rangel1989}.

Desta maneira, os feudos que aqui se iam estabelecendo, através do
instituto da enfiteuse \autocite[726]{rangel1989}, os pactos de
suserania-vassalagem que iam do servo do gleba ao rei, passando por
diversos patamares, muito diferiam dos feudos europeus da Alta Idade
Média, que ao contrário dos pactos aqui estabelecidos, começavam a ser
constituídos pela base, convertendo os escravos libertos em servos e
constituindo a pequena e a grande nobreza, ``tendendo afinal a, com o
tempo, colocar no píncaro o rei'' \autocite[727]{rangel1989}.

A esse respeito também escreveu Alceu Amoroso Lima
\autocite*[51]{amoroso}, na grande obra organizada por Vicente Licínio
Cardoso:
\begin{citacao}
Foi-se vendo pouco a pouco – e até hoje o vemos ainda com surpresa, por vezes –
que o Brasil se formara às avessas, começara pelo fim. Tivera Coroa antes de ter
Povo. Tivera parlamentarismo antes de ter eleições. Tivera escolas superiores
antes de ter alfabetismo. Tivera bancos antes de ter economias. Tivera salões
antes de ter educação popular. Tivera artistas antes de ter arte. Tivera
conceito exterior antes de ter consciência interna. Fizera empréstimos antes de
ter riqueza consolidada. Aspirara a potência mundial antes de ter a paz e a
força interior. Começara em quase tudo pelo fim. Fora uma obra de inversão.
\end{citacao}
Segundo Rangel \autocite*[729]{rangel1989}, as condições em que operavam
os nossos feudos mais se assemelhavam às vigentes na República Romana e
nos primeiros tempos do Império, o que quer dizer que, aqui,
internamente, até que o monopólio da terra estivesse garantido, somente
haveria viabilidade para o sistema escravagista.

Com efeito, é sabido que foi necessário importar o escravo africano, que
era socialmente mais avançado que os índios que aqui habitavam,
fazendo-o prisioneiro do latifúndio, haja vista que o índio estava
habituado a prover o seu sustento de forma natural nas terras que aqui
habitavam.

A Coroa portuguesa \autocite[731]{rangel1989}:
\begin{citacao}
não tinha pressa em dispor de todas as suas terras, mas apenas das suficientes
para implantar fazendas e estâncias, deixando aberta a porta para novas doações,
que comprassem novas vassalagens, aumentando o poder, a riqueza e a glória da
Coroa.
\end{citacao}
Assim, sobravam terras entre uma fazenda e outra, o que impossibilitava
o modo de produção feudal (pela não satisfação do princípio \emph{nulle
terre sans seigneur}), mas apenas o modo de produção escravista. Exceto
por algumas regiões do Brasil onde a pecuária extensiva logrou ocupar
uma vasta extensão contínua de terra, como no Rio Grande do Sul, o
feudalismo só viria a se estabelecer muito tempo depois, com a abolição
da escravidão (1888) e a Proclamação da República (1889)
\autocite[732-733]{rangel1989}.

Porém, para que fosse possível o fim da escravidão sem que houvesse
retorno às formas primitivas de produção pré-escravagistas, foi
necessário um longo processo que teve início com a Lei do Tráfico e a
Lei de Terras, ambas de 1850 \autocite[732]{rangel1989}.

Enquanto a Lei do Tráfico levaria inevitavelmente ao fim da escravidão
em algum ponto futuro, já que a ``lei demográfica peculiar ao escravismo
é a reprodução restrita, o que supõe aportes constantes de mão de obra
alienígena'', a Lei de Terras preparava o território para o novo regime
que teria lugar, o feudalismo, através da promoção da efetiva ocupação
do território, ou seja, de todas as terras acessíves, habitáveis e
agricultáveis \autocite[732-733]{rangel1989}.

Vale dizer que, onde a condição \emph{nulle terre sans seigneur} não
logrou após a abolição da escravidão, como no estado do Maranhão, houve
retrocesso a relações de produção pré-escravistas
\autocite[733-734]{rangel1989}.

\subsection{A crise do feudalismo}\label{a-crise-do-feudalismo}

A crise clássica da sociedade feudal ocorre quando a produção agrícola
não consegue suprir a demanda da superpopulação gerada. Segundo Rangel
\autocite*[219]{rangel1961}:
\begin{citacao} 
"tempo houve em que a expansão do estoque populacional era objetivamente a
maneira mais eficaz de expandir as forças produtivas e o produto social. Nesse
tempo (regime feudal), a riqueza dos príncipes se media pelas almas dos seus
domínios, e aumentar o número destas era a maneira óbvia de expandir aquela
riqueza e também a do corpo social. Este foi forjando para si uma ética, um
direito e uma política conducentes a esse resultado".
\end{citacao}
A crise do feudalismo, sistema eminentemente agrário, e o consequente
surgimento do capitalismo, com o surgimento das cidades modernas, se dá
no contexto da dissolução do Complexo Rural, o que descreve-se nas
seções a seguir.

\subsection{A crise do feudalismo no
Brasil}\label{a-crise-do-feudalismo-no-brasil}

O feudalismo no Brasil desenvolveu-se a partir da Abolição-República,
concomitantemente com a implantação, especialmente no quadro urbano, de
uma vigorosa economia capitalista. No campo, ao lado do velho latifúndio
feudal, logo surgiu outro latifúndio que, em vez de distribuir lotes
entre os agregados -- como seria natural na desintegração do feudalismo
clássico -- empreendeu, ele próprio, a atividade agrícola, usando
mão-de-obra assalariada \autocite[ 738-739]{rangel1989}.

Segundo Rangel \autocite*[739]{rangel1989}, o latifúndio feudal, então,
percebendo-se que havia tendência de seus agregados deixarem de lado o
trabalho nos lotes que haviam recebido no processo de abolição da
escravidão, para trabalhar nas novas fazendas capitalistas, logo comecou
a deslocar esses agregados, dando origem ao processo do êxodo rural.

\subsubsection{O Complexo Rural}\label{o-complexo-rural}

Segundo Rangel \autocite*[p.98]{rangel1956}, a unidade agrícola fechada
é
\begin{citacao}
um microcosmo econômico no qual as pessoas distribuem seu tempo entre numerosas
atividades. Cada uma dessas atividades representa o estado rudimentar daquilo
que, com o desenvolvimento, se tornará uma 'indústria' (...) É evidente que o
camponês não tem consciência da multiplicidade de suas atividades. Ele considera
que elas formam um todo indivisível. Essa inespecialização é sua especialidade.
(\ldots) Chamaremos esse microcosmo econômico, essa 'matriz de insumo-produto' em
miniatura, de 'complexo rural'.
\end{citacao}
O estudo do desenvolvimento do capitalismo não pode ser feito sem o
estudo das bases para o seu desenvolvimento. O capitalismo é um sistema
político-econômico que tem surgimento com a queda do feudalismo, outro
sistema político-econômico cujo enredo se passa, basicamente, no campo.
A classe burguesa, aliás, como diz a história, era formada inicialmente
pelos habitantes dos burgos, que se localizavam no entorno dos feudos.
Estes formavam, no entanto, uma minoria. Durante a idade média, a maior
parte da população vivia nos feudos, que se constituiam de grandes áreas
cercadas e isoladas umas das outras, com economia quase auto-suficiente.

Na economia feudal, portanto, não existia grande grau de especialização
das atividades econômicas, como há hoje. Devido à precariedade do
comércio, era praticamente imperioso que, no interior de cada feudo
todas as atividades econômicas fossem executadas para a própria
sustentabilidade do mesmo.

Segundo Lenin \autocite[\emph{apud}][99]{rangel1954}, mesmo após o
surgimento do capitalismo, nos países periféricos, esta realidade feudal
ou quase-feudal, deve ser levada em consideração:
\begin{citacao} 
A população de um país de economia mercantil debilmente desenvolvida (ou não 
desenvolvida de todo) é quase exclusivamente agrícola. Todavia, não se deve 
deduzir daí que ela se ocupa só da agricultura. Significa apenas que a população 
ocupada na agricultura transforma, ela mesma, os produtos da terra, sendo quase 
inexistentes o intercâmbio e a divisão do trabalho. 
\end{citacao}
\subsubsection{Condições e Métodos de abertura do Complexo
Rural}\label{condiuxe7uxf5es-e-muxe9todos-de-abertura-do-complexo-rural}

Para a abertura do Complexo Rural é necessário que haja vantajosidade
para a economia de mercado e para a economia natural do próprio
Complexo.
\begin{citacao} 
A Abertura do Complexo Rural não é uma operação momentânea, mas sim um largo
processo, com altos e baixos e problemas sempre novos. Sua história está muito
longe de ser idílica. Ao contrário, está cheia de violência. Uma planificação
econômica que não resolva preliminarmente este problema é inconcebível.
Alternadamente, pode conduzir à liberação de mais fatores que aqueles que os
setores não agrícolas podem usar, fazendo toda a economia submergir em uma crise
profunda, ou condenar esses setores à estagnação por insuficiência de
fatores \cite[p.~118]{rangel1954}
\end{citacao}
As medidas tendentes a romper o complexo rural podem ser classificadas
em dois grupos \autocite[113]{rangel1954}:
\begin{enumerate}
\def\labelenumi{\alph{enumi}.}
\item
  as que oferecem um incentivo positivo para a incorporação, à economia
  de mercado, dos fatores usados pelo complexo e;
\item
  as que buscam forçar a abertura do complexo a partir de dentro,
  provocando uma deterioração da produtividade das atividades
  manufatureiras dentro do complexo.
\end{enumerate}
As medidas do tipo a) tem seu exemplo mais típico nos EUA e também na
França, enquanto as medidas do tipo b) predominaram na Inglaterra,
Alemanha e Japão \autocite[114-115]{rangel1954}.

\subsection{Êxodo rural e
industrialização}\label{uxeaxodo-rural-e-industrializauxe7uxe3o}
\begin{citacao}
A revolução democrático-burguesa, nos casos em que a gleba feudal é -- como
aconteceu na Europa Ocidental (principalmente, na França) e nos Estados Unidos
-- substituída pela pequena propriedade familiar ou *homestead*, ao fortalecer
as bases da economia natural ou de autoconsumo, resolve satisfatoriamente o
problema na absorção dos excedentes de mão-de-obra no seio da própria economia
camponesa, estancando ou reduzindo drasticamente o fluxo populacional
responsável pelo êxodo campo-cidade \cite[p.~133]{rangel1986b}. 
\end{citacao}
Segundo Rangel \autocite*[133]{rangel1986b}, no entanto, ``esse tipo de
superação das relações de produção feudais'', ou seja, a revolução
democrático-burguesa, ``não é característico do Brasil. Sem embargo do
surgimento de algumas `ilhas' de pequena propriedade camponesa,
notadamente nas áreas de colonização européia e japonesa nos estados do
Sul, que mais confirmam a regra.''

Pelo contrário, ``o modelo de desenvolvimento do capitalismo na
agricultura brasileira foi e é a grande exploração capitalista, cada dia
mais propensa ao uso de mão-de-obra assalariada e tendendo sempre ao
desmantelamento das bases da economia natural, causando por isso mesmo,
o fenômeno do \textbf{êxodo rural}'' \autocite[134, grifo
nosso]{rangel1986b}.

O caso brasileiro, porém, não é único: ``a industrialização da
Inglaterra fez-se também, originariamente, nas condições de um enorme
excedente de mão de obra, causado pelo \emph{enclosure}\footnote{\emph{Enclosure}
  - literalmente, cercamento. Movimento pelo qual os pequenos
  agricultores ingleses foram expulsos de terras, convertidas estas à
  pecuária, e amontoados nos \emph{slums}, ou favelas das cidades
  industriais nascentes, na primeira metade do século XIX.} \ldots{}''.
No caso inglês, porém, ``o motor primário'' do desenvolvimento foi a
produção manufatureira para exportação, enquanto no Brasil a
industrialização teve seu desenvolvimento estimulado, ``nas condições de
uma crônica crise cambial'', pela política de substituição de
importações \autocite[43-44]{rangel1962}.

Ocorre que, de acordo com Rangel \autocite*[134]{rangel1986b}, ``a
superabundância e a barateza da mão-de-obra não costumam ser bons
condicionantes do processo de industrialização, dado que desestimulam a
formação de capital, isto é, o investimento. Ora, numa economia
capitalista, o investimento é o motor primário do desenvolvimento
\ldots{}''.

Por este motivo, a ``economia brasileira, nas condições de uma crise
agrária profunda e crônica que, entre outras coisas, \textbf{causava uma
urbanização monstruosa}, sem comparação possível com a demanda de
mão-de-obra que a indústria e os serviços não-agrícolas estavam
suscitando nas cidades (perto de três milhões de novos citadinos a cada
ano)\ldots{}'' \autocite[134]{rangel1986b}.

\subsubsection{O êxodo rural como obstáculo ao
desenvolvimento}\label{o-uxeaxodo-rural-como-obstuxe1culo-ao-desenvolvimento}
\begin{citacao}
Ordinariamentem a industrialização pressupõe certa escassez latente de 
mão-de-obra, levando o empresário capitalista a buscar, pelo emprego de bens
modernos de equipamento, economizar o fator trabalho. O resultado é a elevação 
da taxa de investimento, o aumento da procura de bens de capital e de novas
construções, para o que se torna mister empregar mais mão-de-obra 
\cite[p.~43]{rangel1962}.
\end{citacao}
No capitalismo, é conhecido o papel do investimento ou formação de
capital nas taxas de desemprego. Segundo Rangel
\autocite*[156]{rangel1988}, ``por um lado, via efeito multiplicador
(efeito para trás), o investimento cria emprego de mão-de-obra; por
outro lado, via implementação de nova tecnologia, promove dispensa de
mão-de-obra (efeito para frente)''.

Segundo Rangel \autocite[142]{rangel1986c}, um ``\,`exército industrial
de reserva' limitado, isto é, algum desemprego, pode ser considerado
útil, do ponto de vista da produção capitalista, porque serve de
instrumento de coerção para os trabalhadores livres, fortalecendo assim
a disciplina no trabalho''. No entanto, quando este torna-se excessivo,
``pode converter-se em obstáculo ao desenvolvimento da própria economia
capitalista. Ora, aqui está o nosso problema, dado que o `exército
industrial de reserva' brasileiro tornou-se teratologicamente grande.
Por isso mesmo, a questão agrária, que se exprime precipuamente pela
formação desse `exército', não interessa apenas aos camponeses, mas à
sociedade como um todo.''

De acordo com Rangel \autocite*[156]{rangel1988}, ``a via democrática --
divisão dos latifúndios em pequenas propriedades -- ao favorecer uma
distribuição menos desigualitária de renda, cria condições para um
vigoroso efeito multiplicador dos investimentos, isto é, forte efeito
para trás. Inversamente, a via prussiana, ao promover uma distribuição
de renda mais desigualitária, debilita o efeito multiplicador, isto é,
para trás, mas, por força da concentração de renda, aumenta o peso
relativo dos investimentos dispensando mão-de-obra e, por isso mesmo,
aumentando o efeito para diante.''

\subsection{Reforma agrária}\label{reforma-agruxe1ria}

Como foi visto, o assunto é complexo e requer uma análise de todo o
contexto econômico, social e político vigente. A tão sonhada ``reforma
agrária'' a que normalmente se refere a mídia, os movimentos sociais ou
a população em geral, deveria ter tido lugar ainda na década de 1930, ou
seja, em fase anterior ao início da industrialização brasileira. Para
Rangel \autocite[154]{rangel1986a}, a ``reforma agrária, no sentido
convencional da expressão, isto é, a implantação de propriedade familiar
suficientemente ampla, para permitir, ao lado da produção agrícola para
o mercado, uma produção complementar agrícola e não-agrícola, isto é,
para autoconsumo, pode justificar-se em certos casos, especialmente
quando seja possível o renascimento da policultura tradicional e onde a
fazenda capitalista, mono ou oligoculturista, ainda não tenha
aparecido''. No atual contexto econômico, porém, esta reforma, com a
dissolução do latifúndio produtivo para assentamento de famílias,
levaria a uma regressão tecnológica no campo, o que seria altamente
prejudicial para a economia brasileira como um todo.

Isto dito, Rangel então propõe então que, no Brasil, com a agricultura
capitalista plenamente desenvolvida, uma segunda variante de reforma
agrária, ``\emph{não necessariamente rural}'', com a finalidade de
``recompor a economia natural onde quer que isto seja possível'',
viabilizando ``uma produção complementar, deixando a produção agrícola
para o mercado a cargo da fazenda capitalista com mão-de-obra
assalariada'',seja implementada \autocite[155]{rangel1986a}.

\section{Considerações a respeito da valorização da terra no
Brasil}\label{considerauxe7uxf5es-a-respeito-da-valorizauxe7uxe3o-da-terra-no-brasil}

Para Rangel \autocite*[138-139]{rangel1986b}, o problema da terra é uma
questão financeira. Quer com isso dizer que, ultrapassados os problemas
jurídicos da nossa legislação pré-capitalista (Lei de Terras de 1850),
que dificultava a comercialização da terra, o problema do acesso à terra
resume-se ao problema da capacidade do camponês de comprá-la, o que
deveria ter se tornado possível devido ao esperado declínio do preço da
terra que viria com a expansão das fronteiras agrícolas, mas que não
ocorreu, devido à \emph{demanda especulativa}, o que é um \emph{problema
financeiro}.

\subsection{A tendência à elevação de preços nos períodos de recessão
econômica}\label{a-tenduxeancia-uxe0-elevauxe7uxe3o-de-preuxe7os-nos-peruxedodos-de-recessuxe3o-econuxf4mica}

Considerando-se que a \emph{demanda especulativa} aumenta nos períodos
de recessão, quando não há melhores oportunidades de investimento, a
tendência é que o preço da terra varie inversamente à \emph{eficácia
marginal do capital}.

Isto se explica pelo motivo que, em períodos de recessão econômica, a
atratividade dos investimentos na economia real diminui. Os capitais
sobrantes do período anterior de expansão/acumulação, então, na falta de
boas oportunidades de aplicação, torna-se `excessivo' e ocasionando a
diminuição da \emph{eficácia marginal do capital}, o que se reflete na
taxa de juros básica da economia. Esta diminuição da rentabilidade do
capital faz com que os investidores procurem formas alternativas de
alocação financeira destes capitais, ou seja, há um aumento da
\emph{demanda especulativa}, seja no mercado imobiliário, seja no
mercado mobiliário.

Outros fatores também podem agravar o problema. A entrada do Estado no
mercado fundiário como comprador com fins de realização de reforma
agrária, por exemplo, de acordo com Rangel \autocite*[128]{rangel1985},
promoveria a elevação do preço da terra, o que aumentaria ainda mais o
problema agrário.

Desta forma, a reforma agrária viria naturalmente com a solução do
problema financeiro da economia, ou seja, com a abertura de novas
possibilidades de investimentos que diminuíssem a demanda especulativa
sobre a terra. Como a demanda de terra para cultivo e construção seria
pequena em relação à demanda especulativa, o preço da terra tenderia a
cair naturalmente \autocite[139]{rangel1986b}.

\section{Pressupostos da análise de
Rangel}\label{pressupostos-da-anuxe1lise-de-rangel}

Ora, lá se vão quase 35 anos da análise de Rangel. Seria no mínimo
questionável ainda, senão a validade da análise, mas pelo menos a sua
atualidade.

Primeiramente, não consta que a demanda especulativa sobre a terra tenha
caído, pelo menos não ao ponto do preço da terra cair a um nível que
possibilitasse uma verdadeira reforma agrária, conforme previa Rangel.

De acordo com \textcite{rangel1986b}, contudo, estudos levados a cabo
pelo IPEA \emph{a posteriori} confirmaram a sua hipótese da variação do
preço da terra em sentido inverso da eficácia marginal do capital que,
\emph{coeteris paribus}, determina o preço dos valores mobiliários.

Deve-se levar em conta, no entanto, que a análise de Rangel é prévia à
abertura das contas de capital do balanço de pagamento, ou seja, do
estabelecimento do livre (in)fluxo de capital estrangeiro no país, o que
deve ter sido o responsável pela manutenção da demanda especulativa
sobre o preço da terra até os tempos recentes.

\textcite{ADAMS201038} mostrou que, apesar dos bens raízes serem ligados
ao mercado local, os mercados imobiliários de diversos países apresentam
alta correlação, o que pode ser explicado, em parte, pelo livre fluxo de
capitais entre os diversos países desde o Consenso de Washington, na
década de 1990. A figura \ref{fig:adams2} mostra que a alta de preços de
imóveis nos países estudados por \textcite{ADAMS201038} são semelhantes
e diferem mais pela questão de uma assincronia, do que pela tendência,
\emph{i. e.} a tendência em geral é a mesma para todos os países, porém
há um \emph{lag} entre o início de uma tendência nos diversos mercados.
\begin{figure}[H]

{\centering \includegraphics[width=\textwidth]{./images/adams2_crop} 

}

\caption{Interconexão dos preços dos imóveis em diversos países.}\label{fig:adams2}
\end{figure}
\bcenter
Fonte: \textcite{ADAMS201038}. \ecenter

Ainda há de se considerar que a crise agrária crônica não cessou. Pelo
contrário, se agravou e vem se agravando cada vez mais, tendo o Brasil
atingido uma proporção de população urbana muito maior do que o seu grau
de desenvolvimento econômico possibilitaria.

No entanto, outro fator que deve ser levado em conta é estrutural: as
taxas de juros, sejam de curto ou longo prazo, atingiram os patamares
mais baixos da série histórica, desde 1954, como será visto no capítulo
\ref{economia}.

A grave crise econômica de 2008 e seus efeitos que o mundo vive até os
dias atuais teve grandes consequências sobre o preço dos ativos
imobiliários mundo afora. A crise de 2008 gerou uma resposta por parte
dos Bancos Centrais de praticamente todo o mundo, mas em especial os
Bancos Centrais dos países desenvolvidos, no sentido de um aumento nunca
antes visto da liquidez dos sistemas financeiros mundiais, através da
redução aos limites mínimos das taxas de juros de curto prazo, mas
também da aplicação de medidas de recompra de títulos de mais longo
prazo, reduzindo-se assim também a níveis baixíssimos toda a estrutura a
termo da taxa de juros. Talvez Rangel nem tivesse cogitado que as taxas
de juros um dia chegariam a patamares tão baixos que levariam a enormes
valorizações do preço da terra de maneira não-especulativa, mas de
acordo com os fundamentos econômicos.

Estas taxas de juros muito baixas, tanto a curto quanto a longo prazo,
tem estimulado financiamentos imobiliários a taxas de juros negativas no
mundo desenvolvido \autocite{serapicos}, criando um aumento vertiginoso
dos preços dos imóveis, muito acima dos fundamentos econômicos de um
mundo em estagnação econômica crônica, o que vem levando os governos
destes países a tomarem medidas não-usuais para a regulação do mercado
imobiliário, especialmente no que se refere ao controle dos preços dos
aluguéis, devido à crise habitacional que se instalou nestes países
desde a crise de 2008
\autocites{berlim}{londres}{california}{suecia}{newzeland}.

\section{Efeitos da falta de reforma agrária no cadastro
urbano}\label{efeitos-da-falta-de-reforma-agruxe1ria-no-cadastro-urbano}

As altas proporções da população urbana nos países ditos \emph{em
desenvolvimento} em comparação aos \emph{países desenvolvidos} não nos
permite imaginar que as ferramentas de planejamento urbano desenvolvidas
no primeiro mundo surtam o efeito esperado em outras regiões menos
desenvolvidas do globo, pelo menos não na atual realidade
econômico-social. E isto também se aplica, é claro, ao cadastro
territorial multifinalitário.

Enquanto a reforma agrária ideal almejada por muitos infelizmente não
tem lugar, são o cadastro e as outras ferramentas de planejamento urbano
que devem se adaptar a essa outra realidade particular da paisagem dos
países subdesenvolvidos. Tentar, pelo contrário, promover a fórceps a
modificação da terrível paisagem urbana destes locais para que se
enquadrem nos modelos teóricos do mundo desenvolvido passaria,
necessariamente, pela expulsão das classes menos favorecidas das grandes
cidades, sem que exista do outro lado uma porta de saída.

Obviamente, compreendido este contexto histórico do desenvolvimento do
capitalismo no Brasil, que não difere muito do desenvolvimento
capitalista dos outros países da América Latina, não seria de se esperar
que o cadastro territorial multifinalitário, assim como outras
ferramentas do planejamento urbano, como concebidos nos países
desenvolvidos, possam ser replicadas, sem as devidas adaptações, nessa
outra realidade, quase que completamente oposta.

Desta maneira, vem em boa hora a implantação de núcleos de estudos
específicos para o estudo e disseminação do cadastro na América Latina
\autocite{lalan}, em que entende-se que deve-se, contudo, concentrar os
esforços destes núcleos na adaptação das ferramentas clássicas do CTM à
realidade regional.

\hypertarget{economia}{\chapter{O Mercado Imobiliário e a
Economia}\label{economia}}
\begin{epigrafe}
    \vspace*{\fill}
    \begin{flushright}
    \textit{``a economia é uma ciência histórica por excelência -\\
    qualidade que partilha com outras ciências sociais.\\
    Quer isso dizer que está submetida a um duplo processo\\
    evolutivo: o fenomenal (como representação, como ideia \\
    da coisa, como `coisa para nós', no sentido kantiano)\\
    e o numenal (como objeto, coisa representada, `coisa em si')\\
    \ldots e não pode ser estudada senão nesse duplo contexto.''\\
    (Rangel, 1956, p. 204)}
    \end{flushright}
\end{epigrafe}
\section{\texorpdfstring{Os vários significados do termo
\emph{economia}}{Os vários significados do termo }}\label{os-vuxe1rios-significados-do-termo}
\begin{citacao}
A propósito da religião (que, para Marx, era a ideologia por excelência), Hegel
distinguiu três momentos: \emph{doutrina, crença e ritual}; assim, fica-se
tentado a distribuir em torno desses três eixos a multiplicidade de ideias
associadas com o termo `ideologia': a ideologia como um complexo de ideias
(teorias, convicções, crenças, métodos de argumentação); a ideologia em seu
aspecto externo, ou seja, a materialidade da ideologia, os Aparelhos Ideológicos
de Estado; e por fim, o campo mais fugidio, a ideologia `espontânea' que atua
no cerne da própria `realidade' social.
\cite[p.~15]{zizek}
\end{citacao}
Segundo Singer \autocite*[7]{singer}, é possível distinguir pelo menos
três significados do termo \emph{economia}:
\begin{itemize}
\tightlist
\item
  a qualidade de ser estrito ou austero no uso de recursos ou valores;
\item
  a característica comum de uma ampla gama de atividades que compõe a
  \emph{economia} de um país, de uma cidade, etc.
\item
  a ciência que tem por objeto a atividade que dá o segundo significado.
\end{itemize}
A economia (ciência) é a sistematização do conhecimento sobre a economia
(atividade).

\subsection{A economia como atividade}\label{a-economia-como-atividade}

A ciência se divide a respeito da definição de economia como atividade,
entre social (escola \emph{marxista}) e individual (escola
\emph{marginalista}) \autocite[9]{singer}.

Enquanto para os \emph{marxistas} a atividade econômica é sempre
coletiva, praticada mediante a divisão social do trabalho, para os
\emph{marginalistas} a atividade econômica é em sua essência individual,
que atuam autonomamente, tendo em vista apenas seus desejos ou suas
necessidades \autocite[10]{singer}.

\subsection{A economia como ciência}\label{a-economia-como-ciuxeancia}

Também diferem os \emph{marxistas} e os \emph{marginalistas} quanto a
definição de economia como ciência.

Enquanto para os \emph{marxistas} a economia política é a ciência do
social, abrangendo em seu campo de estudo o conjunto de atividades que
formam a vida econômica da sociedade \autocite[14]{singer}, para os
\emph{marginalistas} a ciência econômica tem como modelo as ciências da
natureza, onde cada uma das quais tem como objeto próprio um determinado
``setor'' do universo físico. Analogamente, as ciências do homem teriam
como objeto um ``setor'' do universo humano \autocite[15]{singer}.

Para \textcite{rangel1956}, ``ciência é classificação e medida - não
apenas medida, como pode se depreender do lema econométrico. Se ciência
fosse medida não haveria ciência em Aristóteles ou Hegel''
\autocite[204]{rangel1956}.

A evolução da economia, segundo \textcite{rangel1956}, dá-se através de
um duplo processo evolutivo, o fenomenal e o numenal, seguindo a
filosofia de Kant:
\begin{citacao}
a economia é uma ciência histórica por excelência - qualidade que partilha com
outras ciências sociais. Quer isso dizer que está submetida a um duplo processo
evolutivo: o fenomenal (como representação, como ideia da coisa, como ``coisa
para nós'', no sentido kantiano) e o numenal (como objeto, coisa representada,
``coisa em si'') [\ldots] e não pode ser estudada senão nesse duplo contexto
\cite[p.~204]{rangel1956}.
\end{citacao}
\subsection{A Economia como Ciência
Social}\label{a-economia-como-ciuxeancia-social}

Como Ciência Social, a Economia pode ser definida como a ciência:
\begin{citacao}
que estuda como as pessoas e a sociedade decidem empregar recursos escassos, que
poderiam ter utilização alternativa, na produção de bens e serviços de modo a
distribuí-los  entre as várias pessoas e grupos da sociedade, a fim de
satisfazer as necessidades humanas
\cite[p.~5]{passosnogami}.
\end{citacao}
Para Rosseti \autocite*[31]{rossetti}, no entanto, a Economia não é uma
ciência com limites nitidamente definidos, assim como as demais ciências
sociais:
\begin{citacao}
À semelhança do que ocorre com os demais ramos das ciências sociais, não se pode
considerar a economia como fechada em torno de si mesma. Pelas implicações da
ação econômica sobre outros aspectos da vida humana, o estudo da economia
implica a abertura de suas fronteiras às demais áreas das ciências humanas. Esta
abertura se dá em dupla direção, assumindo assim caráter \textbf{biunívoco}.
\end{citacao}
Segundo \textcites{rosseti}[32]{rossetti}, a separação das ciências
sociais em especialidades distintas é não-rigorosa, ou, ao contrário,
estas especialidades estão entremeadas:
\begin{citacao}
Em síntese, pode-se inferir que as interfaces da economia com outros ramos do
conhecimento social decorrem de que as relações humanas e os problemas nelas
implícitos ou delas decorrentes não são facilmente separáveis segundo níveis de
referência rigorosamente pré-classificados. O referencial econômico deve ser
visto apenas como uma abstração útil, para que se analisem aspectos específicos
da luta humana pela sobrevivência, prosperidade, bem-estar individual e
bem-comum. Ocorre, todavia, que essa mesma luta não se esgota nos limites do que
se convencionou chamar de relações econômicas. Vai muito além, abrangendo
aspectos que dizem respeito à postura ético-religiosa, às formas de organização
política, aos modos de relacionamento social, à estruturação da ordem jurídica,
aos padrões das conquistas tecnológicas, às limitações impostas pelas condições
do meio ambiente e, mais abrangentemente, à formação cultural da sociedade.
\end{citacao}
\section{O Mercado Imobiliário e a Macroeconomia}\label{macro}

Segundo \textcite{LEUNG}, há um reconhecimento relativamente recente e
crescente sobre a importância da interação entre os diversos mercados
imobiliários entre si e entre o mercado imobiliário como um todo e a
macroeconomia.

Pesquisas em economia habitacional convencional e em economia urbana
praticamente ignoram as interações com a macroeconomia. Na melhor das
hipóteses, algumas análises teóricas e empíricas da economia urbana e
habitacional incluem variáveis macroeconômicas (como inflação,
crescimento econômico, PIB, taxa de desemprego etc.) como `variáveis de
controle' exógenas \autocite[3]{LEUNG}.

Por sua vez, os livros de Economia tradicionais ou tratam o mercado
imobiliário como apenas uma dos muitos mercados de bens de consumo ou o
negligenciam como um todo. A Macroeconomia convencional ignora
completamente o mercado imobiliário \autocite[3]{LEUNG}, embora o
mercado habitacional constitua uma grande parte da Macroeconomia
\autocite[5]{LEUNG}. \textcite{krugman} afirmou que um dos segredos da
política monetária reside no fato que a política monetária funciona
através do mercado imobiliário, tendo pouco impacto direto no
investimento em negócios.

Segundo Greenwood e Hercowitz \autocite[\emph{apud}][5]{LEUNG}, o
estoque de capital imobiliário é maior do que o capital de negócios e,
em geral, o valor de mercado investido anualmente no mercado
habitacional é maior do que o investimento em negócios, o que claramente
faz do segmento habitacional muito mais do que apenas um outro mercado
de bens de consumo.

Davis e Heathcote (2001) \autocite[\emph{apud}][6]{LEUNG} afirmam que o
valor de mercado das propriedades imobiliárias em estoque nos EUA é
aproximadamente igual ao PIB médio anual. Segundo a revista britânica
The Economist \autocite{economist}, a maior classe de ativos no mundo é
a propriedade residencial, com valor estimado de 200 trilhões de
dólares, o que equivale a 3 vezes mais o valor de todas as ações
negociadas em bolsa.

No Brasil, segundo \textcites{fnogueira}[155]{fnogueira}, a participação
dos ativos de base imobiliária era cerca de 40\% dos bens e direitos
declarados na Declaração do Imposto de Renda da Pessoa Física. Deve-se
levar em conta, no entanto, que os valores declarados são os valores do
momento da aquisição dos imóveis, que não são atualizados para fins de
declaração de imposto de renda, portanto estes são usualmente menores do
que os valores de mercado.

Segundo \textcite[p.~4]{LEUNG}, no entanto, mais recentemente tem havido
um pequeno porém crescente esforço de pesquisa para preencher a lacuna
entre as duas literaturas e lançar luz sobre questões que são
conjuntamente conseqüentes para a macroeconomia e a habitação.

O mercado imobiliário, diferentemente de outros mercados de capitais,
exibe uma flutuação de valores baixa e não muda imediatamente após a
mudança do noticiário econômico \autocite[3]{ADAMS201038}. O mercado
imobiliário residencial, em particular, mostra forte rigidez pra baixo,
porque os donos de imóveis residenciais tendem a resistir a venda da
propriedade sob períodos de recessão econômica \autocite[
129]{Case2000}.

\textcite{ADAMS201038} mostraram, com a observação de dados em painéis
de 15 países por um período de 30 anos, que variáveis macroeconômicas
apresentam um significativo impacto no preço de imóveis residenciais.

Segundo \textcite[p.~18]{ADAMS201038}, particularmente variáveis como
emprego, produção industrial e aumento da base monetária demonstraram-se
propensas a aumentar a demanda por imóveis residenciais, aumentando
assim os seus preços. Além disto, um aumento na taxa de juros de curto
prazo também afeta positivamente o preço dos imóveis residenciais, pelo
efeito do aumento dos custos de financiamento e pelo desaquecimento do
setor de construção, o que ocasiona também um aumento no preço dos
aluguéis, que também puxa para cima o preço dos imóveis.

Por outro lado, um aumento nas taxas de juros de longo prazo leva a
diminuição da demanda por imóveis residenciais. Por causa da maior
atratividade nos investimentos de renda fixa oriundos do aumento das
taxas de longo prazo, reduz-se a demanda por (investimento em) imóveis
residenciais, o que por conseguinte reduz os seus preços
\autocite[19]{ADAMS201038}.

\textcite{ADAMS201038} notaram também que, devido a diferenças no nível
regulatório e nas características do mercado hipotecário, houve um alto
grau de variação entre os países, embora os resultados sejam muito
similares mesmo com a variação dos métodos de estimação.

A figura \ref{fig:adams} mostra como, teoricamente, as variações nas
variáveis macroeconômicas impactam na formação de preços do mercado
imobiliário. Por exemplo, no painel central é ilustrado como se propaga
para o mercado imobiliário um aumento na taxa de juros de longo prazo:
um aumento nos juros de longo prazo diminui os preços dos imóveis, que
por sua vez diminuem o apetite dos investidores por imóveis, fazendo
diminuir o ritmo das construções, o que por sua vez tende a diminuir o
estoque de imóveis (baixa entrada de imóveis novos no mercado), o que
por sua vez tende a aumentar o valor dos aluguéis.

\subsection{Relação com os fundamentos
econômicos}\label{relauxe7uxe3o-com-os-fundamentos-econuxf4micos}

O mercado imobiliário, assim como acontece com o mercado mobiliário,
deve seguir os fundamentos da Economia no longo prazo. No entanto,
eventuais descolamentos dos preços no mercado imobiliário dos
fundamentos econômicos podem ocorrer. A este descolamento é dado o nome
de bolha especulativa.

Existe no mercado imobiliário uma propensão à formação de bolhas
especulativas, devido a diversos motivos, como ``expectativas
exuberantes'' e problemas de informações de preços, que são difíceis
devido às particularidades do mercado imobiliário
\autocite[3]{ADAMS201038}.
\begin{figure}[H]

{\centering \includegraphics[width=\textwidth]{./images/adams_crop} 

}

\caption{O impacto das variáveis macroeconômicas no preço dos imóveis residenciais.}\label{fig:adams}
\end{figure}
\bcenter
Fonte: \textcite{ADAMS201038}. \ecenter

\subsection{Índices de preços de
imóveis}\label{uxedndices-de-preuxe7os-de-imuxf3veis}

De acordo com o \textcite{rppi}, índices de preços de imóveis são
importantes e podem ser úteis ou necessários:
\begin{itemize}
\tightlist
\item
  como um indicador macroeconômio da atividade econômica;
\item
  para uso na calibração da política monetária;
\item
  como uma ferramenta para estimar o valor de um dos componentes de
  riqueza;
\item
  como um indicador da estabilidade financeira ou de solidez da
  exposição ao risco;
\item
  como um deflator nas contas nacionais;
\item
  para auxílio para a tomada de decisões dos cidadãos na negociação de
  um imóvel;
\item
  como um dos compenentes de um índice de preços ao consumidor; e
\item
  para uso em comparações inter-setoriais ou internacionais.
\end{itemize}
Ainda segundo o \textcite{rppi}, grosseiramente falando, os índices de
preços podem ser divididos em dois grupos:
\begin{itemize}
\tightlist
\item
  índices para a aferição do \emph{estoque} de propriedade imobiliária
  em um determinado tempo; e
\item
  índices para a aferição das \emph{vendas} de propriedade imobiliária
  em um determinado período de tempo.
\end{itemize}
Em síntese, índices do primeiro tipo devem conter informação tanto das
propriedades existentes e das propriedades recém-construídas, enquanto
que para índices do segundo tipo apenas, destinado a medir o
investimento imobiliário, deve se ater às vendas de novos imóveis ou
imóveis recém convertidos em imóveis de novo tipo \autocite[155]{rppi}.

Contudo, deve-se ter em conta que para aferir o investimento bruto em
\emph{real estate} deve-se primeiro decompor o componente estrutural do
componente referente ao preço da terra no valor de venda dos imóveis
\autocite[155]{rppi}. Esta decomposição também deve ser efetuada para
utilização do índice como um componente de um índice de preços ao
consumidor \autocite[156]{rppi}.

\subsubsection{Índices de
capitalização}\label{uxedndices-de-capitalizauxe7uxe3o}

\subsubsection{Metodologias para confecção de índices de preços de
imóveis}\label{metodologias-para-confecuxe7uxe3o-de-uxedndices-de-preuxe7os-de-imuxf3veis}

Diferentes metodologias são conhecidas para a confecção de índices de
preços de imóveis.

\paragraph{Métodos de
estratificação}\label{muxe9todos-de-estratificauxe7uxe3o}

São os métodos mais simples de medição de mudanças nos índices de preços
de imóveis. É feita pela aferição da mudança de uma medida de tendência
central (usualmente média ou mediana) em um determinado período de
tempo, dividindo-se os imóveis em sub-amostras de características
similares.

Um dos problemas da utilização deste tipo de metodologia é que ele é
incapaz de capturar eventuais mudanças quanto a qualidade do estoque de
imóveis ao longo do tempo, ficando viesado para cima, por exemplo,
quando a qualidade dos imóveis aumenta ao longo do tempo e vice-versa. O
índice tambem pode ficar viesado se houver alteração no \emph{mix} de
imóveis entre dois períodos, por exemplo, se num determinado período
houver um maior número de imóveis de alto padrão do que num período
anterior, o índice ficará viesado para cima.

Segundo o Eurostat \autocite*[158]{rppi}, o método da estratificação é
recomendado quando o volume de vendas é grande o suficiente e a
informação das características dos imóveis é detalhada o suficiente para
uma classificação detalhada das propriedades, sendo este método de
confecção de índice somente apropriado quando:
\begin{itemize}
\tightlist
\item
  um nível apropriado de detalhe escolhido é factível para as células
\item
  uma das variáveis de estratificação é a variável \emph{idade} da
  edificação; e
\item
  a decomposição do índice em terra e estrutura não é necessária.
\end{itemize}
\paragraph{Métodos hedônicos}\label{muxe9todos-heduxf4nicos}

É o método mais eficiente de construção de um índice quando há dados
disponíveis. Além explorar o máximo os dados disponíveis, permite a
desagragação dos preços das estruturas do preço da terra e pode ser
elaborado de forma a levar em conta a depreciação dos imóveis, razão
pela qual tal tipo índice pode ser utilizado tanto para o monitoramento
do valor dos estoques de imóveis quanto para aferir o investimento no
setor imobiliário.

Consiste na aplicação de modelos hedônicos, basicamente de duas maneiras
distintas:

\subparagraph{Modelos com variáveis dicotômicas de
tempo}\label{modelos-com-variuxe1veis-dicotuxf4micas-de-tempo}

Trata-se de um método que utiliza um modelo de regressão onde as
variáveis hedônicas dos imóveis são adicionadas de variáveis dicotômicas
de cada período \autocite[158]{rppi}. Apesar da simplicidade, uma vez
que um novo período é adicionado, com o novo ajuste do modelo, os
índices dos períodos anteriores são modificados. Para contornar este
problema, utiliza-se uma janela de tempo móvel, efetuando um ajuste de
modelo para os últimos N períodos.

\subparagraph{Modelos de preços
característicos}\label{modelos-de-preuxe7os-caracteruxedsticos}

Nestes tipos de índices, um modelo hedônico é ajustado a cada período e
o índice é construído através dos preços previstos pelos modelos
\autocite[53]{rppi}. Assim, o índice é construído através da razão entre
os preços previstos nos dois períodos, como mostra a equação
\eqref{eq:cp-approach}.
\begin{equation}
\frac{\hat p^t}{\hat p^0} = \frac{\hat \beta_0^t + \sum \hat \beta_k^t z_k^*}{\hat \beta_0^0 + \sum \hat \beta_k^0 z_k^*}
\label{eq:cp-approach}
\end{equation}
onde as características dos imóveis \(z_k^*\) pode ser estabelecida de
diversas maneiras, como a média das carecterísticas do período base
(\(z_k^* = \bar z_k^0\)), denominado índice de Laspeyres, como média das
características do período posterior (\(z_k^* = \bar z_k^t\)),
denominado índice de Paasche, ou ainda tomando a média geométrica dos
períodos anterior e posterior, denominado índice do tipo Fisher.

\paragraph{Método de vendas repetidas (Repeated
Sales)}\label{muxe9todo-de-vendas-repetidas-repeated-sales}

Embora seja o método mais ``natural'' para a confecção de um índice, e
de ser também o método do índice talvez mais famoso, o de Case e
Shiller, o método de vendas repetidas não é tão eficiente quanto o
método de modelos hedônicos, pois o método de vendas repetidas, ao
utilizar apenas os dados de imóveis transacionados mais de uma vez, por
conseguinte descarta muitos dados de transação que poderiam ser úteis
para a construção de um índice mais robusto \autocite[160]{rppi}.

\paragraph{Métodos baseados em avaliação -- Sales Price Appraisal Ratio
(SPAR)}\label{muxe9todos-baseados-em-avaliauxe7uxe3o-sales-price-appraisal-ratio-spar}

Este tipo de índice utiliza de dados de avaliação feitas para propósito
de tributação ou outros tipos de avaliação, usualmente utilizadas para a
bens similares, com o intuito de resolver alguns problemas relacionados
ao modelo de vendas repetidas (pequeno número de dados repetidos e a
suscetibilidade ao viés). Matematicamente, o índice é calculado de
acordo com a equação \eqref{eq:AP} \autocite[75]{rppi}.
\begin{equation}
P_{AP}^{0t} = \sum_{n \in S(t)} w_n^0(t) \left ( \frac{p_n^t}{a_n^0} \right )
\label{eq:AP}
\end{equation}
onde \(p_n^t\) é o preço de venda no período \(t\), \(a_n^0\) é o valor
da avaliação do bem no período base, \(S(t)\) é a amostra do período
\(t\) e \(w_n^0(t)\) são os pesos utilizados para a confecção do índice.

Ou seja, este método, como uma maneira de contornar o baixo volume de
dados de venda de bens nos diferentes períodos, compensa essa falta de
dados com a utilização de dados de avaliação, nos períodos onde não há
dados de venda.

Como o método acima não utiliza valores observados no período base, os
valores neste período, como acima calculados, serão diferentes de 1.
Como isto é indesejável, uma normalização é feita, obtendo-se assim o
\emph{Sales Price Appraisal Ratio} ou SPAR, cuja formulação pode ser
vista na equação \eqref{eq:SPAR} \autocite[75]{rppi}:
\begin{equation}
P_{SPAR}^{0t} = \frac{\sum\limits_{n \in S(t)} p_n^t}{\sum\limits_{n \in S(t)} a_n^0} \left [ \frac{\sum\limits_{n \in S(0)} p_n^0}{\sum\limits_{n \in S(0)} a_0^n} \right ]^{-1}
\label{eq:SPAR}
\end{equation}
Apesar de ser simples, necessitar apenas de dados de preços e avaliaçõe,
o que simplifica a sua computação, os métodos baseados em avaliações tem
a desvantagem de não lidar adequadamente com mudanças de qualidade entre
os períodos (reparos ou depreciação), além de ser dependente da
qualidade das avaliações.

Uma característica interessante deste tipo de índices é que, quando o
cadastro está atualizado e podem ser utilizados, os dados utilizados
para a confecção do índice representam a totalidade da população e não
apenas uma amostra, como de costume, ou seja, não há erro amostral,
apenas o erro devido à avaliação em si.

Uma limitação importante, porém, é que índices elaborados com tais
métodos não podem decompor o valor total da propriedade imobiliária em
valor de terra e valor de estruturas.

\subsubsection{Índices no mundo}\label{uxedndices-no-mundo}

O \textcite{rppi} elencou uma série de índices ao redor do mundo.

\paragraph{Países desenvolvidos}\label{pauxedses-desenvolvidos}
\begin{enumerate}
\def\labelenumi{\alph{enumi}.}
\tightlist
\item
  Índice de Case-Shiller
\end{enumerate}
\textcite{NBERw2506} desenvolveram\ldots{}

\paragraph{Países em
desenvolvimento}\label{pauxedses-em-desenvolvimento}

Nos países em desenvolvimento, segundo o o Eurostat
\autocite*[110]{rppi}, uma proporção significante dos imóveis advém de
construção própria, em terrenos de família, muitas vezes incompletas,
sendo que a construção de uma casa pode durar anos, as fontes de
financiamento são incertas, não há avaliações disponíveis, o que faz ser
impossível o cálculo dos valores de aquisição do imóvel.

Nestes casos, segundo o o Eurostat \autocite*[110]{rppi}, a única opção
prática para estimar o valor dos imóveis é através da equivalência dos
valores de aluguéis.

\paragraph{Global Real House Price
Index}\label{global-real-house-price-index}

O FMI, através do observatório Global Housing Watch\ldots{}

\subsubsection{Índices na América
Latina}\label{uxedndices-na-amuxe9rica-latina}

\paragraph{IPVU (Colômbia)}\label{ipvu-coluxf4mbia}

Desde 2003 \autocite[130]{rppi}

\subsubsection{Índices no Brasil}\label{uxedndices-no-brasil}

\paragraph{Índice FipeZap}\label{uxedndice-fipezap}

O índice de maior longevidade que se tem conhecimento no Brasil foi
elaborado em conjunto pela Fundação Instituto de Pesquisas Econômicas
(FIPE) e pelo portal de anúncios de imóveis, o chamado índice FipeZap.

Segundo a \textcite{fipezap}, a metodologia utilizada para o cálculo do
índice é a estratificação. Para o índice FipeZap, apenas dois critérios
de estratificação foram adotados, no caso dos imóveis residenciais, a
saber, o número de dormitórios e a área de ponderação, definida pelos
critérios do IBGE. No caso de imóveis comerciais, o único critério de
estratificação é a localização \autocite[7]{fipezap}.

De acordo com a \textcite{fipezap}, não foi possível a inclusão da
variável idade da edificação, por esta não ser uma variável obrigatória
nos anúncios da plataforma zap imóveis.

Infelizmente, devido à sua metodologia e composição, o índice FipeZap
não se presta nem a aferir o estoque de riqueza, nem o investimento
bruto no mercado imobiliário, já que a variável idade não foi inserida
na estratificação e não é possível a decomposição do índice na parte
estrutural e preço da terra, devido à própria metodologia.

Outro problema relacionado a este índice é que ele não é composto de
transações de imóveis, porém de anúncios de imóveis, o que além de poder
ser problemático é questionável, devido às diferentes taxas de descontos
praticadas no mercado em diferentes períodos e pelos diferentes atores,
que neste caso não são mensuráveis. Segundo o o Eurostat
\autocite*[126]{rppi}:
\begin{quote}
Although not related to the issue of timing, a disadvantage of
advertised prices and mortgage approvals is that not all of the prices
included end in transactions, and in the former case, the price will
tend to be higher than the fnal negotiated transaction price.
\end{quote}
Outros problemas relacionados ao índice FipeZap, como se basear
exclusivamente em anúncios em websites, também são elencandos pelo o
Eurostat \autocite*[104]{rppi}:
\begin{quote}
Information collected on a seller's asking price cannot always be easily
verifed and, as well as depending on a balanced and representative
sample, relies on the honesty and knowledge of those being surveyed and
when drawn from advertisements, the accuracy of the information,
especially when it is from a website {[}\ldots{}{]} It has also been
argued that websites will tend to be biased towards properties that have
a competitive asking price to entice potential sellers. All this is, of
course, speculation but it does bring home some of the potential
difculties associated with these sources.
\end{quote}
\paragraph{Índices IGMI-R e IGMI-C}\label{uxedndices-igmi-r-e-igmi-c}

Trata-se de um novo conjunto de índices construídos pela \gls{ABECIP} em
convênio com o \gls{IBRE}/\gls{FGV}, provavelmente através de um método
baseado em avaliações, ``levando em conta os laudos de imóveis
financiados pelos bancos''.
\begin{quote}
Nos Estados Unidos, país de grande dimensão territorial e elevada
mobilidade da população, o índice Case-Schiller faz o cálculo através da
metodologia das transações repetidas. Outros países, que têm uma
realidade diferente da americana, buscam alternativas para fazer o
cálculo.
\end{quote}
\subsection{Comportamento recente do mercado imobiliário e relação com
as variáveis
macroeconômicas}\label{comportamento-recente-do-mercado-imobiliuxe1rio-e-relauxe7uxe3o-com-as-variuxe1veis-macroeconuxf4micas}

No Brasil não existem séries históricas de prazo satisfatórios para uma
análise de longo prazo referente ao preço dos imóveis. No entanto, como
explicitado teoricamente no capítulo \ref{historico}, e mostrado por
\textcite{ADAMS201038} numa série de países, num mundo globalizado, em
que há livre fluxo de capitais, a tendência é que o comportamento dos
preços siga uma mesma tendência, portanto a análise do comportamento dos
preços nos países centrais da economia mundial tende a representar o que
ocorrerá, com maior ou menor \emph{lag}, nos outros países,
influenciados pelas variáveis e decisões ocorridas nestes países.

A figura \ref{fig:nhpi} mostra a evolução do \emph{Home Price Index} de
Case e Shiller, desde 1990.

É possível notar pela análise do índice que os preços dos imóveis que,
após o estouro da bolha do \emph{subprime}, os preços sofrem uma nítida
tendência de alta, desde janeiro de 2012, ultrapassando já, em termos
nominais, os níveis de preços anteriores à crise de 2008.
\begin{figure}[H]

{\centering \includegraphics[width=0.7\linewidth]{images/nhpi-1} 

}

\caption{Home Price Index, de Case e Shiller, em termos nominais.}\label{fig:nhpi}
\end{figure}
\bcenter
Fonte: O autor, à partir de dados de \textcite{QuandlWIKI}. \ecenter

Em julho de 2006 o índice atingiu o valor de 184,62 pontos, entrando
então em tendência de baixa, até atingir os 134,16 pontos, em janeiro de
2012, uma queda de 50.451 pontos, quando se iniciou nova tendência de
alta, que perdura até os dias atuais. No momento em que se escreve esta
dissertação o índice se encontra com 212,20 pontos, maior valor da série
histórica, uma alta de 78,04 pontos.

Em termos reais, no entanto, os preços ainda são inferiores ao pico
registrado, mas há uma nítida tendência de alta, conforme pode ser visto
na figura \ref{fig:rhpi}.
\begin{figure}[H]

{\centering \includegraphics[width=0.7\linewidth]{images/rhpi-1} 

}

\caption{Home Price Index, de Case e Shiller, em termos reais.}\label{fig:rhpi}
\end{figure}
\bcenter
Fonte: O autor, à partir de dados de \textcite{QuandlWIKI}. \ecenter

Na atualidade este índice se encontra em 176,84 pontos, sendo que o
maior valor da série histórica se deu em dezembro de 2005, quando
atingiu 195,83 pontos.

Em janeiro de 1997, o índice estava em 112,26 pontos. Desta data em
diante, o índice entrou em franca tendência de alta até atingir os
195,83 pontos, em dezembro de 2005, auge, em termos reais, da bolha
imobiliária, que no entanto só viria a estourar, nominalmente falando,
em julho de 2006, como se pode ver na figura \ref{fig:nhpi}.

É importante salientar que o fenômeno da alta tão relevante dos preços
dos imóveis é relativamente recente. O gráfico da figura \ref{fig:rhpi2}
ilustra isto.
\begin{figure}[H]

{\centering \includegraphics[width=0.7\linewidth]{images/rhpi2-1} 

}

\caption{HPI, em termos reais.}\label{fig:rhpi2}
\end{figure}
\bcenter
Fonte: O autor, à partir de dados de \textcite{QuandlWIKI}. \ecenter

O valor mínimo da série histórica (não mostrado no gráfico) foi atingido
em\\
Nov de 1921, quando o índice atingiu a marca de 65,61 pontos.

É interessante notar que, em um século, desde 1890, o índice de preços
reais de imóveis atingiu um valor máximo de apenas 130,99 pontos, o que
ocorreu em agosto de 1989.

Ou seja, em relação ao topo histórico de um século (agosto de 1989) à
partir do início da coleta do índice (1890), o índice hoje se encontra
45,85 pontos percentuais acima daquela marca e a apenas -45,85 pontos
percentuais abaixo de atingir o topo histórico de dezembro de 2005.

Em janeiro de 2012, no fundo do vale da curva do índice real, este
atingiu 126,65 pontos, apenas 4,34 pontos abaixo do topo histórico de um
século à partir de 1890.

O \emph{Global Real House Price Index} do \gls{FMI}, que se constitui de
uma média simples de preços reais de residências em 57 países
\autocite{fmitwa}, mostra também uma nítida e forte tendência de alta,
como pode ser observado na figura \ref{fig:global-rhpi}.
\begin{figure}[H]

{\centering \includegraphics[width=0.7\linewidth]{images/global-rhpi-1} 

}

\caption{HPI real do FMI.}\label{fig:global-rhpi}
\end{figure}
Deve-se reparar que, ao contrário do que aconteceu nos EUA, o pico em
termos globais só foi alcançado no último trimestre de 2007. Deve ser
observado ainda que a queda, em termos globais, foi muito menos abrupta
que a queda ocorrida nos EUA. O ponto de retomada, no entanto, tem
grande coincidência: enquanto nos EUA a retomada começa à partir de
Jan/2012, tanto para o índice nominal quanto para o índice real,
globalmente esta retomada ocorre à partir do final do primeiro trimestre
de 2012. Se levar-se em conta que, diferentemente dos índices
norte-americanos, o índice global é trimestral, pode-se dizer que a
retomada global da alta dos preços dos imóveis é praticamente
simultânea.

No entanto, deve-se ter em conta que o preço dos imóveis nos EUA afeta o
índice global.

Isto poderia ser indício da formação de nova bolha imobiliária, nos
moldes da que estourou em meados de 2006, com consequências
catastróficas para a economia global?

Bolhas imobiliárias são difíceis de serem identificadas. Bolhas são
fenômenos de curto prazo que representam descolamentos dos preços em
relação aos seus fundamentos. Logo, para se afirmar que existe uma
bolha, é necessário mostrar que inexiste uma correlação entre os
fundamentos e os preços dos ativos imobiliários.

\textcite{fmitwa} elencam uma série de motivos que podem nos levar a
conclusão que desta vez é diferente, como a falta de sincronicidade em
diversos países, o que ocorreu durante a bolha dos anos 2000, e a maior
vigilância por parte das autoridades monetários no que tange às medidas
macroprudenciais na prevenção, que incluem \autocite{fmiem}:
\begin{itemize}
\tightlist
\item
  Limite a razão empréstimo/valor, o que limita o valor da hipoteca
  relativa ao valor da propriedade.
\item
  Limites a razão dívida/renda, que limita o tamanho do pagamento do
  serviço da dívida a um limite fixo da renda do mutuário.
\item
  Requerimentos de capitais setoriais, que forçam os bancos a manter
  capital extra contra empréstimos a setores específicos, como o mercado
  imobiliário.
\item
  Requerimentos de provisionamento para pagamentos duvidosos.
\end{itemize}
Porém, além das medidas macroprudenciais citadas, considera-se que
existem outros fundamentos que indicam a não existência de uma bolha
especulativa, mas que indicam que o preço dos imóveis estão seguindo os
fundamentos econômicos.

Como anteriormente mencionado, uma das variáveis macroeconômicas que
afetam o mercado imobiliário são as taxas de juros de longo prazo. Uma
diminuição das taxas de juros de longo prazo tende a reduzir a demanda
por títulos públicos de longo prazo e a aumentar a demanda por
investimentos com base imobiliária.

Estas taxas de juros de mais longo prazo tem caído no mundo todo,
especialmente nas economias desenvolvidas, como ilustra o gráfico da
figura \ref{fig:yields}, elaboradas a partir de dados obtidos do
\textcite{fredgs10}, \textcite{fredgs20}, \textcite{fredgs30}. Esta
mostra as taxas de juros dos títulos da dívida do tesouro
norte-americano, em periodicidade mensal. A causa da queda destas taxas
de juros de mais longo prazo (assim como as de curto prazo) são
atribuídas a diversos fatores, o que vai muito além do escopo deste
trabalho. Uma boa discussão pode ser vista em \textcite{bresser2018},
\textcite{krugman2020}.
\begin{figure}[H]

{\centering \includegraphics[width=0.7\linewidth]{images/yields-1} 

}

\caption{Taxas dos \emph{treasuries bonds} desde janeiro/1990.}\label{fig:yields}
\end{figure}
\bcenter
Fonte: O autor, à partir de dados do \gls{FRED}. \ecenter

Mesmo uma análise mais ampla (desde 1953) das taxas de juros mostram que
o período recente é o período de menor taxa de juros em termos
históricos, o que pode ser visto na figura \ref{fig:fred}.
\begin{figure}[H]

{\centering \includegraphics[width=0.7\linewidth]{images/yields2-1} 

}

\caption{Taxas diárias dos \emph{treasuries bonds} desde janeiro/1990.}\label{fig:yields2}
\end{figure}
\bcenter
Fonte: O autor, à partir de dados do \gls{FRED}. \ecenter

No momento em que se escreve deste trabalho, a taxa de juros dos títulos
de mais longa maturidade, ou seja, de 30 anos, está em 1,45\% a.a.

Para os títulos com vencimento em 20 anos, a taxa está em 1,23 \% a.a. e
para os títulos com vencimento em 10 anos está em 0,88\% a.a., todos
muito perto dos seus valores mínimos da série histórica, que foram
atingidos em 2020-03-09, quando os títulos com maturidade de 10 anos
atingiram impressionantes 0.54\% a.a.

Segundo Krugman \autocite*{krugman2020}, no entanto, isto não está a
ocorrer apenas nos EUA, mas em todo o mundo desenvolvido, com maior ou
menor gravidade, já há alguns anos, e não deve ser interpretado apenas
como um efeito de curto prazo de uma crise com a do Corona vírus, que só
fez agravar ainda mais um quadro que já vem de longo prazo:
\begin{citacao}
What this tells us is that the bond market isn’t just pricing in a global
recession driven by the coronavirus, but that it expects the Fed funds rate to
be near zero a lot of the time looking forward. That is, the market sees a
future of secular stagnation, in which the economy is in a liquidity trap, that
is, a situation in which monetary policy loses most of its traction, much if not
most of the time. We were in a liquidity trap for 8 of the past 12 years; the
market now appears to believe that something like this is the new normal.
\end{citacao}
\begin{figure}[H]

{\centering \includegraphics[width=\textwidth]{images/fred-1} 

}

\caption{Taxas mensais dos \emph{treasuries bonds} desde abril/1953.}\label{fig:fred}
\end{figure}
\bcenter
Fonte: O autor, à partir de dados do \gls{FRED}. \ecenter

Esta longa e sustentada tendência de queda de longo prazo, com esta
persistência não pode ser atribuída a uma bolha especulativa,
\emph{i.e.} um desvio de curto prazo de uma tendência de longo prazo,
mas sim a uma tendência estrutural, sistêmica.

As taxas de juros das hipotecas são influenciadas pelas taxas dos
títulos da dívida pública: os títulos da dívida pública são considerados
ativos sem risco, Por isto os investimentos em títulos públicos possuem
as menores taxas do mercado: o governo, que emite a dívida, também emite
a moeda que necessária para pagá-la. Ou seja, não há risco verdadeiro
quando um governo emite um título de dívida, a não ser o risco da
própria desvalorização da moeda (inflação). É natural, portanto, que os
investidores exijam, para que invistam em ativos diferentes dos títulos
da dívida pública, um prêmio de risco. Para o mercado imobiliário, este
risco é mínimo, pois o empréstimo se dá com o imóvel em garantia. É
esperado, portanto, que as taxas de juros das hipotecas sejam apenas um
pouco superiores aos títulos da dívida pública.
\begin{figure}[H]

{\centering \includegraphics[width=\textwidth]{images/mortagages-1} 

}

\caption{Juros hipotecários médios (30 anos) e taxas de títulos de maturidade constante de 30 anos.}\label{fig:mortagages}
\end{figure}
Esta alta queda das taxas de juros dos ativos financeiros de longo prazo
é uma das causadoras da elevação dos preços dos mesmos ativos, assim
como a elevação do preço dos imóveis.

A grande diferença então da recente alta dos preços dos imóveis (desde
2012) da alta anterior (até Dez/2005) é que a alta mais recente é
plenamente justificada pela altíssima liquidez dos mercados. Ou seja,
não se pode atribuir a recente alta dos preços dos imóveis à especulação
imobiliária, ou pelo menos não se pode atribuir a recente alta somente a
especulação: há fundamento para a alta, o que se discute com mais
propriedade na seção \ref{micro}.

Em um cenário de longo prazo, a tendência é que a crise de 2008 apenas
tenha acelerado em demasia, momentaneamente, o que já era uma tendência
de longo prazo, portanto estrutural, da economia.

\subsection{O Mercado Imobiliário e o setor
bancário}\label{MI-e-o-setor-bancario}

A participação dos produtos financeiros relacionados ao setor
imobiliário constituem uma grande parte dos portfolios bancários.
Justamente por isto, crises bancárias estão frequentemente associadas a
superexposição do setor bancário no mercado imobiliário
\autocite[148]{Case2000}. Segundo Claessens \emph{et al.} (2010)
\autocite[\emph{apud}][3]{silver}, de 46 crises bancárias sistêmicas
para quais há dados disponíveis, mais de dois terços foram precedidas
por padrões de aumento-estouro de preços de imóveis.

Não apenas pelo estouro de uma bola especulativa podem haver quedas nos
preços bens imóveis, mas também por conta de uma inversão dos
fundamentos econômicos que sustentavam os preços
\autocite[129]{Case2000}, como aliás parece ser o caso no momento,
devido a um fator exógeno, a saber, a pandemia do corona vírus.

\subsubsection{A crise imobiliária-financeira de
2008}\label{a-crise-imobiliuxe1ria-financeira-de-2008}

\paragraph{A dinâmica da crise}\label{a-dinuxe2mica-da-crise}

Em 2001 o \gls{FED} reduziu a taxa básica da economia americana para
1,5\% a.a., para combater a recessão econômica, atingindo um dos níveis
mais baixos da história. Um dos grandes problemas que emergem com níveis
tão baixos de juros é que os maiores investidores institucionais do
mercado de títulos de dívida pública, como os fundos de pensão, possuem
metas atuariais que só são atingidas com taxas mais altas. Estes
investidores, então, pressionados a atingir estas metas atuariais, são
obrigados a procurar opções de investimentos mais rentáveis do que os
títulos públicos. Estas instituições, no entanto, tem em seus estatutos
alugmas limitações, como investir apenas em produtos financeiros acima
de determinados \emph{ratings}, ou seja, classificações dos produtos,
que são elaboradas pelas agências de classificação de risco.

Desta maneira, havia demanda no mercado para ativos de melhores
classificação de risco, porém com rentabilidade mais altas do que os
ativos da dívida pública. Para suprir esta demanda, houve o advento dos
\gls{CDO}, que são derivativos financeiros compostos de ativos
financeiros, \gls{ABS}, que podem ser compostos de diversos tipos de
empréstimos. No caso do mercado imobiliário, estes ativos são os
\gls{MBS}, que são grupos de ativos constituídos de hipotecas. Este tipo
de derivativos, \gls{CDO}, tem uma estrutura que permite às instituições
agrupar uma série de ativos de menor \emph{rating}, criando assim um
\gls{CDO} compostos de ativos com maior e menor \emph{rating}. A
estrutura destes derivativos se verá com mais detalhes na próxima seção.

No entanto, em virtude das taxas muito baixas praticadas pelo \gls{FED},
esta demanda estava tão grande, que os bancos se sentiram incentivados a
produzir cada vez mais empréstimos imobiliários, com o intuito de vender
estes empréstimos na forma de \gls{CDO}. Desta maneira, os bancos se
livravam dos riscos de crédito e liberavam mais capital para
possibilitar a realização de novos empréstimos.

Neste ínterim, porém, surgiram empréstimos exóticos, em que o mutuário
pagava por alguns anos apenas o valor dos juros. Depois deste período, o
mutuário passa a pagar também um parte do principal. Este tipo de
empréstimos eram chamados de \emph{interest-only loans}. No caso
específico do mercado imobiliário, eram chamados de \emph{interest-only
mortgages}. Convencionou-se chamar este tipo de empréstimos também de
empréstimos \emph{subprime}, o que é terminologicamente incorreto.
Tecnicamente, \emph{subprime} é o nome que se dá aos clientes cujo
histórico de crédito não é bom e, portanto, são mais propensos a
\emph{default}. Como o risco destes empréstimos é maior, os juros destes
empréstimos também são maiores. Com juros maiores, o risco de
\emph{default} aumenta ainda mais. Os empréstimos do tipo
\emph{interest-only} ajudavam a possibilitar a este tipo de cliente a
compra de um imóvel, pois a prestação era bem diminuída no início e, em
tese, enquanto o mercado imobiliário continuasse subindo, o tomador do
empréstimo sempre poderia, em todo caso, revendê-lo com preço maior no
futuro, caso não conseguisse arcar com as prestações, especialmente
quando do início do pagamento do principal. Porém, neste tipo de
empréstimo também era permitido aos emprestadores ajustar a taxa de
juros.

Com a grande liquidez inserida no mercado por estes empréstimos, o preço
dos imóveis tendia a continuar aumentando. Porém, em 2004 o \gls{FED}
iniciou um ciclo de alta dos juros para conter possíver surtos
inflacionários. Em junho de 2006 a taxa básica do \gls{FED} chegou a
5,25\% a.a. Com o aumento das taxas de juros, a demanda por residências
começou a diminuir e os preços dos imóveis começaram a cair. Os
mutuários começaram a perceber que não conseguiriam pagar as suas
prestações e tampouco revender as suas casas a preços maiores. A queda
dos mercados em geral foi grande e com a queda dos mercados, ou seja,
com a queda dos valores das companhias, estas não conseguiam oferecer
garantias suficientes para conseguir novos empréstimos no mercado
financeiro. Ainda, os bancos começaram a ficar receosos de emprestar uns
para os outros, por causa da alta presença de ativos tóxicos nos seus
balanços, os derivativos de crédito que compraram durante o \emph{boom}
imobiliário, o que causou uma crise de liquidez.

\paragraph{A raiz da crise}\label{a-raiz-da-crise}

Segundo Donnelly e Embrechts \autocite*[3]{devil}, a raiz da crise
financeira de 2008 estava nos derivativos de crédito, \gls{CDO},
lastreados em ativos de crédito imobiliário, \gls{MBS}, que tinham por
objetivo transferir o risco de uma hipoteca dos emprestadores ao mercado
financeiro em geral, ou seja, aos bancos, fundos de \emph{hedge} e
companhias seguradoras. Esta transferência de risco, ou securitização,
transformou o que seria apenas uma crise setorial das instituições
hipotecárias, numa crise financeira que atingiu praticamente todos os
setores financeiros e, por consequência, toda a economia real.

O propósito da venda destes seguros era a liberação de capital dos
bancos: uma vez que os bancos necessitam manter provisões de capital em
seus balanços contra os empréstimos realizados, a venda destes
derivativos de risco a outros investidores possibilitava a liberação
desse capital, tornando possível aos bancos efetuar mais empréstimos.

Esta dispersão do risco do setor bancário era vista como um fator de
diminuição da vulnerabilidade do setor aos choques econômicos. Na
prática, porém, a diminuição dos riscos levou os bancos a uma maior
tolerância na análise de risco dos tomadores de empréstimos, o que não
foi percebido de imediato pelos compradores dos \gls{CDO}s.

Segundo Donnelly e Embrechts \autocite*[5]{devil}:
\begin{citacao}
If a bank is not exposed to the risk of mortgage default, then it has no
incentive to control and maintain the quality of the loans it makes. To protect
against this, the theory was that the banks should retain the riskiest part of
the mortgage pool. In practice, this did not always happen, which led to a
reduction in lending standards [...] This possibility was foreseen some fifteen
years before the Crisis with remarkable prescience by Stiglitz, as he points out
in Stiglitz (2008).
\end{citacao}
No mercado financeiro é comum a utilização da expressão ``risco moral''
(\emph{moral hazard}), para descrever os mecanismos que podem levar a um
desequilíbrio de um mercado. Na crise de 2008, este risco moral esteve
presente não apenas neste desincentivo aos emprestadores para a
diminuição dos riscos dos empréstimos, já que estes eram securitizandos,
mas também na crença de que, em última instância, os governos seriam
obrigados a socorrer as instituições denominadas \emph{too big to fail}
(grandes de mais para quebrar) \autocite[4-5]{devil}.

\subsubsection{Derivativos de crédito}\label{derivativos-de-cruxe9dito}

\gls{CDO} são um tipo de instrumento financeiro do grupo dos denominados
derivativos de crédito, lastreados em \gls{ABS}, construídos através do
empacotamento de ativos de crédito financeiros, com o intuito da
diluição do risco. Os \gls{CDO} tiveram papel central na eclosão da
crise imobilário-financeira de 2007-2008.

Segundo \textcite{watts}, \gls{MBS}, \emph{i.e.} seguros baseados em
empréstimos hipotecários, surgiram nos anos 70. Pouco tempo depois,
vieram os \gls{CMO}, que eram derivativos de crédito similares aos
\gls{MBS}, porém com a característica de serem fatiados em
\emph{tranches}.

A diferença entre os \gls{CDO} e \gls{CMO} é que os \gls{CDO} podem ser
baseados em qualquer tipo de ativos ou dívidas, inclusive outros
\gls{CDO}.

Este tipo de instrumento é utilizado tanto no mercado imobiliário como
em geral para a diluição do risco dos empréstimos. No caso do mercado
imobiliário, das hipotecas \autocite[2]{watts}.

Este tipo de instrumento é oferecido pelos emprestadores através de
veículos de propósito específicos, ou \gls{SPV}, que, apesar de
patrocinados pelos bancos, não transferem a ele o risco de falência
institucional. Ou seja, a falência da instituição bancária não afeta o
risco de recebimento dos \gls{CDO}.
\begin{figure}[H]

{\centering \includegraphics[width=0.7\linewidth]{images/CDO-1} 

}

\caption{Esquemático do funcionamento de um CDO.}\label{fig:CDO}
\end{figure}
\bcenter
Fonte: Do autor. Adaptado de \textcite{devil}. \ecenter

Na figura \ref{fig:CDO} pode ser visto um esquemático do funcionamento
de um CDO simples, dividido em três \emph{tranches} ou parcelas. Cada
\emph{tranche} tem uma prioridade no recebimento de cupons. Após o
pagamento das despesas, a prioridade é para o pagamento do \emph{Senior
tranche}, depois do \emph{Mezzanine tranche} e então do \emph{Equity
tranche}. Se ocorrer um \emph{default} em um dos ativos lastreados, esta
perda é descontada primeiramente do \emph{Equity tranche}, que tem os
valores de seus cupons reduzidos. Depois de mais alguns \emph{defaults},
os cupons do \emph{Equity tranche} vão a zero, e perdas adicionais serão
descontadas do \emph{Mezzanine tranche} e assim por diante
\autocite[6]{devil}.

Desta maneira, a cada \emph{tranche} será atribuída pelas agências de
risco uma nota de crédito (\emph{rating}) diferente. Segundo
\textcite{devil}, tipicamente um \gls{CDO} possui composição de em torno
de 80\% de \emph{senior tranches}, o que significa que aproximadamente
20\% do portfolio de ativos base devem entrar em \emph{default} antes do
\emph{senior tranche} ser afetado. Na classificação dos \emph{tranches}
pelas agências de risco normalmente era atribuída ao \emph{senior
tranche} uma nota AAA. Aos \emph{mezzanine tranches} normalmente era
atribuída uma nota BBB-, enquanto que aos \emph{equity tranches} não era
atribuída nenhuma classificação de risco, ou seja, estas parcelas dos
\gls{CDO} eram consideradas de grau especulativo.

Deve-se salientar que existem ainda outros tipos de \gls{CDO}, com maior
número de parcelas, sejam parcelas intermediárias (vários níveis de
\emph{mezzanine tranches}), ou parcelas mais seguras, com as
\emph{super-senior tranches}, aos quais eram atribuídas notas de crédito
superiores à AAA, dos \emph{senior tranches}.

Segundo \textcite{watts}, a duração típica de um contrato de \gls{CDO}
era de 5 anos.
\begin{longtable}[]{@{}lr@{}}
\caption{Parte da matriz de risco de default da Fitch.}\tabularnewline
\toprule
\textbf{Rating at issuance} & \textbf{5-yr default probability
(\%)}\tabularnewline
\midrule
\endfirsthead
\toprule
\textbf{Rating at issuance} & \textbf{5-yr default probability
(\%)}\tabularnewline
\midrule
\endhead
AAA & 0,05\tabularnewline
AA+ & 0,19\tabularnewline
AA & 0,26\tabularnewline
AA- & 0,36\tabularnewline
A+ & 0,56\tabularnewline
A & 0,62\tabularnewline
A- & 0,92\tabularnewline
BBB+ & 1,20\tabularnewline
BBB & 1,89\tabularnewline
BBB- & 3,63\tabularnewline
BB+ & 5,74\tabularnewline
BB & 8,11\tabularnewline
BB- & 12,50\tabularnewline
B+ & 17,09\tabularnewline
B & 21,36\tabularnewline
B- & 27,08\tabularnewline
CCC+ & 33,64\tabularnewline
CCC & 37,64\tabularnewline
\bottomrule
\end{longtable}
\begin{longtable}[]{@{}lc@{}}
\caption{Exemplo de parcelamento de \gls{CDO}.}\tabularnewline
\toprule
\textbf{Tranche (Rating)} & \textbf{Attachment Points}\tabularnewline
\midrule
\endfirsthead
\toprule
\textbf{Tranche (Rating)} & \textbf{Attachment Points}\tabularnewline
\midrule
\endhead
Senior (AAA) & 42\% -- 100\%\tabularnewline
Mezzanine 1 (AA-) & 34\% -- 42\%\tabularnewline
Mezzanine 4 (A-) & 28\% -- 34\%\tabularnewline
Mezzanine 4 (BBB-) & 20\% -- 28\%\tabularnewline
Mezzanine 4 (BB-) & 11\% -- 20\%\tabularnewline
Equity (NA) & 0\% -- 11\%\tabularnewline
\bottomrule
\end{longtable}
Para Donnelly e Embrechts \autocite[20]{devil}, os \emph{equity
tranches} são produtos com muito risco, portanto de pouco valor, o que
interessa aos grandes especuladores, como os \emph{hedge funds},
enquanto os \emph{senior tranches} eram vistos pelas grandes
instituições como ativos muito seguros. Já os \emph{mezzanine tranches}
nem eram tão seguros e nem tão baratos, não interessando nem aos
investidores mais cautelosos, nem aos grandes especuladores. Desta
maneira, o \emph{mezzanine tranches} eram também empacotados em outros
\gls{CDO}, os chamados \gls{CDO}-squared, o que, em tradução literal
seria um \gls{CDO} ao quadrado. Este tipo de instrumento financeiro,
ainda mais complexo, é de precificação ainda mais difícil e isto
contribuiu ainda mais para aumentar o risco moral do mercado.

Segundo Donnelly e Embrechts \autocite[7]{devil}, o fato de que as
grandes instituições financeiras viam os \emph{senior tranches} como
ativos muito seguros teve uma grande influência na crise financeira de
2008. Segundo Donnelly e Embrechts \autocite[24]{devil}, um executivo de
uma subsidiária da AIG chegou a dizer, em agosto de 2007 que era difícil
vislumbrar um cenário onde a AIG poderia perder um dólar sequer com
estes ativos. Em 2008 o prejuízo líquido da AIG foram de US\$99 bilhões,
sendo US\$62 bi apenas no último trimestre, auge da crise do
\emph{subprime}.

\paragraph{Precificação}\label{precificauxe7uxe3o}

De acordo com \textcite{watts}, apesar de terem sido criados há algum
tempo, uma década e meia teve que se passar para que o mercado de
\gls{CDO} se tornasse grande. E o motivo era que faltava um modelo que
servisse de \emph{benchmark}, ou seja, um modelo que permitisse ao
mercado basilar o preço dos \gls{CDO} para que desse uma relativa
segurança aos investidores.

As condições para a precificação destes ativos vieram apenas na primeira
década do presente século, após a publicação do trabalho seminal de
\textcite{Li}.

De acordo com Donnelly e Embrechts \autocite*[7]{devil}, a chave para a
valorização dos \gls{CDO} é a modelagem dos \emph{defaults}. Para isto,
foi adotado pelo mercado (agências de risco de crédito como Fitch,
Moody's e Standard \& Poor's) o modelo da Copula Gaussiana
\autocite[14]{devil}, que tem algumas vantagens, como rapidez de
computação e facilidade de calibração. No entanto, foram desconsideradas
pelo mercado algumas desvantagens do modelo, a saber
\autocite[15]{devil}:
\begin{itemize}
\tightlist
\item
  modelagem insuficiente do agrupamento de \emph{defaults} no portfolio
  (quando uma companhia quebra, é provável que outras companhias do
  mesmo setor também quebrem, isto não pode ser modelado pela Copula
  Gaussiana);
\item
  Diferentes correlações entre os \emph{tranches} do CDO, o que não
  ocorre na prática; e
\item
  ausência de modelagem dos fatores que levam aos \emph{defaults}.
\end{itemize}
Especialmente no que tange à primeira desvantagem, a Copula Gaussiana é
indesejável para a aferição de risco. Como enfatizam Donnelly e
Embrechts \autocite*[16]{devil}: não é sábio confiar em um modelo
baseado na distribuição normal para verificar a probabilidade de
ocorrência de eventos extremos.

Pela própria estrutura dos \gls{CDO}, havia uma concepção que muito
raramente ocorreriam tantos \emph{defaults} simultâneos a ponto de um
\emph{senior tranche} ser afetado. Ocorre que pela própria
característica da Copula Gaussiana, os eventos extremos não são
suficientemente bem representados, o que não ocorre com outros models de
Copulas, como a de Gumbel, Clayton ou Copula-t. A apresentação destes
modelos será vista com detalhe no capítulo \ref{copulas}
(\protect\hyperlink{copulas}{O Método Copulas}).

\section{O Mercado Imobiliário e a Microeconomia}\label{micro}

O mercado imobiliário, como visto na seção anterior, não pode ser
considerado um simples mercado de bens, como outros bens de consumo em
geral, como automóveis, móveis ou eletrodomésticos. Sua análise em nível
macroeconômico, portanto, requer um estudo mais aprofundado das
variáveis da Economia do país, o que não quer dizer que não se dispense
de analisar também o seu comportamento microeconômico, o que aliás a NBR
14.653-01 \autocite*{NBR1465301} parece ter acabado de perceber e
recomendar.

\subsection{Estruturas básicas de
mercado}\label{estruturas-buxe1sicas-de-mercado}

De acordo com a NBR 14.653-01 \autocite*[ix]{NBR1465301}:
\begin{quote}
As estruturas básicas do mercado podem ser, resumidamente:
\begin{itemize}
\item
  \textbf{Concorrência Perfeita}: situação em que o número de vendedores
  e de compradores é suficientemente elevado para que um agente isolado
  não seja capaz de influenciar o comportamento dos preços;
\item
  \textbf{Monopólio}: É constituído por um único vendedor;
\item
  \textbf{Monopsônio}: é constituído por um único comprador;
\item
  \textbf{Oligopólio}: é constituído por um número pequeno de
  vendedores;
\item
  \textbf{Oligopsônio}: é constituído por um número pequeno de
  compradores.
\end{itemize}
\end{quote}
\subsection{Particularidades do mercado
imobiliário}\label{particularidades-do-mercado-imobiliuxe1rio}

Também em nível microeconômico o mercado imobiliário é um mercado
diferenciado dos outros mercados de bens de consumo. Apesar de toda a
indústria da construção ter evoluído muito ao longo das últimas décadas,
ainda persiste no Brasil um forte componente artesanal na construção
civil, o que implica em bens imóveis de características muito distintas,
a depender da mão-de-obra aplicada na sua execução. Também a questão do
projeto arquitetônica implica numa singularidade para cada bem
imobiliário. Mas mais importante ainda é a questão da localização, que
torna cada imóvel único. Isto não ocorre em qualquer outro mercado de
bens de consumo. A não ser por questões de natureza sentimental, um
carro, uma geladeira, ou praticamente qualquer outro bem de consumo é
produzido em série: existem milhares de itens iguais no mercado. Isto
nunca ocorre com os imóveis. Mesmo apartamento vizinhos, em um mesmo
prédio, tem características diferentes, dada a sua posição solar,
localização do andar em relação ao prédio, a vista que cada um possui,
entre outras questões.

Segundo a NBR 14.653-01 \autocite*[x]{NBR1465301}, ``o mercado
imobiliário caracteriza-se como um `mercado imperfeito', com bens não
homogêneos, estoque limitado, liquidez diferenciada e grande influência
de fatores externos.''

De acordo com \textcites{ADAMS201038}[3]{ADAMS201038}, a forte inércia
dos preços do mercado imobiliário influencia o comportamento do mercado
durante os booms econômicos, já que a exuberância das expectativas
(exuberância irracional) dos proprietários de imóveis facilita a
formação de bolhas nestes mercados. Além disto, a falta de informação a
respeito de preços no mercado imobiliário, pelo motivo deste ser um
mercado segmentado, ou seja, os preços obedecem uma lógica local, também
é uma facilitadora da formação de bolhas.

\subsection{Diagnóstico de mercado}\label{diagnuxf3stico-de-mercado}

A NBR 14.653-01 \autocite*[12]{NBR1465301}, estabelece que, ``o
profissional, conforme o tipo de bem, as condições de contratação, o
método empregado e a finalidade da avaliação, pode \textbf{tecer
considerações sobre o mercado do bem avaliando}, de forma a indicar,
tanto quanto possível, \textbf{a estrutura, a conduta e o desempenho do
mercado}.''

\subsection{O imóvel visto como um
investimento}\label{o-imuxf3vel-visto-como-um-investimento}

Na ótica do investidor (e não do especulador, que pretende ganhar com a
volatilidade do mercado, ou seja, comprando na baixa e vendendo na
alta), o imóvel é como um título de longo prazo. Racionalmente ou não, o
comprador de um imóvel com fins de investimento espera que o imóvel
comprado vá gerar um fluxo de aluguéis (constantes ou não) ao longo do
tempo, de maneira que este fluxo de aluguéis compense o investimento
inicial na compra do imóvel.

Diferentemente do que hoje ocorre com os investimentos capitalistas,
onde o payback esperado gira em torno de 5 a 10 anos, o comprador de um
imóvel esperar que este gere um fluxo de renda ao longo de décadas.

Assim, a compra de um imóvel assemelha-se à compra dos títulos de renda
fixa de maior \emph{duration} disponíveis no mercado.

Ora, como se sabe, o valor de face destes títulos, ou seja, o valor do
resgate destes títulos no vencimento, é dado. Porém, estes títulos são
negociados no mercado secundário a valor de mercado, sendo que os
títulos de longo prazo são altamente sensíveis a variações nas taxas de
juros de longo prazo. A saber, o preço destes títulos é inversamente
proporcional às taxas de juros, ou seja, quanto menor as taxas, maior o
valor presente dos títulos, ou valor de mercado, e vice-versa.

\subsection{Rendimentos de aluguel}\label{rendimentos-de-aluguel}

Os rendimentos de aluguel são taxas brutas de retorno dos recebimento de
aluguéis comparados ao valor de venda de um imóvel.

Por exemplo, para um imóvel avaliado em R\$1.000.000,00, com um
rendimento de aluguéis de R\$1.500,00 reais mensais, o rendimento bruto
do aluguel deste imóvel seria:

\[y_r = \frac{12*1.500}{1.000.000} = 3,6\% \ a.a.\] Esta taxa tem sido
utilizada para comparar o rendimento do investimento em imóveis em todo
o planeta, através de sítios de internet especializados em investimento
em imóveis \autocite{gpg}.

\subsubsection{Duration de um título de renda
fixa}\label{duration-de-um-tuxedtulo-de-renda-fixa}

Segundo \textcite{marins1}, a \emph{duration} de um título, ou de um
conjunto de títulos de renda fixa pode ser calculada de acordo com a
fórmula a seguir, de Macaulay:

\[D = \frac{\sum\limits_{t = 1}^{n} t \times \frac{F_t}{(1+i)^t}}{\sum\limits_{t = 1}^{n} \frac{F_t}{(1+i)^t}}\]

\subsubsection{Séries Perpétuas}\label{suxe9ries-perpuxe9tuas}

Uma série perpétua é uma série suficientemente longa de maneira que as
entradas ou saídas de caixa possam ser consideradas infinitas
(geralmente séries acima de 20 anos podem ser consideradas perpétuas).

Assim, o Valor Presente \(VP\) de uma série de pagamentos perpétua de
valor periódico \(A\), descontados de uma taxa de juros \(i\) é igual a:
\begin{equation}
VP = \lim_{n \to \infty} A \frac{(1+i)^n-1}{i(1+i)^n}= \frac{A}{i}
\label{eq:perpetua}
\end{equation}
\subsubsection{Cálculo do valor justo de um
imóvel}\label{cuxe1lculo-do-valor-justo-de-um-imuxf3vel}

O cálculo do valor justo de um imóvel pode ser feito considerando-se o
método do fluxo de caixa descontado, assim como é feito o
\emph{valuation} de uma empresa capitalista.

Por exemplo, se um investidor estima que os rendimentos líquidos de um
determinado imóvel (aqui definida como o valor dos aluguéis descontados
de taxas, custos de manutenção e outras despesas) será de R\$2.000,00
mensais, a uma taxa de juros de 3\% ao ano, o valor presente do imóvel,
considerando-se que este fluxo seja constante ao longo de toda a vida
útil do imóvel (não menor do que 20 anos), é:

\[P = \frac{12 \times 2.000}{0,03} = 800.000\]

Uma queda moderada da taxa de juros de longo prazo, digamos, para 2\%
a.a., teria o seguinte impacto no valor presente deste imóvel:

\[P = \frac{12 \times 2.000}{0,02} = 1.200.000\]

Já uma queda da taxa de juros de longo prazo mais agressiva, digamos
para 1\% a.a., teria o seguinte impacto:

\[P = \frac{12 \times 2.000}{0,01} = 2.400.000\]

A figura \ref{fig:valores-juros} mostra como varia, \emph{coeteris
paribus}, o valor justo de um imóvel em função da taxa de juros de longo
prazo.
\begin{figure}[H]

{\centering \includegraphics[width=0.7\linewidth]{images/valores-juros-1} 

}

\caption{Variação do valor justo de um imóvel em função da taxa de juros.}\label{fig:valores-juros}
\end{figure}
\bcenter
Fonte: Do autor. \ecenter

A tabela \ref{tab:tabela-valor-justo} mostra\ldots{}
\begin{table}[H]

\caption{\label{tab:tabela-valor-justo}Valor justo de um imóvel em função do valor do aluguel.}
\centering
\begin{tabular}[t]{rr}
\toprule
Taxa de Juros(\%) & Valor Justo (R\$)\\
\midrule
0,10 & 24.000.000\\
0,25 & 9.600.000\\
0,50 & 4.800.000\\
0,75 & 3.200.000\\
1,00 & 2.400.000\\
\addlinespace
1,50 & 1.600.000\\
2,00 & 1.200.000\\
3,00 & 800.000\\
4,00 & 600.000\\
5,00 & 480.000\\
\addlinespace
6,00 & 400.000\\
\bottomrule
\multicolumn{2}{l}{\textit{Notas:}}\\
\multicolumn{2}{l}{Supondo um aluguel constante de R\$2.000/mês.}\\
\multicolumn{2}{l}{Taxas de juros anuais.}\\
\end{tabular}
\end{table}
\bcenter
Fonte: Do autor. \ecenter

\subsection{O efeito da política de limitação do valor de
aluguéis}\label{o-efeito-da-poluxedtica-de-limitauxe7uxe3o-do-valor-de-aluguuxe9is}

A política de limitação do valor de aluguéis garante o controle da
inflação, controlando os custos de moradia para a população (aluguéis),
garantindo assim o direito à moradia, que não implica um direito à
propriedade \autocite{fnogueira}.

Obviamente que isto implica de num efeito balizador importante para o
cálculo de um preço justo dos imóveis, mas será que esta política apenas
é uma política capaz de regular com um mínimo de equidade o valor do
solo urbano?

Qualquer instrumento financeiro, como um título público, uma ação de uma
empresa, ou um derivativo, terá um valor de mercado diferente do valor
justo calculado para o instrumento, em virtude dos movimentos do mercado
(oferta vs.~demanda), que se dão não apenas pelos fundamentos
econômicos, mas também pelas expectativas dos diversos agentes
econômicos em relação ao valor futuro daquele instrumento. Em outras
palavras, quem determina o preço é o mercado. No entanto, um modelo de
preços é utilizado para a determinação de um \emph{benchmark}, uma
referência de mercado. Assim como o \emph{valuation} serve para calcular
o ``preço justo'' de uma ação (ou \emph{target price}, ou ainda
preço-alvo), o que por sua vez permite às corretoras efetuarem
recomendações (\emph{outperform} ou \emph{buy}, \emph{neutral} ou
\emph{hold}, \emph{underperform} ou \emph{sell}), assim como o método de
Black\&Scholes \autocite{marins2} permite o cálculo do preço justo das
opções de compra e venda, as séries perpétuas permitem um cálculo
razoável do preço justo de um imóvel, haja vista que permitem, sem
maiores especulações, o cálculo do fluxo de caixa descontado dos
aluguéis, balizando assim as expectativas em torno dos preços dos
imóveis, ajudando a conter dessa maneira a especulação imobiliária.

Em períodos de normalidade econômica poder-se-ia dizer que o
estabelecimento de preços máximos de aluguéis seriam suficientes para
conter uma especulação imobiliária desenfreada.

No entanto, o mundo não vive tempos de normalidade econômica. Desde a
crise de 2008 o mundo vive tempos de uma crise crônica, persistente, que
tem sido enfrentada pelos bancos centrais (especialmente os bancos
centrais dos países desenvolvidos, onde a estagnação é maior), por
enormes aumentos de liquidez do sistema financeiro.

Ou seja, a imposição de um valor máximo aos aluguéis pode ser uma boa
política para contenção da inflação, regulando os custos de moradia das
classes menos privilegiadas, impondo também alguma limitação da
especulação imobiliária desenfreada, porém em um cenário de juros
baixíssimos e estagnação econômica crônica como o que se avizinha e que
cada vez mais se prevê de forma duradoura, não será o suficiente para
conter uma alta expressiva no valor dos imóveis.

Pode-se prever que, apenas com esta política de limitação de aluguéis, o
mundo estará fadado a dividir os cidadãos em proprietários e locatários,
uma vez que o preço do aluguel estará controlado, mas o valor dos
imóveis tende a disparar.

\section{Problemas gerados pelo alto valor dos
imóveis}\label{problemas-gerados-pelo-alto-valor-dos-imuxf3veis}

Além dos problemas relacionados ao setor bancário discutidos na seção
\ref{MI-e-o-setor-bancario}, existe um problema normalmente
desconsiderado pela maioria é que altos preços de imóveis e/ou aluguéis
previnem que os trabalhadores se mudem para cidades com maior
produtividade marginal do trabalho. Ou seja, algumas pessoas tendem a se
manter em ocupações de baixa produtividade em cidades pequenas, no
interior, ao invés de se mudarem para grandes cidades onde certamente
teriam ocupações de maior produtividade, como numa fábrica, por exemplo,
pois o custo adicional de moradia que elas teriam não seria compensado
pelo maior salário que receberiam, devido à maior produtividade da sua
ocupação. Se fosse possível a redução dos custos de moradia em regiões
de maior produtividade da mão de obra, isto resultaria num maior Produto
Interno Bruto \autocite[149]{Case2000}.

Mas o principal problema advindo de uma grande alta no valor dos imóveis
é o problema da reversão das expectativas: uma vez que as expectativas
se revertem, seja por um aumento da taxa de juros, seja pelo estouro de
uma bolha de crédito, como ocorreu em 2008, tendem a causar transtornos
tanto para as pessoas físicas, que podem perder seus imóveis, quanto
para as instituições financeiras expostas ao risco do mercado
imobiliário, seja para os governos que são instados a socorrer as
empresas e as pessoas em dificuldades.

Segundo \textcite{fmiera}, episódios com padrões de boom seguidos de
estouros (de bolhas), ou \emph{boom-bust patterns}, precederam mais de
dois terços das 50 mais recentes crises bancárias sistêmicas.

Uma reversão de expectativas pode ocorrer devido a um fator exógeno. Por
exemplo, no momento em que se escreve esta dissertação, o mundo se
encontra em meio a uma pandemia de proporções globais e ainda sem
qualquer perspectiva que seja encontrada uma cura ou uma vacina contra o
Corona vírus.

Estimativas recentes dão conta de que os níveis de desemprego podem
aumentar substancialmente em todo o mundo, o que pode gerar um
perspectiva de que a demanda por aluguéis diminua muito, puxada pela
diminuição da renda da população devido ao desemprego, \emph{i.e.}, uma
diminução no numerador da equação \eqref{eq:perpetua}.

Por outro lado, a brusca queda das taxas de juros longas, \emph{i.e},
uma diminuição no denominador da equação \eqref{eq:perpetua} deve mais que
compensar a queda no numerador.

Seja, por exemplo um imóvel em que, antes da crise, se imaginava que
produziria uma série de pagamentos de aluguéis de R\$2.000,00 mensais. A
uma taxa de juros de longo prazo de 2\% a.a., o seu valor justo, segundo
a equação \eqref{eq:perpetua} seria de R\$1.200.000,00. Imaginando que,
com a crise do COVID haja uma redução nas expectativas de receitas com o
imóvel, passando para uma prestação mensal de R\$1.000,00. Com a queda
da taxa dos títulos para 1\% a.a., o preço do imóvel permaneceria
constante.

Uma vez alternadas as expectativas, no entanto, ou seja, uma vez que se
resolva a crise sanitária da COVID-19, as taxas de juros podem subir
repentinamente, ainda que permanecendo baixa em níveis históricos, mas
muito dificilmente ocorrerá o mesmo no mercado de trabalho, que tem
recuperação lenta. Este cenário poderia vir a desencadear uma crise
imobiliária similar ou ainda pior do que a da década passada, haja vista
que na atualidade, as ferramentas tradicionais de política monetária já
foram exauridas, além das ferramentas não-tradicionais.

\hypertarget{involutivo}{\chapter{O Método
Involutivo}\label{involutivo}}

Segundo a NBR14653-01 \autocite*[14]{NBR1465301}, o método involutivo
``identifica o valor do bem, alicerçado no seu aproveitamento eficiente,
baseado em modelo de estudo de viabilidade técnico-econômica,
\textbf{mediante hipotético empreendimento compatível} com as
características do bem e \textbf{com as condições de mercado no qual
está inserido}, considerando-se cenários viáveis para a execução e
comercialização do produto. O método involutivo pode identificar o valor
de mercado. No caso da utilização de premissas especiais, o resultado é
um valor especial.''

Apesar de laboriosa, os resultados obtidos com a aplicação do método são
repletos de incertezas, pois as variáveis e os riscos envolvidos não são
facilmente mensuráveis.

As incertezas vão desde à manutenção dos custos de construção previstos,
até aos valores de comercialização do empreendimento e sua velocidade de
vendas, passando pelas incertezas quanto às variáveis macroeconômicas
vigentes.

\section{Incertezas em relação às variáveis de
entrada}\label{incertezas-em-relauxe7uxe3o-uxe0s-variuxe1veis-de-entrada}

A construção civil, especialmente nos países em desenvolvimento, onde a
indústria da construção ainda não é totalmente desenvolvida, uma
atividade com alta participação de trabalho artesanal, onde os custos e
os prazos podem facilmente escapar do controle dos administradores.

Além desta peculiaridade do setor, ainda existem as incertezas inerentes
à situação econômica do país, estado ou cidade em que se desenvolvem as
atividades, que podem alterar as previsões iniciais de custos e receitas
previstas para o empreendimento.

\subsection{Análises de
sensibilidade}\label{anuxe1lises-de-sensibilidade}

A Engenharia de Avaliações, na busca de tentar lidar com estas
incertezas, estabeleceu algumas análises de sensibilidade para as
variáveis de entrada do método involutivo. No entanto, estas análises de
sensibilidade são aplicadas de maneira separada para cada variável,
todas as outras permanecendo constantes, \emph{i.e.} incluída a condição
\emph{coeteris paribus}. Ocorre que, como se sabe, a condição
\emph{coeteris paribus} raramente se aplica, pois uma variável está
sempre de alguma maneira relacionada com a outra.

\subsection{Simulações}\label{simulauxe7uxf5es}

Haja vista o problema relatado com as análises de sensibilidade, onde o
estabelecimento da cláusula \emph{coeteris paribus} é meramente teórica,
tendo pouca validade prática, o uso de simulações, onde são consideradas
relações entre as variáveis de entrada, se fazem necessárias.

Em \textcite{gahochheim} foram realizadas simulações considerando-se
diversas hipóteses, tais como a distribuição de probabilidades \emph{a
priori} e a dependência entre as variáveis. No entanto, foram modeladas
apenas dependências teóricas entre as variáveis (independência total,
dependência total e 50\% de independência).

A escolha da distribuição \emph{a priori} das variáveis influencia no
resultado final das simulações. No entanto, entende-se que uma correta
definição da correlação entre as variáveis seja de maior importância,
especialmente quando se trata de análises de risco de um empreendimento.

Segundo \textcite{matloff2017}, o \emph{princípio da ocorrência
frequente de eventos extremos} afirma que, apesar da pequena
probabilidade de ocorrência de um evento, sendo o número de repetições
de uma experiência suficientemente alto, a tendência é que este evento
ocorrerá. De maneira que considera-se que a distribuição uniforme de
probabilidades \emph{a priori} para cada variável seja uma boa hipótese.

\ldots{}

A relação entre as variáveis, no entanto, pode ser

Outra importante observação que se deve fazer é como o mercado em que o
empreendimento se encontra afeta a correlação entre estas variáveis.

Num mercado de concorrência perfeita, por exemplo, um aumento de custos
não poderá ser repassado ao consumidor sem perda de \emph{market share},
ou seja, a tendência é que as firmas absorvam o aumento de custos
perdendo margem de lucro.

Já num mercado monopolista, a tendência é repassar o aumento de preços
ao consumidor, mantendo-se os lucros do monopolista.

\textcite{gahochheim}, a partir de simulações com diversos cenários,
concluiu que a variável fluxo de vendas foi a variável mais impactante
na formação do valor de um empreendimento hipotético. No entanto, como
salienta \textcites{gahochheim}[119]{gahochheim}, esta variável está
intimamente correlacionada com a variável taxa mínima de atratividade, o
que torna importante a correta definição desta taxa.

\subsection{A variável tempo}\label{a-variuxe1vel-tempo}

O método involutivo é geralmente aplicado em um momento anterior ao
início da implantação de um empreendimento, sendo que muitas vezes este
empreendimento hipotético nem é realizado.

Obviamente que, como em qualquer avaliação de imóveis, seja por qual
método, a avaliação é válida para uma dada data de referência.

Isso não significa, no entanto, que não se possa elaborar diferentes
cenários onde haja alteração de uma ou mais variáveis macroeconomicas ao
longo do decorrer do empreendimento, verificando o impacto da mudança
destas variáveis na viabilidade do mesmo.

Isto seria extramente útil, inclusive, na análise do risco de um
empreendimento.

\subsubsection{A taxa mínima de
atratividade}\label{a-taxa-muxednima-de-atratividade}

Na prática atual da Engenharia de Avaliações convencionou-se o
estabelecimento de uma taxa mínima de atratividade igual a um múltiplo
da taxa de juros de curto prazo, ou taxa Selic, estabelecida pelo
\gls{Bacen}, usualmente feito igual a 3.

No entanto, considera-se que este parâmetro é injustificado e pode ser,
no mínimo, contestado, ainda mais em se tratando de uma possível perícia
judicial para o estabelecimento do valor de um imóvel.

Ora, a taxa mínima de atratividade de um empreendimento qualquer não
pode ser inferior a taxa dos títulos de dívida pública do governo
\textbf{de \emph{duration} similar à duração do empreendimento}, haja
vista que o investimento em títulos públicos pode ser considerado um
investimento livre de risco.
\begin{table}

\caption{\label{tab:TD}Preços e taxas de referência dos títulos públicos}
\centering
\begin{tabular}[t]{>{\raggedright\arraybackslash}p{5cm}cr>{\raggedleft\arraybackslash}p{1.5cm}r}
\toprule
Título & Vencimento & \% a.a. & Valor Mínimo & P. U.\\
\midrule
\rowcolor{gray!6}  \addlinespace[0.3em]
\multicolumn{5}{l}{\textbf{Indexados ao IPCA}}\\
\hspace{1em}Tesouro IPCA+ 2026 & 15/08/2026 & 3,85 & R\$52,06 & R\$2.603,44\\
\hspace{1em}Tesouro IPCA+ 2035 & 15/05/2035 & 4,41 & R\$34,55 & R\$1.727,81\\
\hspace{1em}Tesouro IPCA+ 2045 & 15/05/2045 & 4,41 & R\$33,70 & R\$1.123,56\\
\hspace{1em}Tesouro IPCA+ com Juros Semestrais 2030 & 15/08/2030 & 4,00 & R\$38,84 & R\$3.884,89\\
\hspace{1em}Tesouro IPCA+ com Juros Semestrais 2040 & 15/08/2040 & 4,32 & R\$40,74 & R\$4.074,36\\
\hspace{1em}Tesouro IPCA+ com Juros Semestrais 2055 & 15/05/2055 & 4,42 & R\$43,05 & R\$4.305,38\\
\rowcolor{gray!6}  \addlinespace[0.3em]
\multicolumn{5}{l}{\textbf{Prefixados}}\\
\hspace{1em}Tesouro Prefixado 2023 & 01/01/2023 & 5,63 & R\$34,39 & R\$859,98\\
\hspace{1em}Tesouro Prefixado 2026 & 01/01/2026 & 7,43 & R\$33,10 & R\$662,07\\
\hspace{1em}Tesouro Prefixado com Juros Semestrais 2031 & 01/01/2031 & 8,08 & R\$34,70 & R\$1.156,75\\
\rowcolor{gray!6}  \addlinespace[0.3em]
\multicolumn{5}{l}{\textbf{Indexados à Taxa Selic}}\\
\hspace{1em}Tesouro Selic 2025 & 01/03/2025 & 0,03 & R\$105,57 & R\$10.557,81\\
\bottomrule
\multicolumn{5}{l}{\textsuperscript{*} \href{http://www.tesouro.fazenda.gov.br/tesouro-direto-precos-e-taxas-dos-titulos}{Tesouro Direto}}\\
\end{tabular}
\end{table}
Ou seja, estabelecido um cronograma para o empreendimento como um todo,
incluindo a fase de vendas, a taxa livre de risco pode ser determinada
com precisão. Porém, a taxa livre de risco apenas é estabelecida no
momento da aplicação do método, desconsiderando-se que esta taxa tem
livre flutuação e grande impacto na formação do preço final do imóvel.

Resta ainda a determinação da taxa de risco a ser embutido na taxa
mínima de atratividade, juntamente com a taxa livre de risco, através,
por exemplo, do \emph{Capital Asset Pricing Model}, como descrito em
\textcites{gahochheim}[69-73]{gahochheim}, de acordo com a fórmula
abaixo:

\[\gls{R_e}= \gls{R_f} + \beta(\gls{R_m} - R_f)\]

Inexiste, para tal, método exato de cálculo, como salienta
\textcite{gahochheim}, donde deve-se permitir alguma variação para a
entrada da taxa mínima de atratividade nas simulações.

Porém, não apenas a variação devido à inexatidão da taxa de risco
deveria ser embutida nas simulações, mas também uma variação devido à
possíveis flutuações da taxa livre de risco.

Outra questão de suma importância é que, assim como a taxa mínima de
atratividade influencia no cálculo do valor do imóvel através da
variável fluxo de vendas, através da mudança da taxa de desconto
aplicada ao fluxo de caixa para o cálculo do valor presente, ela também
exerce influencia sobre o preço de venda das unidades, como visto no
capítulo \ref{economia}, impactando também o fluxo de caixa do
empreendimento, desta vez não pela variação da taxa de desconto
aplicada, mas pela variação da magnitude do valor das vendas.

\subsubsection{O valor de venda}\label{o-valor-de-venda}

Usualmente, para aplicação no método involutivo, é considerado um valor
de venda fixo, calculado através do método comparativo direto de dados
de mercado (MCDDM), feito em momento anterior ao empreendimento. Alguma
variação dos preços de venda é permitida apenas dentro do intervalo de
confiança do preço calculado à partir do MCDDM.

Esta análise é questionável, pois o MCDDM é um retrato do mercado para
aquele momento, \emph{i.e.}, ele é válido apenas para a data de
referência em que foi aplicado, mas o valor de venda pode variar durante
a execução do empreendimento, ou ainda durante as vendas dos imóveis,
após a conclusão do mesmo.

Especialmente, o valor de venda, como visto no capítulo \ref{economia},
está intimamente relacionado com as variáveis macroeconômicas, em
especial à taxa de juros de longo prazo, e de maneira não-linear. No
entanto, é fácil demonstrar que a linearização da relação entre o valor
justo de um imóvel e a taxa de juros pode ser realizada pela simples
transformação do valor pela função inversa, como é mostrado na figura
\ref{fig:valores-juros-rec}.
\begin{figure}[H]

{\centering \includegraphics[width=0.7\linewidth]{images/valores-juros-rec-1} 

}

\caption{Variação do valor justo de um imóvel em função da taxa de juros.}\label{fig:valores-juros-rec}
\end{figure}
Outra maneira de se linearizar a relação entre as variáveis é através da
transformação de ambas pela função logarítmica, como mostra a figura
\ref{fig:valores-juros-log}.
\begin{figure}[H]

{\centering \includegraphics[width=0.7\linewidth]{images/valores-juros-log-1} 

}

\caption{Variação do valor justo de um imóvel em função da taxa de juros.}\label{fig:valores-juros-log}
\end{figure}
\subsection{Estimação de intervalos de confiança para os valores à
partir das simulações de Monte
Carlo}\label{estimauxe7uxe3o-de-intervalos-de-confianuxe7a-para-os-valores-uxe0-partir-das-simulauxe7uxf5es-de-monte-carlo}

\hypertarget{copulas}{\chapter{O Método Copulas}\label{copulas}}

Como mostrado no capítlo \ref{economia}, os fatores que levaram à crise
imobiliário financeira pode ser assim resumidos:
\begin{itemize}
\item
  O \textbf{risco moral} presente na análise de risco de financiamento
  de imóveis devido ao surgimento de ativos financeiros (\gls{CDO}) que
  possibilitavam às instituições hipotecárias transferir o risco dos
  seus empréstimos aos compradores destes ativos, o que fez com que a
  análise de capacidade de pagamento dos mutuários fosse negligenciada;
\item
  O \textbf{risco moral} devido à longa cadeia entre os empréstimos
  originais e as pessoas/instituições que acabavam assumindo o risco
  final dos \emph{default}, o que tornou o entendimento do risco por
  parte dos compradores de \gls{CDO} quase impossível, tendo estes que
  confiar quase que cegamente nas classifações das agências de risco de
  crédito;
\item
  O \textbf{risco moral} nas agências de classificação de risco, por
  conflito de interesse, já que ao mesmo tempo em que aconselhavam seus
  clientes como securitizar os seus produtos, também faziam o
  \emph{rating} destes mesmos produtos;
\item
  O \textbf{risco moral} na crença de instituições consideradas grandes
  demais para quebrar, que seriam sempre socorridas pelos governos, em
  última instância;
\item
  O \textbf{excesso de liquidez} do sistema que levou à formação de uma
  bolha especulativa no setor imobiliário;
\item
  A \textbf{falta de controles adequados de risco} por parte de algumas
  instituições que negociavam os derivativos de crédito, assim como as
  agências de classificação de risco de crédito, que se basearam somente
  no modelo da Copula Gaussiana, subestimando a probabilidade de
  \emph{default} em \emph{clusters}, o que eventualmente levou à falta
  de liquidez pelo não pagamento dos coupons dos \emph{senior tranches}
  dos gls\{CDO\}, e portanto à eclosão da crise financeira.
\end{itemize}
Este capítulo tem como objetivo demonstrar os problemas teóricos que
levaram a este último fator, que está na raiz da formação da bolha
imobiliária.

Deve-se lembrar que, se houvesse controle adequado do risco, não haveria
mercado para os \gls{CDO}, especialmente para os lastreados em créditos
do tipo \emph{subprime}, que estão na raiz da crise. Sem possiblidade de
transferência de riscos, as instituições hipotecárias provavelmente não
teriam efetuados tantos empréstimos \emph{subprime}, pois sentiriam a
necessidade de controlar o seu próprio risco de \emph{default}. Por fim,
se não houvesse tanta liquidez devido a todo o crédito oferecido pelas
instituições hipotecárias, muito provavelmente não teria se formado uma
bolha especulativa.

Segundo \textcite{copulas}, Copulas são funções que permitem separar as
distribuições marginais da estrutura de dependência de uma dada
distribuição multivariada.

Por definição, uma copula d-dimensional é uma função de distribuição de
probabilidade cumulativa com distribuições marginais uniformes
\autocite[1]{copulas}.

\section{Teorema de Sklars}\label{teorema-de-sklars}

\section{Tipos de Copulas}\label{tipos-de-copulas}

Em tese, qualquer distribuição multivariada pode ser utilizada para
construir uma Copula.

A Copula Gaussiana, como visto no capítulo \ref{economia}, foi
vastamente utilizada na precificação de derivativos financeiros dos
empréstimos imobiliários, os \gls{CDO}s.

Segundo \textcite{devil}, a Copula Gaussiana foi adotada
``entusiasticamente'' pela indústria (financeira) pela sua simplicidade,
embora ela não seja capaz de captar as principais características do que
se pretendia modelar.

Isto ocorre porque a Copula Gaussina tende a minimizar a probabilidade
de ocorrência de eventos extremos, \emph{i.e} nas caudas da distribuição
de probabilidade.

\subsection{Copulas elípticas}\label{copulas-eluxedpticas}

\subsubsection{A Copula gaussiana}\label{a-copula-gaussiana}

Segundo \textcite{formula}, a primeira aparição de modelos hoje
conhecidos como Copula Gaussiana se deu entre 1987 e 1991, por Oldrich
Vasicek, um matemático teórico.

\subsubsection{A Copula t}\label{a-copula-t}

\subsection{Copulas Arquimedianas}\label{copulas-arquimedianas}

\subsubsection{A Copula Clayton}\label{a-copula-clayton}

\subsubsection{A Copula de Gumbel}\label{a-copula-de-gumbel}

\subsubsection{Outras Copulas
Arquimedianas}\label{outras-copulas-arquimedianas}

\section{Medidas de dependência}\label{medidas-de-dependuxeancia}

\subsection{Coeficiente de correlação de
Pearson}\label{coeficiente-de-correlauxe7uxe3o-de-pearson}

\[Corr(X, Y ) = \frac{Cov(X, Y )}{Var(X)Var(Y)}\] \#\#\#\# Propriedades

\subsection{Coeficiente de correlação local de
Pearson}\label{coeficiente-de-correlauxe7uxe3o-local-de-pearson}

\[\rho_{local}(x, y) = \frac{E[(X - E(X|Y = y))(Y - E(Y|X = x))]}{\sqrt{E(X - E(X|Y = y))^2 E(Y - E(Y |X = x))^2}}\]

\subsubsection{Propriedades}\label{propriedades}

\subsection{Coeficiente de correlação de
Spearman}\label{coeficiente-de-correlauxe7uxe3o-de-spearman}

\[\rho(X, Y ) = 3P[(X - X')(Y - Y') > 0] - P[(X - X')(Y - Y'' ) < 0]\]

\subsubsection{Propriedades}\label{propriedades-1}

\subsection{Coeficiente de Kendall}\label{coeficiente-de-kendall}

\[\tau(X, Y ) = P[(X - X')(Y - Y')) > 0] - P[(X - X')(Y - Y') < 0]\]

\subsubsection{Propriedades}\label{propriedades-2}

\hypertarget{metodologia}{\chapter{Metodologia}\label{metodologia}}

\hypertarget{resultados}{\chapter{Resultados}\label{resultados}}

\hypertarget{conclusao}{\chapter{Conclusão}\label{conclusao}}

As conclusões devem responder às questões da pesquisa, em relação aos
objetivos e às hipóteses. Devem ser breves, podendo apresentar
recomendações e sugestões para trabalhos futuros.

\postextual

\begingroup

\printbibliography[title=REFERÊNCIAS]

\endgroup

\markboth{Referências}{REFERÊNCIAS}

\hypertarget{refs}{}

\bapendices

\chapter{Eficiência Marginal do
Capital}\label{eficiuxeancia-marginal-do-capital}

\section{A função investimento e a eficiência marginal do
capital}\label{a-funuxe7uxe3o-investimento-e-a-eficiuxeancia-marginal-do-capital}

Para Bresser-Pereira \autocite*[3]{Bresser-Pereira1973}, \emph{a
determinação da variável estratégica a determinar o volume de
investimentos torna-se de extraordinária importância}.

Segundo Bresser-Pereira \autocite*[3]{Bresser-Pereira1973}, \emph{a
tradição clássica de dar primazia a taxa de lucros foi abandonada pelos
neoclássicos, que colocaram a taxa de juros no centro do seu sistema}.
Posteriormente, foi Keynes quem \emph{restabeleceu, até um certo ponto,
a importância da taxa de lucros, através do conceito de eficiência
marginal do capital}.

Para \textcite{Bresser-Pereira1973}, ``a teoria ortodoxa\footnote{Bresser
  define como economistas ortodoxos os economistas neoclássicos e os
  keynesianos, no contexto do trabalho citado.} sobre a função
investimento afirma que a acumulação de capital depende da taxa de lucro
prevista (ou eficiência marginal do capital) da taxa de juros, dado o
nível da renda'', com uma relação inversa, ou seja, à medida que aumenta
o volume de investimentos, cai a eficiência marginal do capital,
conforme pode ser observado na figura \ref{fig:eficienciamarginal}
\autocite[4]{Bresser-Pereira1973}:
\begin{figure}[H]

{\centering \includegraphics[width=0.8\linewidth]{images/Page-4-Image-1} 

}

\caption{Eficiência Marginal do Capital e Investimento.}\label{fig:eficienciamarginal}
\end{figure}
Uma das possíveis explicações para esta relação inversa pode ser vista
no trecho abaixo:
\begin{citacao}
Há, portanto, uma relação inversa entre o volume dos investimentos e a
eficiência marginal do capital. Podemos, inclusive, imaginar que as empresas ou
os empresários disponham sempre de um "estoque" de projetos de investimentos,
com taxas diferentes e declinantes de lucro. Quanto maiores fossem os
investimentos efetivamente realizados, mais seria preciso descer na escala de
rentabilidade prevista dos projetos \ldots Será interessante para a empresa
investir enquanto ela puder esperar do novo investimento um retorno superior ou
pelo menos igual ao da taxa de juros do mercado
\cite[p.~5]{Bresser-Pereira1973}.
\end{citacao}
A citação acima implica que também haverá uma relação entre a taxa de
juros de mercado e o volume de investimentos, novamente em uma relação
inversa, haja vista que quanto menor for a taxa de juros de mercado,
maior será o volume de investimentos.

A diferença básica entre a taxa de juros de mercado e a taxa de lucros
(ou eficiência marginal do capital), segundo
\textcite{Bresser-Pereira1973}, é que, enquanto a taxa de lucros é
dependente do volume de investimentos, a taxa de juros de mercado é uma
variável independente.
\begin{citacao}
Em outras palavras, é a variação dos investimentos que leva à variação da
eficiência marginal do capital, enquanto que é a variação da taxa de juros que
leva à variação do volume de investimentos
\cite{Bresser-Pereira1973}.
\end{citacao}
Segundo \textcite{Bresser-Pereira1973}, a eficiência marginal do capital
varia conforme o nível de otimismo dos empresários. A ``distinção entre
a eficiência marginal do capital, dado um determinado nível de otimismo
dos empresários, \(r\), e a eficiência marginal do capital com
diferentes níveis de otimismo, quanto às suas perspectivas de lucro,
\(r’\)'', pode ser vista na figura \ref{fig:eficienciamarginal2}:
\textit{fixada uma taxa de juros em um
determinado nível $j_1$, podemos, então, deduzir graficamente uma nova função
investimento, relacionando positivamente o volume de investimentos, dado um
nível de renda, com a influência marginal do capital, $r’$, a diferentes níveis de
otimismo} \cite[p.~8]{Bresser-Pereira1973}:
\begin{figure}[h]
\begin{center}
\includegraphics[width=.8\textwidth]{images/Page-8-Image-3.png}
\includegraphics[width=.8\textwidth]{images/Page-8-Image-4.png}
\end{center}
\caption{A nova função Investimento.}
\label{fig:eficienciamarginal2}
\end{figure}
\begin{citacao}
Através dos mecanismos ortodoxos da política monetária e fiscal, e dos
mecanismos menos ortodoxos da política salarial, da política cambial, da
política fiscal ampliada, que inclui subsídios os mais variados, o Governo tem
condições crescentes de influenciar direta ou indiretamente as perspectivas de
lucro dos empresários. Por outro lado, as variações no nível de segurança
política para os investimentos, tão grandes no mundo moderno, devem também fazer
variar grandemente o nível de otimismo dos empresários em relação a suas
perspectivas de lucro
\cite[p.~9]{Bresser-Pereira1973}.
\end{citacao}
Segundo Rangel (\emph{apud} \textcite{pereira}), \emph{a eficácia
marginal do capital das empresas com capacidade ociosa é negativa e,
pela lógica, é essa eficácia que deve 3 orientar a taxa de juros}.

\eapendices

\banexos

\chapter{Artigo Valor Econômico}\label{artigo-valor-econuxf4mico}

\eanexos

% ----------------------------------------------------------
% Glossário
% ----------------------------------------------------------
%
% Consulte o manual da classe abntex2 para orientações sobre o glossário.
%
%\glossary

%---------------------------------------------------------------------
% INDICE REMISSIVO
%---------------------------------------------------------------------
\phantompart
\printindex
%---------------------------------------------------------------------

\end{document}
